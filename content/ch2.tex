\section{Category Theory}
\subsubsection{Initial Definitions}
\begin{definition}
    A category $\pazocal{C}$ is a pair $(\Ob(\pazocal{C}),\Hom(\pazocal{C}))$ where 
    \begin{enumerate} 
    \item $\Ob(\pazocal{C})$ denotes a class of \textit{objects}. \item $\Hom(\pazocal{C})$ denotes a class of \textit{morphisms}. 
    \item A morphism $f$ in $\Hom(\pazocal{C})$ is a relation between between elements $A,B$ in $\Ob(\pazocal{C})$. We denote it by $f : A \rightarrow B$. 
    \item For objects $A,B$ in $\Ob(\pazocal{C})$ we denote the class of morphisms from $A$ to $B$ by $\Hom(A,B)$.
    \item There is a binary operation $\circ$ on the class of morphisms called \textit{composition} such that for morphisms $f: B\rightarrow C$ and $g : A \rightarrow B$ we have that
    $$fg := f\circ g : A \rightarrow C$$
    and 
    $$(f\circ g)\circ h = f\circ(g\circ h)$$
    where $f : C \rightarrow D$, $g: B \rightarrow C$ and $h: A \rightarrow B$ for objects $A,B,C,D$ in $\Ob(\pazocal{C})$. Furthermore for each object $X$ in $\Ob(\pazocal{C})$ there is a morphism $\mathbbm{1}_X : X \rightarrow X$ called the \textit{identity morphism} such that 
    $$\mathbbm{1}_B f = f = f \mathbbm{1}_A,$$
    for a morphism $f: A \rightarrow B$.
    \end{enumerate}
\end{definition}
\begin{definition}
    Let $\pazocal{C}$ be a category. An \textit{isomorphism} $f:A \rightarrow B$ is a morphism in $\Hom(\pazocal{C})$ such that there is another morphism $f^{-1} : B\rightarrow A$ satisfying,
    $$ff^{-1} = \mathbbm{1}_B \ \mathrm{ and } \ f^{-1}f = \mathbbm{1}_A.$$
\end{definition}
\begin{definition}
    A category $\pazocal{C}$ is called a \textit{groupoid} if every $f$ in $\Hom(\pazocal{C})$ is an isomorphism
\end{definition}
\begin{example}\label{MonoidIsACategoryGroupIsGroupoid}
    Let $(M,\cdot,e)$ be a monoid. We define a category $\mathrm{B}M$ whose class of objects is $\{M\}$ and whose morphisms are $n,m\in M$ and where composition of morphisms $n,m\in M$ is given by 
    $$nm := n\cdot m.$$  
    Since $M$ is a monoid for $m_1,m_2,m_3\in M$, we have 
    $$m_1(m_2m_3)=(m_1m_2)m_3.$$
    The identity morphism is $e$. When $(G,\dot,e)$ is a group we get that every morphism is an isomorphism, since for each $g\in G$ there is a $g^{-1}\in G$ such that 
    $$gg^{-1}=g^{-1}g = e$$ 
\end{example}
\begin{definition}
    A \textit{subcategory} $\pazocal{D}$ of a category $\pazocal{C}$ is a subclass of $\Ob(\pazocal{C})$ together with a subclass of $\Hom(\pazocal{C})$ that constitutes a category
\end{definition}
\begin{remark}
    equivalently a subcategory of $\pazocal{C}$ is a subclass $\Ob(\pazocal{D})$ of $\Ob(\pazocal{C})$ and a subclass $\Hom(\pazocal{D})$ of $\Hom(\pazocal{C})$ such that each domain $A$ and codomain $B$ for a morphism in $\Hom(\pazocal{D})$, $A,B$ are elements of $\Ob(\pazocal{D})$. In addition such that $\Hom(\pazocal{D})$ is closed under composition. 
\end{remark}
\begin{definition}
    The \textit{maximal groupoid} of a category $\pazocal{C}$ is the subcategory of $\pazocal{C}$ whose objects are $\Ob(\pazocal{C})$ and whose morphisms are the isomorphisms of $\Hom(\pazocal{C})$
\end{definition}
\begin{remark}
    The maximal groupoid is a subcategory. Indeed, The domain and codomain of an isomorphism are trivially in $\Ob(\pazocal{C})$. Suppose $f:A\rightarrow B$ and $g: B \rightarrow C$ are isomorphisms. Then 
    $$f^{-1}g^{-1}gf=f^{-1}\fone_B f = f^{-1}f = \fone_A$$
    and 
    $$gff^{-1}g^{-1} = g\fone_B g^{-1} = gg^{-1} = \fone_C.$$
    Hence $gf: A \rightarrow C$ is an isomorphism with inverse $f^{-1}g^{-1}$. 
\end{remark}
\begin{definition}
    Let $\pazocal{C}$ be a category and $f\in \Hom(A,B)$, $g\in \Hom(B,A)$.  $f$ is called a \textit{retraction} of $g$ and $g$ a section of $f$ if $fg = \fone_B$   
\end{definition}
\begin{lemma}\label{AtMostOneIso}
    Let $\pazocal{C}$ be a category, $f\in \Hom(A,B)$. Suppose $g,h\in \Hom(B,A)$ are respectively a retraction and a section of $f$. Then $g=h$ and $f$ is an isomorphism. It follows that a morphism can have at most one inverse
\end{lemma}
\begin{proof}
    Indeed 
    $$g=g\fone_A = gfh=\fone_B h = h.$$
    Hence $f$ is an isomorphism with $f^{-1} = g=h$. Let $f_1$ and $f_2$ be inverses of an isomorphism $f$. Note that both $f_1$ and $f_2$ is both a section and a retraction. Therefor, by the first statement, $f_1=f_2$.   
\end{proof}
\begin{example}\label{BasicExamplesOfCategories}
    \begin{enumerate}
        \item Let $\mathrm{Set}$ be defined by objects being sets and morphisms being functions. Indeed, letting $\circ$ be composition in the conventional way and letting $\mathbbm{1}_X = \mathrm{id}_X: X\rightarrow X, x\mapsto x$, we see that this indeed defines a category. 
        \item Consider a pair $(X,R)$ of a set $X$ and a transitive, reflexive relation $R$ on $X$. Let $(a,b),(b,c)\in R$. We define 
        $$(a,b)(b,c) := (a,c).$$
        This is indeed well defined since $a R b$ and $bR c$ implies $a R c$, in other words $(a,c)\in R$. Let another pair $(c,d)\in R$. Then $((a,b)(b,c))(c,d)= (a,c)(c,d) = (a,d)$ and $(a,b)((b,c),(c,d))=(a,b)(b,d)=(a,d)$. We define $\mathbbm{1}_a = (a,a)$, which is indeed in $R$. Then $(a,a)(a,b)=(a,b)$ and $(a,b)(b,b)=(a,b)$. So $(X,R)$ indeed defines a morphism. 
    \end{enumerate}
\end{example}
\begin{definition}
    Let $\pazocal{C}$ be a category and $A$ an object in $\pazocal{C}$. \textit{The slice category of $\pazocal{C}$ under $A$} denoted $A/\pazocal{C}$ is the category whose objects are morphisms in $\Hom(\pazocal{C})$ with domain $A$ and where a morphism from $f: A \rightarrow X$ and $g : A \rightarrow Y $ is a map $h : X\rightarrow Y$ such that 
    $$\begin{tikzcd}
       & A \arrow[ld, "f"' ] \arrow[rd, "g"]\\
        X \arrow{rr}{h} & & Y
    \end{tikzcd}$$
    commutes. \textit{The slice category of $\pazocal{C}$ over $A$} denoted $\pazocal{C}/A$ has morphisms with codomain $A$ as objects and a morphism from $f : X \rightarrow A$ to $g : Y\rightarrow A$ is a morphism $h : X \rightarrow Y$ satisfying
    $$\begin{tikzcd}
        X \arrow[rd, "f"'] \arrow[rr, "h"] & & Y\arrow[ld, "g"]\\
        & A 
    \end{tikzcd}$$
\end{definition}
\begin{remark}
    Both these constructions are indeed categories: Consider morphisms $h_{12}$ between $f_1: A\rightarrow X$ \& $f_2 : A\rightarrow Y$ and $h_{23}$ between $f_2 : A\rightarrow Y$ \& $f_3: A\rightarrow Z$. Then we have commutative diagrams
    $$\begin{tikzcd}
        & A \arrow[ld, "f_1"' ] \arrow[rd, "f_2"]\\
        X \arrow{rr}{h_{12}} & & Y
    \end{tikzcd} \quad \begin{tikzcd}  & A \arrow[ld, "f_2"' ] \arrow[rd, "f_3"]\\
        Y \arrow{rr}{h_{23}} & & Z \end{tikzcd}$$
    to obtain the commutative diagram
    $$\begin{tikzcd}
        & A \arrow[ld, "f_1"' ] \arrow[d, "f_2"] \arrow[rd, "f_3"]\\
        X \arrow[r, "h_{12}"] & Y \arrow[r, "h_{23}"] & Z
    \end{tikzcd}
    $$
    hence $h_{23}h_{12}$ is a morphism between $f_{1}$ and $f_3$. For an object $f: A \rightarrow X$ in $A/\pazocal{C}$ define the identity morphism to be $\fone_X$. We thus get that associativity of composition and the identity morphisms being neutral with respect to composition is inherited from this being true in $\pazocal{C}$. Reversing arrows we get that $\pazocal{C}/A$ is also a category.  
\end{remark}
\begin{definition}
    Let $\pazocal{C}_1,\pazocal{C}_2$ be categories. A \textit{Covariant functor} from $\pazocal{C}_1$ to $\pazocal{C}_2$ is a mapping $\pazocal{F}$, denoted $\pazocal{F} : \pazocal{C}_1 \rightarrow \pazocal{C}_2$, which assigns to each object $A$ in $\Ob(\pazocal{C}_1)$ to an object $\pazocal{F}(A)$ in $\Ob(\pazocal{C}_2)$ and to each morphism in $\Hom(\pazocal{C}_1)$, $f : A \rightarrow B$ a morphism in $\Hom(\pazocal{C}_2)$, $\pazocal{F}(f) : \pazocal{F}(A)\rightarrow \pazocal{F}(B)$ such that 
    \begin{enumerate}
        \item for every object $X$ in $\Ob(\pazocal{C}_1)$, $\pazocal{F}(1_X) = 1_{\pazocal{F}(X)}$.
        \item for every pair of morphisms $f: B \rightarrow C$ and $g: A\rightarrow B$ in $\Hom(\pazocal{C}_1)$, $\pazocal{F}(fg) =\pazocal{F}(f)\pazocal{F}(g).$
    \end{enumerate}
\end{definition}
\begin{lemma}\label{FunctorsTransferIsomorphismBetweenCategories}
    Consider two categories $\pazocal{C}_1$ and $\pazocal{C}_2$ with a functor $\pazocal{F}: \pazocal{C}_1 \rightarrow \pazocal{C}_2$. If $f:A \rightarrow B$ is an isomorphism in $\Hom(\pazocal{C}_1)$, then $\pazocal{F}(f): \pazocal{F}(A)\rightarrow\pazocal{F}(A)$ is an isomorphism in $\Hom(\pazocal{C}_2)$. 
\end{lemma}
\begin{proof}
    Indeed, 
    $$\pazocal{F}(f)\pazocal{F}\left(f^{-1}\right) = \pazocal{F}\left(ff^{-1}\right) = \pazocal{F}(\mathbbm{1}_B)= \mathbbm{1}_{\pazocal{F}(B)} \ \mathrm{and} \ \pazocal{F}\left(f^{-1}\right)\pazocal{F}(f) = \pazocal{F}\left(f^{-1}f\right) = \pazocal{F}(\mathbbm{1}_A)= \mathbbm{1}_{\pazocal{F}(A)}.$$
\end{proof}
\begin{definition}
    Let $\pazocal{C}$ be a category. We define \textit{the opposite category of $\pazocal{C}$} denoted $\pazocal{C}\op$ to be the category with $\Ob(\pazocal{C}\op) := \Ob(\pazocal{C})$ and where a morphism $f : A \rightarrow B$ in $\Hom(\pazocal{C}\op)$ is a morphism $f : B\rightarrow A$ in $\Hom(\pazocal{C})$ 
\end{definition}
\begin{remark}
    The above indeed does define a category. We define $\circ\op$ by 
    $$f \circ\op g = g\circ f : C \rightarrow A$$
    where $f: B\rightarrow C$ and $g: A \rightarrow B$ are morphisms in $\Hom(\pazocal{C}\op)$. Then for morphisms $f: C\rightarrow D$, $g: B \rightarrow C$ and $h: A \rightarrow B$
    $$(f\circ\op g)\circ\op h = h(gf)=(hg)f = f\circ\op(g\circ\op h).$$
    Furthermore, we define the identity morphisms in $\Hom(\pazocal{C}\op)$ to be the identity morphisms in $\Hom(\pazocal{C})$, hence 
    $$f\circ\op \mathbbm{1}_A = \mathbbm{1}_Af = f \text{ and } \mathbbm{1}_B \circ\op f = f \mathbbm{1}_B = f.$$
\end{remark}
\begin{definition}
    Consider categories $\pazocal{C}_1$ and $\pazocal{C}_2$. A \textit{contravariant functor} $\pazocal{F}$ between  $\pazocal{C}_1 $ and $ \pazocal{C}_2$ is a covariant functor between $\pazocal{C}_1$ and $\pazocal{C}_2\op$.
\end{definition}
\begin{corollary}
    Consider categories $\pazocal{C}_1$, $\pazocal{C}_2$ and a covariant functor $\pazocal{F}: \pazocal{C}_1 \rightarrow \pazocal{C}_2\op$. If $f: A \rightarrow B$ is an isomorphism in $\Hom(\pazocal{C}_1)$, then $\pazocal{F}(f): \pazocal{F}(B)\rightarrow \pazocal{F}(A)$ is an isomorphism in $\Hom(\pazocal{C}_2\op)$ 
\end{corollary}
\begin{proof}
    This follows immediately from Lemma~\ref{FunctorsTransferIsomorphismBetweenCategories}.
\end{proof}
\begin{example}\label{IntegerInvariants}
    Suppose that there, for a category $\pazocal{C}$, is a well-defined assignment $\pazocal{F}$ of objects in $\pazocal{C}$ to integers and of a morphism $A\rightarrow B$ to $\pazocal{F}(A)\leq \pazocal{F}(B)$. This will define a functor from $\pazocal{C}$ to $(\Z,\leq)$ called \textit{an integer invariant of objects in $\pazocal{C}$}. Indeed, $\pazocal{F}(A) \leq \pazocal{F}(A)$, hence $\pazocal{F}(\mathbbm{1}_A)=\mathbbm{1}_{\pazocal{F}(A)}$. Given morphisms $g:A\rightarrow B$, $f:B\rightarrow C$ in $\Hom(\pazocal{C})$, 
    $$\pazocal{F}(A)\leq\pazocal{F}(B) \text{ and } \pazocal{F}(B)\leq \pazocal{F}(C),$$
    implying $\pazocal{F}(A)\leq \pazocal{F}(C)$, hence $\pazocal{F}(A\overset{fg}{\rightarrow} C) = \pazocal{F}(A\overset{f}{\rightarrow}B)\pazocal{F}(B\overset{g}{\rightarrow}C)$.
    
\end{example}
\begin{definition}
    A category $\pazocal{C}$ is \textit{locally small} if $\Hom(A,B)$ is a set for every object $A,B$ in $\Ob(\pazocal{C})$. It is \textit{small} if $\Ob(\pazocal{C})$ is a set.
\end{definition}
\begin{definition}
    In a category $\pazocal{C}$ a morphism $f\in \Hom(A,B)$ is a \textit{monomorphism} if for every pair of morphisms $g_1,g_2 \in \Hom(C,A)$, 
    $$fg_1=fg_2\implies g_1 = g_2.$$
    It is called an \textit{epimorphism} if for every pair of morphisms $h_1,h_2\in \Hom(B,D)$,
    $$h_1f=h_2f\implies h_1=h_2$$    
\end{definition}
\begin{remark}
    Note that an epimorphism is just a monomorphism in $\pazocal{C}\op$ and vice versa
\end{remark}\label{IsomorphismIsEpiAndMono}
\begin{lemma}
    Let $\pazocal{C}$ be a category and $f:A\rightarrow B$ a morphism. Then if $f$ is an isomorphism, then it is a monomorphism and an epimorphism.
\end{lemma}
\begin{proof}
    Suppose $f$ is an isomorphism. Let $g_1,g_2: C \rightarrow A$ with $fg_1=fg_2$. Then 
    $$
        g_1=f^{-1}fg_1 = f^{-1}fg_2 = g_2.
    $$
    Using the dual result, we get that for every $h_1,h_2: B,D$, if $h_1f=h_2f$ then $h_1=h_2$. We thus conclude that $f$ is both an epimorphism and a monomorphism.\\
\end{proof}
\begin{definition}
    A category in which the converse of the above lemma holds is called \emph{balanced}
\end{definition}
\begin{lemma}
    In the category of $\mathrm{Set}$, we have for a function $f: A\rightarrow B$
    \begin{enumerate}
        \item $f$ is epi if and only if it is surjective.
        \item $f$ is mono if and only if is injective
    \end{enumerate}
    In particular $\mathrm{Set}$ is balanced. 
\end{lemma}
\begin{proof}
    1. Suppose $f$ epimorphism. Note that in general for a subset $C\subset B$, if for each set $X$ and each pair of maps $\lambda_1,\lambda_2: B\rightarrow X$,  $\left.\lambda_1\right|_{C}=\left.\lambda\right|_C$ implies $\lambda_1 = \lambda_2$, then $C=B$. Our assumption tells us that for each pair of functions $h_1,h_2: B\rightarrow X$, $\left.h_1\right|_{\im\ f} = \left.  h_2\right|_{\im\ f}$ implies $h_1 = h_2$, hence $\im \ f = B$, meaning $f$ is surjective.\\
    Suppose conversely that $f$ is surjective. Let $h_1,h_2\in \Hom(X,B)$ and suppose $h_1 f= h_2 f$. Let $b\in B$. Then there is an $a\in A$ such that $b=f(a)$, hence
    $$h_1(b) = h_1(f(a))=h_1f(a)= h_2f(a)=h_2(f(a))=h_2(b).$$
    Since $a$ was chosen arbitrarily it follows that $h_1=h_2$.\\ 
    2. Suppse $f$ is mono. Suppose $a_1,a_2\in A$ are given such that $f(a_1)=f(a_2)$. Consider the functions, $g_i : \{\ast\} \rightarrow A,\ast\mapsto a_i$. Then 
    $$ 
        fg_1 = fg_2,
    $$
    hence $g_1 = g_2$, which implies $a_1=g_1(\ast)=g_2(\ast)= a_2$.\\
    Suppose $f$ is injective. Consider $g_1,g_2 : C\rightarrow A$ with $fg_1 = fg_2$. Let an arbitrary $x\in C$ be given. Then 
    $$
        f(g_1(x)) = f(g_2(x)) \implies g_1(x) = g_2(x).
    $$ 
    Since $x$ was chosen arbitrarily this means $g_1 = g_2$.    
\end{proof}
\begin{lemma}
    Let $\pazocal{C}$ be a category and $f : A\rightarrow B$ a morphism
    \begin{enumerate}
        \item If $f$ is a section of some $h: B\rightarrow A$, then it is a mono.
        \item If $f$ is a retraction of some $h: B\rightarrow A$, then it is epi.
    \end{enumerate} 
\end{lemma}
\begin{proof}
    1. Consider $g_1,g_2: C\rightarrow A$ and suppose $fg_1 = f g_2$. Since $f$ is a section of $h$, then $hf = \fone_A$. Then 
    $$
        g_1 = (hf)g_1 = h(fg_1) = h (fg_2)= (hf)g_2 = g_2.
    $$
    2. Since a retraction is just a section in the opposite category, this result is dual to 1.
\end{proof}
\begin{remark}
    If a morphism is a section of some morphism, then it is called a \emph{split monomorphism}. If a morphism is a retraction of some morphism it is called a \emph{split epimorphism}.
\end{remark}
\begin{lemma}
    For every category $\pazocal{C}$, a morphism $f : A\rightarrow B$ is an isomorphism if and only if it is mono and split epi. Dually it is an isomorphism if and only if it is mono and split epi.
\end{lemma}
\begin{proof}
    One direction follows from Lemma~\ref{IsomorphismIsEpiAndMono}. Suppose $f$ is mono and split epi. Since it is split epi, there is a $g:B\rightarrow A$ such that $fg = \fone_B$. Note that 
    $$f(gf)=(fg)f=\fone_B f= f = f\fone_A,$$
    hence $gf = \fone_A$ by the assumption that $f$ is mono.
\end{proof}
\begin{lemma}
    Consider a category $\pazocal{C}$. Then 
    \begin{enumerate}
        \item The composition of two monos is a mono.
        \item For $f:A\rightarrow B$ and $g : B\rightarrow C$ if $gf$ is a mono, then so is $f$. 
    \end{enumerate}
    Dually 
    \begin{enumerate}
        \item The composition of two epis is an epi.
        \item For For $f:A\rightarrow B$ and $g : B\rightarrow C$ if $gf$ is an epi, then so is $g$.
    \end{enumerate}
\end{lemma}
\begin{proof}
    1. Consider monos $f: B\rightarrow C$ and $g:A\rightarrow B$. Suppose $h_1,h_2 : X\rightarrow A$ are given such that $(fg)h_1 = (fg)h_2$. Then 
    $$f(gh_1)=f(gh_2)\implies gh_1 = gh_2 \implies h_1 =h_2,$$
    hence $fg$ is a mono.\\
    2. Consider $h_1,h_2 : X \rightarrow A$ such that $fh_1 = fh_2$. Then $(gf)h_1 = (gf)h_2$, hence by assumption, $h_1 = h_2$. We thus conclude that $f$ is mono.
\end{proof}
\begin{remark}
    From the above result it follows that $\Ob(\pazocal{C})$ with morphisms being monos/epis in $\Hom(\pazocal{C})$, defines a subcategory of $\pazocal{C}$.
\end{remark}
\begin{definition}
    Let $\pazocal{C}$ be a locally small category and fix an object $X$ in $\pazocal{C}$. Given a morphism $f:A\rightarrow B$, define 
    \begin{gather*}
        f_\ast : \Hom(X,A) \rightarrow \Hom(X,B)\\
        g\mapsto fg
    \end{gather*}  
    we also define the dual concept 
    \begin{gather*}
        f^\ast : \Hom(A,Y) \rightarrow \Hom(B,X)\\
        h\mapsto hf
    \end{gather*}
\end{definition}
\begin{remark}
    Note that $f\op_\ast = f^\ast$, i.e. for $f\op : B \rightarrow A$, the morphism $f$ interpreted as a morphism of $\pazocal{C}\op$ we have 
    \begin{gather*}
        f\op_\ast : \Hom^{\pazocal{C}\op}(X,B)=\Hom(B,X) \rightarrow \Hom^{\pazocal{C}\op}(X,A)=\Hom(A,X)\\
        h: B\rightarrow X \mapsto f\circ\op h= hf
    \end{gather*}
    hence $f\op_\ast = f^\ast$
\end{remark}
\begin{lemma}
    For a locally small category $\pazocal{C}$ and morphism $f:A\rightarrow B$, the following are equivalent
    \begin{enumerate}
        \item $f$ is an isomorphism
        \item For every object $X$ in $\pazocal{C}$, $f_\ast$ is a bijection. 
        \item For every object $X$ in $\pazocal{C}$, $f^\ast$ is a bijection
    \end{enumerate}
\end{lemma}
\begin{proof}
    "1. $\implies$ 2.": Let $g : B\rightarrow A$ be the inverse of $f$. Then given $h\in \Hom(X,A)$,
    $$(g_\ast f_\ast)h = g_\ast(f_\ast h)= g_\ast(fh)=gfh=h \implies g_\ast f_\ast = \fone_{\Hom(X,A)}$$
    and similarly one proves that $f_\ast g_\ast = \fone_{\Hom(X,B)}$.\\
    "2.$\implies$ 1.": Consider $f_\ast : \Hom(B,A)\rightarrow \Hom(B,B)$. This by assumption has an inverse $\lambda : \Hom(B,B)\rightarrow \Hom(B,A)$. Set $g := \lambda(\fone_B)$. Then for each  
    $$fg = f \lambda(\fone_B)= f_\ast(\lambda(\fone_B))=(f_\ast\lambda)(\fone_B)=\fone_{\Hom(B,B)}(\fone_B)=\fone_B.$$
    Now consider the function 
    \begin{gather*}
        \mu : \Hom(A,A)\rightarrow \Hom(A,B)\\
        h\mapsto fh
    \end{gather*}
    which just another instance of an "$f_\ast$", but we use different notation to not confuse it with $f_\ast: \Hom(B,B)\rightarrow \Hom(B,A)$. Then $\mu$ has an inverse $\mu^{-1}$. Note then that
    $$
        \mu(gf)= f(gf)=(fg)f= \fone_B f = f\fone_A= \mu(\fone_A)
    $$
    hence 
    $$
        gf = (\mu^{-1}\mu)(gf) = \mu^{-1}(\mu(gf)) = \mu^{-1}(\mu(\fone_A))= (\mu^{-1}\mu)(\fone_A) = \fone_A.
    $$
    "1. $\iff$ 3.": This is dual to "1. $\iff$ 2.".\\
\end{proof}
\begin{lemma}
    Consider a locally small category $\pazocal{C}$ and a morphism $f:A\rightarrow B$.
    \begin{enumerate}
        \item $f$ is a split epimorphism if and only if for each $X$, $f_\ast: \Hom(X,A) \rightarrow \Hom(X,B)$ is surjective. 
        \item Dually, $f$ is a split monomorphism if and only if for each $X$, $f^\ast : \Hom(B,X)\rightarrow \Hom(A,X)$ is injective. 
    \end{enumerate}
\end{lemma}
\begin{proof}
    1. Suppose $f$ is a split epi. Then $f$ has a right inverse $g$, i.e. $fg = \fone_B$. Let $h: X \rightarrow B$ be given. Then $f_\ast(gh)= f(gh)=(fg)h = h$, hence $f_\ast$ is surjective.\\ 
    Suppose that $f_\ast: \Hom(X,A)\rightarrow \Hom(X,B)$ is surjective for each object $X$. Then $f_\ast: \Hom(B,A) \rightarrow \Hom(B,B)$ is surjective. hence for some $g:B\rightarrow A$, 
    $$
        \fone_B=f_\ast(g) = fg
    $$ 
    hence $f$ is a retraction of $g$ and therefor a split epimorphism. 
    2. This is dual to 1. 
\end{proof}

\subsubsection{Functors}
\begin{definition}
    Consider a locally small category $\pazocal{C}$. We then for each object $X$ in $\pazocal{C}$ get a covariant functor 
    \begin{gather*}
        \Hom(X,\_) : \pazocal{C}\rightarrow \mathrm{Set}\\
        Y\mapsto \Hom(X,Y)\\
        f : A\rightarrow B \mapsto f_\ast: \Hom(X,A)\rightarrow \Hom(X,B) 
    \end{gather*}
    and a contravariant functor
    \begin{gather*}
        \Hom(\_,X) : \pazocal{C} \rightarrow \mathrm{Set}\\
        Y \mapsto \Hom(Y,X)\\
        f : A\rightarrow B \mapsto f^\ast: \Hom(X,B)\rightarrow \Hom(X,A) 
    \end{gather*}
    We say that these are the functors \emph{represented by $X$}.
\end{definition}
\begin{remark}
    Let $f: B\rightarrow C$ and $g:A\rightarrow B$ be given. Then for each $h\in \Hom(X,A)$
    $$
        (fg)_\ast(h)= (fg)h= f(gh)= f_\ast(gh)=f_\ast(g_\ast(h))=f_\ast g_\ast(h) \implies (fg)_\ast=f_\ast g_\ast.
    $$ 
    One easily checks that $(\fone_A) = \fone_{\Hom(X,A)}$. This verifies that $\Hom(X,\_)$ is a covariant functor. Note that $\Hom(\_,X)$ is just the covariant functor 
    \begin{gather*}
        \Hom(X,\_) : \pazocal{C}\op \rightarrow \mathrm{Set}
    \end{gather*}
    and is therefor a contravariant functor from $\pazocal{C}\rightarrow \mathrm{Set}$. 
\end{remark}
\begin{definition}
    For categories $\pazocal{C}$ and $\pazocal{C}'$ we define the \emph{the product category of $\pazocal{C}$ and $\pazocal{C}'$} to be category $\pazocal{C}\times \pazocal{C}'$ whose objects are ordered pairs $(X,X')$ where $X$ is an object of $\pazocal{C}$ and $X'$ is an object of $\pazocal{C}'$ and whose morphisms are ordered pairs of morphisms, denoted $(f,f'):(X,X')\rightarrow (Y,Y')$.\\
    Given a pair of morphisms in the product, $(f,f'):(X,X')\rightarrow (Y,Y')$ and $(g,g'): (Y,Y')\rightarrow (Z,Z')$, composition is defined as 
    $$
        (g,g')(f,f') := (gf,g'f') : (X,X')\rightarrow (Z,Z').
    $$
    Given an object $(A,A')$, we define $\fone_{(A,A')} = (\fone_A, \fone_{A'})$    
    {\Large How do we know such ordered pairs exist in general?}
\end{definition}
\begin{remark}
    Consider morphisms $(f,f'):(Z,Z')\rightarrow (W,W')$, $(g,g'): (Y,Y')\rightarrow (Z,Z')$ and $(h,h') : (X,X')\rightarrow (Y,Y')$. Then 
    \begin{align*}
        (f,f')((g,g')(h,h'))&=(f,f')(gh,g'h')=(f(gh),f'(g'h'))= ((fg)h,(f'g')h')\\ &=(fg,f'g')(h,h')
        =((f,f')(g,g'))(h,h').
    \end{align*} 
    Moreover, 
    \begin{align*}
        (h,h')\fone_{(X,X')}= (h\fone_X, h'\fone_{X'}) = (h,h') = (\fone_{Y} h,\fone_{Y'} h')=\fone_{(Y,Y')}(h,h'). 
    \end{align*}
    One can thus conclude that $\pazocal{C}\times \pazocal{C}'$
\end{remark}
\begin{definition}
    Given a locally small category $\pazocal{C}$ we define the \emph{two-sides represented functor}
    \begin{gather*}
        \Hom(\_,\_) : \pazocal{C}\op\times \pazocal{C}\rightarrow \mathrm{Set}\\
        (X,Y) \mapsto \Hom(X,Y)\\
        (f,h) : (A,B)\rightarrow (C,D) \mapsto (f^\ast,h_\ast) : \Hom(A,B)\rightarrow \Hom(C,D)
    \end{gather*}
    where $f: C\rightarrow A$ as a morphism in $\pazocal{C}$ and $(f^\ast,h_\ast)(g) := hgf$ for $g\in \Hom(A,B)$.
\end{definition}
\begin{lemma}
    Consider categories $\pazocal{C},\pazocal{D},\pazocal{E}$ and covariant functors, $\pazocal{F}: \pazocal{C}\rightarrow \pazocal{D}, \pazocal{G}:\pazocal{D}\rightarrow \pazocal{E}$. Then 
    \begin{gather*}
        \pazocal{G}\circ \pazocal{F} : \pazocal{C}\rightarrow \pazocal{E}\\
        X \mapsto \pazocal{G}(\pazocal{F}(X))\\
        f: Y\rightarrow Z \mapsto \pazocal{G}(\pazocal{F}(f)): \pazocal{G}(\pazocal{F}(Y))\rightarrow \pazocal{G}(\pazocal{F}(Z))
    \end{gather*}
    is a functor called the \emph{composition of $\pazocal{G}$ with $\pazocal{F}$}. 
\end{lemma}
\begin{proof}
    Indeed, for morphisms $f: X\rightarrow Y$ and $g: Y\rightarrow Z$,
    \begin{align*}
        (\pazocal{G}\circ \pazocal{F})(gf) &= \pazocal{G}(\pazocal{F}(gf))=\pazocal{G}(\pazocal{F}(g)\pazocal{F}(f))= \pazocal{G}(\pazocal{F}(g))\pazocal{G}(\pazocal{F}(f))\\
        &=(\pazocal{G}\circ\pazocal{F})(g)(\pazocal{G}\circ\pazocal{F})(f).
    \end{align*}
    Moreover,
    $$
        (\pazocal{G}\circ\pazocal{F})(\fone_X)=\pazocal{G}(\pazocal{F}(\fone_X))=\pazocal{G}(\fone_{\pazocal{F}(X)})=\fone_{\pazocal{G}(\pazocal{F}(X))}=\fone_{(\pazocal{G}\circ\pazocal{F})(X)}.
    $$
\end{proof}
\begin{lemma}\label{FunctorCompositionIsAssociative}
    Composition of functors is associative.
\end{lemma}
\begin{proof}
    Consider functors $\pazocal{F}: \pazocal{C}\rightarrow \pazocal{D},\pazocal{G}:\pazocal{B}\rightarrow \pazocal{C},\pazocal{H}: \pazocal{A}\rightarrow \pazocal{B}$. Given an object $A$ in $\pazocal{A}$,
    \begin{align*}
        (\pazocal{F}\circ (\pazocal{G}\circ\pazocal{H}))(A)&= \pazocal{F}((\pazocal{G}\circ\pazocal{H})(A))= \pazocal{F}(\pazocal{G}(\pazocal{H}(A)))=(\pazocal{F}\circ\pazocal{G})(\pazocal{H}(A))\\
        ((\pazocal{F}\circ \pazocal{G})\circ \pazocal{H})(A).
    \end{align*}
    Once also easily checks that for each morphism $f:X\rightarrow Y$ 
    $$
        (\pazocal{F}\circ (\pazocal{G}\circ\pazocal{H}))(f)=((\pazocal{F}\circ\pazocal{G})\circ\pazocal{H})(f), 
    $$
    hence
    $$\pazocal{F}\circ(\pazocal{G}\circ\pazocal{H})=(\pazocal{F}\circ\pazocal{G})\circ\pazocal{H}.$$
\end{proof}
\begin{remark}\label{IdentityFunctor}
    In addition to functor composition being associative, there is an identity functor for each category 
    \begin{gather*}
        \pazocal{ID}_\pazocal{C}: \pazocal{C}\rightarrow \pazocal{C}\\
        X\mapsto X\\
        f : Y\rightarrow Z \mapsto f : Y\rightarrow Z
    \end{gather*}
    for which 
    $$
        \pazocal{F}\pazocal{ID}_\pazocal{C}=\pazocal{F}=\pazocal{ID}_\pazocal{D}\pazocal{F}.
    $$ 
\end{remark}
\begin{example}
    Note that it is not true that if a functor maps a morphism to an isomorphism, then said morphism is an isomorphism. Indeed, consider for example any functor $\pazocal{F}$ from a category $\pazocal{C}$ to the category with one object and one morphism (which will need to be the identity morphism), $\mathbbm{1}$. This will take any object to the one object and any morphism to identity morphism.   
\end{example}
\begin{definition}
    Let $\pazocal{F}:\pazocal{D}\rightarrow \pazocal{C}$ and $\pazocal{G}:\pazocal{E}\rightarrow \pazocal{C}$ be functors. We define the \emph{comma category} $\pazocal{F}\downarrow \pazocal{G}$ whose objects are triples $(D\in \pazocal{D}, E\in\pazocal{E}, f:\pazocal{F}(D)\rightarrow \pazocal{G}(E))$. A morphism $(D,E,f)\rightarrow (D',E',f')$  is a pair of morphisms $(h: D\rightarrow D',k: E\rightarrow E')$ so that
    $$
        \begin{tikzcd}
            \pazocal{F}(D) \arrow[r, "f"] \arrow[d,"\pazocal{F}(h)"] & \pazocal{G}(E) \arrow[d,"\pazocal{G}(k)"]\\
            \pazocal{F}(D')\arrow[r,"f'"] & \pazocal{G}(E')  
        \end{tikzcd}
    $$
    commutes
\end{definition}
\begin{remark}
    Consider morphisms 
    $$(h:D\rightarrow D',k:E\rightarrow E'):(D,E,f)\rightarrow (D',E',f')$$ 
    and 
    $$(h':D'\rightarrow D'',k':E'\rightarrow E''):(D',E',f')\rightarrow (D'',E'',f'').$$ 
    We define composition by 
    $$
        (h',k')(h,k) := (h'h,k'k)
    $$
    To see that this is well defined note that 
    $$
        \begin{tikzcd}
            \pazocal{F}(D) \arrow[r, "f"] \arrow[d,"\pazocal{F}(h'h)"] & \pazocal{G}(E) \arrow[d,"\pazocal{G}(k'k)"]\\
            \pazocal{F}(D'')\arrow[r,"f''"] & \pazocal{G}(E'')
        \end{tikzcd}
    $$
    commutes since $\pazocal{F}(h'h)=\pazocal{F}(h')\pazocal{F}(h)$, $\pazocal{G}(k'k)=\pazocal{G}(k')\pazocal{G}(k)$ and the diagram
    $$
        \begin{tikzcd}
            \pazocal{F}(D) \arrow[r, "f"] \arrow[d,"\pazocal{F}(h)"] & \pazocal{G}(E) \arrow[d,"\pazocal{G}(k)"]\\
            \pazocal{F}(D')\arrow[r,"f'"]\arrow[d,"\pazocal{F}(h')"] & \pazocal{G}(E') \arrow[d,"\pazocal{G}(k')"]\\
            \pazocal{F}(D'')\arrow[r,"f''"] & \pazocal{G}(E'')
        \end{tikzcd}
    $$
    commutes. Clearly composition is associative. Define the identity morphism for $(D,E,f)$ to be $(\fone_D,\fone_E)$. Note that this is indeed a morphism in $\pazocal{F}\downarrow \pazocal{G}$, since 
    $$
        \begin{tikzcd}
            \pazocal{F}(D) \arrow[r,"f"]\arrow[d,"\pazocal{F}(\fone_D)=\fone_{\pazocal{F}(D)}",swap] & \pazocal{G}(E)\arrow[d,"\pazocal{G}(\fone_{E})=\fone_{\pazocal{G}(E)}"]\\
            \pazocal{F}(D)\arrow[r,"f"] & \pazocal{G}(E) 
        \end{tikzcd}
    $$
    clearly commutes. It is also clear that this is the identity morphism in $\pazocal{F}\downarrow \pazocal{G}$.
\end{remark}
\begin{definition}
    Given functors $\pazocal{F}:\pazocal{D}\rightarrow \pazocal{C}$ and $\pazocal{G}:\pazocal{E}\rightarrow \pazocal{C}$ we define projection functors 
    \begin{gather*}
        \mathrm{dom}: \pazocal{F}\downarrow \pazocal{G} \rightarrow \pazocal{D}\\
        (D,E,f)\mapsto D\\
        (h,k) :(D_1,E_1,f_1) \rightarrow (D_2,E_2,f_2)\mapsto h: D_1\rightarrow D_2
    \end{gather*} 
    and 
    \begin{gather*}
        \mathrm{cod} : \pazocal{F}\downarrow\pazocal{G} \rightarrow \pazocal{E}\\
        (D,E,f)\mapsto E\\
        (h,k) :(D_1,E_1,f_1) \rightarrow (D_2,E_2,f_2)\mapsto k: E_1\rightarrow E_2
    \end{gather*}
\end{definition}
\begin{remark}
    Note that $C/\pazocal{C}$ is just a comma category; $\pazocal{F}_C\downarrow \pazocal{ID}_\pazocal{C}$, where $\pazocal{F}_C: \mathbbm{1} \rightarrow \pazocal{C}$ is the constant functor from the category with one object and one morphism and $\pazocal{ID}_\pazocal{C}$ is the identity functor on $\pazocal{C}$. Indeed an object in this category is a triple $(\ast\in \mathbbm{1},X\in \pazocal{C},f: C\rightarrow X)$, note that there is a unique such triple when given any morphism $f: C\rightarrow X$, so we may represent such a triple more succinctly by just providing a morphism. A morphism in $\pazocal{F}_C\downarrow \pazocal{ID}_\pazocal{C}$ from $f:C\rightarrow X$ to $g: C\rightarrow Y$ is a pair $(h:X\rightarrow Y, \id_C: C\rightarrow C)$ such that 
    $$
        \begin{tikzcd}
            C \arrow[r, "f"]\arrow[d,"\id_C",swap] & X\arrow[d, "h"]\\
            C \arrow[r, "g"] & Y  
        \end{tikzcd}
    $$
    commutes. Note that the second data point in this pair is redundant and that this is equivalent to 
    $$
        \begin{tikzcd}
            & C\arrow[ld, "f"]\arrow[rd, "g"]\\
            X \arrow{rr}{h} & & Y 
        \end{tikzcd}
    $$
    commuting. So the data specifying the two categories is the same. Note that projection functor simply map a morphism $f: C \rightarrow X$ to its domain and codomain respectively. We can also construct $\pazocal{C}/C$ as $\pazocal{ID}_\pazocal{C}\downarrow \pazocal{F}_C$.  
\end{remark} 
\subsubsection{Categories of Categories}
When our foundations are suitably developed we may consider a category of categories when we have the last lemma of the prior section in mind. Naively one cannot consider such a category, since the question of whether such a category is an object in the category of categories leads to an instance of Russel's paradox. Instead one needs to consider some more refined concepts such as the following, 
\begin{proposition}
    Consider the class of small categories with morphisms being functors. This defines a locally small category denoted $\Cat$. 
\end{proposition}
\begin{proof}
    Note that the class of subclasses of the proper class of all sets is itself a class (in e.g. Morse-Kelley set theory). Note that a functor between categories, whose objects and class of morphisms are sets, is a function. I.e. such a functor 
    is a function 
    $$\Ob(\pazocal{C})\times \Hom(\pazocal{C}) \rightarrow \Ob(\pazocal{C}')\times \Hom(\pazocal{C}').$$
    Then the class of all such functors is a subclass of the class of all functions{\Large ?}. By Lemma~\ref{FunctorCompositionIsAssociative} and Remark~\ref{IdentityFunctor}, $\Cat$ is a category and it is locally small since $\Hom(\pazocal{C},\pazocal{D})$ is a set for each pair of small categories $\pazocal{C},\pazocal{D}$. 
\end{proof}
\begin{remark}
    The collection of all locally small categories with functors as morphisms, which will be denoted $\mathrm{CAT}$, will similarly define a category, which won't be locally small.  
\end{remark}
From the above we get a notion of isomorphism of categories.
\begin{proposition}
    Given a category $\pazocal{C}$ and an object $A$, then 
    $$\pazocal{C}/A \simeq \left(A/\pazocal{C}\op\right)\op$$
\end{proposition}
\begin{proof}
    Define a functor 
    \begin{gather*}
        \pazocal{F} : \pazocal{C}/A \rightarrow \left(A/\pazocal{C}\op\right)\op
    \end{gather*}
    which maps an object $f: X\rightarrow A$ to itself. Note that the objects of $\left(A/\pazocal{C}\op\right)\op$ is the same as the objects of $\left(A/\pazocal{C}\op\right)$, which are morphisms of $\pazocal{C}\op$ with domain $A$. Such morphisms are exactly morphisms of $\pazocal{C}$ with codomain $A$. It follows that the assignment of objects is well-defined. Moreover $\pazocal{F}$ maps a morphism  
    $$
        \begin{tikzcd}
            X \arrow[rd, "f"'] \arrow[rr, "h"] & & Y\arrow[ld, "g"]\\
            & A 
        \end{tikzcd}
    $$ 
    to itself. Indeed, a morphism from $\lambda\in \Hom^{\pazocal{C}\op}(A,X)$ to $\mu\in \Hom^{\pazocal{C}\op}(B,X)$ in $\left(A/\pazocal{C}\op\right)\op$ is a morphism in $A/\pazocal{C}\op$ from $\mu$ to $\lambda$, $\nu\in \Hom^{\pazocal{C}\op}(Y,X)$ such that $\lambda = \nu\circ\op\mu$. In $\pazocal{C}$, this corresponds to $\nu: X\rightarrow Y$ being a morphism satisfying $\lambda = \mu\circ \nu$. This is exactly a morphism in $\pazocal{C}/A$. It follows that $\pazocal{F}$ is well-defined on morphisms as well. The above arguments also clearly shows that the functor 
    \begin{gather*}
        \pazocal{G}: \left(A/\pazocal{C}\op\right)\op \rightarrow \pazocal{C}/A 
    \end{gather*}
    taking objects and morphisms to themselves gives a well-defined functor whose left and right inverse is $\pazocal{F}$. It follows that $\pazocal{C}/A\simeq \left(A/\pazocal{C}\op\right)\op$ at least when this category can be shown to live in some category.  
\end{proof}
\subsubsection{Natural Transformations}
\begin{definition}
    A \emph{natural transformation} of functors $\pazocal{F},\pazocal{G} : \pazocal{C}\rightarrow \pazocal{D}$,
    $$\alpha : \pazocal{F} \Rightarrow \pazocal{G}$$
    is a collection of morphisms 
    $$\left\{\alpha_X : \pazocal{F}(X)\rightarrow \pazocal{G}(X) : X\in\Ob(\pazocal{C})\right\} $$
    called \emph{components} such that for any morphism $f: X\rightarrow X'$,
    $$
        \begin{tikzcd}
            \pazocal{F}(X) \arrow[r, "\alpha_X"] \arrow[d,"\pazocal{F}(f)"] & \pazocal{G}(X)\arrow[d,"\pazocal{G}(f)"]\\
            \pazocal{F}(X') \arrow[r,"\alpha_{X'}"] & \pazocal{G}(X')
        \end{tikzcd}
    $$ 
    commutes
\end{definition}
\begin{remark}
    A natural transformation is also denoted 
    $$
        \begin{tikzcd}
            \pazocal{C} \arrow[r,bend left = 50, "\pazocal{F}"{name=F}] & \pazocal{D} \arrow[l, bend left = 50, "\pazocal{G}"{name=G}] \arrow[Rightarrow,from=F, to=G,"\alpha",shorten >=8pt, shorten <=8pt]
        \end{tikzcd}
    $$
\end{remark}
\begin{definition}
    A \emph{natural isomorphism} is a natural transformation for which every component is an isomorphism. 
\end{definition}
\begin{lemma}
    Suppose $\alpha : \pazocal{F}\Rightarrow \pazocal{G}$ is a natural isomorphism with components $\{\alpha_X\}$. Then $\{\alpha_X^{-1}\}$ defines the components of a natural isomorphism $\alpha^{-1} : \pazocal{G} \Rightarrow \pazocal{F}$. 
\end{lemma}
\begin{proof}
    Let $f: X\rightarrow X'$ be a morphism. We need to check that the diagram 
    $$
        \begin{tikzcd}
            \pazocal{G}(X) \arrow[r,"\alpha_X^{-1}"] \arrow[d,"\pazocal{G}(f)"] & \pazocal{F}(X)\arrow[d,"\pazocal{F}(f)"] \\
            \pazocal{G}(X')\arrow[r,"\alpha_{X'}^{-1}"] & \pazocal{F}(X')
        \end{tikzcd}
    $$
    commutes. Indeed, using the fact that $\alpha_{X'}\pazocal{F}(f)= \pazocal{G}(f)\alpha_X$,
    \begin{align*}
        \pazocal{F}(f)\alpha_{X}^{-1} = \alpha_{X'}^{-1}\alpha_{X'}\pazocal{F}(f)\alpha_X^{-1} = \alpha_{X'}^{-1}\pazocal{G}(f)\alpha_X\alpha_X^{-1}= \alpha_{X'}^{-1}\pazocal{G}(f).
    \end{align*}
\end{proof}
\begin{example}
    Let $\pazocal{C}$ be a locally small category. Consider distinct morphisms... {\Large Do exercise 1.4 iv.}
\end{example}
\subsubsection{Equivalence of Categories}
    \begin{definition}
        Let $\mathbbm{2}$ denote the category with two objects $0,1$ and a single non-identity arrow $0\rightarrow 1$. 
    \end{definition}
    We consider two functors 
    \begin{gather*}
        \iota_i : \mathbbm{1}\rightarrow \mathbbm{2}\\
        \ast \mapsto i\\
        \id_\ast \mapsto \id_i 
    \end{gather*}
    for $i=0,1$. 
    \begin{lemma}
        Fix functors $\pazocal{F},\pazocal{G}:\pazocal{C}\rightarrow \pazocal{D}$. There is a bijection of natural transformations $\alpha: \pazocal{F} \Rightarrow \pazocal{G}$ and \textbf{witnessing functors} $\pazocal{H}:\pazocal{C}\times \mathbbm{2}\rightarrow \pazocal{D}$ such that 
        $$
            \begin{tikzcd}
                \pazocal{C} \arrow[r,"\pazocal{I}_0"]\arrow[rd,"\pazocal{F}",swap] & \pazocal{C}\times \mathbbm{2} \arrow[d,"\pazocal{H}"] & \pazocal{C}\arrow[ld,"\pazocal{G}"]\arrow[l,"\pazocal{I}_1",swap]\\
                & \pazocal{D}
            \end{tikzcd}
        $$
        commutes. Here $\pazocal{I}_i: X\mapsto (X,i), f:Y\rightarrow Z\mapsto (f,\id_i)$
    \end{lemma}
    \begin{proof}
        Let a functor $\pazocal{H}$ be given as described. For each object $X\in \pazocal{C}$, define 
        \begin{gather*}
            \alpha_X^\pazocal{H} = \pazocal{H}(\id_X,0\rightarrow 1) 
        \end{gather*}
        This is a morphism with domain 
        $$
            \pazocal{H}(X,0)=\pazocal{H}\pazocal{I}_0(X)=\pazocal{F}(X)
        $$
        and codomain
        $$
            \pazocal{H}(X,1)=\pazocal{H}\pazocal{I}_1(X)=\pazocal{G}(X).
        $$
        We see that the collection 
        $$ 
            \left\{\alpha_X^\pazocal{H} : \pazocal{F}(X)\rightarrow \pazocal{G}(X) \mid X\in \pazocal{C}\right\}
        $$
        defines the components of a natural transformation, as
        \begin{align*}
            \pazocal{G}(f)\alpha_X &= \pazocal{G}(f)\pazocal{H}(\id_X,0\rightarrow 1) = \pazocal{H\pazocal{I}_1}(f)\pazocal{H}(\id_X,0\rightarrow 1)\\
            &= \pazocal{H}(f,\id_1)\pazocal{H}(\id_X,0\rightarrow 1)\\ 
            &= \pazocal{H}(f,0\rightarrow 1) = \pazocal{H}(\id_{X'},0\rightarrow 1)\pazocal{H}(f,\id_0)\\
            &= \alpha_{X'}\pazocal{H}\pazocal{I}_0(f)=\alpha_{X'}\pazocal{F}(f).
        \end{align*}
        We denote this natural transformation by $\alpha^\pazocal{H}$.
        Given a natural transformation $\alpha: \pazocal{F}\Rightarrow \pazocal{G}$, define a functor 
        \begin{gather*} 
            \pazocal{H}_\alpha : \pazocal{C}\times \mathbbm{2}\rightarrow \pazocal{D}\\
            (X,0) \mapsto \pazocal{F}(X)\\
            (X,1)\mapsto \pazocal{G}(X)\\
            (f,\id_0)\mapsto \pazocal{F}(f)\\
            (f,\id_1) \mapsto \pazocal{G}(f)\\
            (f: X\rightarrow Y,0\rightarrow 1) \mapsto \alpha_X 
        \end{gather*}
        One readily verifies that $\pazocal{F}=\pazocal{H}\pazocal{I}_0$ and that $\pazocal{G}=\pazocal{H}\pazocal{I}_1$. Note that by definition
        $$
            \alpha_X^{\pazocal{H}_\alpha} = \pazocal{H}(\id_X,0\rightarrow 1) = \alpha_X.
        $$
        Suppose we are given two witnessing functors $\pazocal{H},\pazocal{H}'$ with $\alpha^\pazocal{H} = \alpha^{\pazocal{H}'}$.
        For each object $X\in \pazocal{C}$
        $$\pazocal{H}(X,0)=\pazocal{H}\pazocal{I}_0(X)=\pazocal{F}(X)=\pazocal{H}'\pazocal{I}_0(X)=\pazocal{H}'(X,0).$$
        By a similar computation $\pazocal{H}(X,1)=\pazocal{H}'(X,1)$. Moreover, similar computations show that $\pazocal{H}(f,\id_i) = \pazocal{H}'(f,\id_i)$ for $i=0,1$. It remains to check that $\pazocal{H}(f,0\rightarrow 1) = \pazocal{H}'(f,0\rightarrow)$ which readily follows from the fact that $\alpha_X^\pazocal{H}=\alpha_X^{\pazocal{H}'}$ for each $X$.
    \end{proof}
    \begin{definition}
        Consider categories $\pazocal{C}$ and $\pazocal{D}$. An \emph{equivalence of categories} is a pair of functors $\pazocal{F}: \pazocal{C}\rightarrow \pazocal{D}$, $\pazocal{G}:\pazocal{D}\rightarrow \pazocal{C}$ and natural isomorphisms $\alpha : \pazocal{ID}_\pazocal{C} \simeq \pazocal{G}\pazocal{F}$ and $\beta: \pazocal{ID}_\pazocal{D} \simeq \pazocal{F}\pazocal{G}$. We say two categories are \emph{equivalent} if there is an equivalence of categories between them and in this case we may write $\pazocal{C}\simeq \pazocal{D}$. 
    \end{definition}
    \begin{remark}
        Equivalence of categories is an equivalence relation. 
        \begin{itemize}
            \item Two copies of $\pazocal{ID}_\pazocal{C}$ together with two copies of components $\{\id_X : X\in \pazocal{C}\}$ defines an equivalence of categories of $\pazocal{C}$ with itself.
            \item Suppose $\pazocal{C}\simeq \pazocal{D}$. Then there are functors $\pazocal{F}: \pazocal{C}\rightarrow \pazocal{D}$, $\pazocal{G}: \pazocal{D}\rightarrow \pazocal{C}$ together with natural isomorphisms $\alpha: \pazocal{ID}_\pazocal{C}\simeq \pazocal{G}\pazocal{F}$ and $\beta:\pazocal{ID}_\pazocal{D}\simeq \pazocal{F}\pazocal{G}$. This data by definition also expresses that $\pazocal{D}\simeq \pazocal{C}$, when we implicitly use the commutativity of conjuction. 
            \item Suppose $\pazocal{C}\simeq \pazocal{D}$ and $\pazocal{D}\simeq \pazocal{E}$. There are then functors  $\pazocal{F}: \pazocal{C}\rightarrow \pazocal{D}$, $\pazocal{G}: \pazocal{D}\rightarrow \pazocal{C}$ together with natural isomorphisms $\alpha: \pazocal{ID}_\pazocal{C}\simeq \pazocal{G}\pazocal{F}$ and $\beta:\pazocal{ID}_\pazocal{D}\simeq \pazocal{F}\pazocal{G}$. Moreover there are functors $\pazocal{H} : \pazocal{D}\rightarrow\pazocal{E}$, $\pazocal{I}: \pazocal{E}\rightarrow \pazocal{D}$ with natural isomorphisms $\gamma : \pazocal{ID}_{\pazocal{D}} \simeq \pazocal{I}\pazocal{H}$ and $\kappa : \pazocal{ID}_\pazocal{E} \simeq \pazocal{H}\pazocal{I}$. Set 
            $$
                \mathfrak{F} := \pazocal{H}\pazocal{F} \text{ and } \mathfrak{G} := \pazocal{G}\pazocal{I}.
            $$
            Consider
            $$
                \mu := \left\{\mu_X:=\pazocal{G}(\gamma_{\pazocal{F}(X)})\alpha_X : X\rightarrow \mathfrak{GF}(X) \mid X\in \pazocal{C}\right\}
            $$
            This is a natural isomorphism. Indeed, 
            \begin{align*}
                \mathfrak{GF}(f)\mu_X&= \pazocal{GIHF}(f)\pazocal{G}(\gamma_{\pazocal{F}(X)})\alpha_X = \pazocal{G}(\pazocal{IH}(\pazocal{F}(f))\gamma_{\pazocal{F}(X)})\alpha_X\\ 
                &= \pazocal{G}(\gamma_{\pazocal{F}(X')}\pazocal{F}(f))\alpha_X = \pazocal{G}(\gamma_{\pazocal{F}(X')})\pazocal{GF}(f)\alpha_X\\
                &= \pazocal{G}(\gamma_{\pazocal{F}(X')})\alpha_{X'}f = \mu_{X'}f. 
            \end{align*}
            so $\mu$ is a natural transformation of $\pazocal{ID}_{\pazocal{C}}$ to $\mathfrak{GF}$. Moreover, every component of $\mu$ is a composition of isomorphism, hence every component of $\mu$ is an isomorphism. We construct a natural isomorphims $\nu : \pazocal{ID}_\pazocal{E}\simeq \mathfrak{F}\mathfrak{G}$, by considering the collection of morphisms 
            $$
                \nu := \left\{ \nu_Y:= \pazocal{H}(\beta_{\pazocal{I}(Y)})\kappa_Y :Y \rightarrow \pazocal{HFGI}(Y)\mid Y\in\pazocal{E}\right\}.
            $$
            We thus conclude that $\pazocal{C}\simeq \pazocal{E}$.
        \end{itemize}
    \end{remark}
    \begin{definition}
        Let $\pazocal{C},\pazocal{D}$ be locally small categories and $\pazocal{F} : \pazocal{C}\rightarrow \pazocal{D}$ a functor. We say that $\pazocal{F}$ is \emph{full} if $\pazocal{F}(\bullet):\Hom(X,Y)\rightarrow \Hom(\pazocal{F}(X),\pazocal{F}(Y)), f\mapsto \pazocal{F}(f)$ is surjective for each pair of objects $X,Y$ in $\pazocal{C}$. It is \emph{faithful} if $\pazocal{F}(\bullet)$ is injective. If $\pazocal{F}(\bullet)$ is both full and faithful, we say that $\pazocal{F}(\bullet)$ is \emph{fully faithful}.
    \end{definition}
    \begin{definition}
        A functor $\pazocal{F} : \pazocal{C}\rightarrow \pazocal{D}$ is \emph{essentially surjective on objects} if for every $D\in \Ob(\pazocal{D})$ there is a $C\in \Ob(\pazocal{C})$ such that $\pazocal{F}(C)\simeq D$. 
    \end{definition}
    \begin{definition}
        A faithful functor $\pazocal{F}:\pazocal{C}\rightarrow\pazocal{D}$ is an \emph{embedding} if it is injective on objects, i.e. if $\pazocal{F}(C)=\pazocal{F}(C')$ implies $C=C'$ for every pair of objects $C,C'\in \pazocal{C}$. If $\pazocal{F}$ is fully faithful and injective on objects, then it is a \emph{full embedding}.
    \end{definition}
    \begin{lemma}\label{UniqueChangeOfDomainAndCodomainViaIsos}
        Given a morphism $f: A\rightarrow B$ and isomorphisms $\alpha: A\simeq A'$, $\beta : B\simeq B'$ there is a unique morphism $f: A' \rightarrow B'$ so that the diagrams
        $$
            \begin{tikzcd}
                A\arrow[d, "f"] & A'\arrow[l,"\alpha",swap]\arrow[d,"f'"]\\
                B \arrow[r,"\beta",swap] & B' 
            \end{tikzcd}
            \begin{tikzcd}
                A\arrow[r,"\alpha"]\arrow[d, "f"] & A'\arrow[d,"f'"]\\
                B \arrow[r,"\beta",swap] & B'
            \end{tikzcd}
            \begin{tikzcd}
                A\arrow[d, "f"] & A'\arrow[l,"\alpha",swap]\arrow[d,"f'"]\\
                B & B' \arrow[l,"\beta"]
            \end{tikzcd}
            \begin{tikzcd}
                A\arrow[d, "f"]\arrow[r,"\alpha"] & A' \arrow[d,"f'"] \\
                B & B' \arrow[l,"\beta"]
            \end{tikzcd}
        $$
        commute.
    \end{lemma}
    \begin{proof}
        \textbf{Diagram 1:} Pick $f':= \beta f\alpha$. Suppose $f'': A'\rightarrow B'$ is another morphism making the diagram commute. Then $f'' = \beta f \alpha = f'$.\\
        \textbf{Diagram 2:} Note that $f' = \beta f \alpha^{-1}$ is the unique morphism making  
        $$
            \begin{tikzcd}
                A\arrow[d, "f"] & A'\arrow[l,"\alpha^{-1}",swap]\arrow[d,"f'"]\\
                B \arrow[r,"\beta",swap] & B'
            \end{tikzcd}
        $$
        commute, hence it is the unique morphism such that $ \alpha f' =\alpha\alpha^{-1}\beta f = \beta f$.\\
        \textbf{Diagram 3:} Apply diagram 1 to $\alpha : A'\rightarrow A $, $f$ and $\beta^{-1}: B'\rightarrow B$.\\
        \textbf{Diagram 4:} Apply diagram 1 to $\alpha^{-1}$, $f$ and $\beta^{-1}$.
    \end{proof}
    \begin{lemma}
        Consider a fully faithful functor $\pazocal{F}:\pazocal{C}\rightarrow \pazocal{D}$. For each $X,Y\in \Ob(\pazocal{C})$
            $$
                X\simeq Y \iff \pazocal{F}(X) \simeq \pazocal{F}(Y).
            $$
    \end{lemma}
    \begin{proof}
        Consider an isomorphism $h : \pazocal{F}(X)\rightarrow \pazocal{F}(Y)$. By fullness, there is an $f:X\rightarrow Y$ such that $\pazocal{F}(f)=h$. Moreover there is a $g:Y\rightarrow X$ such that $\pazocal{F}(g)=h^{-1}$. Then 
        $$\pazocal{F}(gf) = \pazocal{F}(g)\pazocal{F}(f)=h^{-1}h=\fone_{\pazocal{F}(X)} = \pazocal{F}(\fone_X) \implies gf = \fone_X.$$
        By symmetry, $fg = \fone_Y$.   
    \end{proof}
    \begin{theorem}\label{EquivOfCatIsFFFESO}
        A functor that is one functor comprising an equivalence of categories is fully faithful and essentially surjective on objects. For locally small categories, assuming the axiom of choice any fully faithful functor that is essentially surjective on objects is one functor in an equivalence of categories.  
    \end{theorem}
    \begin{proof}
        Consider functors $\pazocal{F}:\pazocal{C}\rightarrow \pazocal{D}$ and $\pazocal{G}:\pazocal{D}\rightarrow \pazocal{C}$ with natural isomorphism $\alpha: \pazocal{ID}_\pazocal{C}\simeq \pazocal{GF}$ and $\beta : \pazocal{ID}_\pazocal{D}\simeq \pazocal{FG}$.\\
         Suppose $f,g\in\Hom(X,Y)$ are given such that $\pazocal{F}(f)=\pazocal{F}(g)$. Note that
        $$
            \alpha_Y f =  \pazocal{F}(f)\alpha_X = \pazocal{F}(g)\alpha_X = \alpha_Y g \implies f=g 
        $$
        since $\alpha_Y$ is an isomorphism and thus a monomorphism. We thus conclude that $\pazocal{F}$ is faithful, and by a symmetric argument so is $\pazocal{G}$.\\
        Consider any morphism $h\in \Hom(\pazocal{F}(X),\pazocal{F}(Y))$. Then we have commutative diagrams 
        $$
            \begin{tikzcd}
                \pazocal{GF}(X) \arrow[d,"\pazocal{G}(h)"] & X \arrow[l,"\alpha_X"]\arrow[d,"f"]\\ 
                \pazocal{GF}(Y) & Y \arrow[l,"\alpha_Y"]   
            \end{tikzcd}
        $$  
        for some unique $f : X\rightarrow Y$. We thus by naturality get a commutative diagram
        $$
            \begin{tikzcd}
                \pazocal{GF}(X) \arrow[d,"\pazocal{GF}(f)"] & X \arrow[l,"\alpha_X"]\arrow[d,"f"]\\ 
                \pazocal{GF}(Y) & Y \arrow[l,"\alpha_Y"]   
            \end{tikzcd}
        $$
        Using the result of Lemma~\ref{UniqueChangeOfDomainAndCodomainViaIsos} diagram 2 to these two diagrams, by uniqueness, $\pazocal{G}(\pazocal{F}(f))=\pazocal{G}(h)$. By the faithfulness of $\pazocal{G}$, 
        $$
            \pazocal{F}(f) = h.
        $$  
        We thus conclude that $\pazocal{F}$ (and $\pazocal{G}$) are fully faithful functors.\\ 
        Consider an object $D\in \pazocal{D}$. Set $C= \pazocal{G}(D)$. Then 
        $$
            \begin{tikzcd}
                D \arrow[r,"\beta_D"]\arrow[d,"\id_D"] & \pazocal{F}(C) = \pazocal{FG}(D) \arrow[d,"\id_{\pazocal{F}(C)}"]\\
                D \arrow[r,"\beta_D"] & \pazocal{F}(C) = \pazocal{FG}(D)
            \end{tikzcd}
        $$
        hence $D\simeq \pazocal{F}(C)$.\\ 
        Consider a fully faithul functor $\pazocal{F} : \pazocal{C} \rightarrow \pazocal{D}$ that is essentially surjective on objects. We construct the functor 
        \begin{gather*}
            \pazocal{G} : \pazocal{D} \rightarrow \pazocal{C}\\ 
            D \mapsto \pazocal{G}(D)\\
            f : D_1 \rightarrow D_2 \mapsto \pazocal{G}(f) : \pazocal{G}(D_1) \rightarrow \pazocal{G}(D_2)
        \end{gather*}
       to make $\pazocal{F}$ and $\pazocal{G}$ equivalence of categories. For objects, we do this by for each $D\in \Ob(\pazocal{D})$ considering 
       $$
            \pazocal{C}_D := \{ C\in \Ob(\pazocal{C}) : \pazocal{F}(C)\simeq D\} 
       $$
       to obtain a family of sets indexed by objects in $\pazocal{D}$
       $$
            \pazocal{X} := \{ \pazocal{C}_D : D\in \Ob(\pazocal{D})\}.  
       $$
       Since $\pazocal{F}$ is essentially surjective on objects, this is a family of non-empty set, hence by the axiom of choice there is a well-defined assigment
       \begin{gather*}
            \pazocal{G}:\Ob(\pazocal{D}) \rightarrow \pazocal{X} \rightarrow \bigcup \pazocal{X}\subset \Ob(\pazocal{C})\\
            D \mapsto \pazocal{C}_D \mapsto \pazocal{G}(D).
       \end{gather*}
        Hence $\pazocal{FG}(D)\simeq D$. Denote this isomorphism by $\alpha_D$. Fix a morphism $f:D\rightarrow D'$ 
        $$
            \begin{tikzcd}
                \pazocal{FG}(D) \arrow[r,"\alpha_D"]\arrow[d,"\exists!h",dashed,swap] & D \arrow[d,"f"]\\
                \pazocal{FG}(D') \arrow[r,"\alpha_{D'}",swap] & D'
            \end{tikzcd}
        $$
        Since $\pazocal{F}$ is full there is a morphism $\pazocal{G}(f) : \pazocal{G}(D) \rightarrow \pazocal{G}(D')$, such that $\pazocal{F}(\pazocal{G}(f)) = h$. This assignment is well-defined, since if $k: \pazocal{G}(D)\rightarrow \pazocal{G}(D')$ is another morphism with $\pazocal{F}(k)= h = \pazocal{F}(\pazocal{G}(f))$, then by faithfulness of $\pazocal{F}$, $k = \pazocal{G}(f)$.\\
        We now check that $\pazocal{G}$ is a functor. Consider morphisms $f:X\rightarrow Y$ and $g:Y\rightarrow Z$. Note that 
        \begin{align*}
            \pazocal{F}(\pazocal{G}(g)\pazocal{G}(f))\alpha_X &=\pazocal{F}(\pazocal{G}(g))\pazocal{F}(\pazocal{G}(f))\alpha_X = \pazocal{F}(\pazocal{G}(g))\alpha_Yf = \alpha_Z gf, 
        \end{align*}
        hence $\pazocal{G}(g)\pazocal{G}(f)=\pazocal{G}(gf)$. We also have that 
        $$\pazocal{F}(\id_{\pazocal{G}(X)})\alpha_X = \id_{\pazocal{FG}(X)} \alpha_X = \alpha_X \id_X = \pazocal{F}(\pazocal{G}(\id_X))\alpha_X\implies \id_{\pazocal{G}(X)} = \pazocal{G}(\id_X). $$
        Note that for any morphism $f:D\rightarrow D'$ by definition 
        $$
            \begin{tikzcd}
                D \arrow[r,"\alpha_D^{-1}"]\arrow[d,"f"] & \pazocal{FG}(D)\arrow[d,"\pazocal{FG}(f)"]\\
                D' \arrow[r,"(\alpha_{D'})^{-1}"] & \pazocal{FG}(D')
            \end{tikzcd}
        $$
        commutes, so we have a natural isomorphism $\alpha: \pazocal{ID}_\pazocal{D}\simeq \pazocal{FG}$.\\
        Fix an object $C\in \pazocal{C}$. Since $\pazocal{F}$ is fully faithful, there is an isomorphism $\beta_C : C\rightarrow \pazocal{GF}(C)$ such that $\pazocal{F}(\beta_C) = \alpha_{\pazocal{F}(C)}^{-1} : \pazocal{F}(C)\rightarrow \pazocal{FGF}(C)$. Fix a morphism $f: C\rightarrow C'$. Then 
        \begin{align*} 
            \pazocal{F}(\pazocal{GF}(f)\beta_C) &= \pazocal{F}(\pazocal{GF}(f))\pazocal{F}(\beta_C)= \pazocal{FG}(\pazocal{F}(f))\alpha_{\pazocal{F}(C)}^{-1} = \alpha_{\pazocal{F}(C')}^{-1}\pazocal{F}(f)= \pazocal{F}(\beta_{C'})\pazocal{F}(f)\\
            &= \pazocal{F}(\beta_{C'} f).
        \end{align*}
        By the faithfulness of $\pazocal{F}$ we thus have that 
        $$
            \pazocal{GF}(f)\beta_C = \beta_{C'}f,
        $$
        hence $\beta: \pazocal{ID}_{\pazocal{D}} \simeq \pazocal{GF}$. We thus conclude that $\pazocal{C}\simeq \pazocal{D}$.
    \end{proof}
    \begin{example}
        We aim to show a an equivalence of categories implicitly known from the linear algebra of finite dimensional vector space, which is for a fixed field $K$ that the study of $n\times n$ matrices over $K$ is the "same" as the study of finite dimensional vector spaces. To make this concrete, we consider two locally small categories:
        \begin{itemize}
            \item $\mathrm{Mat}_K$, whose objects are $\Z_{\geq 1}$ and whose morphisms are $\{\mathrm{Mat}_{n\times m}(K) : n,m\in \Z_{\geq 1}\}$. composition is given by matrix multiplication. Note we know (or will know) that for an $n\times m$ and an $m\times l$ matrix the resulting matrix from matrix multiplication is an $n\times l$ matrix. We also know that matrix multiplication is associative and the identity matrix is the neutral element with respect to matrix multiplication and thus plays the role of identity morphism in this context.
            \item $\mathrm{Vect}^{\mathrm{fd}}_K$, whose objects is the proper class of finite dimensional vector spaces and whose morphism are linear maps. This is a subcategory of $\mathrm{Set}$.  
        \end{itemize}    
        We aim to show that $\mathrm{Mat}_K\simeq \mathrm{Vect}^{\mathrm{fd}}_K$. To do this, we first introduce the category $\mathrm{Vect}_K^{\mathrm{basis}}$. An object in this category is a pair $V,\pazocal{V}$, where $V$ is a finite dimensional vector space and $\pazocal{V}$ is a basis of $V$. The morphisms are linear maps. Let $\pazocal{E}_n$ denote the standard basis of $K^n$. Consider the assignment
        \begin{gather*}
            K^{(\_)} : \mathrm{Mat}_K \rightarrow \mathrm{Vect}^{\mathrm{basis}}_K\\
            n \mapsto (K^n,\pazocal{E}_n)\\
            \mathrm{Mat}_{n\times m}(K)\ni A \mapsto (l_A : K^m\rightarrow K^n, v\mapsto A v)
        \end{gather*}  
        one readily verifies that constitutes a functor. With some general linear algebra knowledge and it also easy to verify that it is fully faithful and essentially surjective on objects. Indeed, if $V$ is an $n$-dimensional vector space with basis $\pazocal{V}=\{v_1,\dots,v_n\}$, $(V,\pazocal{V})\simeq K^n$, by the map 
        $$K^n\ni (a_1,\dots,a_n)\mapsto \sum_1^n a_iv_i,$$
        showing that $K^{(\_)}$ is essentially surjective on objects.\\ 
        If $L_A = L_B$ for some $n\times m$ matrices $A,B$, we know that $A_{ij}=(L_A e_j)_i = (L_B e_j)_i = B_{ij}$, for each $i$ and $j$, hence $A=B$. This means $K^{(\_)}$ is faithful.\\
        Consider a linear map $L : (K^m,\pazocal{E}_m)\rightarrow (K^n,\pazocal{E}_n)$. Set $A:= (L(e_j)_i)$. For an arbitrary $v\in K^m$ we have 
        $$
            L_A (v) = (L(e_j)_i) v = (\sum_1^m L(e_j) v_j)= \sum_1^m L(v_j e_j) = L(v). 
        $$
        We thus get that $L = L_A$, which means $K^{(\_)}$ is full.\\
        Consider the forgetful functor $\mathrm{Vect}_K^\mathrm{basis}\rightarrow \mathrm{Vect}_K^\mathrm{fd}$ which forgets the basis on objects. Clearly this is also a full faithful functor that is essentially surjective on objects. By Theorem~\ref{EquivOfCatIsFFFESO} we then get that $\mathrm{Mat}_K\simeq \mathrm{Vect}_K^\mathrm{fd}$.
    \end{example}

\subsubsection{Diagrams}
    \begin{definition}
        A \emph{(commutative) diagram} in a category $\pazocal{C}$ is a functor from a small category, called the \emph{indexing category}, to $\pazocal{C}$. 
    \end{definition}
    \begin{example}
        An example arises the category $\mathbbm{2}\times \mathbbm{2}$. This category has objects $(0,0),(0,1),(1,0),(1,1)$ and arrows $(\id_i,\id_j)$ where $i,j=0,1$ and $(\id_i,0\rightarrow 1)$, $(0\rightarrow 1,\id_i)$ and $(0\rightarrow 1,0\rightarrow 1)$. We thus have a category with four objects, and five non-trivial arrows: one arrow from $(0,0)$ to each of the other objects, one arrow from $(0,1)$ and $(1,0)$ to $(1,1)$. We may illustrate this as a directed graph.
        $$
            \begin{tikzcd}
                (0,0) \arrow[r]\arrow[d]\arrow[rd] & (0,1)\arrow[d]\\
                (1,0) \arrow[r] & (1,1)
            \end{tikzcd}
        $$
        A diagram $D : \mathbbm{2}\times \mathbbm{2} \rightarrow \pazocal{C}$ maps the four objects in $\mathbbm{2} \times \mathbbm{2}$ to four objects in $\pazocal{C}$, $A,B,C,D$ say. Moreover the five non-trivial arrows of $\mathbbm{2}\times \mathbbm{2}$ to morphisms $f_{0001} : A\rightarrow B$, $f_{0010}: A\rightarrow C$, $f_{0011}: A\rightarrow D$, $f_{0111}: B \rightarrow D$ and $f_{1011} : C\rightarrow D$. Functoriality assures that the diagram (in the informal sense),
        $$
            \begin{tikzcd} 
                A \arrow[r,"f_{0001}"]\arrow[rd, "f_{0011}",swap]\arrow[d,"f_{0010}",swap] & B\arrow[d,"f_{0111}"]\\
                C \arrow[r,"f_{1011}",swap] & D
            \end{tikzcd}
        $$ 
        commutes, i.e. $f_{0111}f_{0001}=f_{0011}=f_{1011}f_{0010}$
    \end{example}
    In general, the abstract definition of diagram lets us study the consequences of a list of morphism relations in terms of directed graph underlying these. This proof technique is called \emph{diagram chasing}.
    \begin{lemma}
        Functors preserve commutative diagram.
    \end{lemma}
    \begin{proof}
        Indeed given a functor $\pazocal{F} : \pazocal{C}\rightarrow \pazocal{D}$ and a diagram $D : \pazocal{I} \rightarrow \pazocal{C}$. Then $\pazocal{F}D : \pazocal{I}\rightarrow \pazocal{D}$ is a diagram in $\pazocal{D}$.
    \end{proof}
    \begin{lemma}
        Consider a composable sequence of morphisms $f_1,\dots,f_n$ in $\pazocal{C}$. If 
        $$
            f_jf_{j-1}\cdots f_{i+1}f_i = g_m\cdots g_1
        $$
        for some composable morphisms $g_1,\dots,g_m$ in $\pazocal{C}$, then 
        $$
            f_n \cdots f_1 = f_n\cdots f_{j+1} g_m\cdots g_1 f_{i-1}\cdots f_1
        $$
    \end{lemma}
    \begin{proof}
        Indeed, this is just a result of pre-composing with $f_{i-1}\cdots f_1$ and post-composing with $f_n\cdots f_{j+1}$ on both sides of the equality. 
    \end{proof}
    
    
\subsubsection{Currying}