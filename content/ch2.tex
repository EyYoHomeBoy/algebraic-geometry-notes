\section{Category Theory}
\subsubsection{Initial Definitions}
\begin{definition}
    A category $\pazocal{C}$ is a pair $(\Ob(\pazocal{C}),\Hom(\pazocal{C}))$ where 
    \begin{enumerate} 
    \item $\Ob(\pazocal{C})$ denotes a class of \textit{objects}. \item $\Hom(\pazocal{C})$ denotes a class of \textit{morphisms}. 
    \item A morphism $f$ in $\Hom(\pazocal{C})$ is a relation between between elements $A,B$ in $\Ob(\pazocal{C})$. We denote it by $f : A \rightarrow B$. 
    \item For objects $A,B$ in $\Ob(\pazocal{C})$ we denote the class of morphisms from $A$ to $B$ by $\Hom(A,B)$.
    \item There is a binary operation $\circ$ on the class of \emph{composable} morphisms called \textit{composition} such that for composable morphisms $f: B\rightarrow C$ and $g : A \rightarrow B$ we have that
    $$fg := f\circ g : A \rightarrow C$$
    and 
    $$(f\circ g)\circ h = f\circ(g\circ h)$$
    where $f : C \rightarrow D$, $g: B \rightarrow C$ and $h: A \rightarrow B$ for objects $A,B,C,D$ in $\Ob(\pazocal{C})$. Furthermore for each object $X$ in $\Ob(\pazocal{C})$ there is a morphism $\mathbbm{1}_X : X \rightarrow X$ called the \textit{identity morphism} such that 
    $$\mathbbm{1}_B f = f = f \mathbbm{1}_A,$$
    for a morphism $f: A \rightarrow B$.
    \end{enumerate}
\end{definition}
\begin{remark}
    Here the notion of "class" depends foundations. For instance, it could mean a set in the sense of ZFC or it could mean a proper class in the sense of NGB {\Large I guess what I really mean is that objects and morphisms live in a model of ZFC/NGB but I am not sure}. 
\end{remark}
\begin{definition}
    In a category, a morphism $f : X\rightarrow X$ is called an \emph{endomorphism}.
\end{definition}
\begin{definition}
    Let $\pazocal{C}$ be a category. An \textit{isomorphism} $f:A \rightarrow B$ is a morphism in $\Hom(\pazocal{C})$ such that there is another morphism $f^{-1} : B\rightarrow A$ satisfying,
    $$ff^{-1} = \mathbbm{1}_B \ \mathrm{ and } \ f^{-1}f = \mathbbm{1}_A.$$
    An endomorphism that is an isomorphism is called an \emph{automorphism}. 
\end{definition}
\begin{definition}
    A \textit{subcategory} $\pazocal{D}$ of a category $\pazocal{C}$ is a subclass of $\Ob(\pazocal{C})$ together with a subclass of $\Hom(\pazocal{C})$ that constitutes a category.
\end{definition}
\begin{remark}
    equivalently a subcategory of $\pazocal{C}$ is a subclass $\Ob(\pazocal{D})$ of $\Ob(\pazocal{C})$ and a subclass $\Hom(\pazocal{D})$ of $\Hom(\pazocal{C})$ such that each domain $A$ and codomain $B$ for a morphism in $\Hom(\pazocal{D})$, $A,B$ are elements of $\Ob(\pazocal{D})$. In addition such that $\Hom(\pazocal{D})$ is closed under composition. 
\end{remark}
\begin{definition}
    Let $\pazocal{C}$ be a category and $f\in \Hom(A,B)$, $g\in \Hom(B,A)$.  $f$ is called a \textit{retraction} of $g$ and $g$ a section of $f$ if $fg = \fone_B$   
\end{definition}
\begin{lemma}\label{AtMostOneIso}
    Let $\pazocal{C}$ be a category, $f\in \Hom(A,B)$. Suppose $g,h\in \Hom(B,A)$ are respectively a retraction and a section of $f$. Then $g=h$ and $f$ is an isomorphism. It follows that a morphism can have at most one inverse
\end{lemma}
\begin{proof}
    Indeed 
    $$g=g\fone_A = gfh=\fone_B h = h.$$
    Hence $f$ is an isomorphism with $f^{-1} = g=h$. Let $f_1$ and $f_2$ be inverses of an isomorphism $f$. Note that both $f_1$ and $f_2$ is both a section and a retraction. Therefor, by the first statement, $f_1=f_2$.   
\end{proof}
\begin{definition}
    Let $\pazocal{C}$ be a category. We define \textit{the opposite category of $\pazocal{C}$} denoted $\pazocal{C}\op$ to be the category with $\Ob(\pazocal{C}\op) := \Ob(\pazocal{C})$ and where a morphism $f : A \rightarrow B$ in $\Hom(\pazocal{C}\op)$ is a morphism $f : B\rightarrow A$ in $\Hom(\pazocal{C})$ 
\end{definition}
\begin{remark}
    The above indeed does define a category. We define $\circ\op$ by 
    $$f \circ\op g = g\circ f : C \rightarrow A$$
    where $f: B\rightarrow C$ and $g: A \rightarrow B$ are morphisms in $\Hom(\pazocal{C}\op)$. Then for morphisms $f: C\rightarrow D$, $g: B \rightarrow C$ and $h: A \rightarrow B$
    $$(f\circ\op g)\circ\op h = h(gf)=(hg)f = f\circ\op(g\circ\op h).$$
    Furthermore, we define the identity morphisms in $\Hom(\pazocal{C}\op)$ to be the identity morphisms in $\Hom(\pazocal{C})$, hence 
    $$f\circ\op \mathbbm{1}_A = \mathbbm{1}_Af = f \text{ and } \mathbbm{1}_B \circ\op f = f \mathbbm{1}_B = f.$$
\end{remark}
\begin{definition}
    A category $\pazocal{C}$ is \textit{locally small} if $\Hom(A,B)$ is a set for every object $A,B$ in $\Ob(\pazocal{C})$. It is \textit{small} if $\Ob(\pazocal{C})$ is a set.
\end{definition}
\begin{definition}
    In a category $\pazocal{C}$ a morphism $f\in \Hom(A,B)$ is a \textit{monomorphism} if for every pair of morphisms $g_1,g_2 \in \Hom(C,A)$, 
    $$fg_1=fg_2\implies g_1 = g_2.$$
    It is called an \textit{epimorphism} if for every pair of morphisms $h_1,h_2\in \Hom(B,D)$,
    $$h_1f=h_2f\implies h_1=h_2$$    
\end{definition}
\begin{remark}
    Note that an epimorphism is just a monomorphism in $\pazocal{C}\op$ and vice versa
\end{remark}\label{IsomorphismIsEpiAndMono}
\begin{lemma}
    Let $\pazocal{C}$ be a category and $f:A\rightarrow B$ a morphism. Then if $f$ is an isomorphism, then it is a monomorphism and an epimorphism.
\end{lemma}
\begin{proof}
    Suppose $f$ is an isomorphism. Let $g_1,g_2: C \rightarrow A$ with $fg_1=fg_2$. Then 
    $$
        g_1=f^{-1}fg_1 = f^{-1}fg_2 = g_2.
    $$
    Using the dual result, we get that for every $h_1,h_2: B,D$, if $h_1f=h_2f$ then $h_1=h_2$. We thus conclude that $f$ is both an epimorphism and a monomorphism.\\
\end{proof}

\begin{definition}
    A category for which the converse of the above lemma holds is called \emph{balanced}.
\end{definition}

\begin{lemma}
    In the category of $\mathrm{Set}$, we have for a function $f: A\rightarrow B$
    \begin{enumerate}
        \item $f$ is epi if and only if it is surjective.
        \item $f$ is mono if and only if is injective
    \end{enumerate}
    In particular $\mathrm{Set}$ is balanced. 
\end{lemma}
\begin{proof}
    1. Suppose $f$ epimorphism. Note that in general for a subset $C\subset B$, if for each set $X$ and each pair of maps $\lambda_1,\lambda_2: B\rightarrow X$,  $\left.\lambda_1\right|_{C}=\left.\lambda\right|_C$ implies $\lambda_1 = \lambda_2$, then $C=B$. Our assumption tells us that for each pair of functions $h_1,h_2: B\rightarrow X$, $\left.h_1\right|_{\im\ f} = \left.  h_2\right|_{\im\ f}$ implies $h_1 = h_2$, hence $\im \ f = B$, meaning $f$ is surjective.\\
    Suppose conversely that $f$ is surjective. Let $h_1,h_2\in \Hom(X,B)$ and suppose $h_1 f= h_2 f$. Let $b\in B$. Then there is an $a\in A$ such that $b=f(a)$, hence
    $$h_1(b) = h_1(f(a))=h_1f(a)= h_2f(a)=h_2(f(a))=h_2(b).$$
    Since $b$ was chosen arbitrarily it follows that $h_1=h_2$.\\ 
    2. Suppse $f$ is mono. Suppose $a_1,a_2\in A$ are given such that $f(a_1)=f(a_2)$. Consider the functions, $g_i : \{\ast\} \rightarrow A,\ast\mapsto a_i$. Then 
    $$ 
        fg_1 = fg_2,
    $$
    hence $g_1 = g_2$, which implies $a_1=g_1(\ast)=g_2(\ast)= a_2$.\\
    Suppose $f$ is injective. Consider $g_1,g_2 : C\rightarrow A$ with $fg_1 = fg_2$. Let an arbitrary $x\in C$ be given. Then 
    $$
        f(g_1(x)) = f(g_2(x)) \implies g_1(x) = g_2(x).
    $$ 
    Since $x$ was chosen arbitrarily this means $g_1 = g_2$.    
\end{proof}
\begin{remark}
    Note that surjectivity implies epi, and dually injectivity, in any subcategory of $\mathrm{Set}$. 
\end{remark}
\begin{example}
    There are non balanced categories. Consider the canonincal embedding 
    \begin{gather*}
        f_\Q: \Z \hookrightarrow \Q\\
        n \mapsto n
    \end{gather*}
    a morphism in the category of rings. Since it is injective it is a monomorphism. Suppose $h_1,h_2: \Q \rightarrow R$ are ring homomorphism such that 
    $$
        h_1 f_\Q = h_2 f_\Q 
    $$
    these are both ring homomorphisms $\Z\rightarrow \Q$. We shall later see that there is only one such homomorphism, namely
    \begin{gather*}
        f_R : \Z \rightarrow R\\
        n\mapsto \sum_1^n 1
    \end{gather*}
    The vocabulary that captures this fact is that $\Z$ is the initial object in $\mathrm{Ring}$. We conclude that $h_1 =h_2$ and hence that $f_\Q$ is epi. However, since $f_\Q$ is not surjective it cannot be an isomorphism.  
\end{example}
\begin{lemma}
    Let $\pazocal{C}$ be a category and $f : A\rightarrow B$ a morphism
    \begin{enumerate}
        \item If $f$ is a section of some $h: B\rightarrow A$, then it is a mono.
        \item If $f$ is a retraction of some $h: B\rightarrow A$, then it is epi.
    \end{enumerate} 
\end{lemma}
\begin{proof}
    1. Consider $g_1,g_2: C\rightarrow A$ and suppose $fg_1 = f g_2$. Since $f$ is a section of $h$, then $hf = \fone_A$. Then 
    $$
        g_1 = (hf)g_1 = h(fg_1) = h (fg_2)= (hf)g_2 = g_2.
    $$
    2. Since a retraction is just a section in the opposite category, this result is dual to 1.
\end{proof}
\begin{remark}
    If a morphism is a section of some morphism, then it is called a \emph{split monomorphism}. If a morphism is a retraction of some morphism it is called a \emph{split epimorphism}.
\end{remark}
\begin{lemma}
    For every category $\pazocal{C}$, a morphism $f : A\rightarrow B$ is an isomorphism if and only if it is mono and split epi. Dually it is an isomorphism if and only if it is mono and split epi.
\end{lemma}
\begin{proof}
    One direction follows from Lemma~\ref{IsomorphismIsEpiAndMono}. Suppose $f$ is mono and split epi. Since it is split epi, there is a $g:B\rightarrow A$ such that $fg = \fone_B$. Note that 
    $$f(gf)=(fg)f=\fone_B f= f = f\fone_A,$$
    hence $gf = \fone_A$ by the assumption that $f$ is mono.
\end{proof}
\begin{lemma}
    Consider a category $\pazocal{C}$. Then 
    \begin{enumerate}
        \item The composition of two monos is a mono.
        \item For $f:A\rightarrow B$ and $g : B\rightarrow C$ if $gf$ is a mono, then so is $f$. 
    \end{enumerate}
    Dually 
    \begin{enumerate}
        \item The composition of two epis is an epi.
        \item For For $f:A\rightarrow B$ and $g : B\rightarrow C$ if $gf$ is an epi, then so is $g$.
    \end{enumerate}
\end{lemma}
\begin{proof}
    1. Consider monos $f: B\rightarrow C$ and $g:A\rightarrow B$. Suppose $h_1,h_2 : X\rightarrow A$ are given such that $(fg)h_1 = (fg)h_2$. Then 
    $$f(gh_1)=f(gh_2)\implies gh_1 = gh_2 \implies h_1 =h_2,$$
    hence $fg$ is a mono.\\
    2. Consider $h_1,h_2 : X \rightarrow A$ such that $fh_1 = fh_2$. Then $(gf)h_1 = (gf)h_2$, hence by assumption, $h_1 = h_2$. We thus conclude that $f$ is mono.
\end{proof}
\begin{remark}
    From the above result it follows that $\Ob(\pazocal{C})$ with morphisms being monos/epis in $\Hom(\pazocal{C})$, defines a subcategory of $\pazocal{C}$.
\end{remark}
\begin{definition}
    Let $\pazocal{C}$ be a locally small category and fix an object $X$ in $\pazocal{C}$. Given a morphism $f:A\rightarrow B$, define 
    \begin{gather*}
        f_\ast : \Hom(X,A) \rightarrow \Hom(X,B)\\
        g\mapsto fg
    \end{gather*}  
    we also define the dual concept 
    \begin{gather*}
        f^\ast : \Hom(A,Y) \rightarrow \Hom(B,X)\\
        h\mapsto hf
    \end{gather*}
\end{definition}
\begin{remark}
    Note that $f\op_\ast = f^\ast$, i.e. for $f\op : B \rightarrow A$, the morphism $f$ interpreted as a morphism of $\pazocal{C}\op$ we have 
    \begin{gather*}
        f\op_\ast : \Hom^{\pazocal{C}\op}(X,B)=\Hom(B,X) \rightarrow \Hom^{\pazocal{C}\op}(X,A)=\Hom(A,X)\\
        h: B\rightarrow X \mapsto f\circ\op h= hf
    \end{gather*}
    hence $f\op_\ast = f^\ast$
\end{remark}
\begin{lemma}
    For a locally small category $\pazocal{C}$ and morphism $f:A\rightarrow B$, the following are equivalent
    \begin{enumerate}
        \item $f$ is an isomorphism
        \item For every object $X$ in $\pazocal{C}$, $f_\ast$ is a bijection. 
        \item For every object $X$ in $\pazocal{C}$, $f^\ast$ is a bijection
    \end{enumerate}
\end{lemma}
\begin{proof}
    "1. $\implies$ 2.": Let $g : B\rightarrow A$ be the inverse of $f$. Then given $h\in \Hom(X,A)$,
    $$(g_\ast f_\ast)h = g_\ast(f_\ast h)= g_\ast(fh)=gfh=h \implies g_\ast f_\ast = \fone_{\Hom(X,A)}$$
    and similarly one proves that $f_\ast g_\ast = \fone_{\Hom(X,B)}$.\\
    "2.$\implies$ 1.": Consider $f_\ast : \Hom(B,A)\rightarrow \Hom(B,B)$. This by assumption has an inverse $\lambda : \Hom(B,B)\rightarrow \Hom(B,A)$. Set $g := \lambda(\fone_B)$. Then for each  
    $$fg = f \lambda(\fone_B)= f_\ast(\lambda(\fone_B))=(f_\ast\lambda)(\fone_B)=\fone_{\Hom(B,B)}(\fone_B)=\fone_B.$$
    Now consider the function 
    \begin{gather*}
        \mu : \Hom(A,A)\rightarrow \Hom(A,B)\\
        h\mapsto fh
    \end{gather*}
    which is just another instance of an "$f_\ast$", but we use different notation to not confuse it with $f_\ast: \Hom(B,B)\rightarrow \Hom(B,A)$. Then $\mu$ has an inverse $\mu^{-1}$. Note then thats
    $$
        \mu(gf)= f(gf)=(fg)f= \fone_B f = f\fone_A= \mu(\fone_A)
    $$
    hence 
    $$
        gf = (\mu^{-1}\mu)(gf) = \mu^{-1}(\mu(gf)) = \mu^{-1}(\mu(\fone_A))= (\mu^{-1}\mu)(\fone_A) = \fone_A.
    $$
    "1. $\iff$ 3.": This is dual to "1. $\iff$ 2.".\\
\end{proof}
\begin{lemma}
    Consider a locally small category $\pazocal{C}$ and a morphism $f:A\rightarrow B$.
    \begin{enumerate}
        \item $f$ is a split epimorphism if and only if for each $X$, $f_\ast: \Hom(X,A) \rightarrow \Hom(X,B)$ is surjective. 
        \item Dually, $f$ is a split monomorphism if and only if for each $X$, $f^\ast : \Hom(B,X)\rightarrow \Hom(A,X)$ is injective. 
    \end{enumerate}
\end{lemma}
\begin{proof}
    1. Suppose $f$ is a split epi. Then $f$ has a right inverse $g$, i.e. $fg = \fone_B$. Let $h: X \rightarrow B$ be given. Then $f_\ast(gh)= f(gh)=(fg)h = h$, hence $f_\ast$ is surjective.\\ 
    Suppose that $f_\ast: \Hom(X,A)\rightarrow \Hom(X,B)$ is surjective for each object $X$. Then $f_\ast: \Hom(B,A) \rightarrow \Hom(B,B)$ is surjective. hence for some $g:B\rightarrow A$, 
    $$
        \fone_B=f_\ast(g) = fg
    $$ 
    hence $f$ is a retraction of $g$ and therefor a split epimorphism. 
    2. This is dual to 1. 
\end{proof}
\subsubsection{Examples of Categories}
\begin{example}
    Consider the proper class of sets and the class of functions. This defines a category $\mathrm{Set}$ where composition is given by function composition and the identity morphism is the identity function.  
\end{example}
\begin{example}
    The following are some examples of subcategories of $\mathrm{Set}$.
    \begin{enumerate}  
        \item There is the category of finite sets $\mathrm{Fin}$, whose objects are finite sets and whose morphisms are maps. 
        \item $\mathrm{Top}$ has the proper class of topological spaces as objects and continuous functions as morphisms.
        \item $\mathrm{Htpy}$ is the category whose objects are topological spaces and whose morphisms are homotopy classes of continuous maps. That is given continous maps $f,g: X\rightarrow Y$, these maps are \emph{homotopic} if there exists a continuous map $H: X\times [0,1]\rightarrow Y$ such that $H(x,0)=f(x)$ and $H(x,1)=g(x)$ for each $x\in X$. This defines an equivalence relation, $\sim$ say. We define $[g]_\sim[f]_\sim: = [gf]_\sim$. Suppose $f,f': X\rightarrow Y$ are homotopic via a map $I : X\times [0,1]\rightarrow Y$  and $g,g': Y\rightarrow Z$ are homotopic via $J: Y\times [0,1]\rightarrow Z$. We define 
        \begin{gather*}
            H : X\times[0,1] \rightarrow Z\\
            (x,t) \mapsto J(I(x,t),t)
        \end{gather*} 
        Then $H(x,0)= J(I(x,0),0)=J(f(x),0)=gf(x)$ and $H(x,1)=J(I(x,1),1)=J(f'(x),1)=g'f'(x)$, so this is homotopy, hence composition is well-defined. One readily verifies associativity. The identity morphism is the homotopy class of the identity map. 
        \item Given a (preferably algebraically closed) field $K$, a multiprojective variety over $K$ is a solution to a set of polynomial equations (a socalled algebraic set) in a set of the form $\A^n(K)\times \Pp^{n_1}(K)\times \cdots \times \Pp^{n_m}(K)$ that cannot be written as the union of two proper subsets that are sets of solutions to systems of polynomial equations. The set of varieties in such a set forms a subbasis for a topology called the Zariski topology. A subset $U$ of a multiprojective variety $V$ is called a variety if it is open in Zariski subspace topology on $V$. The collection of all such varieties is the objects in a category $\mathrm{Var}_K$. To discuss how the morphisms work in this category one has to go into a long explanation about some constructions in algebraic geometry. The jist is the following: Given a set $A$, $\Hom^\mathrm{Set}(A,K)$ defines a ring. In our context we are interested in the subring $\Gamma(A)$ of this ring consisting of maps that are essentially evaluation of fractions of polynomials. The morphisms of $\mathrm{Var}_K$ are then continuous maps of varieties $\varphi: X\rightarrow Y$ such given any open subset $U\subset Y$ and $f\in \Gamma(U)$, we have that $f\circ \varphi \in \Gamma\left(\varphi^{-1}(U)\right)$. One checks that this defines a subcategory of $\mathrm{Top}$.
        \item As a subcategory of $\mathrm{Var}_K$ we have $\mathrm{AffVar}_K$, the category of affine varieties, whose objects are varieties isomorphic to irreducible algebraic sets in $\A^n$ for some $n$.
        \item As a subcategory of $\mathrm{Var}_K$ we have $\mathrm{ProjVar}_K$, the category of projective varieties, whose objects are varieties isomorphic to irreducible algebraic sets in $\P^n$ for some $n$.
        \item As another subcategory of $\mathrm{Top}$, we have $\mathrm{Man}_\R$, the category of smooth manifolds over $\R$ with morphisms being smooth maps. One can also consider $\mathrm{Man}_\C$ and as a subcategory $\mathrm{RS}$ of riemann surfaces.  
        \item There is the category $\mathrm{Meas}$ of measure spaces and measurable functions.
        \item $\mathrm{Mon}$ has the proper class of monoids as objects and monoid homomorphisms as morphisms.
        \item $\mathrm{Group}$ has groups as objects and group homomorphisms as morphisms.
        \item $\mathrm{Ab}$ has abelian groups as objects and group homomorphisms as morphisms.
        \item $\mathrm{Ring}$ has (unital) rings as objects and ring homomorphism as morphisms. 
        \item $\mathrm{Rng}$ has (not necessarily unital rings) as objects and maps that are monoid homomorphisms with respect to both addition and multiplication. 
        \item Given a ring, $R$ $_R\mathrm{Mod}$ is the category with objects being left $R$-modules with left $R$-module homomorphisms.
        \item There is a dual notion of a category of right $R$-modules $\mathrm{Mod}_R$. 
        \item Given rings $R,S$, there is a category $_R\mathrm{Mod}_S$ of $(R,S)$-bimodules and $(R,S)$-bimodule homomorphisms. 
        \item Given a field $K$ we have the category of vector spaces, $\mathrm{Vect}_K$. As a subcategory there is the category of finite dimensional vector spaces $\mathrm{Vect}_K^\mathrm{fd}$
        \item Given a commutative ring $R$, we have the category of algebras over $R$, denoted $\mathrm{Alg}_R$. 
        \item Indeed, asking that a set is endowed with some structure and finding a class of maps that preserves this structure and is closed under composition readily leads to a category. 
    \end{enumerate}
\end{example}
\begin{example}
    Given any subcategory $\pazocal{C}$ of $\mathrm{Set}$. We can form $\pazocal{C}_\ast$ consisting of a pair $(X,x)$ where $X\in \Ob(\pazocal{C})$ and $x\in X$ where a morphism $f: (X,x)\rightarrow (Y,y)$ is a map $f:X\rightarrow Y$ such that $f(x)=y$. This is the category of \emph{pointed sets in $\pazocal{C}$}. For example, we could consider the category of pointed sets $\mathrm{Set}_\ast$ or the category of pointed topological spaces $\mathrm{Top}_\ast$.
\end{example}
\begin{example}
    Given a (commutative say) ring $R$ we can form the category of \emph{chain complexes} $\mathrm{Ch}_R$ whose objects are families of $R$-modules $\{M_i\}_{i\in\Z}$ together with a family of $R$-module maps $\{d_i : M_i\rightarrow M_{i-1}\}_{i\in \Z}$ such that $d_{i-1}d_i=0$ which are called \emph{boundary maps}. A morphism of chain complexes $f_\bullet : M_\bullet \rightarrow M_\bullet'$ is a collection of maps $\{f_i : M_i \rightarrow M_i'\}$ such that the square
    $$
        \begin{tikzcd}
            M_i \arrow[r,"d_i"]\arrow[d,"f_i"] & M_{i-1}\arrow[d,"f_{i-1}"]\\
            M_i' \arrow[r,"d_i'"] & M_{i-1}'
        \end{tikzcd}
    $$
    commutes. Composition of $f_\bullet : M_\bullet \rightarrow M_\bullet'$ and $g_\bullet : M'_\bullet \rightarrow M_\bullet''$ is given by $\{g_if_i : M_i\rightarrow M_i''\}$. Indeed, the square 
    $$
        \begin{tikzcd}
            M_i \arrow[r,"d_i"]\arrow[d,"g_if_i"] & M_{i-1}\arrow[d,"g_{i-1}f_{i-1}"]\\
            M_i'' \arrow[r,"d_i''"] & M_{i-1}''
        \end{tikzcd}
    $$ 
    commutes, since 
    $$
        \begin{tikzcd}
            M_i \arrow[r,"d_i"]\arrow[d,"f_i"] & M_{i-1}\arrow[d,"f_{i-1}"]\\
            M_i' \arrow[r,"d_i'"]\arrow[d,"g_i"] & M_{i-1}'\arrow[d,"g_{i-1}"]\\
            M_i'' \arrow[r,"d_i''"] & M_{i-1}''
        \end{tikzcd}
    $$
    commutes. The identity morphism is $\{\id_{M_i}\}_{i\in\Z}$.
\end{example} 
\begin{example}
    Given a preordered set $(X,\leq)$, we can construct a category whose objects are $X$ and where a morphisms are $\leq$. Composition of morphisms $a\leq b$ and $b\leq c$ is $a\leq c$. Indeed if $a\leq b$ and $b\leq c$, then $a\leq c$, so this is well-defined. One readily verifies that this is associative using transitivity a few times. The identity morphism is $a\leq a$ with exists by reflexivity.\\
    This could for instance be an ordinal $\alpha\in \Omega$.
\end{example}
\begin{definition}
    A small category $\pazocal{C}$ is called a \textit{groupoid} if every $f$ in $\Hom(\pazocal{C})$ is an isomorphism
\end{definition}
\begin{example}\label{MonoidIsACategoryGroupIsGroupoid}
    Let $(M,\cdot,e)$ be a monoid. We define a category $\mathrm{B}M$ whose class of objects is $\{M\}$ and whose morphisms are $n,m\in M$ and where composition of morphisms $n,m\in M$ is given by 
    $$nm := n\cdot m.$$  
    Since $M$ is a monoid for $m_1,m_2,m_3\in M$, we have 
    $$m_1(m_2m_3)=(m_1m_2)m_3.$$
    The identity morphism is $e$. When $(G,\dot,e)$ is a group we get that every morphism is an isomorphism, since for each $g\in G$ there is a $g^{-1}\in G$ such that 
    $$gg^{-1}=g^{-1}g = e$$ 
    Note that a groupoid with one object $X$ is a group in the following sense: The set $\Hom(X,X)$ forms a group.
\end{example}
\begin{example}
    Consider a topological space $X$. We construct the \emph{fundemental groupoid of $X$} to be the category with objects $X$ and morphisms being homotopy classes of paths $f_{x,y}:[0,1]\rightarrow X$ such that for $f_1,f_2\in [f]_{x,y}$, $f_1(0)=x=f_2(0)$ and  $f_1(1)=y=f_2(1)$. Here composition of paths $[f]_{x,y}$ and $[g]_{y,z}$ is the homotopy class of $g\bullet f$, where 
    $$
        (g\bullet f)(t) := 
        \begin{cases}
            f(2t) & \text{if } t\in [0,1/2]\\
            g(2t-1) & \text{ if} t\in (1/2,1]
        \end{cases}
    $$
    one proves that this is independent of representatives by showing that one can construct a homotopy of any pair of composed paths. The identity morphism for an object $x\in X$ is $[0,1]\rightarrow X, t\mapsto x$. It is clear that this is a groupoid, since given a path $f:[0,1]\rightarrow X$ we construct $f^{-1}:[0,1]\rightarrow X, t\mapsto f(1-t)$.
\end{example}
\begin{definition}
    The \textit{maximal groupoid} of a category $\pazocal{C}$ is the subcategory of $\pazocal{C}$ whose objects are $\Ob(\pazocal{C})$ and whose morphisms are the isomorphisms of $\Hom(\pazocal{C})$
\end{definition}
\begin{remark}
    The maximal groupoid is a subcategory. Indeed, The domain and codomain of an isomorphism are trivially in $\Ob(\pazocal{C})$. Suppose $f:A\rightarrow B$ and $g: B \rightarrow C$ are isomorphisms. Then 
    $$f^{-1}g^{-1}gf=f^{-1}\fone_B f = f^{-1}f = \fone_A$$
    and 
    $$gff^{-1}g^{-1} = g\fone_B g^{-1} = gg^{-1} = \fone_C.$$
    Hence $gf: A \rightarrow C$ is an isomorphism with inverse $f^{-1}g^{-1}$. 
\end{remark}

\begin{definition}
    A category $\pazocal{C}$ is \emph{discrete}, if for each morphism $f:X\rightarrow Y$, $Y=X$ and $f=\id_X$.
\end{definition}
\begin{example}\label{SetIsDiscreteCategory}
    Take any non-empty set $X$. We then have a category whose objects are $X$ and whose morphisms are 
    $$\Delta_X := \{ (x,x) \in X\times X : x\in X\}.$$
    Upon defining $(x,x)(y,y)= (x,x)$, we thus have that 
    $$
        ((x,x)(y,y))(z,z)=(x,x)(z,z)= (x,x) = (x,x)(y,y)=(x,x)((y,y)(z,z)).
    $$
    We define $\fone_x = (x,x)$. Clearly this is neutral with respect to composition. Any morphism is by definition the identity morphism, so this is an example of a discrete category. 
\end{example}
\begin{definition}
    Let $\pazocal{C}$ be a category and $A$ an object in $\pazocal{C}$. \textit{The slice category of $\pazocal{C}$ under $A$} denoted $A/\pazocal{C}$ is the category whose objects are morphisms in $\Hom(\pazocal{C})$ with domain $A$ and where a morphism from $f: A \rightarrow X$ and $g : A \rightarrow Y $ is a map $h : X\rightarrow Y$ such that 
    $$\begin{tikzcd}
       & A \arrow[ld, "f"' ] \arrow[rd, "g"]\\
        X \arrow{rr}{h} & & Y
    \end{tikzcd}$$
    commutes. \textit{The slice category of $\pazocal{C}$ over $A$} denoted $\pazocal{C}/A$ has morphisms with codomain $A$ as objects and a morphism from $f : X \rightarrow A$ to $g : Y\rightarrow A$ is a morphism $h : X \rightarrow Y$ satisfying
    $$\begin{tikzcd}
        X \arrow[rd, "f"'] \arrow[rr, "h"] & & Y\arrow[ld, "g"]\\
        & A 
    \end{tikzcd}$$
\end{definition}
\begin{remark}
    Both these constructions are indeed categories: Consider morphisms $h_{12}$ between $f_1: A\rightarrow X$ \& $f_2 : A\rightarrow Y$ and $h_{23}$ between $f_2 : A\rightarrow Y$ \& $f_3: A\rightarrow Z$. Then we have commutative diagrams
    $$\begin{tikzcd}
        & A \arrow[ld, "f_1"' ] \arrow[rd, "f_2"]\\
        X \arrow{rr}{h_{12}} & & Y
    \end{tikzcd} \quad \begin{tikzcd}  & A \arrow[ld, "f_2"' ] \arrow[rd, "f_3"]\\
        Y \arrow{rr}{h_{23}} & & Z \end{tikzcd}$$
    to obtain the commutative diagram
    $$\begin{tikzcd}
        & A \arrow[ld, "f_1"' ] \arrow[d, "f_2"] \arrow[rd, "f_3"]\\
        X \arrow[r, "h_{12}"] & Y \arrow[r, "h_{23}"] & Z
    \end{tikzcd}
    $$
    hence $h_{23}h_{12}$ is a morphism between $f_{1}$ and $f_3$. For an object $f: A \rightarrow X$ in $A/\pazocal{C}$ define the identity morphism to be $\fone_X$. We thus get that associativity of composition and the identity morphisms being neutral with respect to composition is inherited from this being true in $\pazocal{C}$. Reversing arrows we get that $\pazocal{C}/A$ is also a category.  
\end{remark}

\subsubsection{Functors}
\begin{definition}
    Let $\pazocal{C}_1,\pazocal{C}_2$ be categories. A \textit{Covariant functor} from $\pazocal{C}_1$ to $\pazocal{C}_2$ is a mapping $\pazocal{F}$, denoted $\pazocal{F} : \pazocal{C}_1 \rightarrow \pazocal{C}_2$, which assigns to each object $A$ in $\Ob(\pazocal{C}_1)$ to an object $\pazocal{F}(A)$ in $\Ob(\pazocal{C}_2)$ and to each morphism in $\Hom(\pazocal{C}_1)$, $f : A \rightarrow B$ a morphism in $\Hom(\pazocal{C}_2)$, $\pazocal{F}(f) : \pazocal{F}(A)\rightarrow \pazocal{F}(B)$ such that 
    \begin{enumerate}
        \item for every object $X$ in $\Ob(\pazocal{C}_1)$, $\pazocal{F}(1_X) = 1_{\pazocal{F}(X)}$.
        \item for every pair of morphisms $f: B \rightarrow C$ and $g: A\rightarrow B$ in $\Hom(\pazocal{C}_1)$, $\pazocal{F}(fg) =\pazocal{F}(f)\pazocal{F}(g).$
    \end{enumerate}
\end{definition}
\begin{definition}
    Consider categories $\pazocal{C}_1$ and $\pazocal{C}_2$. A \textit{contravariant functor} $\pazocal{F}$ between  $\pazocal{C}_1 $ and $ \pazocal{C}_2$ is a covariant functor between $\pazocal{C}_1\op$ and $\pazocal{C}_2$.
\end{definition}
\begin{lemma}\label{FunctorsTransferIsomorphismBetweenCategories}
    Consider two categories $\pazocal{C}_1$ and $\pazocal{C}_2$ with a functor $\pazocal{F}: \pazocal{C}_1 \rightarrow \pazocal{C}_2$. If $f:A \rightarrow B$ is an isomorphism in $\Hom(\pazocal{C}_1)$, then $\pazocal{F}(f): \pazocal{F}(A)\rightarrow\pazocal{F}(A)$ is an isomorphism in $\Hom(\pazocal{C}_2)$. 
\end{lemma}
\begin{proof}
    Indeed, 
    $$\pazocal{F}(f)\pazocal{F}\left(f^{-1}\right) = \pazocal{F}\left(ff^{-1}\right) = \pazocal{F}(\mathbbm{1}_B)= \mathbbm{1}_{\pazocal{F}(B)} \ \mathrm{and} \ \pazocal{F}\left(f^{-1}\right)\pazocal{F}(f) = \pazocal{F}\left(f^{-1}f\right) = \pazocal{F}(\mathbbm{1}_A)= \mathbbm{1}_{\pazocal{F}(A)}.$$
\end{proof}
\begin{example}\label{IntegerInvariants}
    Suppose that there, for a category $\pazocal{C}$, is a well-defined assignment $\pazocal{F}$ of objects in $\pazocal{C}$ to integers and of morphisms $A\rightarrow B$ to $\pazocal{F}(A)\leq \pazocal{F}(B)$. This will define a functor from $\pazocal{C}$ to $(\Z,\leq)$ called \textit{an integer invariant of objects in $\pazocal{C}$}. Indeed, $\pazocal{F}(A) \leq \pazocal{F}(A)$, hence $\pazocal{F}(\mathbbm{1}_A)=\mathbbm{1}_{\pazocal{F}(A)}$, so $\fone_{\pazocal{F}(A)}= \pazocal{F}(A)$. Given morphisms $g:A\rightarrow B$, $f:B\rightarrow C$ in $\Hom(\pazocal{C})$, 
    $$\pazocal{F}(A)\leq\pazocal{F}(B) \text{ and } \pazocal{F}(B)\leq \pazocal{F}(C),$$
    implying $\pazocal{F}(A)\leq \pazocal{F}(C)$, hence $\pazocal{F}(A\overset{fg}{\rightarrow} C) = \pazocal{F}(A\overset{f}{\rightarrow}B)\pazocal{F}(B\overset{g}{\rightarrow}C)$.
    Integer invariants are useful in a large variety of contexts. Some examples are:
    \begin{enumerate}
        \item Dimension (of vector spaces) defines an integer invariant of objects in $\mathrm{Vect}^\mathrm{fd}$.
        \item Transcendence degree over a field $K$ of field extensions over $K$ is an integer the category of algebraic extensions over $K$. A closely related notion, when $K$ is algebraically closed, is dimension of varieties in $\mathrm{Var}_K$.
        \item For a commutative ring $R$, $\mid$ defines a preorder on $R[x_1,\dots,x_n]\setminus 0$. An integer invariant on the category associated with this poset is the degree. 
        \item Genus in the category of category of compact smooth manifolds.
        \item For a set $X$, the finite subsets of $X$ is a poset with respect to inclusion
        \item For the the posets $(\R, \leq)$ or $(\Q,\leq)$, the floor- and ceiling functions define an integer invariant. 
    \end{enumerate}
    One can generalize the above to any poset $X$ to a notion of \emph{invariants over $X$ of objects in $\pazocal{C}$}, which is useful if we for intance want to consider invariants over $\Z \cup \{\infty\}$, relevant to for instance dimension of vector spaces in $\mathrm{Vect}$, degree for the poset $(R[x_1,\dots,x_n],\mid)$ with the convention that $\deg \ 0 = \infty$ or cardinality for a family of subsets of some $X$. For a measure space $(X,\Omega,\mu)$, inclusion defines a poset on $\Omega$ and $\mu$ defines an invariant over $[0,\infty]$ of $\Omega$-measurable sets. 
\end{example}
\begin{example}
    The assignment 
    \begin{gather*}
        \pazocal{P}:\mathrm{Set} \rightarrow \mathrm{Set}\\
        X \mapsto \pazocal{P}(X):= \{A : A\subset X\}\\
        (f : X\rightarrow Y) \mapsto (A\mapsto f(A))
    \end{gather*}
    where for each $A\in \pazocal{P}(X)$,
    $$f(A):=\{f(x)\in Y : x \in A\}$$
    defines a functor. Indeed, for $f: X\rightarrow Y$, $g:Y\rightarrow Z$,
    $$gf(A)= \{gf(x)\in Z : x\in A\} = \{g(f(x)) \in Z : x\in A\} = g(\{f(x)\in Y : x\in A \})=g(f(A)). $$
    Moreover, 
    $$\id_X(A) = A = \id_{\pazocal{P}(X)}(A).$$
\end{example}
\begin{example}
    Given any function $f: X\rightarrow Y$, when considering both $X$ and $Y$ with the structure of discrete category, we get a functor 
    \begin{gather*}
        X\rightarrow Y\\
        x\mapsto f(x)\\
        \fone_x \mapsto \fone_{f(x)}
    \end{gather*}  
\end{example}
\begin{example}
    Consider a pair of monoids $M,N$ and a functor 
    $\pazocal{F} : \mathrm{B}M \rightarrow \mathrm[B]N$ this is just a function $M\rightarrow N,x\mapsto \pazocal{F}(x)$. By functoriality $\pazocal{F}(e_M)=e_N$ and $\pazocal{F}(xy)=\pazocal{F}(x)\pazocal{F}(y)$, so in particular $\pazocal{F}$ corresponds to a monoid homomorphism. Conversely, any monoid homomorphism gives rise to functor of monoids. Similar considerations can be made to a ring $R$.
\end{example}
\begin{definition}
    A functor $\pazocal{F} : \pazocal{C}\rightarrow \pazocal{C}$ is called an \emph{endofunctor}. 
\end{definition}
\begin{definition}
    For each subcategory $\pazocal{C}$ of $\mathrm{Set}$ there is a \emph{forgetful functor} 
    \begin{gather*}
        \pazocal{C} \rightarrow \mathrm{Set}\\
        X \mapsto X\\
        f : X \rightarrow Y \mapsto f : X \rightarrow Y
    \end{gather*}
    In general, given a pair of subcategories of $\mathrm{Set}$, $\pazocal{C}$ and $\pazocal{D}$ say, such that $\pazocal{D}$ is a subcategory of $\pazocal{C}$, we can construct a forgetful functor from $\pazocal{D}$ to $\pazocal{C}$ in the same fashion.   
\end{definition}
\begin{example}
    Consider for instance the forgetful functor taking rings $(R,+,\cdot)$ to the underlying set $R$, or intermediately, the underlying additive group $(R,+)$ in $\mathrm{Group}$ and ring maps to maps or intermediately to group maps.  
\end{example}
\begin{example}
    Another example is the forgetful functor 
    \begin{gather*}
        \mathrm{Mon} \rightarrow \mathrm{Set}_\ast\\
        (M,\cdot,e)\mapsto (M,e)\\
        f : (M,\cdot_M,e_M)\rightarrow (N,\cdot_N,e_N) \mapsto (f: (M,e_M)\rightarrow (N,e_N)), x\mapsto f(x)
    \end{gather*}
    which is well-defined since monoid homomorphisms fix the neutral element.
\end{example}
\begin{example}
    Given an algebraically closed field $K$, there is a contravariant functor
    \begin{gather*}
        \Gamma : \mathrm{Var}_K\rightarrow \mathrm{FinGenAlg}_K\\
        X \mapsto \Gamma(X)\\
        \varphi : X\rightarrow Y \mapsto \widetilde{\varphi} : \Gamma(Y)\rightarrow \Gamma(X)
    \end{gather*}
    where $\mathrm{FinGenAlg}_K$ is the category of finitely generated algebras over $K$.
\end{example}
\begin{definition}
    Consider a locally small category $\pazocal{C}$. We then for each object $X$ in $\pazocal{C}$ get a covariant functor 
    \begin{gather*}
        \Hom(X,\_) : \pazocal{C}\rightarrow \mathrm{Set}\\
        Y\mapsto \Hom(X,Y)\\
        f : A\rightarrow B \mapsto f_\ast: \Hom(X,A)\rightarrow \Hom(X,B) 
    \end{gather*}
    and a contravariant functor
    \begin{gather*}
        \Hom(\_,X) : \pazocal{C} \rightarrow \mathrm{Set}\\
        Y \mapsto \Hom(Y,X)\\
        f : A\rightarrow B \mapsto f^\ast: \Hom(X,B)\rightarrow \Hom(X,A) 
    \end{gather*}
    We say that these are the functors \emph{represented by $X$}.
\end{definition}
\begin{remark}
    Let $f: B\rightarrow C$ and $g:A\rightarrow B$ be given. Then for each $h\in \Hom(X,A)$
    $$
        (fg)_\ast(h)= (fg)h= f(gh)= f_\ast(gh)=f_\ast(g_\ast(h))=f_\ast g_\ast(h) \implies (fg)_\ast=f_\ast g_\ast.
    $$ 
    One easily checks that $(\fone_A) = \fone_{\Hom(X,A)}$. This verifies that $\Hom(X,\_)$ is a covariant functor. Note that $\Hom(\_,X)$ is just the covariant functor 
    \begin{gather*}
        \Hom(X,\_) : \pazocal{C}\op \rightarrow \mathrm{Set}
    \end{gather*}
    and is therefor a contravariant functor from $\pazocal{C}\rightarrow \mathrm{Set}$. 
\end{remark}
\begin{definition}
    For categories $\pazocal{C}$ and $\pazocal{C}'$ we define the \emph{the product category of $\pazocal{C}$ and $\pazocal{C}'$} to be category $\pazocal{C}\times \pazocal{C}'$ whose objects are ordered pairs $(X,X')$ where $X$ is an object of $\pazocal{C}$ and $X'$ is an object of $\pazocal{C}'$ and whose morphisms are ordered pairs of morphisms, denoted $(f,f'):(X,X')\rightarrow (Y,Y')$.\\
    Given a pair of morphisms in the product, $(f,f'):(X,X')\rightarrow (Y,Y')$ and $(g,g'): (Y,Y')\rightarrow (Z,Z')$, composition is defined as 
    $$
        (g,g')(f,f') := (gf,g'f') : (X,X')\rightarrow (Z,Z').
    $$
    Given an object $(A,A')$, we define $\fone_{(A,A')} = (\fone_A, \fone_{A'})$    
    {\Large How do we know such ordered pairs exist in general?}
\end{definition}
\begin{remark}
    Consider morphisms $(f,f'):(Z,Z')\rightarrow (W,W')$, $(g,g'): (Y,Y')\rightarrow (Z,Z')$ and $(h,h') : (X,X')\rightarrow (Y,Y')$. Then 
    \begin{align*}
        (f,f')((g,g')(h,h'))&=(f,f')(gh,g'h')=(f(gh),f'(g'h'))= ((fg)h,(f'g')h')\\ &=(fg,f'g')(h,h')
        =((f,f')(g,g'))(h,h').
    \end{align*} 
    Moreover, 
    \begin{align*}
        (h,h')\fone_{(X,X')}= (h\fone_X, h'\fone_{X'}) = (h,h') = (\fone_{Y} h,\fone_{Y'} h')=\fone_{(Y,Y')}(h,h'). 
    \end{align*}
    One can thus conclude that $\pazocal{C}\times \pazocal{C}'$
\end{remark}
\begin{definition}
    Given a locally small category $\pazocal{C}$ we define the \emph{two-sides represented functor}
    \begin{gather*}
        \Hom(\_,\_) : \pazocal{C}\op\times \pazocal{C}\rightarrow \mathrm{Set}\\
        (X,Y) \mapsto \Hom(X,Y)\\
        (f,h) : (A,B)\rightarrow (C,D) \mapsto (f^\ast,h_\ast) : \Hom(A,B)\rightarrow \Hom(C,D)
    \end{gather*}
    where $f: C\rightarrow A$ as a morphism in $\pazocal{C}$ and $(f^\ast,h_\ast)(g) := hgf$ for $g\in \Hom(A,B)$.
\end{definition}
\begin{lemma}
    Consider categories $\pazocal{C},\pazocal{D},\pazocal{E}$ and covariant functors, $\pazocal{F}: \pazocal{C}\rightarrow \pazocal{D}, \pazocal{G}:\pazocal{D}\rightarrow \pazocal{E}$. Then 
    \begin{gather*}
        \pazocal{G} \pazocal{F} : \pazocal{C}\rightarrow \pazocal{E}\\
        X \mapsto \pazocal{G}(\pazocal{F}(X))\\
        f: Y\rightarrow Z \mapsto \pazocal{G}(\pazocal{F}(f)): \pazocal{G}(\pazocal{F}(Y))\rightarrow \pazocal{G}(\pazocal{F}(Z))
    \end{gather*}
    is a functor called the \emph{composition of $\pazocal{G}$ with $\pazocal{F}$}. 
\end{lemma}
\begin{proof}
    Indeed, for morphisms $f: X\rightarrow Y$ and $g: Y\rightarrow Z$,
    \begin{align*}
        (\pazocal{G} \pazocal{F})(gf) &= \pazocal{G}(\pazocal{F}(gf))=\pazocal{G}(\pazocal{F}(g)\pazocal{F}(f))= \pazocal{G}(\pazocal{F}(g))\pazocal{G}(\pazocal{F}(f))\\
        &=(\pazocal{G}\pazocal{F})(g)(\pazocal{G}\pazocal{F})(f).
    \end{align*}
    Moreover,
    $$
        (\pazocal{G}\circ\pazocal{F})(\fone_X)=\pazocal{G}(\pazocal{F}(\fone_X))=\pazocal{G}(\fone_{\pazocal{F}(X)})=\fone_{\pazocal{G}(\pazocal{F}(X))}=\fone_{(\pazocal{G}\pazocal{F})(X)}.
    $$
\end{proof}
\begin{lemma}\label{FunctorCompositionIsAssociative}
    Composition of functors is associative.
\end{lemma}
\begin{proof}
    Consider functors $\pazocal{F}: \pazocal{C}\rightarrow \pazocal{D},\pazocal{G}:\pazocal{B}\rightarrow \pazocal{C},\pazocal{H}: \pazocal{A}\rightarrow \pazocal{B}$. Given an object $A$ in $\pazocal{A}$,
    \begin{align*}
        (\pazocal{F}\circ (\pazocal{G}\circ\pazocal{H}))(A)&= \pazocal{F}((\pazocal{G}\circ\pazocal{H})(A))= \pazocal{F}(\pazocal{G}(\pazocal{H}(A)))=(\pazocal{F}\circ\pazocal{G})(\pazocal{H}(A))\\
        ((\pazocal{F}\circ \pazocal{G})\circ \pazocal{H})(A).
    \end{align*}
    Once also easily checks that for each morphism $f:X\rightarrow Y$ 
    $$
        (\pazocal{F}\circ (\pazocal{G}\circ\pazocal{H}))(f)=((\pazocal{F}\circ\pazocal{G})\circ\pazocal{H})(f), 
    $$
    hence
    $$\pazocal{F}\circ(\pazocal{G}\circ\pazocal{H})=(\pazocal{F}\circ\pazocal{G})\circ\pazocal{H}.$$
\end{proof}
\begin{remark}\label{IdentityFunctor}
    In addition to functor composition being associative, there is an identity functor for each category 
    \begin{gather*}
        \pazocal{ID}_\pazocal{C}: \pazocal{C}\rightarrow \pazocal{C}\\
        X\mapsto X\\
        f : Y\rightarrow Z \mapsto f : Y\rightarrow Z
    \end{gather*}
    for which 
    $$
        \pazocal{F}\pazocal{ID}_\pazocal{C}=\pazocal{F}=\pazocal{ID}_\pazocal{D}\pazocal{F}.
    $$ 
\end{remark}
\begin{example}
    Note that it is not true that if a functor maps a morphism to an isomorphism, then said morphism is an isomorphism. Indeed, consider for example any functor $\pazocal{F}$ from a category $\pazocal{C}$ to the category with one object and one morphism (which will need to be the identity morphism), $\mathbbm{1}$. This will take any object to the one object and any morphism to identity morphism.   
\end{example}
\begin{definition}
    Let $\pazocal{F}:\pazocal{D}\rightarrow \pazocal{C}$ and $\pazocal{G}:\pazocal{E}\rightarrow \pazocal{C}$ be functors. We define the \emph{comma category} $\pazocal{F}\downarrow \pazocal{G}$ whose objects are triples $(D\in \pazocal{D}, E\in\pazocal{E}, f:\pazocal{F}(D)\rightarrow \pazocal{G}(E))$. A morphism $(D,E,f)\rightarrow (D',E',f')$  is a pair of morphisms $(h: D\rightarrow D',k: E\rightarrow E')$ so that
    $$
        \begin{tikzcd}
            \pazocal{F}(D) \arrow[r, "f"] \arrow[d,"\pazocal{F}(h)"] & \pazocal{G}(E) \arrow[d,"\pazocal{G}(k)"]\\
            \pazocal{F}(D')\arrow[r,"f'"] & \pazocal{G}(E')  
        \end{tikzcd}
    $$
    commutes
\end{definition}
\begin{remark}
    Consider morphisms 
    $$(h:D\rightarrow D',k:E\rightarrow E'):(D,E,f)\rightarrow (D',E',f')$$ 
    and 
    $$(h':D'\rightarrow D'',k':E'\rightarrow E''):(D',E',f')\rightarrow (D'',E'',f'').$$ 
    We define composition by 
    $$
        (h',k')(h,k) := (h'h,k'k)
    $$
    To see that this is well defined note that 
    $$
        \begin{tikzcd}
            \pazocal{F}(D) \arrow[r, "f"] \arrow[d,"\pazocal{F}(h'h)"] & \pazocal{G}(E) \arrow[d,"\pazocal{G}(k'k)"]\\
            \pazocal{F}(D'')\arrow[r,"f''"] & \pazocal{G}(E'')
        \end{tikzcd}
    $$
    commutes since $\pazocal{F}(h'h)=\pazocal{F}(h')\pazocal{F}(h)$, $\pazocal{G}(k'k)=\pazocal{G}(k')\pazocal{G}(k)$ and the diagram
    $$
        \begin{tikzcd}
            \pazocal{F}(D) \arrow[r, "f"] \arrow[d,"\pazocal{F}(h)"] & \pazocal{G}(E) \arrow[d,"\pazocal{G}(k)"]\\
            \pazocal{F}(D')\arrow[r,"f'"]\arrow[d,"\pazocal{F}(h')"] & \pazocal{G}(E') \arrow[d,"\pazocal{G}(k')"]\\
            \pazocal{F}(D'')\arrow[r,"f''"] & \pazocal{G}(E'')
        \end{tikzcd}
    $$
    commutes. Clearly composition is associative. Define the identity morphism for $(D,E,f)$ to be $(\fone_D,\fone_E)$. Note that this is indeed a morphism in $\pazocal{F}\downarrow \pazocal{G}$, since 
    $$
        \begin{tikzcd}
            \pazocal{F}(D) \arrow[r,"f"]\arrow[d,"\pazocal{F}(\fone_D)=\fone_{\pazocal{F}(D)}",swap] & \pazocal{G}(E)\arrow[d,"\pazocal{G}(\fone_{E})=\fone_{\pazocal{G}(E)}"]\\
            \pazocal{F}(D)\arrow[r,"f"] & \pazocal{G}(E) 
        \end{tikzcd}
    $$
    clearly commutes. It is also clear that this is the identity morphism in $\pazocal{F}\downarrow \pazocal{G}$.
\end{remark}
\begin{definition}
    Given functors $\pazocal{F}:\pazocal{D}\rightarrow \pazocal{C}$ and $\pazocal{G}:\pazocal{E}\rightarrow \pazocal{C}$ we define projection functors 
    \begin{gather*}
        \mathrm{dom}: \pazocal{F}\downarrow \pazocal{G} \rightarrow \pazocal{D}\\
        (D,E,f)\mapsto D\\
        (h,k) :(D_1,E_1,f_1) \rightarrow (D_2,E_2,f_2)\mapsto h: D_1\rightarrow D_2
    \end{gather*} 
    and 
    \begin{gather*}
        \mathrm{cod} : \pazocal{F}\downarrow\pazocal{G} \rightarrow \pazocal{E}\\
        (D,E,f)\mapsto E\\
        (h,k) :(D_1,E_1,f_1) \rightarrow (D_2,E_2,f_2)\mapsto k: E_1\rightarrow E_2
    \end{gather*}
\end{definition}
\begin{remark}
    Note that $C/\pazocal{C}$ is just a comma category; $\pazocal{F}_C\downarrow \pazocal{ID}_\pazocal{C}$, where $\pazocal{F}_C: \mathbbm{1} \rightarrow \pazocal{C}$ is the constant functor from the category with one object and one morphism and $\pazocal{ID}_\pazocal{C}$ is the identity functor on $\pazocal{C}$. Indeed an object in this category is a triple $(\ast\in \mathbbm{1},X\in \pazocal{C},f: C\rightarrow X)$, note that there is a unique such triple when given any morphism $f: C\rightarrow X$, so we may represent such a triple more succinctly by just providing a morphism. A morphism in $\pazocal{F}_C\downarrow \pazocal{ID}_\pazocal{C}$ from $f:C\rightarrow X$ to $g: C\rightarrow Y$ is a pair $(h:X\rightarrow Y, \id_C: C\rightarrow C)$ such that 
    $$
        \begin{tikzcd}
            C \arrow[r, "f"]\arrow[d,"\id_C",swap] & X\arrow[d, "h"]\\
            C \arrow[r, "g"] & Y  
        \end{tikzcd}
    $$
    commutes. Note that the second data point in this pair is redundant and that this is equivalent to 
    $$
        \begin{tikzcd}
            & C\arrow[ld, "f"]\arrow[rd, "g"]\\
            X \arrow{rr}{h} & & Y 
        \end{tikzcd}
    $$
    commuting. So the data specifying the two categories is the same. Note that projection functor simply map a morphism $f: C \rightarrow X$ to its domain and codomain respectively. We can also construct $\pazocal{C}/C$ as $\pazocal{ID}_\pazocal{C}\downarrow \pazocal{F}_C$.  
\end{remark} 
\subsubsection{Functors on Groups}
Fix a group $G$ and a category $\pazocal{C}$. A functors $\mathrm{B}G \rightarrow \pazocal{C}$, maps $G$ to an object $X$ in $\pazocal{C}$ and a group element $g\in G$ to an endomorphism $g^\star : X\rightarrow X$ with 
$$(gh)_\star = g_\star h_\star$$
and $e_\star = \fone_X$. Such a functor is called an \emph{action}. When $\pazocal{C} = \mathrm{Set}$, it is a $G$-set. When $\pazocal{C} = \mathrm{Vect}_K$ for some field $K$ it is called a \emph{$G$-representation}. Each endomorphism induced by $G$ is an isomorphism, again, by functoriality. 
\subsubsection{Categories of Categories}
When our foundations are suitably developed we may consider a category of categories when we have the last lemma of the prior section in mind. Naively one cannot consider such a category, since the question of whether such a category is an object in the category of categories leads to an instance of Russel's paradox. Instead one needs to consider some more refined concepts such as the following, 
\begin{proposition}
    Consider the class of small categories with morphisms being functors. This defines a locally small category denoted $\Cat$. 
\end{proposition}
\begin{proof}
    Note that the class of subclasses of the proper class of all sets is itself a class (in e.g. Morse-Kelley set theory). Note that a functor between categories, whose objects and class of morphisms are sets, is a function. I.e. such a functor 
    is a function 
    $$\Ob(\pazocal{C})\times \Hom(\pazocal{C}) \rightarrow \Ob(\pazocal{C}')\times \Hom(\pazocal{C}').$$
    Then the class of all such functors is a subclass of the class of all functions{\Large ?}. By Lemma~\ref{FunctorCompositionIsAssociative} and Remark~\ref{IdentityFunctor}, $\Cat$ is a category and it is locally small since $\Hom(\pazocal{C},\pazocal{D})$ is a set for each pair of small categories $\pazocal{C},\pazocal{D}$. 
\end{proof}
\begin{remark}
    The collection of all locally small categories with functors as morphisms, which will be denoted $\mathrm{CAT}$, will similarly define a category, which won't be locally small.  
\end{remark}
From the above we get a notion of isomorphism of categories with this we have for instance the following result
\begin{proposition}
    Given a category $\pazocal{C}$ and an object $A$, then 
    $$\pazocal{C}/A \simeq \left(A/\pazocal{C}\op\right)\op$$
\end{proposition}
\begin{proof}
    Define a functor 
    \begin{gather*}
        \pazocal{F} : \pazocal{C}/A \rightarrow \left(A/\pazocal{C}\op\right)\op
    \end{gather*}
    which maps an object $f: X\rightarrow A$ to itself. Note that the objects of $\left(A/\pazocal{C}\op\right)\op$ is the same as the objects of $\left(A/\pazocal{C}\op\right)$, which are morphisms of $\pazocal{C}\op$ with domain $A$. Such morphisms are exactly morphisms of $\pazocal{C}$ with codomain $A$. It follows that the assignment of objects is well-defined. Moreover $\pazocal{F}$ maps a morphism  
    $$
        \begin{tikzcd}
            X \arrow[rd, "f"'] \arrow[rr, "h"] & & Y\arrow[ld, "g"]\\
            & A 
        \end{tikzcd}
    $$ 
    to itself. Indeed, a morphism from $\lambda\in \Hom^{\pazocal{C}\op}(A,X)$ to $\mu\in \Hom^{\pazocal{C}\op}(B,X)$ in $\left(A/\pazocal{C}\op\right)\op$ is a morphism in $A/\pazocal{C}\op$ from $\mu$ to $\lambda$, $\nu\in \Hom^{\pazocal{C}\op}(Y,X)$ such that $\lambda = \nu\circ\op\mu$. In $\pazocal{C}$, this corresponds to $\nu: X\rightarrow Y$ being a morphism satisfying $\lambda = \mu\circ \nu$. This is exactly a morphism in $\pazocal{C}/A$. It follows that $\pazocal{F}$ is well-defined on morphisms as well. The above arguments also clearly shows that the functor 
    \begin{gather*}
        \pazocal{G}: \left(A/\pazocal{C}\op\right)\op \rightarrow \pazocal{C}/A 
    \end{gather*}
    taking objects and morphisms to themselves gives a well-defined functor whose left and right inverse is $\pazocal{F}$. It follows that $\pazocal{C}/A\simeq \left(A/\pazocal{C}\op\right)\op$ at least when this category can be shown to live in some category.  
\end{proof}
\begin{example}
    Another example of an isomorphism of categories is the functor 
    \begin{gather*}
        (\_)\op : \mathrm{CAT} \rightarrow \mathrm{CAT}\\
        \pazocal{C} \mapsto \pazocal{C}\op\\
        \pazocal{F} : \pazocal{C}\op \rightarrow \pazocal{C}\op \mapsto \pazocal{F}\op : \pazocal{C}\op \rightarrow \pazocal{D}\op
    \end{gather*}
    where for a given functor $\pazocal{F}:\pazocal{C}\rightarrow \pazocal{D}$\\
    \begin{gather*}
        \pazocal{F}\op : \pazocal{C}\op \rightarrow \pazocal{D}\op\\
        X \mapsto \pazocal{F}(X)\\
        f : X \rightarrow Y \mapsto f\op : Y \rightarrow X
    \end{gather*}
    Note that clearly $(\_)\op(\_)\op = \pazocal{ID}_\mathrm{CAT}$.
\end{example}
\begin{example}
    Consider the category $\mathrm{Set}^\partial$ whose objects are sets and whose morphisms are partial functions. For a given set $X$, we define 
    $$
        X_{+} := X\cup \{X\}
    $$
    and for a given partial map $f : X\dottedarrow Y$ we define 
    \begin{gather*}
        f_{+} : X_{+} \rightarrow Y_{+}
    \end{gather*}
    where 
    $$
        f_{+}(x) := 
        \begin{cases}
            f(x) & \text{if } x\in \dom\ f\\
            Y & \text{otherwise}
        \end{cases}
    $$
    We thus get a functor
    \begin{gather*}
        (\_)_{+} : \mathrm{Set}^\partial \rightarrow \mathrm{Set}_\ast\\
        X \mapsto X_{+}\\
        f : X \dottedarrow Y \mapsto f : X_{+}\rightarrow Y_{+}
    \end{gather*}
    Indeed, given partial maps $f: X\dottedarrow Y$ and $g : Y \dottedarrow Z$, note that we have a partial map $gf : X\dottedarrow Z$ with 
    $$\dom \ gf = \dom \ f \cap f^{-1}(\dom\ g).$$
    If $x\in \dom \ gf$, then $f(x)\in \dom \ g$, hence 
    $$
        (gf)_{+}(x)= gf(x) = g(f(x)) = g_{+}(f(x))= g_{+}(f_{+}(x))=g_{+}f_{+}(x).
    $$ 
    Suppose $x\notin \dom \ gf \subset X_{+}$. Then $x\notin \dom\ f$ or $x\notin f^{-1}(\dom \ g)$. In the first case $f(x)=Y$, hence
    $$
        (gf)_{+}(x) = Z = g_{+}(Y)=g_{+}(f_{+}(x))= g_{+}f_{+}(x). 
    $$
    Suppose $x\in \dom \ f$. Then  $f(x)\notin \dom \ g$, hence $g_{+}(f_{+} (x)) = Z$, hence 
    $$
        (gf)_{+}(x) = Z = g_{+}(f_{+}(x)) = g_{+}f_{+}(x).   
    $$
    We conclude that $(gf)_{+} = g_{+}f_{+}$. We define a left inverse to $(\_)_{+}$, by 
    \begin{gather*}
        \pazocal{U} : \mathrm{Set}_\ast \rightarrow \mathrm{Set}^\partial\\
        (X,x)\mapsto X\setminus \{x\}\\
        f : (X,x)\rightarrow (Y,y) \mapsto \pazocal{U}(f) : X\setminus\{x\} \dottedarrow Y\setminus \{y\}
    \end{gather*}
    where 
    $$\dom \ \pazocal{U}(f) := X\setminus f^{-1}(y)$$
    and on the domain of $\pazocal{U}(f)$, $\pazocal{U}(f)(a) = f(a)$. On objects $X$,
    $$
        \pazocal{U}(X_{+})=\pazocal{U}(X\cup\{X\},X) = (X\cup\{X\})\setminus\{X\} = X.
    $$
    and on morphisms $f : X\dottedarrow Y$, 
    $$
        \dom\ \pazocal{U}(f_{+}) = (X\cup\{X\})\setminus (f_{+})^{-1}(Y) = (X\cup\{X\})\setminus (X\setminus \dom\ f \cup \{X\}) = \dom\ f \cap X = \dom \ f.
    $$ 
    Moreover for $x\in \dom \ f$
    \begin{align*}
        \pazocal{U}(f_{+})(x) = f_{+}(x) = f(x),
    \end{align*}
    hence 
    $$
        \pazocal{U}(f_{+})= f.
    $$
    On the other hand 
    $$
        (\pazocal{U}(X,x))_{+} = (X\setminus \{x\})_{+} = (X\setminus \{x\})\cup\{X\setminus \{x\}\}.
    $$
    Thus $\pazocal{U}$ and $(\_)_{+}$ are not mutual inverses. However, 
    \begin{gather*}
        (X\setminus\{x\})\cup\{X\setminus\{x\}\}\\
        X\ni y\mapsto y\\
        X\setminus \{x\} \mapsto x
    \end{gather*}
    Defines a bijection. So every set $X\in \mathrm{Set}^\partial$ is isomorphic to $\pazocal{U}(Y,y)$ for some $(Y,y)\in \mathrm{Set}_{+}$. Moreover, given a map of pointed sets $f: (X\cup\{X\},X) \rightarrow (Y\cup\{Y\}, Y)$, $\left. f\right|_{X\setminus f^{-1}(Y)} : X\setminus f^{-1}(Y)\rightarrow Y$ gives rise to partial map $g : X\dottedarrow Y$ where for $x\in \dom\ g$,
    $$
        g_{+}(x) = f(x)
    $$
    and for $x\notin \dom \ g$,
    $$
        g_{+}(x) = Y = f(x).
    $$
    It thus follows that $g_{+}= f$. Additionally, suppose for partial maps $f,g : X \dottedarrow Y$  that 
    $$
        f_{+} = g_{+}.
    $$
    If $x\in \dom \ f$, then $Y\neq f(x)=f_{+}(x)=g_{+}(x)$, hence $g_{+}(x)=g(x)$, meaning $x\in \dom\ g$. It follows by symmetry that $\dom\ f = \dom \ g$ and thus readily that $ f= \left. f_{+} \right|_{\dom \ f} = \left. g_{+}\right|_{\dom\ g} = g$. Later we will see that this shows that $(\_)_{+}$ and $\pazocal{U}$ defines and equivalence of categories between $\mathrm{Set}^\partial$ and $\mathrm{Set}_{+}$. 
\end{example}
\subsubsection{Natural Transformations}
\begin{definition}
    A \emph{natural transformation} of functors $\pazocal{F},\pazocal{G} : \pazocal{C}\rightarrow \pazocal{D}$,
    $$\alpha : \pazocal{F} \Rightarrow \pazocal{G}$$
    is a collection of morphisms 
    $$\left\{\alpha_X : \pazocal{F}(X)\rightarrow \pazocal{G}(X) : X\in\Ob(\pazocal{C})\right\} $$
    called \emph{components} such that for any morphism $f: X\rightarrow X'$,
    $$
        \begin{tikzcd}
            \pazocal{F}(X) \arrow[r, "\alpha_X"] \arrow[d,"\pazocal{F}(f)"] & \pazocal{G}(X)\arrow[d,"\pazocal{G}(f)"]\\
            \pazocal{F}(X') \arrow[r,"\alpha_{X'}"] & \pazocal{G}(X')
        \end{tikzcd}
    $$ 
    commutes
\end{definition}
\begin{remark}
    A natural transformation is also denoted 
    $$
        \begin{tikzcd}
            \pazocal{C} \arrow[r,bend left = 50, "\pazocal{F}"{name=F}] \arrow[r, bend right = 50, "\pazocal{G}"{name=G},swap] & \pazocal{D}  \arrow[Rightarrow,from=F, to=G,"\alpha", shorten <= 10pt, shorten >= 10 pt]
        \end{tikzcd}
    $$
\end{remark}
\begin{definition}
    A \emph{natural isomorphism} is a natural transformation for which every component is an isomorphism. If two functors $\pazocal{F},\pazocal{G}: \pazocal{C}\rightarrow \pazocal{D}$ are naturally isomorphic via a natural transformation $\alpha$, we write $\alpha: \pazocal{F}\cong \pazocal{G}$ 
\end{definition}
\begin{lemma}
    Suppose $\alpha : \pazocal{F}\Rightarrow \pazocal{G}$ is a natural isomorphism with components $\{\alpha_X\}$. Then $\{\alpha_X^{-1}\}$ defines the components of a natural isomorphism $\alpha^{-1} : \pazocal{G} \Rightarrow \pazocal{F}$. 
\end{lemma}
\begin{proof}
    Let $f: X\rightarrow X'$ be a morphism. We need to check that the diagram 
    $$
        \begin{tikzcd}
            \pazocal{G}(X) \arrow[r,"\alpha_X^{-1}"] \arrow[d,"\pazocal{G}(f)"] & \pazocal{F}(X)\arrow[d,"\pazocal{F}(f)"] \\
            \pazocal{G}(X')\arrow[r,"\alpha_{X'}^{-1}"] & \pazocal{F}(X')
        \end{tikzcd}
    $$
    commutes. Indeed, using the fact that $\alpha_{X'}\pazocal{F}(f)= \pazocal{G}(f)\alpha_X$,
    \begin{align*}
        \pazocal{F}(f)\alpha_{X}^{-1} = \alpha_{X'}^{-1}\alpha_{X'}\pazocal{F}(f)\alpha_X^{-1} = \alpha_{X'}^{-1}\pazocal{G}(f)\alpha_X\alpha_X^{-1}= \alpha_{X'}^{-1}\pazocal{G}(f).
    \end{align*}
\end{proof}
\begin{lemma}\label{PreCompositionIsANaturalTransformationAndThisAssignmentIsInjective}
    Let $\pazocal{C}$ be a locally small category. Consider a morphism $f\in \Hom(X,Y)$. Then the collection of maps
    \begin{gather*}
        \{ f_\ast : \Hom(Z,X)\rightarrow \Hom(Z,Y), \Hom(Z,X)\ni g \mapsto fg \in \Hom(Z,Y) : Z\in \Ob(\pazocal{C})\}
    \end{gather*}
    defines a natural transformation $f_\ast : \Hom(\_,X)\Rightarrow \Hom(\_,Y)$. By duality $f^\ast : \Hom(Y,\_)\Rightarrow \Hom(X,\_)$ defines a natural transformation. Moreover if $f\neq f'$ for parallel morphisms $f,f'$, then $f_\ast \neq f'_\ast$
\end{lemma}
\begin{proof}
    Let a morphism $h : A\rightarrow B$ be given. Then
    $$
        \begin{tikzcd}
            \Hom(A,X)\arrow[r,"_Af_\ast"]\arrow[d,"{_Y}h^\ast"] & \Hom(A,Y)\arrow[d,"{_X}h^\ast"]\\
            \Hom(B,X) \arrow[r,"{_B}f_\ast"] & \Hom(B,Y)
        \end{tikzcd}
    $$
    commutes by associativity of composition. Suppose $f\neq f'$. Then ${_X}f_\ast(\fone_X) \neq {_X}f_\ast'(\fone_X)$, hence ${_X}f_\ast\neq {_X}f_\ast'$, which implies $f_\ast \neq f_\ast'$.
\end{proof}
\begin{example}
    Let $V$ a vector space over some field $K$. Let for any vector space $W$ over $K$, $W^\ast$ denote $\Hom(W,K)$, \emph{the dual vector space of $W$}, i.e. $(\_)^\ast= \Hom(\_,K)$.
    Consider the linear maps 
    \begin{gather*}
        {_V}\ev: V \rightarrow V^{\ast\ast}\\
        v \mapsto \ev_v
    \end{gather*}
    where $\ev_v$ is the linear map
    \begin{gather*}
        V^\ast \rightarrow K\\
        (l: V\rightarrow K)\mapsto l(v)
    \end{gather*}
    The collection $\{{_W}\ev : W\rightarrow W^{\ast\ast} : W\in\Ob(\mathrm{Vect}_K)\}$, defines a natural transformation of $\pazocal{ID}_{\mathrm{Vect}_K}$ with  $(\_)^{\ast\ast}$. The square 
    $$
        \begin{tikzcd}
            V \arrow[r,"{_V}\ev"] \arrow[d,"l"] &V^{\ast\ast}\arrow[d,"l^{\ast\ast}"]\\
            W \arrow[r,"{_W}\ev"] & W^{\ast\ast}
        \end{tikzcd}
    $$
    commutes. Indeed, 
    \begin{align*}
        (({_W}\ev l)(v))(\phi) &= \ev_{l(v)}(\phi) = \phi(l(v))= \phi l(v)=\ev_v(\phi l)=(\ev_vl^\ast)(\phi)=(l^{\ast\ast}(\ev_v))(\phi)\\ 
        &=((l^{\ast\ast}{_V}\ev)(v))(\phi) 
    \end{align*}
\end{example}
\subsubsection{Equivalence of Categories}
    \begin{definition}
        Let $\mathbbm{2}$ denote the category with two objects $0,1$ and a single non-identity arrow $0\rightarrow 1$. 
    \end{definition}
    We consider two functors 
    \begin{gather*}
        \iota_i : \mathbbm{1}\rightarrow \mathbbm{2}\\
        \ast \mapsto i\\
        \id_\ast \mapsto \id_i 
    \end{gather*}
    for $i=0,1$. 
    \begin{lemma}
        Fix functors $\pazocal{F},\pazocal{G}:\pazocal{C}\rightarrow \pazocal{D}$. There is a bijection of natural transformations $\alpha: \pazocal{F} \Rightarrow \pazocal{G}$ and \textbf{witnessing functors} $\pazocal{H}:\pazocal{C}\times \mathbbm{2}\rightarrow \pazocal{D}$ such that 
        $$
            \begin{tikzcd}
                \pazocal{C} \arrow[r,"\pazocal{I}_0"]\arrow[rd,"\pazocal{F}",swap] & \pazocal{C}\times \mathbbm{2} \arrow[d,"\pazocal{H}"] & \pazocal{C}\arrow[ld,"\pazocal{G}"]\arrow[l,"\pazocal{I}_1",swap]\\
                & \pazocal{D}
            \end{tikzcd}
        $$
        commutes. Here $\pazocal{I}_i: X\mapsto (X,i), f:Y\rightarrow Z\mapsto (f,\id_i)$
    \end{lemma}
    \begin{proof}
        Let a functor $\pazocal{H}$ be given as described. For each object $X\in \pazocal{C}$, define 
        \begin{gather*}
            \alpha_X^\pazocal{H} = \pazocal{H}(\id_X,0\rightarrow 1) 
        \end{gather*}
        This is a morphism with domain 
        $$
            \pazocal{H}(X,0)=\pazocal{H}\pazocal{I}_0(X)=\pazocal{F}(X)
        $$
        and codomain
        $$
            \pazocal{H}(X,1)=\pazocal{H}\pazocal{I}_1(X)=\pazocal{G}(X).
        $$
        We see that the collection 
        $$ 
            \left\{\alpha_X^\pazocal{H} : \pazocal{F}(X)\rightarrow \pazocal{G}(X) \mid X\in \pazocal{C}\right\}
        $$
        defines the components of a natural transformation, as
        \begin{align*}
            \pazocal{G}(f)\alpha_X &= \pazocal{G}(f)\pazocal{H}(\id_X,0\rightarrow 1) = \pazocal{H\pazocal{I}_1}(f)\pazocal{H}(\id_X,0\rightarrow 1)\\
            &= \pazocal{H}(f,\id_1)\pazocal{H}(\id_X,0\rightarrow 1)\\ 
            &= \pazocal{H}(f,0\rightarrow 1) = \pazocal{H}(\id_{X'},0\rightarrow 1)\pazocal{H}(f,\id_0)\\
            &= \alpha_{X'}\pazocal{H}\pazocal{I}_0(f)=\alpha_{X'}\pazocal{F}(f).
        \end{align*}
        We denote this natural transformation by $\alpha^\pazocal{H}$.
        Given a natural transformation $\alpha: \pazocal{F}\Rightarrow \pazocal{G}$, define a functor 
        \begin{gather*} 
            \pazocal{H}_\alpha : \pazocal{C}\times \mathbbm{2}\rightarrow \pazocal{D}\\
            (X,0) \mapsto \pazocal{F}(X)\\
            (X,1)\mapsto \pazocal{G}(X)\\
            (f,\id_0)\mapsto \pazocal{F}(f)\\
            (f,\id_1) \mapsto \pazocal{G}(f)\\
            (f: X\rightarrow Y,0\rightarrow 1) \mapsto \alpha_X 
        \end{gather*}
        One readily verifies that $\pazocal{F}=\pazocal{H}\pazocal{I}_0$ and that $\pazocal{G}=\pazocal{H}\pazocal{I}_1$. Note that by definition
        $$
            \alpha_X^{\pazocal{H}_\alpha} = \pazocal{H}(\id_X,0\rightarrow 1) = \alpha_X.
        $$
        Suppose we are given two witnessing functors $\pazocal{H},\pazocal{H}'$ with $\alpha^\pazocal{H} = \alpha^{\pazocal{H}'}$.
        For each object $X\in \pazocal{C}$
        $$\pazocal{H}(X,0)=\pazocal{H}\pazocal{I}_0(X)=\pazocal{F}(X)=\pazocal{H}'\pazocal{I}_0(X)=\pazocal{H}'(X,0).$$
        By a similar computation $\pazocal{H}(X,1)=\pazocal{H}'(X,1)$. Moreover, similar computations show that $\pazocal{H}(f,\id_i) = \pazocal{H}'(f,\id_i)$ for $i=0,1$. It remains to check that $\pazocal{H}(f,0\rightarrow 1) = \pazocal{H}'(f,0\rightarrow)$ which readily follows from the fact that $\alpha_X^\pazocal{H}=\alpha_X^{\pazocal{H}'}$ for each $X$.
    \end{proof}
    \begin{definition}
        Consider categories $\pazocal{C}$ and $\pazocal{D}$. An \emph{equivalence of categories} is a pair of functors $\pazocal{F}: \pazocal{C}\rightarrow \pazocal{D}$, $\pazocal{G}:\pazocal{D}\rightarrow \pazocal{C}$ and natural isomorphisms $\alpha : \pazocal{ID}_\pazocal{C} \cong \pazocal{G}\pazocal{F}$ and $\beta: \pazocal{ID}_\pazocal{D} \cong \pazocal{F}\pazocal{G}$. We say two categories are \emph{equivalent} if there is an equivalence of categories between them and in this case we may write $\pazocal{C}\cong \pazocal{D}$. 
    \end{definition}
    \begin{remark}
        Equivalence of categories is an equivalence relation. 
        \begin{itemize}
            \item Two copies of $\pazocal{ID}_\pazocal{C}$ together with two copies of components $\{\id_X : X\in \pazocal{C}\}$ defines an equivalence of categories of $\pazocal{C}$ with itself.
            \item Suppose $\pazocal{C}\cong \pazocal{D}$. Then there are functors $\pazocal{F}: \pazocal{C}\rightarrow \pazocal{D}$, $\pazocal{G}: \pazocal{D}\rightarrow \pazocal{C}$ together with natural isomorphisms $\alpha: \pazocal{ID}_\pazocal{C}\cong \pazocal{G}\pazocal{F}$ and $\beta:\pazocal{ID}_\pazocal{D}\cong \pazocal{F}\pazocal{G}$. This data by definition also expresses that $\pazocal{D}\cong \pazocal{C}$, when we implicitly use the commutativity of conjuction. 
            \item Suppose $\pazocal{C}\cong \pazocal{D}$ and $\pazocal{D}\cong \pazocal{E}$. There are then functors  $\pazocal{F}: \pazocal{C}\rightarrow \pazocal{D}$, $\pazocal{G}: \pazocal{D}\rightarrow \pazocal{C}$ together with natural isomorphisms $\alpha: \pazocal{ID}_\pazocal{C}\cong \pazocal{G}\pazocal{F}$ and $\beta:\pazocal{ID}_\pazocal{D}\cong \pazocal{F}\pazocal{G}$. Moreover there are functors $\pazocal{H} : \pazocal{D}\rightarrow\pazocal{E}$, $\pazocal{I}: \pazocal{E}\rightarrow \pazocal{D}$ with natural isomorphisms $\gamma : \pazocal{ID}_{\pazocal{D}} \cong \pazocal{I}\pazocal{H}$ and $\kappa : \pazocal{ID}_\pazocal{E} \cong \pazocal{H}\pazocal{I}$. Set 
            $$
                \mathfrak{F} := \pazocal{H}\pazocal{F} \text{ and } \mathfrak{G} := \pazocal{G}\pazocal{I}.
            $$
            Consider
            $$
                \mu := \left\{\mu_X:=\pazocal{G}(\gamma_{\pazocal{F}(X)})\alpha_X : X\rightarrow \mathfrak{GF}(X) \mid X\in \pazocal{C}\right\}
            $$
            This is a natural isomorphism. Indeed, 
            \begin{align*}
                \mathfrak{GF}(f)\mu_X&= \pazocal{GIHF}(f)\pazocal{G}(\gamma_{\pazocal{F}(X)})\alpha_X = \pazocal{G}(\pazocal{IH}(\pazocal{F}(f))\gamma_{\pazocal{F}(X)})\alpha_X\\ 
                &= \pazocal{G}(\gamma_{\pazocal{F}(X')}\pazocal{F}(f))\alpha_X = \pazocal{G}(\gamma_{\pazocal{F}(X')})\pazocal{GF}(f)\alpha_X\\
                &= \pazocal{G}(\gamma_{\pazocal{F}(X')})\alpha_{X'}f = \mu_{X'}f. 
            \end{align*}
            so $\mu$ is a natural transformation of $\pazocal{ID}_{\pazocal{C}}$ to $\mathfrak{GF}$. Moreover, every component of $\mu$ is a composition of isomorphism, hence every component of $\mu$ is an isomorphism. We construct a natural isomorphims $\nu : \pazocal{ID}_\pazocal{E}\cong \mathfrak{F}\mathfrak{G}$, by considering the collection of morphisms 
            $$
                \nu := \left\{ \nu_Y:= \pazocal{H}(\beta_{\pazocal{I}(Y)})\kappa_Y :Y \rightarrow \pazocal{HFGI}(Y)\mid Y\in\pazocal{E}\right\}.
            $$
            We thus conclude that $\pazocal{C}\cong \pazocal{E}$.
        \end{itemize}
    \end{remark}
    \begin{definition}
        Let $\pazocal{C},\pazocal{D}$ be locally small categories and $\pazocal{F} : \pazocal{C}\rightarrow \pazocal{D}$ a functor. We say that $\pazocal{F}$ is \emph{full} if $\pazocal{F}(\bullet):\Hom(X,Y)\rightarrow \Hom(\pazocal{F}(X),\pazocal{F}(Y)), f\mapsto \pazocal{F}(f)$ is surjective for each pair of objects $X,Y$ in $\pazocal{C}$. It is \emph{faithful} if $\pazocal{F}(\bullet)$ is injective. If $\pazocal{F}(\bullet)$ is both full and faithful, we say that $\pazocal{F}(\bullet)$ is \emph{fully faithful}.
    \end{definition}
    \begin{definition}
        A functor $\pazocal{F} : \pazocal{C}\rightarrow \pazocal{D}$ is \emph{essentially surjective on objects} if for every $D\in \Ob(\pazocal{D})$ there is a $C\in \Ob(\pazocal{C})$ such that $\pazocal{F}(C)\simeq D$. 
    \end{definition}
    \begin{definition}
        A faithful functor $\pazocal{F}:\pazocal{C}\rightarrow\pazocal{D}$ is an \emph{embedding} if it is injective on objects, i.e. if $\pazocal{F}(C)=\pazocal{F}(C')$ implies $C=C'$ for every pair of objects $C,C'\in \pazocal{C}$. If $\pazocal{F}$ is fully faithful and injective on objects, then it is a \emph{full embedding}.
    \end{definition}
    \begin{lemma}\label{UniqueChangeOfDomainAndCodomainViaIsos}
        Given a morphism $f: A\rightarrow B$ and isomorphisms $\alpha: A\simeq A'$, $\beta : B\simeq B'$ there is a unique morphism $f: A' \rightarrow B'$ so that the diagrams
        $$
            \begin{tikzcd}
                A\arrow[d, "f"] & A'\arrow[l,"\alpha",swap]\arrow[d,"f'"]\\
                B \arrow[r,"\beta",swap] & B' 
            \end{tikzcd}
            \begin{tikzcd}
                A\arrow[r,"\alpha"]\arrow[d, "f"] & A'\arrow[d,"f'"]\\
                B \arrow[r,"\beta",swap] & B'
            \end{tikzcd}
            \begin{tikzcd}
                A\arrow[d, "f"] & A'\arrow[l,"\alpha",swap]\arrow[d,"f'"]\\
                B & B' \arrow[l,"\beta"]
            \end{tikzcd}
            \begin{tikzcd}
                A\arrow[d, "f"]\arrow[r,"\alpha"] & A' \arrow[d,"f'"] \\
                B & B' \arrow[l,"\beta"]
            \end{tikzcd}
        $$
        commute.
    \end{lemma}
    \begin{proof}
        \textbf{Diagram 1:} Pick $f':= \beta f\alpha$. Suppose $f'': A'\rightarrow B'$ is another morphism making the diagram commute. Then $f'' = \beta f \alpha = f'$.\\
        \textbf{Diagram 2:} Note that $f' = \beta f \alpha^{-1}$ is the unique morphism making  
        $$
            \begin{tikzcd}
                A\arrow[d, "f"] & A'\arrow[l,"\alpha^{-1}",swap]\arrow[d,"f'"]\\
                B \arrow[r,"\beta",swap] & B'
            \end{tikzcd}
        $$
        commute, hence it is the unique morphism such that $ \alpha f' =\alpha\alpha^{-1}\beta f = \beta f$.\\
        \textbf{Diagram 3:} Apply diagram 1 to $\alpha : A'\rightarrow A $, $f$ and $\beta^{-1}: B'\rightarrow B$.\\
        \textbf{Diagram 4:} Apply diagram 1 to $\alpha^{-1}$, $f$ and $\beta^{-1}$.
    \end{proof}
    \begin{lemma}
        Consider a fully faithful functor $\pazocal{F}:\pazocal{C}\rightarrow \pazocal{D}$. For each $X,Y\in \Ob(\pazocal{C})$
            $$
                X\simeq Y \iff \pazocal{F}(X) \simeq \pazocal{F}(Y).
            $$
    \end{lemma}
    \begin{proof}
        Consider an isomorphism $h : \pazocal{F}(X)\rightarrow \pazocal{F}(Y)$. By fullness, there is an $f:X\rightarrow Y$ such that $\pazocal{F}(f)=h$. Moreover there is a $g:Y\rightarrow X$ such that $\pazocal{F}(g)=h^{-1}$. Then 
        $$\pazocal{F}(gf) = \pazocal{F}(g)\pazocal{F}(f)=h^{-1}h=\fone_{\pazocal{F}(X)} = \pazocal{F}(\fone_X) \implies gf = \fone_X.$$
        By symmetry, $fg = \fone_Y$.   
    \end{proof}
    \begin{theorem}\label{EquivOfCatIsFFFESO}
        A functor that is one functor comprising an equivalence of categories is fully faithful and essentially surjective on objects. For locally small categories, assuming the axiom of choice any fully faithful functor that is essentially surjective on objects is one functor in an equivalence of categories.  
    \end{theorem}
    \begin{proof}
        Consider functors $\pazocal{F}:\pazocal{C}\rightarrow \pazocal{D}$ and $\pazocal{G}:\pazocal{D}\rightarrow \pazocal{C}$ with natural isomorphism $\alpha: \pazocal{ID}_\pazocal{C}\cong \pazocal{GF}$ and $\beta : \pazocal{ID}_\pazocal{D}\cong \pazocal{FG}$.\\
         Suppose $f,g\in\Hom(X,Y)$ are given such that $\pazocal{F}(f)=\pazocal{F}(g)$. Note that
        $$
            \alpha_Y f =  \pazocal{F}(f)\alpha_X = \pazocal{F}(g)\alpha_X = \alpha_Y g \implies f=g 
        $$
        since $\alpha_Y$ is an isomorphism and thus a monomorphism. We thus conclude that $\pazocal{F}$ is faithful, and by a symmetric argument so is $\pazocal{G}$.\\
        Consider any morphism $h\in \Hom(\pazocal{F}(X),\pazocal{F}(Y))$. Then we have commutative diagrams 
        $$
            \begin{tikzcd}
                \pazocal{GF}(X) \arrow[d,"\pazocal{G}(h)"] & X \arrow[l,"\alpha_X"]\arrow[d,"f"]\\ 
                \pazocal{GF}(Y) & Y \arrow[l,"\alpha_Y"]   
            \end{tikzcd}
        $$  
        for some unique $f : X\rightarrow Y$. We thus by naturality get a commutative diagram
        $$
            \begin{tikzcd}
                \pazocal{GF}(X) \arrow[d,"\pazocal{GF}(f)"] & X \arrow[l,"\alpha_X"]\arrow[d,"f"]\\ 
                \pazocal{GF}(Y) & Y \arrow[l,"\alpha_Y"]   
            \end{tikzcd}
        $$
        Using the result of Lemma~\ref{UniqueChangeOfDomainAndCodomainViaIsos} diagram 2 to these two diagrams, by uniqueness, $\pazocal{G}(\pazocal{F}(f))=\pazocal{G}(h)$. By the faithfulness of $\pazocal{G}$, 
        $$
            \pazocal{F}(f) = h.
        $$  
        We thus conclude that $\pazocal{F}$ (and $\pazocal{G}$) are fully faithful functors.\\ 
        Consider an object $D\in \pazocal{D}$. Set $C= \pazocal{G}(D)$. Then 
        $$
            \begin{tikzcd}
                D \arrow[r,"\beta_D"]\arrow[d,"\id_D"] & \pazocal{F}(C) = \pazocal{FG}(D) \arrow[d,"\id_{\pazocal{F}(C)}"]\\
                D \arrow[r,"\beta_D"] & \pazocal{F}(C) = \pazocal{FG}(D)
            \end{tikzcd}
        $$
        hence $D\simeq \pazocal{F}(C)$.\\ 
        Consider a fully faithul functor $\pazocal{F} : \pazocal{C} \rightarrow \pazocal{D}$ that is essentially surjective on objects. We construct the functor 
        \begin{gather*}
            \pazocal{G} : \pazocal{D} \rightarrow \pazocal{C}\\ 
            D \mapsto \pazocal{G}(D)\\
            f : D_1 \rightarrow D_2 \mapsto \pazocal{G}(f) : \pazocal{G}(D_1) \rightarrow \pazocal{G}(D_2)
        \end{gather*}
       to make $\pazocal{F}$ and $\pazocal{G}$ equivalence of categories. For objects, we do this by for each $D\in \Ob(\pazocal{D})$ considering 
       $$
            \pazocal{C}_D := \{ C\in \Ob(\pazocal{C}) : \pazocal{F}(C)\simeq D\} 
       $$
       to obtain a family of sets indexed by objects in $\pazocal{D}$
       $$
            \pazocal{X} := \{ \pazocal{C}_D : D\in \Ob(\pazocal{D})\}.  
       $$
       Since $\pazocal{F}$ is essentially surjective on objects, this is a family of non-empty set, hence by the axiom of choice there is a well-defined assigment
       \begin{gather*}
            \pazocal{G}:\Ob(\pazocal{D}) \rightarrow \pazocal{X} \rightarrow \bigcup \pazocal{X}\subset \Ob(\pazocal{C})\\
            D \mapsto \pazocal{C}_D \mapsto \pazocal{G}(D).
       \end{gather*}
        Hence $\pazocal{FG}(D)\simeq D$. Denote this isomorphism by $\alpha_D$. Fix a morphism $f:D\rightarrow D'$ 
        $$
            \begin{tikzcd}
                \pazocal{FG}(D) \arrow[r,"\alpha_D"]\arrow[d,"\exists!h",dashed,swap] & D \arrow[d,"f"]\\
                \pazocal{FG}(D') \arrow[r,"\alpha_{D'}",swap] & D'
            \end{tikzcd}
        $$
        Since $\pazocal{F}$ is full there is a morphism $\pazocal{G}(f) : \pazocal{G}(D) \rightarrow \pazocal{G}(D')$, such that $\pazocal{F}(\pazocal{G}(f)) = h$. This assignment is well-defined, since if $k: \pazocal{G}(D)\rightarrow \pazocal{G}(D')$ is another morphism with $\pazocal{F}(k)= h = \pazocal{F}(\pazocal{G}(f))$, then by faithfulness of $\pazocal{F}$, $k = \pazocal{G}(f)$.\\
        We now check that $\pazocal{G}$ is a functor. Consider morphisms $f:X\rightarrow Y$ and $g:Y\rightarrow Z$. Note that 
        \begin{align*}
            \pazocal{F}(\pazocal{G}(g)\pazocal{G}(f))\alpha_X &=\pazocal{F}(\pazocal{G}(g))\pazocal{F}(\pazocal{G}(f))\alpha_X = \pazocal{F}(\pazocal{G}(g))\alpha_Yf = \alpha_Z gf, 
        \end{align*}
        hence $\pazocal{G}(g)\pazocal{G}(f)=\pazocal{G}(gf)$. We also have that 
        $$\pazocal{F}(\id_{\pazocal{G}(X)})\alpha_X = \id_{\pazocal{FG}(X)} \alpha_X = \alpha_X \id_X = \pazocal{F}(\pazocal{G}(\id_X))\alpha_X\implies \id_{\pazocal{G}(X)} = \pazocal{G}(\id_X). $$
        Note that for any morphism $f:D\rightarrow D'$ by definition 
        $$
            \begin{tikzcd}
                D \arrow[r,"\alpha_D^{-1}"]\arrow[d,"f"] & \pazocal{FG}(D)\arrow[d,"\pazocal{FG}(f)"]\\
                D' \arrow[r,"(\alpha_{D'})^{-1}"] & \pazocal{FG}(D')
            \end{tikzcd}
        $$
        commutes, so we have a natural isomorphism $\alpha: \pazocal{ID}_\pazocal{D}\simeq \pazocal{FG}$.\\
        Fix an object $C\in \pazocal{C}$. Since $\pazocal{F}$ is fully faithful, there is an isomorphism $\beta_C : C\rightarrow \pazocal{GF}(C)$ such that $\pazocal{F}(\beta_C) = \alpha_{\pazocal{F}(C)}^{-1} : \pazocal{F}(C)\rightarrow \pazocal{FGF}(C)$. Fix a morphism $f: C\rightarrow C'$. Then 
        \begin{align*} 
            \pazocal{F}(\pazocal{GF}(f)\beta_C) &= \pazocal{F}(\pazocal{GF}(f))\pazocal{F}(\beta_C)= \pazocal{FG}(\pazocal{F}(f))\alpha_{\pazocal{F}(C)}^{-1} = \alpha_{\pazocal{F}(C')}^{-1}\pazocal{F}(f)= \pazocal{F}(\beta_{C'})\pazocal{F}(f)\\
            &= \pazocal{F}(\beta_{C'} f).
        \end{align*}
        By the faithfulness of $\pazocal{F}$ we thus have that 
        $$
            \pazocal{GF}(f)\beta_C = \beta_{C'}f,
        $$
        hence $\beta: \pazocal{ID}_{\pazocal{D}} \cong\pazocal{GF}$. We thus conclude that $\pazocal{C}\cong \pazocal{D}$.
    \end{proof}
    \begin{example}
        We aim to show a an equivalence of categories implicitly known from the linear algebra of finite dimensional vector space, which is for a fixed field $K$ that the study of $n\times n$ matrices over $K$ is the "same" as the study of finite dimensional vector spaces. To make this concrete, we consider two locally small categories:
        \begin{itemize}
            \item $\mathrm{Mat}_K$, whose objects are $\Z_{\geq 1}$ and whose morphisms are $\{\mathrm{Mat}_{n\times m}(K) : n,m\in \Z_{\geq 1}\}$. composition is given by matrix multiplication. Note we know (or will know) that for an $n\times m$ and an $m\times l$ matrix the resulting matrix from matrix multiplication is an $n\times l$ matrix. We also know that matrix multiplication is associative and the identity matrix is the neutral element with respect to matrix multiplication and thus plays the role of identity morphism in this context.
            \item $\mathrm{Vect}^{\mathrm{fd}}_K$, whose objects is the proper class of finite dimensional vector spaces and whose morphism are linear maps. This is a subcategory of $\mathrm{Set}$.  
        \end{itemize}    
        We aim to show that $\mathrm{Mat}_K\cong \mathrm{Vect}^{\mathrm{fd}}_K$. To do this, we first introduce the category $\mathrm{Vect}_K^{\mathrm{basis}}$. An object in this category is a pair $V,\pazocal{V}$, where $V$ is a finite dimensional vector space and $\pazocal{V}$ is a basis of $V$. The morphisms are linear maps. Let $\pazocal{E}_n$ denote the standard basis of $K^n$. Consider the assignment
        \begin{gather*}
            K^{(\_)} : \mathrm{Mat}_K \rightarrow \mathrm{Vect}^{\mathrm{basis}}_K\\
            n \mapsto (K^n,\pazocal{E}_n)\\
            \mathrm{Mat}_{n\times m}(K)\ni A \mapsto (L_A : K^m\rightarrow K^n, v\mapsto A v)
        \end{gather*}  
        one readily verifies that constitutes a functor. With some general linear algebra knowledge and it also easy to verify that it is fully faithful and essentially surjective on objects. Indeed, if $V$ is an $n$-dimensional vector space with basis $\pazocal{V}=\{v_1,\dots,v_n\}$, $(V,\pazocal{V})\simeq K^n$, by the map 
        $$K^n\ni (a_1,\dots,a_n)\mapsto \sum_1^n a_iv_i,$$
        showing that $K^{(\_)}$ is essentially surjective on objects.\\ 
        If $L_A = L_B$ for some $n\times m$ matrices $A,B$, we know that $A_{ij}=(L_A e_j)_i = (L_B e_j)_i = B_{ij}$, for each $i$ and $j$, hence $A=B$. This means $K^{(\_)}$ is faithful.\\
        Consider a linear map $L : (K^m,\pazocal{E}_m)\rightarrow (K^n,\pazocal{E}_n)$. Set $A:= (L(e_j)_i)$. For an arbitrary $v\in K^m$ we have 
        $$
            L_A (v) = (L(e_j)_i) v = (\sum_1^m L(e_j) v_j)= \sum_1^m L(v_j e_j) = L(v). 
        $$
        We thus get that $L = L_A$, which means $K^{(\_)}$ is full.\\
        Consider the forgetful functor $\mathrm{Vect}_K^\mathrm{basis}\rightarrow \mathrm{Vect}_K^\mathrm{fd}$ which forgets the basis on objects. Clearly this is also a full faithful functor that is essentially surjective on objects. By Theorem~\ref{EquivOfCatIsFFFESO} we then get that $\mathrm{Mat}_K\simeq \mathrm{Vect}_K^\mathrm{fd}$.
    \end{example}
    \begin{definition}
        A category is \emph{connected} if for each pair of objects $X,Y$ there is a finite sequence of objects
        $$
            X_0:= X, X_1,\dots,X_{n-1},X_n := Y
        $$
        and for each $i\in \{0,\dots,n-1\}$ there is a morphism $f : X_i\rightarrow X_{i+1}$ or $f: X_{i+1}\rightarrow X_i$.
    \end{definition}
    \begin{proposition}
        A connected groupoid $\pazocal{G}$ is equivalent to the category induced by the automorphism group $G_X:= \mathrm{Aut}(X)$ for any object $X\in \Ob(\pazocal{G})$.
    \end{proposition}
    \begin{proof}
        We construct a functor, 
        \begin{gather*}
           \pazocal{F}:\mathrm{B}G_X \rightarrow \pazocal{G}\\
           G_X \mapsto X\\ 
           (f : G_X\rightarrow G_X) \mapsto (f: X \rightarrow X) 
        \end{gather*}
        This is full since any morphism $f:X\rightarrow X$ in $\pazocal{G}$ is an isomorphism, hence $f$ is a morphism in $G_X$ and by definition $\pazocal{F}(f: G_X\rightarrow G_X)= f:X\rightarrow X$. It is also faithful, since if $\pazocal{f},\pazocal{g}: X\rightarrow Y$ are such that $\pazocal{f}=\pazocal{g}$, then trivially $f=g$. Note that given an object $Y$ in $\pazocal{G}$, there is a sequence of objects $X_0 = X,\dots,X_n = Y$ and a morphism $f_i : X_i\rightarrow X_{i+1}$ or $f: X_{i+1}\rightarrow X_i$ for $i=0,\dots,n-1$. Each $f_i$ is an isomorphism, hence $\pazocal{F}(G_X)=X\simeq Y$. This implies that $\pazocal{F}$ is essentially surjective on objects. We conclude that $\pazocal{G} \cong \mathrm{B}G_X$ for each $X\in \Ob(\pazocal{G})$.
    \end{proof}
    The upshot of the above result is that, as categories, a connected groupoid is just an automorphism group. In particular, a one object groupoid is, again, as categories, the same as the automorphism group of its one object. This makes precise the fact that, in category theory, groups are just one object groupoids. 
    \begin{corollary}
        The category of one object groupoids, $\pazocal{GRP}_1$, is equivalent to the category of groups $\mathrm{Grp}$. 
    \end{corollary}   
    \begin{proof}
        We construct the functor
        \begin{gather*}
            \Omega: \pazocal{GRP}_{1} \rightarrow \mathrm{Grp}\\
            (\pazocal{G},X)\mapsto \mathrm{Aut}(X)\\
            \pazocal{F} : (\pazocal{G},X)\rightarrow (\pazocal{G}',X') \mapsto \mathrm{Aut}(X)\ni f\mapsto \pazocal{F}(f) \in \mathrm{Aut}(X')
        \end{gather*}
        This is a full. Indeed consider $f : \mathrm{Aut}(X)\rightarrow \mathrm{Aut}(X')$. We construct a functor
        \begin{gather*}
            \pazocal{F}_f : (\pazocal{G},X)\rightarrow (\pazocal{G},X')\\
            X\mapsto X'\\
            g : X\rightarrow X \mapsto f(g) : X'\rightarrow X'
        \end{gather*}
        clearly $\Omega(\pazocal{F}_f)= f$. Suppose $\Omega(\pazocal{F}_0) = \Omega(\pazocal{F}_1)$. Take a morphism $g:X\rightarrow X$. Then $\pazocal{F}_0(g) = \Omega(\pazocal{F}_0)(g)=\Omega(\pazocal{F}_1(g)) = \pazocal{F}_1(g)$, hence $\pazocal{F}_0=\pazocal{F}_1$. We conclude that $\Omega$ is faithful. Take an arbitrary group $G\in \mathrm{Grp}$. Then $G\simeq \Omega(\mathrm{B}G,G)$. 
    \end{proof}
    \begin{definition}
        Consider a locally small category $\pazocal{C}$. We define the \emph{isomorphism class} of an object $X$ in $\pazocal{C}$ to be the class of all objects isomorphic to $X$.
    \end{definition}
    \begin{definition}
        A category $\pazocal{C}$ is \emph{skeletal}, if each isomorphism class of $\pazocal{C}$ contains a single object. 
    \end{definition}
    \begin{lemma}
        An equivalence of skeletal categories is an isomorphism of categories. 
    \end{lemma}
    \begin{proof}
        
    \end{proof}
    \begin{proposition}
        Consider a category $\pazocal{C}$. There is a skeletal category $\mathrm{Sk}\pazocal{C}$ equivalent to $\pazocal{C}$. It is unique up to isomorphism.   
    \end{proposition}
    \begin{proof}
        The objects of $\mathrm{Sk}\pazocal{C}$ are given by choosing one object from each isomorphism class (here implicitly we use some version of the axiom of choice, so some assumption on smallness is probably appropriate as to not discuss foundations). The morphisms are those with domain and codomain in the aforementioned objects. The inclusion of $\mathrm{Sk}\pazocal{C}$ is trivially fully faithful and essentially surjective on objects. We thus conclude that $\mathrm{Sk}\pazocal{C}\cong \pazocal{C}$. Consider another skeletal category $\pazocal{D}$ equivalent to $\pazocal{C}$ via a fully faithful functor $\pazocal{F}$ that is essentially surjective on objects. Any object $X$ in $\pazocal{C}$ has a unique object $X_\pazocal{D}$ such that $X\simeq \pazocal{F}(X_\pazocal{D})$. The assignment $\mathrm{Sk}\pazocal{C}\rightarrow \pazocal{D}, X \mapsto X_{\pazocal{D}}$ is thus a bijective assignment of objects. Given a morphism $f: X \rightarrow Y$ in $\mathrm{Sk}\pazocal{C}$, we make the assignment $f\mapsto f_\pazocal{D} : X_\pazocal{D}\rightarrow Y_\pazocal{D}$, where $f_\pazocal{D}$ is the unique morphism with $\pazocal{F}(f_{\pazocal{D}})=f$. The inverse assignment is simply $\pazocal{F}$ on morphisms.  
    \end{proof}
    We have seen that equivalence of categories does not necessarily preserve smallness. For instance $\mathrm{Mat}_K$ is small while $\mathrm{Vect}_K^\mathrm{fd}$ is locally small but not small since the collection of finite dimensional vector spaces is a proper class. Indeed for any non-empty set $X$, we can form $K[X]$, the free vector space generated by $X$. If $\mathrm{Vect}_K^\mathrm{fd}$ was a set, then $K[\{X\}]$ would be an element of $\mathrm{Vect}_K^\mathrm{fd}$ for every set $X$. Then $S:=\{K[\{X\}] : X \text{ is a set}\}$ is a subset of $\mathrm{Vect}_K^\mathrm{fd}$. But then $ \bigcup S$ is a set, which every set is an element of. But no such set can exist as an immediate consequence axiom of foundation (cf. Example~\ref{ASetCannotHaveItselfAsAnElement}). In this sense smallness is an ugly notion. We can introduce a weaker notion of smallness, which is preserved by equivalence of categories. 
    \begin{definition}
        A category is \emph{essentially small} if it is equivalent to a small category.
    \end{definition}
    Another property not preserved by equivalence is discreteness. Take for instance the example of a poset $(X,\leq)$ with non-trivial arrows, eg. $X=\N$. This also has an associated discrete category (cf. Example~\ref{SetIsDiscreteCategory}). There is a fully faithful functor, surjective on objects, 
    \begin{gather*}
        X\rightarrow (X,\leq)\\
        x\mapsto x\\
        x\mapsto x\leq x
    \end{gather*}
    hence $X\cong (X,\leq)$. Again we may consider the class of \emph{essentially discrete categories}, i.e. categories that are equivalent to a discrete category.
    \begin{proposition}
        If $\pazocal{C}$ is a locally small category that is equivalent to some category $\pazocal{D}$, then $\pazocal{D}$ is locally small. 
    \end{proposition}
    \begin{proof}
        Consider a morphism $f: X\rightarrow Y$ in $\pazocal{D}$. We find objects $A,B$ in $\pazocal{C}$ such that $\alpha_X: X\simeq \pazocal{F}(A)$ and $\alpha_Y : Y\simeq \pazocal{F}(B)$. Consider $f' = \alpha_Y f \alpha_X^{-1} : \pazocal{F}(A)\rightarrow \pazocal{F}(B)$. For some unique $g: A\rightarrow B$, $\pazocal{F}(g)=f'$, hence 
        $$
            \beta^{-1}...
        $$ 
    \end{proof}
\subsubsection{Diagrams}
    \begin{definition}
        A \emph{(commutative) diagram} in a category $\pazocal{C}$ is a functor from a small category, called the \emph{indexing category}, to $\pazocal{C}$. 
    \end{definition}
    \begin{example}
        An example arises the category $\mathbbm{2}\times \mathbbm{2}$. This category has objects $(0,0),(0,1),(1,0),(1,1)$ and arrows $(\id_i,\id_j)$ where $i,j=0,1$ and $(\id_i,0\rightarrow 1)$, $(0\rightarrow 1,\id_i)$ and $(0\rightarrow 1,0\rightarrow 1)$. We thus have a category with four objects, and five non-trivial arrows: one arrow from $(0,0)$ to each of the other objects, one arrow from $(0,1)$ and $(1,0)$ to $(1,1)$. We may illustrate this as a directed graph.
        $$
            \begin{tikzcd}
                (0,0) \arrow[r]\arrow[d]\arrow[rd] & (0,1)\arrow[d]\\
                (1,0) \arrow[r] & (1,1)
            \end{tikzcd}
        $$
        A diagram $D : \mathbbm{2}\times \mathbbm{2} \rightarrow \pazocal{C}$ maps the four objects in $\mathbbm{2} \times \mathbbm{2}$ to four objects in $\pazocal{C}$, $A,B,C,D$ say. Moreover the five non-trivial arrows of $\mathbbm{2}\times \mathbbm{2}$ to morphisms $f_{0001} : A\rightarrow B$, $f_{0010}: A\rightarrow C$, $f_{0011}: A\rightarrow D$, $f_{0111}: B \rightarrow D$ and $f_{1011} : C\rightarrow D$. Functoriality assures that the diagram (in the informal sense),
        $$
            \begin{tikzcd} 
                A \arrow[r,"f_{0001}"]\arrow[rd, "f_{0011}",swap]\arrow[d,"f_{0010}",swap] & B\arrow[d,"f_{0111}"]\\
                C \arrow[r,"f_{1011}",swap] & D
            \end{tikzcd}
        $$ 
        commutes, i.e. $f_{0111}f_{0001}=f_{0011}=f_{1011}f_{0010}$
    \end{example}
    In general, the abstract definition of diagram lets us study the consequences of a list of morphism relations in terms of directed graph underlying these. This proof technique is called \emph{diagram chasing}.
    \begin{lemma}
        Functors preserve commutative diagram.
    \end{lemma}
    \begin{proof}
        Indeed given a functor $\pazocal{F} : \pazocal{C}\rightarrow \pazocal{D}$ and a diagram $D : \pazocal{I} \rightarrow \pazocal{C}$. Then $\pazocal{F}D : \pazocal{I}\rightarrow \pazocal{D}$ is a diagram in $\pazocal{D}$.
    \end{proof}
    \begin{lemma}
        Consider a composable sequence of morphisms $f_1,\dots,f_n$ in $\pazocal{C}$. If 
        $$
            f_jf_{j-1}\cdots f_{i+1}f_i = g_m\cdots g_1
        $$
        for some composable morphisms $g_1,\dots,g_m$ in $\pazocal{C}$, then 
        $$
            f_n \cdots f_1 = f_n\cdots f_{j+1} g_m\cdots g_1 f_{i-1}\cdots f_1
        $$
    \end{lemma}
    \begin{proof}
        Indeed, this is just a result of pre-composing with $f_{i-1}\cdots f_1$ and post-composing with $f_n\cdots f_{j+1}$ on both sides of the equality. 
    \end{proof}
   
    \begin{definition}
        An object $I\in \Ob(\pazocal{C})$ is \emph{initial} if for each $X\in \Ob(\pazocal{C})$ there is a unique morphism $I\rightarrow X$.\\
        An object $T\in \Ob(\pazocal{C})$ is \emph{terminal} if for each $X\in\Ob(\pazocal{C})$ there is a unique morphism $X\rightarrow T$, i.e. $T$ is an initial object in $\pazocal{C}\op$.
    \end{definition}
    \begin{example}
        \begin{enumerate}
            \item $0$ is both initial and terminal in $\mathrm{Group},\mathrm{Ab},\mathrm{Mod}_R,\mathrm{Vect}_K$. It is also terminal in $\mathrm{Ring}$ (the category of initial rings) but not initial since there is no ring homomorphism from the zero ring. 
            \item $\Z$ is an initial object in $\mathrm{Ring}$. Indeed 
            \begin{gather*}
                f_R : \Z \rightarrow R\\
                n \mapsto \sum_1^n 1_R
            \end{gather*}
            defines a ring homomorphism and for any ring homomorphism $f : \Z \rightarrow R$ and integer $n\in\Z$, 
            $$
                f(n)= f\left(\sum_1^n 1 \right) = \sum_1^n f(1)= \sum_1^n 1_R=f_R(n).
            $$
            \item The rational numbers are initial in $\mathrm{Field}_{0}$, the category of fields of characteristic $0$. Indeed, there is a functor from the category of integral domains to the category of fields given by 
            \begin{gather*}
                R\mapsto \mathrm{Frac}(R)\\
                f : R\rightarrow S \mapsto \mathrm{Frac}(f):\mathrm{Frac}(R)\ni\frac{x}{y}\mapsto \frac{f(x)}{f(y)} \in \mathrm{Frac}(S) 
            \end{gather*}
            We claim that $\mathrm{Frac}(f_R)$ is the unique ring map from $\mathrm{Frac}(\Z) = \Q$ to $K$ for any characteristic $0$ field $K$. Note that given a homomorphism $f: \Q\rightarrow K$, $\left. f \right|_\Z : \Z \rightarrow K$ is equal to $f_K$, hence 
            $$f(\frac{a}{b})= f(a)f(b)^{-1}=f_K(a)f_K(b)^{-1}=\frac{f_K(a)}{f_K(b)}=\mathrm{Frac}(f_K)(\frac{a}{b}),$$
            for each $\frac{a}{b}\in \Q$, hence $\mathrm{Frac}(f_K)=f$.
            \item If a preordered set has a minimum this will be the initial object. For instance $0\in \N$ is the initial object.
            \item For a monoid $M$, the category $\mathrm{B}M$ only has an initial or terminal object when $M$ is trivial. 
            \item The empty set, $\emptyset$, is the initial object in $\mathrm{Set}$. Indeed, the only function with domain $\emptyset$ is $\emptyset$.
            \item In $\mathrm{Set}_\ast$, $(\{\ast\},\ast)$ is both initial and terminal.
            \item In $\mathrm{Cat}$, the category $(\emptyset,\emptyset)$ is the initial object. Indeed, there is a functor that maps objects and morphisms via the empty function and these are the only mappings with the empty set as domain. $\fone$ is the terminal object. Indeed, for any locally small category $\pazocal{C}$, there is a functor 
            \begin{gather*}
                \ast : \pazocal{C}\rightarrow\fone\\
                X\mapsto \ast\\
                f: X\rightarrow Y \mapsto \fone_\ast
            \end{gather*}  
            Since $\fone$ have only one object and one morphism any such functor must map every object to this one object and every morphism to this one morphism, hence it is equal to $\ast$.  
        \end{enumerate}
    \end{example}
    \begin{example}
        Faithful functors need not preserve epimorphism. Indeed, the ring map $\Z\hookrightarrow \Q$ is epi but not surjective, but under the forgetful functor $\mathrm{Ring}\rightarrow \mathrm{Set}$, it is not an epi, since $\mathrm{Set}$ is balanced. In $\mathrm{Ring}\op$, it is a mono but in $\mathrm{Set}\op$ it is not. 
    \end{example}
    \begin{proposition}
        Consider a category with an intial object $I$ and a terminal object $T$. If there is a map $g: T\rightarrow I$, then $I\simeq T$. It follows that $T$ is initial.
    \end{proposition}
    \begin{proof}
        There is a unique $f: I \rightarrow T$. Then the morphism $fg : T\rightarrow T$ is the identity on $T$ since $T$ is terminal and $gf : I\rightarrow I$ is the identity on $I$ since $I$ is initial.
    \end{proof}
    \begin{definition}
        An object that is both initial and terminal is called a zero object
    \end{definition}
    \begin{remark}
        Note that initial, terminal and zero objects are unique up to unique isomorphism, so we may aswell say \emph{THE} initial, terminal and zero object. Indeed, take a initial objects $I_1,I_2$. Then $f: I_1\rightarrow I_2$ and $g: I_2 \rightarrow I_1$ define mutual inverses. A dual argument is made for terminal objects
    \end{remark}
    \begin{definition}
        Consider an endofunctor $\pazocal{T}: \pazocal{C}\rightarrow \pazocal{C}$. A \emph{coalgebra for $\pazocal{T}$} is an object $X$ in $\pazocal{C}$ with a morphism $\gamma : X\rightarrow \pazocal{T}(X)$. We define a category of coalgebras for $\pazocal{T}$ by letting a morphism of coalgebras $(X,\gamma)\rightarrow (X', \gamma')$ be a morphism $f: X\rightarrow X'$ such that 
        $$
            \begin{tikzcd}
                X \arrow[r,"f"]\arrow[d,"\gamma",swap] & X'\arrow[d,"\gamma'"]\\
                \pazocal{T}(X) \arrow[r,"\pazocal{T}(f)",swap] & \pazocal{T}(X')
            \end{tikzcd}
        $$
        commutes. Composition is given by the composition inherited by $\pazocal{C}$ and the identity morphisms are also inherited by $\pazocal{C}$. 
    \end{definition}
    \begin{proposition}
        For terminal coalgebras $(X,\gamma)$, we have that $\gamma: X\rightarrow \pazocal{T}(X)$ is an isomorphism. 
    \end{proposition}
    \begin{proof}
        There is a unique morphism $\delta: \pazocal{T}(X)\rightarrow X$...
    \end{proof}
    \begin{definition}
        A \emph{concrete category} is a category $\pazocal{C}$ for which there is a faithful functor $\pazocal{C} \rightarrow \mathrm{Set}$.
    \end{definition}
    \begin{proposition}
        Consider a morphism $f: X\rightarrow Y$ in a category $\pazocal{C}$. Consider a faithful functor $\pazocal{F} : \pazocal{C}\rightarrow \pazocal{D}$. If $\pazocal{F}(f)$ is mono so in $f$. Dually if $\pazocal{F}(f)$ is epi so is $f$.
    \end{proposition}
    \begin{proof}
        Indeed, suppose $fg = fh$ for a pair of parallel morphisms $g,h$, composable with $f$. Then 
        $$\pazocal{F}(f)\pazocal{F}(g)= \pazocal{F}(fg) =\pazocal{F}(fh)  = \pazocal{F}(f)\pazocal{F}(h) \implies \pazocal{F}(g) = \pazocal{F}(h) \implies g=h.$$ 
        To make the dual argument, note the functor $\pazocal{F}\op : \pazocal{C}\op \rightarrow \pazocal{D}\op$ which maps a morphism $f\in \Hom(\pazocal{C}\op)$ with domain $X$ and codomain $Y$ to $\pazocal{F}: \pazocal{F}(Y) \rightarrow \pazocal{F}(X)$, which is a morphism from $\pazocal{F}(X)$ to $\pazocal{F}(Y)$ in $\pazocal{D}\op$, is again a faithful functor. Since an epi is the dual concept to mono the result follows by applying the statement for monos to $\pazocal{F}\op$.   
    \end{proof}
    \begin{corollary}
        If $\pazocal{C}$ is a concrete category, if the image of a morphism $f\in \Hom(\pazocal{C})$ in $\mathrm{Set}$ is an injection/surjection, then $f$ is mono/epi. 
    \end{corollary}
    \begin{example}
        It may not always be that faithful functors preserve epimorphisms. Take any non-zero integer $a$. Then for any $x,y\in\Z$, if $xa = ya$, then $x=y$. That is in category $\mathrm{B}(\Z,\cdot)$, $a$ is an epimorphism. Consider the injective monoid homomorpism 
        $$
        \Z \hookrightarrow \mathrm{Mat}_2(\Z), n \mapsto \begin{pmatrix}
            n & 0 \\
            0 & 0
        \end{pmatrix}.
        $$
        This corresponds to a faithful functor 
        \begin{gather*}
            \mathrm{B}\Z \rightarrow \mathrm{B}\mathrm{Mat}_2(\Z)\\
            \Z \mapsto \mathrm{Mat}_2(\Z)\\
            n \mapsto \begin{pmatrix}
            n & 0 \\
            0 & 0
        \end{pmatrix}
        \end{gather*}
        Note that 
        $$
            \begin{pmatrix}
                1 & 0 \\
                0 & 0
            \end{pmatrix}
            \begin{pmatrix}
                a & 0\\
                0 & 0  
            \end{pmatrix}
            = 
            \begin{pmatrix}
                1 & 0\\
                0 & 1
            \end{pmatrix}
            \begin{pmatrix}
                a & 0\\
                0 & 0  
            \end{pmatrix}.
        $$
        So since 
        $$
            \begin{pmatrix}
                1 & 0 \\
                0 & 0
            \end{pmatrix} 
            \neq
            \begin{pmatrix}
                1 & 0\\
                0 & 1
            \end{pmatrix},
        $$
        we conclude that 
        $$
            \begin{pmatrix}
                a & 0\\
                0 & 0
            \end{pmatrix}
        $$
        is not an epimorphism in $\mathrm{B}\mathrm{Mat}_2(\Z)$. Dually, a faithful functor does not preserve monos, since these are just epis in $\Z\op$.
    \end{example}
    \begin{lemma}
        Consider a faithful functor $\pazocal{F}: \pazocal{C} \rightarrow \pazocal{D}$. If a diagram in $\pazocal{C}$ (a directed graph of morphisms) commutes in the image of $\pazocal{F}$, then it commutes in $\pazocal{C}$
    \end{lemma}
    \begin{proof}
        Indeed given sequences $f_1,\dots, f_n$ and $g_1,\dots,g_m$ of composable morphism in $\pazocal{C}$ with the same domain and codomain with $\pazocal{F}(f_n) \cdots \pazocal{F}(f_n) = \pazocal{F}(g_m)\cdots \pazocal{F}(g_1)$, we have that 
        $$\pazocal{F}(f_n\cdots f_1) = \pazocal{F}(g_m\cdots g_1) \implies f_n\cdots f_1 = g_m\cdots g_1.$$
    \end{proof}
\subsubsection{Natural Transformations as Morphisms}
\begin{proposition}
    Consider categories $\pazocal{C}$ and $\pazocal{D}$. $\Hom(\pazocal{C},\pazocal{D})$, i.e. functors between $\pazocal{C}$ and $\pazocal{D}$ define the objects in category where morphisms are natural transformations.
\end{proposition}
\begin{proof}
    Given natural transformations $\alpha: \pazocal{F} \Rightarrow \pazocal{G}$ and $\beta : \pazocal{G}\Rightarrow \pazocal{H}$, we define 
    $$\beta\alpha : \pazocal{F} \Rightarrow \pazocal{H}$$
    to be the collection of morphisms
    $$\left\{\beta_{X}\alpha_X : \pazocal{F}(X)\rightarrow \pazocal{H}(X) : X\in \Ob(\pazocal{C})\right\}.$$
    This is seen to be a natural transformation by  
    $$
        \begin{tikzcd}
            \pazocal{F}(X) \arrow[r, "\alpha_X"] \arrow[d,"\pazocal{F}(f)"] & \pazocal{G}(X)\arrow[d,"\pazocal{G}(f)"]\arrow[r,"\beta_X"] & \pazocal{H}(X)\arrow[d,"\pazocal{H}(f)"]\\
            \pazocal{F}(X') \arrow[r,"\alpha_{X'}"] & \pazocal{G}(X') \arrow[r,"\beta_{X'}"] & \pazocal{H}(X')
        \end{tikzcd}
    $$ 
    commuting for each morphism $f: X\rightarrow X'$. This composition is clearly associative. The identity morphism is 
    $$\{\id_{\pazocal{F}(X)}: \pazocal{F}(X)\rightarrow \pazocal{F}(X): X\in \Ob(\pazocal{C})\}.$$
\end{proof}
\begin{remark}
    We call the composition of natural transformations defined above \emph{vertical composition}.
\end{remark}
An important instance of the above category is $\Hom(\Hom(X,\_),\pazocal{F})$ where $X$ lives in a locally small category $\pazocal{C}$ and $\pazocal{F}$ is another functor from $\pazocal{C}$ to $\mathrm{Set}$. Its importance is due to the Yoneda lemma, which will tell us that this is a set that is isomorphic to the set $\pazocal{F}(X)$.  
\begin{definition}
    Given a pair of natural transformations
    $$
        \begin{tikzcd}
            \pazocal{C} \arrow[r,bend left = 50, "\pazocal{F}"{name=F}] \arrow[r, bend right = 50, "\pazocal{G}"{name=G},swap] & \pazocal{D}  \arrow[Rightarrow,from=F, to=G,"\alpha", shorten <= 10pt, shorten >= 10 pt]
        \end{tikzcd}
        \begin{tikzcd}
            \pazocal{D} \arrow[r,bend left = 50, "\pazocal{H}"{name=F}] \arrow[r, bend right = 50, "\pazocal{K}"{name=G},swap] & \pazocal{E}  \arrow[Rightarrow,from=F, to=G,"\beta", shorten <= 10pt, shorten >= 10 pt]
        \end{tikzcd}
    $$
    We define \emph{horizontal composition of $\beta$ and $\alpha$} denoted $\beta\ast\alpha$ to be the natural transformations whose components is the diagonal of the commutative square
    $$
        \begin{tikzcd}
            \pazocal{HF}(X) \arrow[rd, dashed, "(\beta\ast \alpha)_X"] \arrow[r,"\beta_{\pazocal{F}(X)}"] \arrow[d,"\pazocal{H}(\alpha_X)",swap] & \pazocal{KF}(X)\arrow[d,"\pazocal{K}(\alpha_X)"]\\
            \pazocal{HG}(X) \arrow[r,"\beta_{\pazocal{G}(X)}",swap] & \pazocal{KG}(X)
        \end{tikzcd}
    $$
\end{definition}
\begin{proposition}
    horizontal composition is well-defined.
\end{proposition}
\begin{proof}
    The square commutes by naturality of $\beta$. We need to check that $\beta\ast\alpha : \pazocal{HF} \Rightarrow \pazocal{KG}$ is indeed a natural transformation. Indeed, consider the rectangle 
    $$
        \begin{tikzcd}
            \pazocal{HF}(X) \arrow[d,"\pazocal{HF}(f)"]\arrow[r,"\pazocal{H}(\alpha_X)"] & \pazocal{HG}(X) \arrow[r,"\beta_{\pazocal{G}(X)}"] \arrow[d,"\pazocal{HG}(f)"] & \pazocal{KG}(X)\arrow[d,"\pazocal{KG}(f)"]\\
            \pazocal{HF}(X') \arrow[r,"\pazocal{H}(\alpha_{X'})"]& \pazocal{HG}(X') \arrow[r,"\beta_{\pazocal{G}(X')}"] & \pazocal{KG}(X')
        \end{tikzcd}
    $$ 
    commutes by the naturality of $\alpha$. This shows that 
    \begin{align*} 
        \pazocal{KG}(f)(\beta\ast\alpha)_X&=\pazocal{KG}(f)\beta_{\pazocal{G}(X)}(\beta_{\pazocal{G}(X)}\pazocal{H}(\alpha_X))\\
         &= (\beta_{\pazocal{G}(X')}\pazocal{H}(\alpha_{X'}))\pazocal{HF}(f)= (\beta\ast\alpha)_{X'}\pazocal{HF}(f)
    \end{align*}
    hence $\beta\ast \alpha$ is a natural tranformation. Associativity follows by the trivial componentwise associativity
\end{proof}
\begin{lemma}
    Consider the functors and natural transformations
    $$
        \begin{tikzcd}
            \pazocal{C} \arrow[r, bend left=70, "\pazocal{F}"{name=F}] \arrow[r, "\pazocal{G}"{name=G, description}]
            \arrow[r, bend right=70, "\pazocal{H}"{name=H,swap}] & \pazocal{D} \arrow[Rightarrow, from=F, to=G, "\alpha"]\arrow[Rightarrow,from=G, to=H, "\beta"] \arrow[r, bend left=70, "\pazocal{J}"{name=J}] \arrow[r, "\pazocal{K}"{name=K, description}]
            \arrow[r, bend right=70, "\pazocal{L}"{name=L,swap}] & \pazocal{E} \arrow[Rightarrow, from=J, to=K,"\gamma"]\arrow[Rightarrow, from=K, to=L, "\delta"]
        \end{tikzcd}
    $$
    Then 
    $$
        \delta\gamma \ast \beta\alpha = (\delta\ast \beta)(\gamma\ast \alpha).
    $$
\end{lemma}
\begin{proof}
    By the naturality of $\delta$ we get that the square 
    $$
        \begin{tikzcd}
            \pazocal{K}(\pazocal{F}(X)) \arrow[r, "\delta_{\pazocal{F}(X)}"] \arrow[d,"\pazocal{K}(\alpha_X)"] & \pazocal{L}(\pazocal{F}(X)) \arrow[d, "\pazocal{L}(\alpha_X)"]\\
            \pazocal{K}(\pazocal{G}(X)) \arrow[r,"\delta_{\pazocal{G}(X)}"] & \pazocal{L}(\pazocal{G}(X))
        \end{tikzcd}
    $$
    commutes. It thus follows that 
    \begin{align*}
        (\delta\gamma \ast \beta\alpha)_X &= \pazocal{L}(\beta_X\alpha_X)(\delta\gamma)_{\pazocal{F}(X)} = \pazocal{L}(\beta_X)\pazocal{L}(\alpha_X)\delta_{\pazocal{F}(X)}\gamma_{\pazocal{F}(X)}\\
        &= \pazocal{L}(\beta_X)\delta_{\pazocal{G}(X)} \pazocal{K}(\alpha_X)\gamma_{\pazocal{F}(X)}= (\delta \ast \beta)_X (\gamma\ast\alpha)_X\\
        &=((\delta\ast\beta)(\gamma\ast \alpha))_X.
    \end{align*}
\end{proof}
\begin{proposition}
    The objects of $\mathrm{Cat}$ form a category with morphisms being \textbf{cells} of the form 
    $$
        \begin{tikzcd}
            \pazocal{C} \arrow[r, bend right=50,""{name=1}]\arrow[r,bend left=50,""{name=2}] & \pazocal{D} \arrow[Rightarrow, from=2, to=1, shorten <= 6, shorten >= 6] 
        \end{tikzcd}
    $$ 
    where a composition of cells is horinzontal composition of the natural transformations. 
\end{proposition}
\begin{proof}
    This follows directly from the prior lemma, i.e. after one checks that horizontal composition is associative, which is readily verifiable. The identity cell is a pair of identity functors together with the identity transformation of these functors. 
\end{proof}
\begin{definition}
    A \emph{2-category} $\pazocal{C}$ is a class of objects, a class of 1-morphisms between objects, together with a class of 2-morphisms between parallel morphisms, where for $f,g : X\rightarrow Y$, we denote such an arrow $\alpha$ by $\alpha : f\Rightarrow g$. Moreover, this data has to adhere to the following restrictions: 
    \begin{enumerate}
        \item objects and 1-morphisms form a category
        \item There is an operation of 2-morphisms between 1-morphisms of any fixed pair of objects $X,Y$ called \emph{vertical composition} such that 1-morphisms between $X$ and $Y$ are objects in a category with $2$-morphisms form a category. 
        \item There is a category whose objects are objects in $\pazocal{C}$ where morphisms are \emph{2-cells} of the form 
        $$
            \begin{tikzcd}
                X \arrow[r, bend right=50,""{name=a}] \arrow[r,bend left=50,""{name=b}] & Y \arrow[Rightarrow,from=b, to=a,shorten <= 6, shorten >= 6]
            \end{tikzcd}
        $$
        Morphisms are 2-morphisms and these compose by a binary operation called \emph{horizontal composition}. Given a pair of $2$-morphisms
        $$
            \begin{tikzcd}
                X
                \arrow[r, bend left=50]
                \arrow[r, bend right=50]
                \arrow[phantom, from=1-1, to=1-2, "\Downarrow\,\alpha" description,
                shorten >=4pt, shorten <=4pt]
                & Y
                \arrow[r, bend left=50]
                \arrow[r, bend right=50]
                \arrow[phantom, from=1-2, to=1-3, "\Downarrow\,\beta" description,
                shorten >=4pt, shorten <=4pt]
                & Z
            \end{tikzcd}
        $$
        we denote this composition by $\beta\ast\alpha$.
        \item For 1-morphisms $f : X\rightarrow Y$ and $g: Y\rightarrow Z$, 
        $$
            \fone_g\ast\fone_f = \fone_{gf}.
        $$
        \item 
        For 2-cells,
        $$
        \begin{tikzcd}
            X \arrow[r, bend left=70, "f"{name=F}] \arrow[r, "g"{name=G, description}]
            \arrow[r, bend right=70, "h"{name=H,swap}] & Y \arrow[Rightarrow, from=F, to=G, "\alpha"]\arrow[Rightarrow,from=G, to=H, "\beta"] \arrow[r, bend left=70, "j"{name=J}] \arrow[r, "k"{name=K, description}]
            \arrow[r, bend right=70, "l"{name=L,swap}] & Z \arrow[Rightarrow, from=J, to=K,"\gamma"]\arrow[Rightarrow, from=K, to=L, "\delta"]
        \end{tikzcd}
    $$
    Then 
    $$
        \delta\gamma \ast \beta\alpha = (\delta\ast \beta)(\gamma\ast \alpha).
    $$
   \end{enumerate} 
\end{definition}
Readily, we see that we have seen one example of a 2-category, namely the 2-category of small categories. 
\subsection{The Yoneda Lemma}
\subsubsection{Representable Functors}
We give an alternative definition of initial and terminal objects.
\begin{lemma}
    \begin{enumerate}
        \item An object $I$ in a locally small category $\pazocal{C}$ is initial if and only if $\Hom(I,\_)\cong \ast$, where $\ast : \pazocal{C}\rightarrow \mathrm{Set}, S\mapsto \{\ast\}$. 
        \item An object $T$ in a locally small category $\pazocal{C}$ is terminal if and only if $\Hom(\_,T)\cong \ast\op$, where $\ast\op : \pazocal{C}\op\rightarrow \mathrm{Set}, S\mapsto \{\ast\}$. 
    \end{enumerate}
\end{lemma}
\begin{proof}
    1. Suppose $I$ is initial. For each object $X$ in $\pazocal{C}$, $\Hom(I,X)$ is a singleton $\{f_X\}$. Then 
    $$
        \{ \ast:\Hom(I,X) \rightarrow \ast(X), f_X\mapsto to \{\ast\} : X\in \Ob(\pazocal{C})\}
    $$
    defines a natural isomorphism $\Hom(I,\_)\cong \ast$. Indeed, each of these maps is a bijection with inverse, $\{\ast\}\mapsto f_X$. Note that any square
    $$
        \begin{tikzcd}
            \Hom(I,X) \arrow[r,"\ast"] \arrow[d] & \{\ast\} \arrow[d] \\
            \Hom(I,Y) \arrow[r,"\ast"] &\{\ast\} 
        \end{tikzcd}
    $$
    must have vertical maps that are constant maps, so naturality follows.\\
    Suppose now that $\Hom(I,\_)\cong \ast$. Then there is a collection of bijective maps 
    $$
        \{\Hom(I,X) \simeq \{\ast\} : X\in \Ob(\pazocal{C})\}.
    $$
    In particular, $\vert \Hom(I,X) \vert  =\vert \{\ast\}\vert = 1$, for each object $X$ in $\pazocal{C}$. It follows that $\Hom(I,X)$ is a singleton for each object and thus that $I$ is the initial object in $\pazocal{C}$.\\
    2. is to dual to 1.
\end{proof}
\begin{definition}
    A co-/contravariant functor $\pazocal{F}$ whose domain is locally small is \emph{representable} if there is an object $X$ such that $\Hom(X,\_)$ resp. $\Hom(\_,X)$ is naturally isomorphic to $\pazocal{F}$. We then say that \emph{$\pazocal{F}$ is represented by $X$}
\end{definition}
\begin{remark}
    $\pazocal{C}$ has an initial resp. termninal object if and only if $\ast : \pazocal{C}\rightarrow \mathrm{Set}$ resp. $\ast\op: \pazocal{C}\op \rightarrow \mathrm{Set}$.  
\end{remark}
\begin{definition}
    A \emph{representation} of a functor $\pazocal{F}$ whose domain is locally small is a pair of an object $X$ and a natural isomorphism of $\pazocal{F}$ with $Hom(X,\_)$ and $\Hom(\_,X)$ depending on variance. 
\end{definition}
The motivation for restating the notion of initial and terminal object is that these are special cases of objects with universal properties. We shall define what this means exactly.
\begin{definition}
    A \emph{universal property} for an object $X$ in a locally small category $\pazocal{C}$ is a co/contravariant representable functor $\pazocal{C}\rightarrow \mathrm{Set}$ represented by $X$. 
\end{definition}
\begin{example}
    Consider the functor
    \begin{gather*}
        \Ob: \mathrm{Cat} \rightarrow \mathrm{Set}\\
        \pazocal{C} \mapsto \Ob(\pazocal{C})\\
        (\pazocal{F} :\pazocal{C}\rightarrow \pazocal{D})\mapsto (\Ob(\pazocal{C})\rightarrow \Ob(\pazocal{D}), X\mapsto \pazocal{F}(X))
    \end{gather*}
    Consider also the functor $\Hom(\fone,\_)$. For any category $\pazocal{C}$, a functor $\fone\rightarrow \pazocal{C}$ is of the form 
    \begin{gather*}
        \fone \rightarrow \pazocal{C}\\
        \ast \mapsto X\\
        \fone_\ast \mapsto \fone_X
    \end{gather*}
    for some object $X$ in $\pazocal{C}$.  The functor $\mathrm{Ob}$ is represented by $\fone_X$. Indeed, 
    $$
        \{f_\pazocal{X}:\Hom(\fone,\pazocal{X})\rightarrow \Ob(\pazocal{X}), \pazocal{F}_X \mapsto X \mid \pazocal{X}\in \mathrm{Cat}\}
    $$
    is a family of isomorphisms, where $X\mapsto \pazocal{F}_X$ is the inverse. One should note that this is a collection of well defined maps, since if $\pazocal{F}_X=\pazocal{F}_Y$, then $X=\pazocal{F}_X(\ast)=\pazocal{F}_Y(\ast)=Y$. We check that for an arbitrary functor $\pazocal{G}: \pazocal{X}\rightarrow \pazocal{Y}$, the square 
    $$
        \begin{tikzcd}
            \Hom(\fone,\pazocal{X}) \arrow[r,"f_\pazocal{X}"]\arrow[d, "\pazocal{G}_\ast"] & \Ob(\pazocal{X})\arrow[d,"\Ob(\pazocal{G})"]\\
            \Hom(\fone,\pazocal{Y}) \arrow[r,"f_\pazocal{Y}"] & \Ob(\pazocal{Y})
        \end{tikzcd}
    $$
    commutes. Indeed, given a functor $\pazocal{F}_X : \fone \rightarrow \pazocal{X}$ for some object $X\in \pazocal{X}$,
    $$\Ob(\pazocal{G})f_\pazocal{X}(\pazocal{F}_X)= \Ob(\pazocal{G})(X)= \pazocal{G}(X) = f_\pazocal{Y}(\pazocal{H}_{\pazocal{G}(X)})= f_\pazocal{Y}(\pazocal{G}\pazocal{F}_X)= f_\pazocal{Y}\pazocal{G}_\ast(\pazocal{F}_X).$$
\end{example}
\begin{example}
    
\end{example}
\subsubsection{Statement \& Proof of The Yoneda Lemma}
\begin{theorem}(The Yoneda Lemma) For any $\pazocal{C}$ in $\mathrm{CAT}$, given a functor $\pazocal{F}:\pazocal{C}\rightarrow \mathrm{Set}$ and an object $X\in \Ob(\pazocal{C})$
$$
    \Hom^{\pazocal{C}^\mathrm{Set}}(\Hom(X,\_),\pazocal{F}) \simeq \pazocal{F}(X).
$$
This bijection maps a natural transformation $\alpha: \Hom(X,\_) \Rightarrow \pazocal{F}$ to $\alpha_X(\fone_X)$. The assignment of $X$ resp. $\pazocal{F}$ is natural. I.e.
\begin{enumerate}
    \item \textbf{Naturality in Functors:} Fix an object $X$. For each functor $\pazocal{F}$ we write $\Phi^{\pazocal{F}} : \Hom^{\pazocal{C}^{\mathrm{Set}}}(\Hom(X,\_),\pazocal{F}) \rightarrow \pazocal{F}(X)$ for the bijection. The claim is that 
    $$\Phi^{\bullet}  : \Hom^{\pazocal{C}^\mathrm{Set}}(\Hom(X,\_),\bullet) \Rightarrow \bullet(X)$$
    given by the collection of morphisms
    $$\left\{\Phi^\pazocal{F} : \pazocal{F} \in \pazocal{C}^\mathrm{Set}\right\}$$
    is natural. Here $\bullet(X)$ denotes the functor 
    \begin{gather*}
        \pazocal{C}^\mathrm{Set} \rightarrow \mathrm{Set}\\
        \pazocal{F} \mapsto \pazocal{F}(X)\\
        \beta : \pazocal{F}\Rightarrow \pazocal{G} \mapsto \beta_X
    \end{gather*}
    \item \textbf{Naturality in Objects:} Fix a functor $\pazocal{F}\in \pazocal{C}^\mathrm{Set}$. For each object $X$ we write $\Phi_X : \Hom^{\pazocal{C}^\mathrm{Set}}(\Hom(X,\_),\pazocal{F}) \rightarrow \pazocal{F}(X)$ for the bijection. The claim is that 
    $$\Phi_\bullet : \Hom^{\pazocal{C}^\mathrm{Set}}(\Hom(\bullet,\_),\pazocal{F})\Rightarrow \pazocal{F}(\bullet)$$
    given by the collection of morphisms
    $$\left\{ \Phi_X : X\in \Ob(\pazocal{C})\right\}$$
    is natural. Here $\Hom^{\pazocal{C}^\mathrm{Set}}(\Hom(\bullet,\_),\pazocal{F})$ is the functor 
    \begin{gather*}
        \pazocal{C}\rightarrow \mathrm{Set}\\
        X \mapsto \Hom^{\pazocal{C}^\mathrm{Set}}(\Hom(X,\_),\pazocal{F})\\
        f : X\rightarrow Y \mapsto \left((f^\ast)^\ast: \Hom^{\pazocal{C}^\mathrm{Set}}(\Hom(X,\_),\pazocal{F}) \Rightarrow \Hom^{\pazocal{C}^\mathrm{Set}}(\Hom(Y,\_),\pazocal{F}),\alpha \mapsto \alpha f^\ast\right)
    \end{gather*}

\end{enumerate}
\end{theorem}
\begin{proof}
    Note that the functor category $\pazocal{C}^\mathrm{Set}$ is locally small, hence $\Hom^{\pazocal{C}^\mathrm{Set}}(\Hom(X,\_),\pazocal{F})$ is a set.\\
    \textbf{Explicit construction of bijection:}
    We define
    \begin{gather*}
        \Phi : \Hom^{\pazocal{C}^\mathrm{Set}}(\Hom(X,\_),\pazocal{F}) \rightarrow \pazocal{F}(X)\\
        (\alpha : \Hom(X,\_) \Rightarrow \pazocal{F})  \mapsto \alpha_X(\fone_X)
    \end{gather*}
    We aim to construct an inverse to this function. Let $y\in \pazocal{F}(X)$. We wish to find a natural transformation $\Psi(y) : \Hom(X,\_)\Rightarrow \pazocal{F}$. For this to be a natural transformation we must in particular have that for any object $Y$ and any morphism $f: X\rightarrow Y$, it must be that the square 
    $$
        \begin{tikzcd}
            \Hom(X,X) \arrow[r, "\Psi(y)_X"] \arrow[d,"f_\ast"] & \pazocal{F}(X) \arrow[d,"\pazocal{F}(f)"]\\
            \Hom(X,Y) \arrow[r,"\Psi(y)_Y"] & \pazocal{F}(Y)
        \end{tikzcd}    
    $$
    commutes. Note that in particular
    $$
        \pazocal{F}(f)(\Psi(y)_X(\fone_X))= \Psi(y)_Yf_\ast(\fone_X)= \Psi(y)_Y(f)
    $$
    We want 
    $$X= \Phi(\Psi(y)) = \Psi(y)_X(\fone_X).$$
    For this to hold we must then have that 
    $$\pazocal{F}(f)(X) = \Psi(y)_Y(f).$$
    That is to say, we must define the inverse of $\Phi$, to be 
    \begin{gather*}
        \Psi:\pazocal{F}(X)\rightarrow \Hom^{\pazocal{C}^\mathrm{Set}}(\Hom(X,\_),\pazocal{F})\\
        y \mapsto \Psi(y):= \{\left.\Psi(y)_Y : \Hom(X,Y)\rightarrow \pazocal{F}(Y), f\mapsto \pazocal{F}(f)(y)\right| Y\in \Ob(\pazocal{C}) \}
    \end{gather*}
    We check that this is indeed a natural transformation. Let a morphism $g: Z\rightarrow W$ be given. Then for any $f\in \Hom(X,Z)$,
    \begin{align*}
        \Psi(y)_Wg_\ast(f) = \Psi(y)_W(gf) =\pazocal{F}(gf) \pazocal{F}(g)\pazocal{F}(f)(y) = \pazocal{F}(g)\Psi(y)_Z(f).
    \end{align*}
    hence the square
    $$
        \begin{tikzcd}
            \Hom(X,Z) \arrow[r,"\Psi(y)_Z"]\arrow[d,"g_\ast"] & \pazocal{F}(Z)\arrow[d,"\pazocal{F}(g)"] \\
            \Hom(X,W) \arrow[r,"\Psi(y)_W"] & \pazocal{F}(W)
        \end{tikzcd}
    $$
    commutes. Note that we constructed $\Psi$ to be a right inverse of $\Phi$, so it remains to check that it is also a left inverse. Let a natural transformation $\alpha: \Hom(X,\_)\Rightarrow \pazocal{F}$ be given. Then for every $f\in \Hom(X,Y)$
    $$
        \begin{tikzcd} 
            \Hom(X,X) \arrow[r,"\alpha_X"] \arrow[d,"f_\ast"] & \pazocal{F}(X) \arrow[d,"\pazocal{F}(f)"]\\
            \Hom(X,Y) \arrow[r, "\alpha_Y"] & \pazocal{F}(Y)
        \end{tikzcd}
    $$ 
    commutes, hence 
    $$
        \Psi\Phi(\alpha)(f)= \Psi(\alpha_X(\fone_X))(f) = \pazocal{F}(f)(\alpha_X(\fone_X)) = \pazocal{F}(f)\alpha_X(\fone_X)=\alpha_Y f_\ast(\fone_X)=\alpha_Y(f).
    $$
    We can thus conclude that $\Psi\Phi(\alpha)=\alpha$.\\
    \textbf{Naturality on functors:} Let a natural transformation $\beta : \pazocal{F}\Rightarrow \pazocal{G}$ be given. We need to check that the square
    $$
        \begin{tikzcd}
            \Hom^{\pazocal{C}^\mathrm{Set}}(\Hom(X,\_),\pazocal{F}) \arrow[r,"\Phi^\pazocal{F}"]\arrow[d,"\beta_\ast"] & \pazocal{F}(X)\arrow[d,"\beta_X"]\\
            \Hom^{\pazocal{C}^\mathrm{Set}}(\Hom(X,\_),\pazocal{G}) \arrow[r,"\Phi^\pazocal{G}"] & \pazocal{G}(X)
        \end{tikzcd}
    $$
    Indeed for any natural transformation $\alpha : \Hom(X,\_)\Rightarrow \pazocal{F}$,
    \begin{align*}
        \Phi^\pazocal{G}\beta_\ast(\alpha) = \Phi^\pazocal{G}(\beta\alpha) = (\beta\alpha)_X(\fone_X)= \beta_X(\alpha_X(\fone_X))= \beta_X\Phi^\pazocal{F}(\alpha).
    \end{align*}
    \textbf{Naturality on Objects:} Let a morphism be given $f:X\rightarrow Y$. We need to check that 
    $$
        \begin{tikzcd}
            \Hom^{\pazocal{C}^\mathrm{Set}}(\Hom(X,\_),\pazocal{F}) \arrow[r,"\Phi_X"]\arrow[d,"(f^\ast)^\ast"] & \pazocal{F}(X)\arrow[d,"\pazocal{F}(f)"]\\
            \Hom^{\pazocal{C}^\mathrm{Set}}(\Hom(Y,\_),\pazocal{F}) \arrow[r,"\Phi_Y"] & \pazocal{F}(Y)
        \end{tikzcd}
    $$
    commutes. Indeed, for a natural transformation $\alpha: \Hom(X,\_)\Rightarrow \pazocal{F}$, 
    \begin{align*}
        \Phi_Y(f^\ast)^\ast(\alpha)= \Phi_Y(\alpha f^\ast) = \alpha_Y (f^\ast)_Y(\fone_Y) = \alpha_Y(f) = \pazocal{F}(f)(\alpha_X(\fone_X))= \pazocal{F}(f)\Phi_X(\alpha).
    \end{align*}
\end{proof}
\subsubsection{Corollaries of The Yoneda Lemma}
We first give an alternative characterization of the Yoneda lemma as a natural isomorphism of a pair of functors. 
\begin{definition} 
    The first of these functors is the \emph{evaluation functor}: Given a locally small category $\pazocal{C}$, we consider 
    \begin{gather*}
        \ev: \pazocal{C}\times \mathrm{Set}^\pazocal{C}\rightarrow \mathrm{Set}\\
        (X,\pazocal{F})\mapsto \pazocal{F}(X)\\
        (f,\alpha): (X,\pazocal{F})\rightarrow (Y,\pazocal{G})\mapsto \alpha_X\pazocal{F}(f) : \pazocal{F}(X)\rightarrow \pazocal{G}(Y)
    \end{gather*}
\end{definition} 
Before constructing the second functor to consider we prove a lemma to provide a different characterization of bifunctors
\begin{lemma}\label{BifunctorAsFunctorToFunctorCategory}
    Consider locally small categories $\pazocal{C},\pazocal{D},\pazocal{E}$. There is a one-to-one correspondence of bifunctors $\pazocal{C}\times \pazocal{D}\rightarrow \pazocal{E}$ onto functors $\pazocal{C}\rightarrow\pazocal{E}^\pazocal{D}$.
\end{lemma}
\begin{proof}
    Let a bifunctor $\pazocal{F} : \pazocal{C}\times \pazocal{D}\rightarrow \pazocal{E}$ be given. We define a functor 
    \begin{gather*}
        \mathrm{Cur}: \pazocal{C}\rightarrow \pazocal{E}^\pazocal{D}\\
        X \mapsto \pazocal{F}(X,\_)\\
        f : X\rightarrow Y  \mapsto \pazocal{F}(f,\_): \pazocal{F}(X,\_)\Rightarrow  \pazocal{F}(Y,\_)
    \end{gather*}
    where 
    \begin{gather*}
        \pazocal{F}(X,\_): \pazocal{D}\rightarrow \pazocal{E}\\
        S \mapsto \pazocal{F}(X,S)\\
        g : S\rightarrow T \mapsto \pazocal{F}(\fone_X,g) 
    \end{gather*}
    and $\pazocal{F}(f,\_)$ is the collection
    $$\{\pazocal{F}(f,S):= \pazocal{F}(f,\fone_S): \pazocal{F}(X,S)\rightarrow \pazocal{F}(Y,S)\}.$$
    The square
    $$
        \begin{tikzcd}
            \pazocal{F}(X,S) \arrow[r,"\pazocal{F}(f{,}S)"]\arrow[d,"\pazocal{F}(X{,}g)"] & \pazocal{F}(Y,S) \arrow[d, "\pazocal{F}(Y{,}g)"]\\
            \pazocal{F}(X,T) \arrow[r,"\pazocal{F}(f{,}T)"] & \pazocal{F}(Y,T)
        \end{tikzcd}
    $$
    commutes since 
    \begin{align*}
        \pazocal{F}(Y,g)\pazocal{F}(f,S) &= \pazocal{F}(\fone_Y, g)\pazocal{F}(f,\fone_S) = \pazocal{F}(\fone_Yf,g\fone_S)= \pazocal{F}(f\fone_X,\fone_T g)\\
        &= \pazocal{F}((f,\fone_T)(\fone_X,g)) = \pazocal{F}(f,\fone_T)\pazocal{F}(\fone_X,g)= \pazocal{F}(f,T)\pazocal{F}(X,g).
    \end{align*}
    So $\pazocal{F}(f,\_)$ is a natural transformation. Given a functor $\pazocal{G} : \pazocal{C}\rightarrow \pazocal{E}^\pazocal{D}$, the functor 
    \begin{gather*}
        \pazocal{C}\times \pazocal{D}\rightarrow \pazocal{E}\\
        (X,S)\mapsto \pazocal{G}(X)(S)\\
        (f,g): (X,S)\rightarrow (Y,T)\mapsto (\pazocal{G}(f))_T(\pazocal{G}(X)(g))
    \end{gather*}
    is the bifunctor associated with $\pazocal{G}$. Given two bifunctors $\pazocal{F}$ and $\pazocal{F}'$ with the same image under this association, note that $$\pazocal{F}(X,S) = \pazocal{F}(X,\_)(S) =\pazocal{F}'(X,\_)(S)=\pazocal{F}'(X,S).$$
    Moreover,
    $$\pazocal{F}(f,g) = \pazocal{F}(\fone_Y,g)\pazocal{F}(f,\fone_S) = \pazocal{F}(Y,g)\pazocal{F}(f,S) = \pazocal{F}'(Y,g)\pazocal{F}'(f,S) = \pazocal{F}'(f,g),$$ 
    hence $\pazocal{F}=\pazocal{F}'$.
\end{proof}
Consider the bifunctor 
$$
    \Hom(\_,\_) : \pazocal{C}\op \times \pazocal{C} \rightarrow \mathrm{Set}.
$$
We can then apply the prior lemma to obtain a contravariant functor 
\begin{gather*}
    y: \pazocal{C}\op \rightarrow \mathrm{Set}^\pazocal{C}\\
    X\mapsto \Hom(X,\_)\\
    f : X\rightarrow Y \mapsto f^\ast : \Hom(Y,\_) \rightarrow \Hom(X,\_)
\end{gather*}
We may regard this as a  functor $\pazocal{C} \rightarrow (\mathrm{Set}^\pazocal{C})\op$ to obtain a functor 
\begin{gather}
    \Hom(y(\_),\_) : \pazocal{C}\times \mathrm{Set}^\pazocal{C}\rightarrow (\mathrm{Set^\pazocal{C}})\op\times \mathrm{Set}^\pazocal{C}\rightarrow \mathrm{SET}\\
    (X,\pazocal{F})\mapsto (\Hom(X,\_),\pazocal{F}) \mapsto \Hom(\Hom(X,\_),\pazocal{F})
\end{gather}
where $\mathrm{SET}$ is the category of sufficient notion of collection of objects to contain the collection of all natural transformations between of objects.\\
By the Yoneda lemma we are in a position where we know that we can replace $\mathrm{SET}$ by $\mathrm{Set}$. We thus more succinctly express the Yoneda Lemma as the existence of a natural isomorphism 
$$
    \begin{tikzcd}
        \pazocal{C}\times \mathrm{Set}^\pazocal{C} \arrow[r,bend left=50,"\Hom(y(\_){,}\_)"{name=1, above}]\arrow[r,bend right=50,"\ev"{name=2,below}] & \mathrm{Set} \arrow[Rightarrow, from=1,to=2,shorten <= 8, shorten >= 8, "\cong\Phi"]
    \end{tikzcd}
$$ 

\begin{corollary}\label{YonedaEmbedding}
    For a locally small category $\pazocal{C}$ consider the functor
    \begin{gather*}
        y: \pazocal{C} \hookrightarrow \mathrm{Set}^{\pazocal{C}\op}\\
        X \mapsto \Hom(\_,X)\\
        f : X\rightarrow Y \mapsto f_\ast : \Hom(\_,X)\Rightarrow \Hom(\_,Y)
    \end{gather*}
    which we by Lemma~\ref{BifunctorAsFunctorToFunctorCategory} may be identified with the functor 
    \begin{gather*}
        y: \pazocal{C}\op \hookrightarrow \mathrm{Set}^{\pazocal{C}}\\
        X \mapsto \Hom(X,\_)\\
        f : X\rightarrow Y \mapsto f^\ast : \Hom(Y,\_)\Rightarrow \Hom(X,\_)
    \end{gather*}
    and $\Hom(\_,\_)$. This (or these) functor(s) is a fully faithful embedding called the \textbf{Yoneda embedding}
\end{corollary}
\begin{proof}
    It is clearly an embedding since if $X\neg Y$, then $\fone_X\in \Hom(X,X)$ and $\fone_X\notin \Hom(X,Y)$, hence $\Hom(\_,X)\neq \Hom(\_,Y)$. To prove fullness and faithfullness, it is equivalent to show that the mapping 
    $$
        \Hom(X,Y)\ni f \mapsto f_\ast\in \Hom(\Hom(\_,X),\Hom(\_,Y))
    $$
    is a bijection. By Lemma~\ref{PreCompositionIsANaturalTransformationAndThisAssignmentIsInjective}, it is injective. Let a natural transformation $\alpha\in \Hom(\Hom(\_,X),\Hom(\_,Y))$ be given. For an object $A$, set $f:= \alpha_X(\fone_X) : X \rightarrow Y$. Then $f_\ast$, takes a $\fone_X$ to 
    $$f\fone_X = f = \alpha_X(\fone_X),$$
    i.e. $\Phi(f_\ast)=\alpha_X(\fone_X)=\Phi(\alpha)$, hence by the Yoneda Lemma, 
    $$f_\ast = \alpha.$$

\end{proof}
\begin{corollary}
    Given a ring $R$ and an $n\times m$ matrix $A$, a row operation on $A$ is uniquely determined by left multiplication by the matrix resulting from the same row operation performed on $I_n$. 
\end{corollary}
\begin{proof}
    Fix a positive integer $n$. For each $i,j\in\{ 1,\dots,n\}$ and $a,b\in R$ consider the family of maps 
    \begin{gather*}
        {_m}\alpha_{ij}^{a,b}: \mathrm{Mat}_{n\times m}(R) \simeq (R^m)^n \rightarrow \mathrm{Mat}_{n\times m}(R)\simeq (R^m)^n\\
        (r_1,\dots,r_n) \mapsto (r_1,\dots,r_{i-1}, ar_i+b r_j,\dots,r_n) = (r_1,\dots,ar_i,\dots,r_n) + (b\delta_{ip}r_{pq}) 
    \end{gather*}
    over positive integers $m$. Clearly every row operation is a special case of such maps. We claim that ${_\bullet}\alpha_{ij}^{a,b}$ defines a natural endomorphism of $\Hom(\_,n)$. Indeed, given a matrix $A\in \mathrm{Mat}_{kl}=\Hom(l,k)$, the square 
    $$
        \begin{tikzcd}
            \mathrm{Mat}_{nk} \arrow[r,"{_k}\alpha_{ij}^{a{,}b}"]\arrow[d,"A^\ast"] & \mathrm{Mat}_{nk}\arrow[d,"A^\ast"]\\
            \mathrm{Mat}_{nl} \arrow[r, "{_l}\alpha^{a{,}b}_{ij}"] & \mathrm{Mat}_{nl}
        \end{tikzcd}
    $$
    commutes. To see this let $B\in \mathrm{Mat}_{nk}$ be given. We shorten notation and let ${_\bullet}\alpha$ denote horinzontal maps and $r(C)_i$ denote the $i$'th row of a matrix $C$. We thus get that 
    \begin{align*}
        {_l}\alpha A^\ast(B)&={_l}\alpha(BA)= (r(BA)_1,\dots, ar(BA)_i+br(BA)_j,\dots, r(BA)_n)\\
        &= (r(B)_1A,\dots, ar(B)_iA+br(B)_jA,\dots, r(B)_nA)\\
        &= (r(B)_1A,\dots, (ar(B)_i+br(B)_j)A,\dots, r(B)_nA)\\
        &= {_k}\alpha(B)A = A^\ast {_k}\alpha(B).    
    \end{align*}
    This proves naturality. To spell it out, we now have a natural transformation 
    $$
        \alpha_{ij}^{a,b}: \Hom(\_,n) \Rightarrow \Hom(\_,n)
    $$ 
    This is given by the Yoneda embedding applied to some $n\times n$ matrix. The proof of fullness of the Yoneda embedding or simply applying the Yoneda Lemma tells that this $n\times n$ matrix is ${_n}\alpha_{ij}^{a,b}(I_n)$. Thus in the special case where the map in question is a row operation, $\rho$ say, it is via the Yoneda embedding uniquely determined by $\rho(I_n)$.   
\end{proof}
\subsection{Universal Properties \& Universal Elements}
\begin{definition}
    two objects $X$ and $Y$ in the same small category are \emph{representably isomorphic} if $\Hom(X,\_)\cong \Hom(Y,\_)$ and $\Hom(\_,X)\cong\Hom(\_,Y)$.
\end{definition}
\begin{lemma}
    If $X$ and $Y$ in the same locally small category are isomorphic, then they are representably isomorphic.  
\end{lemma}
\begin{proof}
    This is a consequence functoriality of the Yoneda embeddings. 
\end{proof}
\begin{lemma}
    Let $X$ and $Y$ be a pair of objects in a locally small category $\pazocal{C}$. If $\Hom(\_,X)\cong \Hom(\_,Y)$ or $\Hom(X,\_)\cong \Hom(Y,\_)$, then $X\simeq Y$. As a consequence, if $\pazocal{F} : \pazocal{C}\rightarrow \mathrm{Set}$ is a functor represented by $X$ and $Y$, then $X\simeq Y$.
\end{lemma}
\begin{proof}
    Suppose $\alpha: \Hom(\_,X)\cong \Hom(\_,Y)$ or $\alpha: \Hom(Y,\_)\cong \Hom(X,\_)$, then since the Yoneda embedding is fully faithful, then $\alpha = y(f)$ for some unique isomorphism $f: X\rightarrow Y$.  
\end{proof}
\begin{remark}
    With this result in mind, we have motivated the use of the definite article when refering to an object representing a representable functor, since given a pair of representations there is a unique isomorphism between the objects. In this way there is a canonical way of identifying objects in this context.   
\end{remark}
\begin{definition}
    A \emph{universal property} of an object $X\in \pazocal{C}$, where $\pazocal{C}$ is locally small, is a pair consisting of a representable functor $\pazocal{F}$ together with a \emph{universal element} $x\in \pazocal{F}(X)$ that defines a natural isomorphism of $\Hom(X,\_)$ and $\pazocal{F}$ or $\Hom(\_,X)$ and $\pazocal{F}$ via $\Phi$ (as seen in the Yoneda Lemma). 
\end{definition}
\subsubsection{The Category of Elements}
\begin{definition}
    Let $\pazocal{F}: \pazocal{C}\rightarrow\mathrm{Set}$, where $\pazocal{C}$ is a locally small category, be a covariant functor. Then $\int \pazocal{F}$ is the category characterised by the following data: 
    \begin{enumerate}
        \item Objects are pairs $(A,X)$ where $A\in \Ob(\pazocal{C})$ and $X\in \pazocal{F}(A)$.
        \item A morphism $(A,X)\rightarrow (B,Y)$ is a morphism $f: A\rightarrow B$ in $\pazocal{C}$ so that $(\pazocal{F}(f))(X)=Y$.
    \end{enumerate} 
    This category is called \emph{the category of elements}.
\end{definition}
\begin{remark}
    Note that $(\pazocal{F}(\fone_A))(X)=\id_{\pazocal{F}(X)}(X)=X$, so $\fone_A\in \Hom(\int\pazocal{F})$. Consider morphisms $f\in \Hom((A,X),(B,Y))$ and $g\in \Hom((B,Y),(C,Z))$. Then 
    $$(\pazocal{F}(gf))(X)=(\pazocal{F}(g)\circ \pazocal{F}(f))(X)= \pazocal{F}(g)(\pazocal{F}(f)(X))=\pazocal{F}(g)(Y)=Z,$$
    so $gf\in \Hom(\int\pazocal{F})$. Clearly composition is associative and $\fone_(A,X):=\fone_A$ is the identity morphism for each object $A$. So $\int\pazocal{F}$ indeed does define a category.
\end{remark}
We denote the forgetful functor 
\begin{gather*}
    \int\pazocal{F} \rightarrow \pazocal{C}\\
    (A,X)\mapsto A\\
    f: (A,X)\rightarrow (B,X) \mapsto f: A\rightarrow B
\end{gather*}
by $\Pi$. 
\begin{definition}
    The \emph{category of elements} for a contravariant functor $\pazocal{F}:\pazocal{C}\rightarrow \mathrm{Set}$ is just the category of elements of $\pazocal{F}$ regarded as covariant functor $\pazocal{F} :\pazocal{C}\op\rightarrow \mathrm{Set}$.
\end{definition}
\begin{lemma}
    For a covariant functor $\pazocal{F}:\pazocal{C}\op \rightarrow \mathrm{Set}$, 
    $$\int\pazocal{F} \cong y \downarrow \pazocal{F}.$$
    Here $y: \pazocal{C}\rightarrow \mathrm{Set}^{\pazocal{C}\op}$ is the yoneda embedding and $\pazocal{F}$ is regarded as functor $\fone\rightarrow \mathrm{Set}^{\pazocal{C}\op}$.
\end{lemma}
\begin{proof}
    
\end{proof}
\begin{proposition}
    A covariant functor $\pazocal{F} : \pazocal{C}\rightarrow \mathrm{Set}$ is representable if and only if $\int \pazocal{F}$ has an initial object. For a contravariant functor, the same but the category of elements has a terminal object.  
\end{proposition}
\begin{proof}
    
\end{proof}
\subsection{Limits \& Colimits}

\subsubsection{Currying}