\newpage
\appendix
\section{Logic}
\section{First Order Predicate Logic}
The purpose of this section is to introduce a framework for doing mathematics. To be precise we want to define a formal language in which, we in principle could formalise the math described in the main sections of these note. We will choose first order logic \verb|FOL|.
\subsection{The Metatheory of First Order Logic}
\subsubsection{Classical Metatheory}
We assume a notion of \emph{finiteness}. With this notion consider symbols $\mathbf{0}$ and $\mathbf{s}$, from which we build a \emph{potentially infinite} list $\mathbf{N}$ of \emph{natural numbers} of finite strings formed by appending $\mathbf{s}$ to the left to the prior string in the list, where the list starts with $\mathbf{0}$. In other words the have a list
$$\mathbf{N} = [\mathbf{0},\mathbf{s0},\mathbf{ss0},\dots]$$ 
where $\dots$ signify that to look up the next string in the list, we concatenate from the right $\mathbf{s}$ the last formed string in the list. Each string is either the string $\mathbf{0}$ or $\sigma\mathbf{0}$, where $\sigma$ is some non-empty finite string of $\mathbf{s}$'s. We thus have the following for possibly empty strings of $\mathbf{s}$'s, $\sigma, \pi,\rho$
\begin{gather*}
    \sigma\pi \mathbf{0} = \pi\sigma \mathbf{0},\\
    \mathbf{s}\sigma\pi\mathbf{0} \equiv \mathbf{s}\pi\sigma\mathbf{0},\\
    \mathbf{s}\sigma\pi\mathbf{0}\equiv \sigma \mathbf{s}\pi\mathbf{0},\\
    \sigma\mathbf{0} \equiv \pi\mathbf{0} \metaiff \mathbf{s}\sigma\mathbf{0}\equiv \mathbf{s}\pi\mathbf{0},\\
    \sigma\mathbf{0}\equiv \pi\mathbf{0} \metaiff \sigma\rho\mathbf{0}\equiv \sigma\rho\mathbf{0}, 
\end{gather*}
where "$\equiv$" means "identical to". For two elements in $\mathbf{N}$, $n,m$ we write $n<m$ if $n$ is formed earlier in the list than $m$ and $n>m$ if $n$ is formed later in the list than $m$. By $n+1$ we mean the next element to be formed in the list. And if $n$ is not $\mathbf{0}$ we write $n-1$ to be the string formed immediately before $n$. We define 
$$\sigma\mathbf{0}+\mathbf{0} :\equiv \sigma\mathbf{0} \text{ and } 0 + \pi\mathbf{0} :\equiv \pi\mathbf{0}$$
and 
$$\sigma\mathbf{0} + \pi\mathbf{0} :\equiv \sigma\pi \mathbf{0}.$$
We also have a principle of induction. I.e. if some statement $P$ holds for $\mathbf{0}$ and whenever $P$ holds for $n$, then it holds for $n+1$, then $P$ holds for all strings in $\mathbf{N}$.\\ 
We also have a principle of recursion. I.e. If we define $X_0$ and whenever $X_n$ is defined, so is $X_{n+1}$, then $X_m$ is defined for every string $m$ in $\mathbf{N}$.\\
Note here that we have implicitly introduced the notion of indexing. Given a finite list of objects, initialize a point on $\mathbf{N}$ at $\mathbf{0}$, pick out an element in the list of objects and assign the value point at in $\mathbf{N}$, move the point to the next element in $\mathbf{N}$ and move pick out the next element in the list of objects. This lets us consider a length $n$ list of object, e.g. $t_1,\dots,t_n$. These informal metamathematical concepts form a sufficient \emph{classical metatheory} for the syntax of  first order logic. 
\subsubsection{Strong Metatheory}
Sometimes we need to strengthen our metatheory with \emph{the law of excluded middle} and \emph{naive set theory}. The first of these notions is that a statement is either true or false. The last of which is the notion that we can consider a sufficiently small collection of objects and ask whether a given object is a member of such a collection. We will need this to define the semantics of first order logic.
\subsection{The Alphabet of First Order Logic}
\subsubsection*{The Logical Symbols}
\verb|FOL| is a framework for formalizing different theories. It is therefor has to flexible enough to be augmented to work in a large variety of contexts and domains. We therefor make distinctions between the role of the symbols that comprise \verb|FOL|. The \emph{logical symbols} in the language of \verb|FOL| are those symbols that are not context specific. These are 
\begin{enumerate}
    \item \textbf{Variables}: These are play the role of placeholders for objects in the domain to which we want to apply \verb|FOL|. Eg. $x$ might be a placeholder for a set in Zermol-Fraenkel set theory or $n$ might be a placeholder for a natural number in the standard model of Peano arithemetic or $\mathcal{C}$ could be a placeholder for a category in Category Theory. We presuppose a potentially infinite list of such symbols. That is we may always generate sufficiently large number of distinct variables.
    \item \textbf{Logical operators:} $\neg$ (not), $\wedge$ (and), $\vee$ (or), $\to$ (implies).
    \item \textbf{Logical quantifiers:} $\forall$ (universal quantifier, to be read as "for all" or "for every") and $\exists$ (existential quantifier, to be read as "exists" or "there is").
    \item \textbf{Equality symbol:} $=$ a \emph{relational symbol} (scroll down a little for a definition) which is not context specific.    
\end{enumerate}
\subsubsection{The Non-logical Symbols}
Those symbols that are introduced in some context to use \verb|FOL| to talk about some domain. these are 
\begin{enumerate}
    \item \textbf{Constant symbols:} These are symbols that signify a specific object in a specific domain. Examples are $\emptyset$ in for example Zermelo-Fraenkel set theory or $0$ in Peano arithmetic.
    \item \textbf{Function symbols:} A symbol that is a placeholder for an object that depends on finite list of objects in a domain called arguments. For instance if $F$ is a function symbol that takes $n$ arguments, then given variables $x_1,\dots, x_n$, $F(x_1,\dots,x_n)$ is some object. A function symbol has a natural number attached to them called an \emph{arity}. A function symbol that takes $n$ arguments is an $n$-ary function. Examples are $+$ in Peano Arithmetic or $\cup$ in Zermelo-fraenkel set theory. 
    \item \textbf{Relation symbols:} Symbol that signify relations of objects. Like function symbols these have a natural number arity and given a relation symbol $R$ and objects $x_1,\dots, x_n$, $R(x_1,\dots,x_n)$ signifies that $x_1,\dots,x_n$ are in relation by $R$.  
\end{enumerate}
To actually give these syntactical meaning, we need to define rules for building well-formed. I.e. we need to define what strings of symbols we are allowed to write in the language. A set of non-logical symbols is called a \textit{signature} often denoted $\mathcal{L}$. 
\subsubsection{Terms and Formulae}
For a signature $\mathcal{L}$, a string of symbols is an \emph{$\mathcal{L}$-term} if it is a result of finitely applying these rules in some order
\begin{align*}
    \mathrm{T0}\colon\quad & \text{Each variable is an } \mathcal{L}\text{-term}.\\
    \mathrm{T1}\colon\quad & \text{Each constant symbol in }\mathcal{L} \text{ is an } \mathcal{L}\text{-term}.\\
    \mathrm{T2}\colon\quad & \text{If } \tau_1,\dots,\tau_n \text{ are any } \mathcal{L}\text{-terms and }F \text{ is an } n \text{-ary function symbol in } \mathcal{L}\text{,}\\
    &\text{then } F(\tau_1,\dots,\tau_n) \text{ is an } \mathcal{L}\text{-term}. 
\end{align*} 
When the symbols are not involving any non-logical symbols, it is called a \emph{term}. Those terms of the form $\mathrm{T0}$ or $\mathrm{T1}$ are called \emph{atomic terms}. Note that in the above, the $\tau$'s are placeholders for objects in the language \textbf{not} for objects in a domain of interest. If we want to prove a property $\Phi$ of terms, then we do that by proving that each of the three categories of terms satisfy $\Phi$, which is called the \emph{induction on term construction}.\\
An \emph{$\mathcal{L}$-formula} is a string of symbols resulting from finite application of the rules
\begin{align*}
    \mathrm{F0}\colon\quad & \text{For } \mathcal{L}\text{-terms } \tau_1,\tau_2\text{ then } \tau_1 = \tau_2\text{ is an } \mathcal{L}\text{-formula.}\\
    \mathrm{F1}\colon\quad & \text{For } \mathcal{L}\text{-terms } \tau_1,\dots,\tau_n \text{ and } R \text{ a non-logical } n\text{-ary relation symbol in }\mathcal{L},\\
     &\text{then } R(\tau_1,\dots,\tau_n) \text{ is an } \mathcal{L}\text{-formula.}\\
     \mathrm{F2}\colon\quad & \text{if }\varphi \text{ is an }\mathcal{L}\text{-formula, then } \neg\varphi \text{ is an }\mathcal{L}\text{-formula.}\\
     \mathrm{F3}\colon\quad & \text{if }\varphi \text{ and } \psi \text{ are } \mathcal{L}\text{-formulae, then } \varphi\to \psi, \varphi\wedge \psi \text{ and } \varphi\vee \psi \text{ are } \mathcal{L}\text{-formulae.}\\
     \mathrm{F4}\colon\quad &  \text{if }\varphi \text{ is an }\mathcal{L}\text{-formula, then given a variable }x\text{, } \exists x\varphi \text{ and } \forall x\varphi \text{ are } \mathcal{L}\text{-formulae.}  
\end{align*}
When an $\mathcal{L}$-formula is build from terms, it is simply called a \emph{formula}. $\mathrm{F0}$- and $\mathrm{F1}$-formulae are called \emph{atomic formulae}. The \emph{induction on formula construction} refers to the fact that proving a property $\Phi$ is satisfied by formulae, we can prove that it is satisfied for atomic formulae, and that for $\mathcal{L}$-formulae $\varphi,\psi$ satisfying $\Phi$, given a variable $x$, then $\neg\varphi,\varphi \wedge \psi,\varphi\vee\psi,\varphi\to \psi,\exists x\varphi,\forall x\varphi$ also satisfy $\Phi$.\\
Consider a variable $x$ and a formula $\varphi$ of the form $\forall x\psi$ or $\exists x\psi$. If $x$ is used to construct $\psi$ and does not immediately appear after a logical quantifier at some position in $\psi$, it is said to be in the \emph{range of} said logical quantifier. Such a variable is said to \emph{bound at that position} by the last quantifier that it is in the range of. A variable that is not bound by a quantifier at a particular position is said to be \emph{free} at position. A variable may appear in a formula as both bounded and free. For instance $(\forall x(x\neq x))\to \exists z(x=x)$. The first two occurences of $x$ are bound by the $\forall$ and the last two are bound by the $\exists$. The set of free variables of a formula (a rule given later will show that this set is uniquely defined) is denoted $\mathrm{free}(\varphi)$. A formula $\varphi$ is a \emph{sentence} if $\mathrm{free}(\varphi)=\emptyset$, so $x\neq x\to x=x$ is a formula while $\forall x(x\neq x\to x=x)$ is a sentence. A term is \emph{closed} if it contains no variables. For a formula $\phi$ and variables $x_1,\dots,x_n$ with $\{x_1,\dots, x_n\}\subset \mathrm{free}(\phi)$, we denote $\phi$ by $\phi(x_1,\dots,x_n)$. In the metalanguage of \verb|FOL|, we write $\varphi \equiv \psi$ for formulas, if the strings comprising these are identical.\\ 
When $\tau$ is a term and $x$ is a variable in $\tau$, for a term $\omega$, we get a new term by \emph{substituting} $x$ with $\omega$, i.e. by replacing every instance of $x$ by $\omega$, which we may do considering $\tau$ as unary function symbol in $\mathcal{L}$. We denote this new term by $\tau(x/\omega)$.\\ 
For a formula $\varphi$ and a variable $x$ and $\tau$ a term, $\varphi(x/\tau)$ is the formula obtained by replacing every free instance of $x$ in $\varphi$ by $\tau$. So we obtain a notion of substitution for formulas as well. A substition of $x$ in a formula $\varphi$ with a term $\tau$ is called \emph{admissible} if it is not in the range of a quantifier that binds a variable in $\tau$. If $x$ does not occur as a free variable in $\varphi$, then trivially $x$ is not bound hence, $\varphi(x/\tau)$ is admissible. In this case the string $\varphi$ is unchanged, since no instance $x$ is replaced by $\tau$. So $\varphi\equiv \varphi(x/\tau)$. In general, for an admissible substitution, we write $\varphi(\tau)$ instead of $\varphi(x/\tau)$. When we declare that a symbol, $\phi$, in the metalanguage denotes another, $\psi$, we write $\phi :\equiv \psi$. For instance, $\varphi(\tau) :=  \varphi(x/\tau)$. 
\subsection{Axioms and Inference Rules}
An \emph{axiom} is a special formula, which will be one ingredient in producing new formulas. In a looser sense (for now), think of axioms as statements that are valid in some context. Sometimes we also consider \emph{axiom schemae}, defined to be a collection axioms taken to be valid for every instance of some fixed list of function symbols, a fixed list of relation symbols and a fixed list of formulae.
\subsubsection{Logical Axioms: Assigning Truth Values to Formulae}
The \emph{logical axioms} are those axioms that are valid in any context. These are given by the following axiom schemae. Consider $\varphi,\varphi_1,\varphi_2,\varphi_3,\psi$ arbitrary formulae. We have formulae
\begin{align*}
    \mathrm{L0}\colon\quad & \varphi\vee \neg\varphi\\
    \mathrm{L1}\colon\quad & \varphi\to (\psi\to \varphi)\\
    \mathrm{L2}\colon \quad & (\psi\to (\varphi_1\to \varphi_2))\to ((\psi\to \varphi_1)\to (\psi\to \varphi_2))\\
    \mathrm{L3}\colon \quad & (\varphi\wedge \psi)\to \varphi\\
    \mathrm{L4}\colon \quad & (\varphi\wedge \psi)\to \psi\\
    \mathrm{L5}\colon \quad & \varphi\to (\psi\to (\varphi\wedge \psi))\\
    \mathrm{L6}\colon \quad & \varphi \to (\varphi\vee \psi)\\
    \mathrm{L7}\colon \quad & \psi \to (\varphi \vee \psi)\\
    \mathrm{L8}\colon \quad & (\varphi_1\to \varphi_3)\to ((\varphi_2\to \varphi_3)\to ((\varphi_1\vee \varphi_2)\to \varphi_3 ))\\
    \mathrm{L9}\colon \quad & \neg\varphi \to (\varphi \to \psi).
\end{align*}
Moreover if $x$ is a variable in $\varphi$ and $\tau$ a term such that $\varphi(x/\tau)$ is admissible, then 
\begin{align*}
    \mathrm{L10}\colon \quad & \forall x\varphi(x) \to \varphi(\tau)\\
    \mathrm{L11}\colon \quad & \varphi(\tau)\to \exists x\varphi(x). 
\end{align*}
Suppose $x$ is a variable of $\psi$ with $x\notin\mathrm{free}(\psi)$. Then 
\begin{align*}
    \mathrm{L12}\colon \quad & \forall x(\psi\to \varphi(x))\to (\psi\to \forall x\varphi(x))\\
    \mathrm{L13}\colon \quad & \forall x(\varphi(x)\to \psi)\to (\exists\varphi(x)\to \psi).
\end{align*} 
Lastly we have axioms for the binary relational symbol $=$, for general $n$-ary relational symbols and for $n$-ary functional symbols. Let $\tau,\tau_1,\dots,\tau_n,\tau_1',\dots,\tau_n'$ be terms and $R$ an $n$-ary relation symbol and $F$ an $n$-ary function symbol.
\begin{align*}
    \mathrm{L14}\colon \quad & \tau = \tau\\
    \mathrm{L15}\colon \quad & \tau_1 = \tau_1' \wedge \dots \wedge \tau_n = \tau_n' \to (R(\tau_1,\dots,\tau_n)\to R(\tau_1',\dots,\tau_n'))\\
    \mathrm{L16}\colon \quad & \tau_1 = \tau_1' \wedge \dots \wedge \tau_n = \tau_n' \to (F(\tau_1,\dots,\tau_n)= F(\tau_1',\dots,\tau_n'))  
\end{align*}
The above axioms when $\mathrm{L}_0$ is excluded is called the logical axioms of \emph{intuitionistic logic}. The inclusion of $\mathrm{L}_0$ comprises the logical axioms of \emph{classical logic}.
We use $:\iff$ to define relations on symbols in metalanguage of \verb|FOL|. For instance, we define for formulae $\varphi,\psi$, terms $\tau,\tau'$ and a variable $x$, 
\begin{gather*}
    \varphi \leftrightarrow \psi :\metaiff (\varphi \to \psi \wedge \psi \to \varphi)\\
    \exists! x\varphi :\metaiff \exists x\varphi(x) \wedge \forall y(\varphi(y)\to x=y)\\
    \tau \neq \tau' :\metaiff \neg(\tau = \tau').
\end{gather*}

\subsubsection{Non-logical Axioms: Defining a Theory}
To the logical axioms we may add a set of formulae (non-logical axioms), formed from a signature $\mathcal{L}$, which we call an \emph{$\mathcal{L}$-theory}. Examples of non-logical axioms are those of, \verb|ZF|-axioms which forms Zermelo-Fraenkel set theory, the axioms of topology which forms the theory of topology, \verb|PA|-axioms which forms number theory, etc. For mathematical theories such axioms are always sentences.  
\subsubsection{Proofs}
To deduce new formulas from the axioms of some $\mathcal{L}$-theory, we introduce two \emph{inference rules}: 
$$\infer{\psi}{\varphi \to \psi & \varphi} \quad \quad \quad\text{and}\quad \quad \quad  \infer{\forall x \varphi}{\varphi}.$$
These are called \emph{modus ponens} and \emph{generalization} respectively.\\
We want to use the notion of axiom and inference rule to prove statements in different $\mathcal{L}$-theories. Let $\Phi$ be a finite set of $\mathcal{L}$-formulae and $\psi$ be an $\mathcal{L}$-formula. We say that \emph{$\psi$ is provable from $\Phi$}, denoted $\Phi\vdash \psi$ if there are $\mathcal{L}$-formulae $\varphi_1,\dots,\varphi_n$, called \emph{a formal proof of $\psi$} satisfying $\varphi_n\equiv \psi$ and for $i\leq n$ $\varphi_i$ satisfies one of the following conditions 
\begin{enumerate}
    \item It is a logical axiom.
    \item It is an element of $\Phi$
    \item There are $j,k< i$ such that $\varphi_j\equiv \varphi_k \to \varphi_i$.
    \item There is a $j<i$ with $\varphi_i \equiv \forall x \varphi_j$, where $x\notin \mathrm{free}(\phi)$ for any of the $\phi\in \Phi$.
\end{enumerate} 
We take $\vdash \psi$ to mean $\emptyset \vdash \psi$. In this case $\psi$ is called a \emph{tautology}. $\Phi$ is called a \emph{context}. If $\psi$ has no formal proof in a fixed context $\Phi$, we write $\Phi \not\vdash \psi$. Rather than writing down formal proofs down in, for instance, a box proof, we will just note that this is possible to do and refer to tools like \href{https://leanprover-community.github.io/learn.html}{Lean} and \href{https://rocq-prover.org/about}{Rocq} which uses an embedding of $\verb|FOL|$ in type theory to enable formal proof writing that can be checked automatically. Instead, we will write proofs in natural language in such a way that it to a sufficiently knowledgeable reader is somewhat clear what a formal proof would look like. At a certain point we also formulate formulae in natural language. Additionally we take certain basic proofs in \verb|FOL| for granted. This treatment of \verb|FOL| only serves to provide a clear and precise language for building mathematical theory and we will quickly become satisfied that it exists and be happy that if we properly formulate theory in natural language it will be formalizable in \verb|FOL|.  
\subsubsection{Tautology and Logical Equivalence}
We say that formulae $\varphi$ and $\psi$ are \emph{logically equivalent} if $\vdash \varphi \leftrightarrow \psi$. We write 
$$\varphi \iff \psi \metaiff \vdash \varphi \leftrightarrow \psi.$$
If $\Phi \vdash \varphi \leftrightarrow \psi$, we write $\varphi \iff_\Phi \psi$. In a natural language context, $\Phi$ will be a sequence of sentences in natural language and will not explicitly be written as a finite set, so therefore we usually just write $\varphi \iff \psi$ if we want to symbolically express that $\varphi$ is equivalent to $\psi$ in a given context (it should then be clear what the context is). In the same vein we write $\varphi\implies_\Phi \psi$ to mean $\Phi \vdash \varphi \to \psi$ and in a mathematical context we shorten it to $\varphi \implies \psi$. 
\subsubsection{Proofs in Natural Deduction}
A different approach to building a system of deduction in \verb|FOL| is to write inference rules which tells us how to introduce and eliminate certain formulae to obtain new formulae. There is a different notion of proof in natural deduction. Luckily this approach is equivalent to that of introducing logical axioms, modus ponens and generalization. So when building proofs, we can also apply introduction and elimination rules. 
\subsubsection{The Deduction Theorem}
Beyond the extra proof techniques that are provided by natural deduction which remain unmentioned we give a useful metatheoretic principical 
\begin{theorem}(Deduction Theorem)
    Let a signature $\mathcal{L}$ be given and $\Phi$ be a set of $\mathcal{L}$-formulae. Consider $\mathcal{L}$-formulae $\varphi,\psi$. Then 
    $$\Phi+\varphi \vdash \psi \metaiff \Phi \vdash \varphi\to \psi,$$
    where $\Phi+\varphi$ is $\Phi\cup\{\varphi\}.$
\end{theorem}
\begin{proofsketch}
    "$\impliedby$": Suppose $\Phi\vdash \varphi\to \psi$. With the assumption $\Phi+\varphi$, we then get $\varphi\to \psi$, by the metatheoretic assumption, there is a formal proof $\varphi_1,\dots,\varphi_n:\equiv \varphi\to \psi$ from $\Phi$. Then using modus ponens on $\varphi_n$ and $\varphi\in \Phi+\varphi$ we get $\psi$. Then from $\Phi+\varphi$, $\varphi_1,\dots,\varphi_n,\psi$ is a sequence of formulae such that $\Phi+\varphi \vdash \psi$.\\
    "$\implies$": Suppose $\Phi+\varphi\vdash \psi$. Then there is a formal proof $\varphi_1,\dots,\varphi_n$ of $\psi$ from $\Phi+\varphi$. We now aim to show $\Phi \vdash \varphi\to \varphi_n$ (note that $\varphi_n\equiv \psi$) for each $i\leq n$. Note that if $\varphi_i\in \Phi$ or is a logical axiom, we first using the logical axiom $\mathrm{L}_1$ get $\sigma\equiv \varphi_i \to (\varphi\to \varphi_i)$ and using modus ponens on $\sigma$ and $\varphi_i$ we get $\varphi\to \varphi_i$. If $\varphi_i\equiv \varphi$, we get $\Phi\vdash \varphi\to\varphi$, since $\vdash \varphi\to \varphi$. From this we conclude $\Phi\vdash \varphi\to \varphi_0$, since $\varphi_0$ can only be obtained in the three ways described above. Suppose now that for $j< i$, $\Phi\vdash \varphi\to \varphi_j$. If we then can show that when $\varphi_i$ is obtained from modus ponens or generalization, it will follow by (metatheoretical) induction on $i\leq n$ that $\Phi\vdash \varphi\to\varphi_n$.
\end{proofsketch}
\subsubsection{Consistency and Compactness}
To discuss the notion of consistency we first introduce the notion of \emph{ex falso quodlibet}. given formulae $\varphi$ and $\psi$, this is $\{\varphi,\neg \varphi\}\vdash \psi$. It is proven in the following way: we get $\varphi$ and $\neg \varphi$ from the context. From modus ponens applied to $\neg\varphi \to (\varphi \to \psi)$ (from $\mathrm{L}9$) and $\neg \varphi$ we get $\varphi\to \psi$. Applying modus ponens to $\varphi\to \psi$ and $\varphi$ we get $\psi$ as desired. An \emph{ex falso quadlibet introduction} is $\Phi\vdash \varphi\wedge \neg\varphi$ where $\varphi$ is some $\mathcal{L}$-formula. We denote the existence of such an instance, $\varphi\wedge \neg\varphi$ by $\mathrm{False}$ \\
A set of formulae $\Phi$ is called \emph{consistent} if $\Phi\not\vdash \mathrm{False}$. We denote this by $\mathrm{Con}(\varphi)$. A set of formulae that is not consistent is called \emph{inconsistent}, denoted $\neg\mathrm{Con}(\Phi)$. Note that if a set of formulae is not consistent then we may prove anything from that set of formulae. I.e. for a formula $\varphi$ 
$$\neg\mathrm{Con}(\Phi)\metaimplies \Phi \vdash \mathrm{False} \metaimplies \Phi \vdash \varphi.$$
If on the other hand $\Phi$ is a consistent set of formulae and we can prove a formula $\varphi$ from $\Phi$, then then there is no way to prove $\neg\varphi$ from $\Phi$,
$$\mathrm{Con}(\Phi) \text{ and } \Phi\vdash \varphi\metaimplies \Phi\not\vdash \neg\varphi.$$
Indeed if $\Phi\vdash \neg\varphi$, and we suppose $\Phi\vdash \varphi$, then $\Phi\vdash \mathrm{False}$, hence $\neg\mathrm{Con}(\Phi)$.\\

A use case for the principle of ex falso quodlibet is contraposition
\begin{theorem}(Proof by Contraposition)
    Given $\mathcal{L}$-formulae $\varphi$ and $\psi$,
    $$\vdash (\varphi \to \psi) \leftrightarrow (\neg\psi\to \neg\varphi).$$
\end{theorem}
\begin{proof}
    We aim to apply $\wedge$-introduction. To do this we need to show $\vdash (\varphi\to \psi) \to (\neg \psi\to \varphi)$ and $\vdash (\neg \psi\to \neg \varphi)\to (\varphi\to \psi)$. First we show $\varphi \to \psi \vdash (\neg\psi\to \neg\varphi)$ which can be shown by $\varphi \to \psi + \neg \psi\vdash \neg \varphi$. Now note that $\{\varphi\to \psi,\neg\psi\}+\varphi\vdash \psi\wedge \neg\psi$, since by modus ponens on $\varphi\to\psi$ and $\varphi$ results in $\psi$, hence upon applying $\wedge$-introduction we get $\psi\wedge \neg\psi$. We can therefor use ex falso to get $\neg \varphi$.\\
    Next we show $\neg \psi \to \neg \varphi \vdash \varphi\to \psi$ which can be shown by $\{\neg\psi\to \neg \varphi,\varphi\}\vdash \psi$. Using $\mathrm{L}_0$ we get $\psi\vee \neg\psi$. If $\psi$, then we are done. If $\neg \psi$, then using modus ponens on $\neg\psi\to \neg \varphi$ and $\neg \psi$ we get $\neg \varphi$, hence $\varphi\wedge \neg \varphi$. Then using ex falso we obtain $\psi$.
\end{proof}
\begin{remark}
    Note that $\vdash (\varphi\to \psi)\to (\neg \psi\to \neg \varphi)$ can be done fully in intuitionistic logic. A more general statement then the above is 
    $$\Phi+\varphi \vdash \psi \metaiff \Phi+\neg\psi\vdash \neg\varphi.$$
    With the above in mind, it is easy to see how one proves this. 
\end{remark}
\begin{theorem}(Proof by Contradiction) Let $\Phi$ be a set of $\mathcal{L}$-formulae and $\varphi,\psi $ be $\mathcal{L}$-formulae. Then 
$$
    \Phi+\neg\varphi \vdash \mathrm{False} \metaimplies \Phi \vdash \varphi
$$
\end{theorem}
\begin{proof}
    There is some way to prove $\psi \wedge \neg \psi$ for some $\mathcal{L}$-formula from $\Phi+\neg\varphi$, hence $\Phi\vdash \neg\varphi \to \psi\wedge \neg \psi$. We then apply $\mathrm{L}_0$ to get $\varphi\vee \neg \varphi$. If $\varphi$, we are done. If $\neg\varphi$ we get $\psi\wedge \neg \psi$ using modus ponens, hence using exfalso we get $\varphi$. 
\end{proof}
Given some $\mathcal{L}$-formula $\varphi$ and a set of $\mathcal{L}$-formulae $\Phi$ we also have that 
$$\neg\mathrm{Con}(\Phi+\varphi)\metaimplies \Phi \vdash \neg\varphi,$$
since then $\Phi+\varphi \vdash \mathrm{False}$ which is equivalent to $\Phi +\neg(\neg \varphi) \vdash \mathrm{False}$, so using contradiction we get $\Phi\vdash \neg\varphi$. 
Conversely if 
$$(\text{there is an }\mathcal{L}\text{-formula }\varphi \text{ such that } \Phi\vdash \neg \varphi)\metaimplies \neg\mathrm{Con}(\Phi+\varphi),$$
since then get $\Phi+\varphi\vdash \varphi\wedge \neg \varphi$, hence by contradiction, $\Phi+\varphi\vdash \mathrm{False}$.
\begin{lemma}\label{TheoryWithInconsistentSubTheoryIsInconsistent}
    Let $\Phi$ be a set of $\mathcal{L}$-formulae and $\Phi'\subset \Phi$. Then $\neg\mathrm{Con}(\Phi')$ implies $\neg\mathrm{Con}(\Phi)$
\end{lemma}
\begin{proof}
    Indeed, if there is a formula $\varphi$ such that $\Phi'\vdash \varphi\wedge \neg\varphi$, then trivially $\Phi\vdash \varphi\wedge \neg\varphi$.
\end{proof}
\begin{theorem}(Compactness Theorem)\label{CompactnessTheorem}
    A set of $\mathcal{L}$-formulae $\Phi$ is consistent if and only if every finite subset $\Phi'\subset \Phi$ is consistent
\end{theorem}
\begin{proof}
    One implication follows immediately from the above lemma.
    Suppose $\Phi$ is not consistent. Then there is a formula $\varphi$ and a sequence of formulae $\varphi_1,\dots,\varphi_n$ with $\varphi_n = \varphi\wedge \neg\varphi$ that is a formal proof of $\varphi\wedge \neg\varphi$ from $\Phi$. Now consider the finite subsequence $\varphi_{i_1},\dots,\varphi_{i_m}\in \Phi$, with $1\leq i_1<\dots <i_m\leq n$. Then setting $\Phi':=\{\varphi_{i_1},\dots,\varphi_{i_m}\}\subset \Phi$ we get $\Phi'\vdash \varphi\wedge \neg\varphi$, hence $\neg\mathrm{Con}(\Phi')$.  
\end{proof}
\subsubsection{Sentences in PNF and sPNF}
Let $\quantifs$ denote either $\exists$ or $\forall$.
\begin{definition}
    For an $\mathcal{L}$-sentence $\sigma$, it is in \emph{Prenex Normal Form of $\sigma$}, denoted \emph{PNF}, if it is on the form
    $$\quantifs_0 x_0\quantifs_1x_1\dots \quantifs_n x_n\sigma'$$
    where $\sigma'$ is quantifier free.\\
    $\sigma$ is in \emph{special Prenex Normal Form}, denoted \emph{sPNF}, if it is in PNF as above where $x_1,\dots,x_n$ appear free in $\sigma'$.  
\end{definition}
\begin{lemma}\label{EverySentenceHasAnEquivalentSentenceInsPNF}
    Given an $\mathcal{L}$-sentence $\sigma$ there is a sentence $\tau$ in sPNF such that
    $$\sigma \iff \tau.$$ 
\end{lemma}
\begin{proof}
    We don't prove this result.
\end{proof}
\subsection{Semantics of First Order Logic}
We are now going to assign truth values to formulae. That is we want for some signature $\mathcal{L}$ to assign either false or true to each $\mathcal{L}$-formula. In propositional logic this can be done by asserting truth tabels for logical connectives, $\wedge$, $\vee$, $\neg$, $\to$. This simple approach does not extend to \verb|FOL|. Take for instance the formula, $\forall x \exists y \varphi(x,y)$. Could we possibly assign a truthtable to such a formula. Suppose for instance that $\varphi\equiv x<y$. Then there is a dependency on what $x$ and $y$ is and we are therefor at a loss in computing a truth table in terms of $\varphi$ in general. Introducing an algorithm for determining truth for $\forall$ and $\exists$ is not terrible, but we need a way to map formulas onto some sort of context. To discuss how to do this we need the notion of an \emph{interpretatation} for which we will assume some notion of naive set theory and a notion of law of excluded middle, i.e. we assume that a statement (to be interpreted informally) is either true or false. We are thus working in the strong metatheory.
\subsubsection{Structures and Interpretations}
given a signature $\mathcal{L}$ an $\mathcal{L}$-structure $M$ is a pair of a non-empty set $A$ (the domain of $M$) and a mapping from $\mathcal{L}$ to $A$ taking constant symbols $c\in \mathcal{L}$ to an element $c^M\in A$, $n$-ary relation symbols $R\in \mathcal{L}$ to a set of $n$-tuples $R^M\subset A^n$ and $n$-ary function symbols $F$ to a function $F^M: A^n \rightarrow A$. An \emph{assignment} on an $\mathcal{L}$-structure $M$ is a mapping, $\iota$, taking variables $x$ in $\mathcal{L}$ to $A$. An $\mathcal{L}$-interpretation $I$ is then a pair consisting of an $\mathcal{L}$-structure $M$ and an assignment $\iota$ on $M$ where to a variable $x\in \mathcal{L}$ and an element $a\in A$, we define for variables $y$ in $\mathcal{L}$
$$
    \iota[x\mapsto a] (y)=
    \begin{cases}
        a & \text{if } x\equiv y\\
        \iota(y) & \text{else}
    \end{cases}
$$
$\iota[x\mapsto a]$ is thus itself an assigment on $M$. We define 
$$I[x\mapsto a] := (M,\iota[x\mapsto a]).$$
For an $\mathcal{L}$-term $\tau$ we define $I(\tau)\in A$ by recursion on terms in the following way: On a variable $x$, $I(x):=\iota(x)$, on a constant $c$, $I(c):= c^M$, for an $n$-ary function symbol and terms $\tau_1,\dots,\tau_n$, $I(F(\tau_1,\dots,\tau_n)):= F^M(I(\tau_1),\dots,I(\tau_n)).$ We now define how to assign truth value to formulae under an interpretation. When a formula $\varphi$ is true in an interpretatation $I$ we write $I\vDash \varphi$. This mapping is constructed inductively on symbols that are used to construct formulae. So for instance given terms $\tau_1,\tau_2$ and variable $x$, $I\vDash \tau_1=\tau_2 :\iff_{\mathrm{meta}} I(\tau_1) = I(\tau_2)$ and $I\vDash \exists x \varphi: \iff_{\mathrm{meta}}\ \mathrm{there}\ \mathrm{exists}\ a\ \mathrm{in}\ A: \ I[x\mapsto a]\vDash \varphi$ and $I\vDash \neg\varphi :\iff_{\mathrm{meta}} I\vDash \varphi \text{is not true}$. I.e. we translate truth of a formula in some interpretation to statements in the metatheory. I feel that there is only a need to illustrate the idea as I don't want to write all of the rules down. Under the assumption of classical logic in the meta theory, we now have that $I \vDash \varphi \ \mathrm{or} \ I\vDash \neg \varphi$. It is still necessarily NOT true that $\Phi \vdash \varphi$ or $\Phi \vdash \neg\varphi$. That is there may be statements in some theory that is true but not provable or a statement that is true but where $\Phi \vdash \neg\varphi$.\\
Given a set of $\mathcal{L}$-formulae $\Phi$ and an $\mathcal{L}$-structure $M$, we say that \emph{$M$ is a model of $\Phi$} if for for every assignment $\iota$ and for each $\varphi\in \Phi$, 
$$(M,j) \vDash \varphi.$$
If this is the case, we write 
$$M\vDash \Phi.$$
We note that one can construct a signature $\mathcal{L}$ and a domain $A$ for which there is a set of sentences $\Phi$ and two $\mathcal{L}$-structures $M_1$ and $M_2$ where $M_1$ is a model of $\Phi$ and $M_2$ is not. Take $\mathcal{L}=\{c,f\}$ where $c$ is constant and $f$ is a unary function symbol, $A=\{0,1\}$ and $\Phi = \{\varphi_1,\varphi_2\}$ with $\varphi_1 :\equiv\forall x(x=c\vee x=f(c))$ and $\varphi_2:\equiv \exists(x\neq c)$. Now we define $c^{M_1} :=0$, $f^{M_1}(0)=1, f^{M_2}(1)=0$ and $c^{M_2}:=0, f^{M_2}(0)=0,f^{M_2}(1)=1$. Fix an assignment and consider the associated interpretatations $I_1$ and $I_2$. It turns out that $I_i\vDash \varphi_2$ for $i=1,2$ while $I_1 \vDash\varphi_1$ and $I_2\vDash \neg\varphi_1$. Indeed, $I_i\vdash \varphi_2$ is equivalent to $I_i[x\mapsto 0] \vDash x\neq c$ or $I_i[x\mapsto 1] \vDash x\neq c$. We check that $I_i[x\mapsto 1] \vDash x\neq c$. To do this we have to check that it isn't true that $I_i[x\mapsto 1](x)$ is equal to $I_i[x\mapsto 1](c)$. This is obvious since $I_i[x\mapsto 1](x)=1$ and $I_i[x\mapsto 1](c)=c^{I_i}=0$. To check that $M_1\vDash \varphi_1$ it is sufficient to check that $I_1[x\mapsto 0] \vDash x=c$ and $I_1[x\mapsto 1]\vDash x = f(c)$. To check the first we simply note that $I_1[x\mapsto 0](x)=0$ and $I[x\mapsto 0](c)=c^{M_1}=0$. To check the second statement we again note that $I_1[x\mapsto1](x)=1$ and also note that $I_1[x\mapsto 1](f(c))=f^{M_1}(I[x\mapsto 1](c))=f^{M_1}(c^{M_1})=f^{M_1}(0)= 1$. To see that $M_2\vDash \neg \varphi_1$ (which is in fact equivalent to $M_2\not\vDash \varphi_1$, since $\varphi_1$ is a sentence), we need to check that $I_2 \vDash \exists x(x\neq c \wedge x\neq f(c))$ and for this it is sufficient to check that $I_2[x\mapsto 1]\vDash x\neq c \wedge x\neg f(c)$. Indeed, $I_2[x\mapsto 1](x)=1$ and $I_2[x\mapsto 1](c) =0$. Furthermore, $I_2[x\mapsto 1](f(c))=f^{M_2}(I_2[x\mapsto 1](c))=f^{M_2}(c^{M_2})=f^{M_2}(0)=0$.
\subsubsection{Universal closure}
Consider an $\mathcal{L}$-formula $\varphi$ and a model $M$ and some variable $x$. Then $M\vDash \varphi$ if and only if $M\vDash \forall x \varphi$. Upon defining $\overline{\varphi}:\equiv \forall x_1\forall x_2\forall \dots \forall x_n \varphi$ where $x_1,\dots,x_n$ are those variables that appear free in $\varphi$ at some position, then we get that $$M\vDash \varphi \iff_{\mathrm{meta}} M\vDash \overline{\varphi}.$$
this $\overline{\cdot}$-construction is called the \emph{universal closure of $\varphi$}.\\
We also introduce the following notation. Let $\varphi$ be an $\mathcal{L}$-formula with free variables $x_1,\dots,x_n$ and consider a sequence of elements in the domain $A$, $a_1,\dots,a_n$. Then we define $M\vDash \varphi(a_1,\dots,a_n)$ to mean that $(M,j [x\mapsto a_1][x\mapsto a_2]\dots[x\mapsto a_n]) \vDash \varphi(a_1,\dots,a_n)$ for every assignment $j$ in $M$.\\
\subsubsection{Isomorphisms of $\mathcal{L}$-structures}
We now give following notion of checking that two models are essentially the same:
\begin{definition}
    Fix a signature $\mathcal{L}$. Consider two $\mathcal{L}$-structures $M$ and $N$ with domains $A$ resp. $B$. We say that $M$ and $N$ are \emph{isomorphic}, denoted $M\simeq N$, if there is a bijection $f:A\rightarrow B$ satisfying
    \begin{enumerate}
        \item For every constant symbol $c\in \mathcal{L}$ 
        $$f(c^M)=f(c^N)$$
        \item For $n$ and every $n$-ary function symbol $F\in \mathcal{L}$ and all $n$-ary relation symbols $R\in\mathcal{L}$ and every $a_1,\dots,a_n\in A$, we have that 
        $$
            f(F^M(a_1,\dots,a_n))=F^{N}(f(a_1),\dots,f(a_n)),
        $$
        and that 
        $$
            (a_1,\dots,a_n)\in R^M \iff (f(a_1),\dots,f(a_n)) \in R^N.
        $$
    \end{enumerate}
    Two models are \emph{isomorphic} if there respective underlying $\mathcal{L}$-structures are isomorphic.   
\end{definition}
\begin{remark}
    An immediate consequence is that if two models $M$ and $N$ are isomorphic, then for each $\mathcal{L}$-formula $\varphi$, 
    $$M\vDash \varphi \iff_{\mathrm{meta}} N\vDash \varphi.$$
    It may happen that there is a pair of models $M'$ and $N'$ for which
    $$M'\vDash\varphi \iff_{\mathrm{meta}} N' \vDash \varphi$$
    for every $\mathcal{L}$-sentence $\varphi$ where $M'$ and $N'$ are NOT isomorphic. This leads us to the give the following definition:\\
    We say that an $\mathcal{L}$-structures $M$ is \emph{elementarily equivalent} to an $\mathcal{L}$-structure $N$, denoted $M\equiv_e N$ if 
    $$ M\vDash \varphi \implies_{\mathrm{meta}} N \vDash \varphi$$
    It turns out that with this definition we also have that $N\equiv _e M$ when $M\equiv_e N$. Applying the fact about universal closure it sufficient to check $M\vDash \varphi \iff_{\mathrm{meta}} N\vDash \varphi$ for every $\mathcal{L}$-sentence $\varphi$.    
\end{remark}
\subsubsection{Soundness and the Soundness Theorem}
\begin{definition}
    A logical calculus $\verb|L|$ is sound if every statement that on syntactical side can be proven is true on the semantical side
\end{definition}
\begin{theorem}
    Let $\Phi$ be a set of $\mathcal{L}$-formulae and $M$ a model of $\Phi$. Then for every $\mathcal{L}$-formula $\varphi$,
    $$\Phi \vdash \varphi \implies_{\mathrm{meta}}\ M \vDash \varphi.$$
\end{theorem}
\begin{proofsketch}
    We first prove that given any model $M$, we have that $M\vDash \theta_i$ for $i\in \{0,\dots,16\}$, where $\theta_i$ is an instance of $\mathrm{L}_i$, which are the logical axioms of $\verb|FOL|$. For $\mathrm{L}_0$ this follows from the assumption that we have classical logic in the metatheory.  Given the logical symbols $\vee,\wedge, \neg, \to$, and formulae $\varphi,\psi$ to for instance show that $M\vDash \varphi \vee \psi$ we have to show that $M\vDash \varphi$ or $M\vDash \psi$. Setting $\Theta:\equiv_{\mathrm{meta}} M\vDash \varphi \vee \psi$, $\Phi :\equiv_{\mathrm{meta}} M\vDash \varphi$ and $\Psi :\equiv_{\mathrm{meta}} M\vDash \psi$, we note that the truth value of $\Theta$ is dependent on the truth values of $\Phi$ and $\Psi$, by definition, via a truth table. We can for $\mathrm{L}_i$, $i=1,\dots,9$, set $\Theta_i :\equiv_{\mathrm{meta}} M\vDash \theta_i$ and see that the truth of this formula can be verified using such truth tables.\\
    We now consider the domain of $M$, $A$ and fix an assignment $j$, denote the associated interpretation $(M,j)$ by $I$.\\ 
    \textbf{"$\mathrm{L}_{10}$ is valid in $M$":} Fix an $\mathcal{L}$-formula $\varphi$, a term $\tau$ and a variable $x$ such that $\varphi(x/\tau)$ is admissible. We need to prove that assuming $I\vDash \forall x\varphi(x)$ that $I\vdash \varphi(\tau)$. Our assumption is equivalent to for every $a\in A$, $I[x\mapsto a]\vDash \varphi(x)$. Then $I[x\mapsto I(\tau)] \vDash \varphi(x)$, which is equivalent to {\Large explain why} $I \vdash  \varphi(\tau)$. So $(M,j)\vDash \forall x\varphi(x) \to \varphi(\tau)$ for every assignment $j$, hence $M\vDash \forall x\varphi(x)\to \varphi(\tau)$.\\
    \textbf{"$\mathrm{L}_{11}$ is valid in $M$":} Fix $\varphi,x,\tau$ as before. We need to show assuming $I \vDash \varphi(\tau)$ that there is an $a\in A $ such that $I[x\mapsto a]\vDash \varphi(x)$. Using the same fact as before, our assumption is equivalent to $I[x\mapsto I(\tau)] \vDash \varphi(x)$. Then $(M,j) \vDash \varphi(\tau)\to \exists x\varphi(x)$ for every interpretation $j$, hence $M\vDash \varphi(\tau)\to \exists x\varphi(x)$.\\
    \textbf{"$\mathrm{L}_{12}$ is valid in $M$":} Fix a non-free variable $x$ of a formula $\psi$ and let $\varphi(x)$ be an arbitrary formula. Suppose that forall $a\in A$, if $I[x\mapsto a]\vDash \psi$, then $I[x\mapsto a]\vDash \varphi(x)$. Note that by {\large insert fact later} $I[x\mapsto a]\vDash \psi$ is equivalent to $I\vDash \psi$ and that by definition $I[x\mapsto a]\vDash \varphi(x)$ is equivalent to $I\vDash \forall x\varphi(x)$. Assume then $I\vDash \psi$. From what have just noted it follows then that $I\vDash \forall x \varphi(x)$. So we conclude that $I \vDash \forall x(\psi \to \varphi(x))\to (\psi \to \forall x \varphi(x))$, and since $j$ was arbitrary, it follows that $M\vDash  \forall x(\psi \to \varphi(x))\to (\psi \to \forall x \varphi(x))$.\\
    \textbf{"$\mathrm{L}_{13}$ is valid in $M$":} Assume that for every $a\in A$ if $I[x\mapsto a] \vDash\varphi(x)$, then $I[x\mapsto a]\vDash \psi$. Suppose there is a $b\in A$ such that $I[x\mapsto b]\vDash \varphi(x)$. Then by assumption $I[x\mapsto b]\vDash \psi$ which using the same fact as before is equivalent to $I\vDash \psi$. We thus conclude that $I \vDash \forall x(\varphi(x)\to \psi) \to (\exists x \varphi x \to \psi)$, and since $j$ was arbitrarily chosen, $M\vDash\forall x(\varphi(x)\to \psi) \to (\exists x \varphi x \to \psi)$.\\
    \textbf{"$\mathrm{L}_{14}$ is valid in $M$":}
    Let $\tau$ be a term. Clearly $I(\tau)$ is the same element as $I(\tau)$, $I\vDash \tau = \tau$ and it immediately follows that $M\vDash \tau = \tau$.\\
    \textbf{"$\mathrm{L}_{14}$ is valid in $M$":} Let $R$ be an $n$-ary relation symbol in $\mathcal{L}$ and $\tau_1,\dots,\tau_n,\tau_1',\dots,\tau_n'$ be $\mathcal{L}$-terms. Suppose $I(\tau_i)$ is the same object as $I(\tau_i')$ for each $i$. Suppose moreover that $I\vDash R(\tau_1,\dots,\tau_i)$. Then $(I(\tau_1'),\dots, I(\tau_n'))=(I(\tau_1),\dots,I(\tau_n))\in R^M\subset A$, hence by definition $I \vDash R(\tau_1',\dots,\tau_n')$. This shows that 
    $$I \vDash \tau_1 = \tau_1'\wedge \dots \wedge \tau_n=\tau_n' \to (R(\tau_1,\dots,\tau_n)\to R(\tau_1',\dots,\tau_n'))$$
    which means 
    $$M\vDash \tau_1 = \tau_1'\wedge \dots \wedge \tau_n=\tau_n' \to (R(\tau_1,\dots,\tau_n)\to R(\tau_1',\dots,\tau_n')).$$
    \textbf{"$\mathrm{L}_{15}$ is valid in $M$":} Let $F$ be an $n$-ary function symbol. Let $\tau_1,\dots,\tau_n,\tau_1',\dots,\tau_n'$ be given as before. Suppose $I(\tau_i)$ is the same object as $I(\tau_i')$ for each $i$. Then 
    $$F^M(I(\tau_1),\dots,I(\tau_n)) = F^{M}(I(\tau_1'),\dots,I(\tau_n'))$$
    hence 
    $$I\vDash F(\tau_1,\dots,\tau_n) = F(\tau_1',\dots,\tau_n') \implies_{\mathrm{meta}} M \vDash F(\tau_1,\dots,\tau_n) = F(\tau_1',\dots,\tau_n').$$
    Now, let $\Phi$ be a set of $\mathcal{L}$-formulae and $M$ a model of $\Phi$. Let $\varphi$ be an $\mathcal{L}$-formula. Suppose $\Phi\vdash \varphi$. Note now that every instance of modus ponens is obviously valid in $M$ and that by {\large insert fact} every instance of generalization is valid in $M$. It thus follows that every application of these rules to conclude that $\Phi\vdash \varphi$ is valid in $M$. Since we know that the logical axioms of $\verb|FOL|$ are valid in $M$, it follows that if the formulae $\varphi_1,\dots,\varphi_n$ that make up a formal proof of $\varphi$ from $\Phi$, are all valid. In particular, since $\varphi_n \equiv \varphi$, it follows that $M\vDash \varphi$, which finishes the proof (sketch) 
\end{proofsketch}
\begin{corollary}\label{ConsequencesOfSoundnessTheorem}
    \begin{enumerate}
        \item For every tautology $\varphi$ we get that $M\vDash \varphi$ for any model $M$.
        \item For a set of $\mathcal{L}$-formulae $\Phi$, if there is a model $M$ with $M\vDash \Phi$, then $\mathrm{Con}(\Phi)$.
        \item Every instance of a logical axiom is consistent.
        \item For any sentence $\sigma$, model $M$ and set of formulae $\Phi$,
        $$\text{if } (M\not\vDash \sigma \text{ and } M\vDash \Phi) \text{ then } \Phi\not\vdash \sigma$$ 
    \end{enumerate}
\end{corollary}
\begin{proof}
    1. Indeed, $M$ is a model of the empty set (of formulae), hence by an application of the above theorem the result follows.\\
    2. Indeed, given a formula $\varphi\in\Phi$, 
    $$M\not\vDash \varphi \wedge \neg\varphi,$$
    means $\Phi \not\vdash \varphi\wedge \neg\varphi$ by the soundness theorem {\large note to self: write something about contraposition in metatheory}.\\
    3. Any model of a set of formulae is a model of any instance of logical axiom, hence by 2. such an instance is consistent.\\
    4. This is just a special case of the contrapositive of the soundness theorem. 
\end{proof}
\subsection{Gödels Completeness Theorem}
We now refine the definition of an $\mathcal{L}$-theory. A set $T$ of $\mathcal{L}$-sentences is said to be an \emph{$\mathcal{L}$-theory}. In mathematics a set of axioms will always consist of sentences. Note that as far as semantics is concerned we may replace every formula in a set of $\mathcal{L}$-formulae by the set of the universal closures of these formulae and nothing will change. 
\begin{definition}
    An $\mathcal{L}$-theory $T$ is \emph{(syntactically) complete} if for every $\mathcal{L}$-sentence $\varphi$, either $T \vdash \varphi$ or $T \vdash \neg\varphi$. An $\mathcal{L}$-theory that is not complete is called \emph{incomplete}
\end{definition}
\begin{remark}
    Note that an inconsistent theory is incomplete. 
\end{remark}
\begin{definition}
    For an $\mathcal{L}$-theory $T$, let $\mathrm{Th}(T)$ denote the set of $\mathcal{L}$-sentences $\sigma$ with $T\vdash \sigma$.
\end{definition}
\begin{remark}
    Note that a consistent theory $T$ is complete if and only if either $\sigma\in \mathrm{Th}(T)$ or $\neg \sigma \in\mathrm{Th}(T)$ for every $\mathcal{L}$-sentence $\sigma$.  
\end{remark}
\begin{theorem}
    Let $T$ be an $\mathcal{L}$-theory such that there is a model $M$ of $T$. Then there is a complete $\mathcal{L}$-theory $\overline{T}$ with $\overline{T}\supset T$. 
\end{theorem}
\begin{proof}
    Define $\overline{T}$ to be the sentences, $\sigma$, which satisfy $M\vDash \sigma$. Since for any sentence $\tau$, either $M\vDash \tau$ or $M\vDash \neg\tau$, we get that either $\tau\in \overline{T}$ or $\neg \tau\in \overline{T}$. Let $\sigma$ be any $\mathcal{L}$-sentence. If $\sigma\in \overline{T}$, then since $\sigma\vdash \sigma$, it follows that $\overline{T}\vdash \sigma$. Similarly, if $\neg\sigma\in \overline{T}$, then $\overline{T} \vdash \neg\sigma$. We thus conclude that $\overline{T}$ is complete. Since $M\vDash T$, it follows that $T\subset \overline{T}$.
\end{proof}
\begin{definition}
    Let $M$ be a model of some theory $T$. Then $\mathrm{Th}(M)=\overline{T}$ is called the \emph{model of $M$} 
\end{definition}
\begin{remark}
    Note that $\mathrm{Th}(M)$ is complete.
\end{remark}

\subsubsection{Maximally Consistent Theories}
\begin{definition}
    An $\mathcal{L}$-theory $T$ is \emph{maximally consistent} if for $\mathrm{Con}(T)$ and for every $\mathcal{L}$-sentence $\sigma\in T$ or $\neg\mathrm{Con}(T+\sigma)$. 
\end{definition}
\begin{remark}\label{ConsistentTheoryIsInconsistentUponBeingExtendedByOneFormulaIffItCanProveSaidTheory}
    From the last two facts gathered about consistency in the section on consistency and compactness we get that for a consistent theory $T$,
    $$\neg\mathrm{Con}(T+\sigma)\metaiff T\vdash \neg \sigma,$$
    so a consistent $\mathcal{L}$-theory $T$ is maximally consistent if and only if $\sigma\in T$ or $T\vdash \neg\sigma$. 
\end{remark}
\begin{lemma}
    For a consistent $\mathcal{L}$-theory $T$, $T$ is maximally consistent if and only if $\sigma \in T$ or $\neg\sigma\in T$ for every $\mathcal{L}$-formula $\sigma$.
\end{lemma}
\begin{proof}
    Suppose first that $T$ is maximally consistent. Let $\sigma$ be an $\mathcal{L}$-sentence. By assumption $\sigma\in T$ or $T\vdash \neg \sigma$. Suppose $\sigma\notin T$. Then $T\vdash \neg \sigma$, and since $\mathrm{Con}(T)$, $T\not\vdash \sigma$. This means $T\not\vdash \neg\neg\sigma$, hence $\neg\sigma\in T$.\\
    Assume now $\sigma\in T$ or $\neg\sigma\in T$ for every $\mathcal{L}$-sentence $\sigma$. Let $\sigma$ be an arbitrary $\mathcal{L}$-sentence. Suppose $\sigma\notin T$. Then $\neg \sigma\in T$. Hence, $\neg \sigma$ is by definition a formal proof of $T\vdash \neg\sigma$.  
\end{proof}
\begin{lemma}
    Let $T$ be a consistent $\mathcal{L}$-theory. Then 
    \begin{enumerate}
        \item If $T$ is complete, then $\mathrm{Th}(T)$ is maximally consistent.
        \item If $T$ is maximally consistent, then $T=\mathrm{Th}(T)$. 
    \end{enumerate}
\end{lemma}
\begin{proof}
    1. Since $T$ is complete, it is consistent. Let $\sigma$ be an $\mathcal{L}$-sentence. Since $T$ is completeness we also get $T\vdash \sigma$ or $T\vdash \neg\sigma$ which is equivalent to $\sigma\in \mathrm{Th}(T)$ or $\neg \sigma\in \mathrm{Th}(T)$ which by the prior lemma means that $\mathrm{Th}(T)$ is maximally consistent.\\
    2. Trivially $T\subset \mathrm{Th}(T)$. Let $\sigma \in \mathrm{Th}(T)$. Then $T\vdash \sigma$. By assumption $\sigma\in T$ or $\neg\sigma\in T$.\\ 
    For any there $S$ where $\tau\in S$ or $\neg\tau\in S$ for a  sentence $\tau\in \mathrm{Th}(S)$, if $\neg\tau\in S$, then $\neg\mathrm{Con}(T)$. So we see that $\neg\sigma\notin T$ which implies $\sigma \in T$. We thus conclude that $T=\mathrm{Th}(T)$. 
\end{proof}
\begin{lemma}
    If an $\mathcal{L}$-theory has a model $M$, then $\mathrm{Th}(M)$ is a maximally consistent extension of $T$. 
\end{lemma}
\begin{proof}
    Let $\sigma\in T$. Then $T\vdash \sigma$, hence by soundness, $M\vDash \sigma$ which means $\sigma\in \mathrm{Th}(M)$. $\mathrm{Th}(M)=\overline{T}$ is readily seen to be maximally consistent. It is consistent since a theory with a model is automatically consistent (cf. Corollary~\ref{ConsequencesOfSoundnessTheorem} 2.) and for any $\mathcal{L}$-sentence $\sigma$, either $M\vDash \sigma$ or $M\vDash \neg\sigma$ which by definition is equivalent to either $\sigma\in \mathrm{Th}(M)$ or $\neg\sigma \in \mathrm{Th}(M)$
\end{proof}
\subsubsection{Universal List of Sentences}
For this section and the one's on Lindenbaum's Lemma and Gödel's Completeness Theorem any signature $\mathcal{L}$ is countable. I.e. the symbols of $\mathcal{L}$ can be arranged in a list that is at most potentially infinite list $L_\mathcal{L}$. We now aim to encode every $\mathcal{L}$-sentence in an ordered list called the \emph{universal list of $\mathcal{L}$}. Let $\xi_i$ be the symbol at index $i$ in $L_\mathcal{L}$. We encode this as
$$\#\xi_i := \underbrace{22\dots2}_{i\ 2\text{'s}},$$
i.e. we set $\#\xi_0:=2$ and recursively define $\#\xi_{i+1} := \#\xi_i 2$. For logical symbols we also define an encoding using $1$'s. First we define encodings for the logical connectives
\begin{table}[h]
\center
\begin{tabular}{lllll}
Symbol $\xi$    & Encoding $\#\xi$         &  &  &  \\
$=$       & $11$             &  &  &  \\
$\neg$    & $1111$           &  &  &  \\
$\wedge$  & $111111$         &  &  &  \\
$\vee$    & $11111111$       &  &  &  \\
$\to$     & $1111111111$     &  &  &  \\
$\exists$ & $111111111111$   &  &  &  \\
$\forall$ & $11111111111111$ &  &  & 
\end{tabular}
\end{table}
We can order the variables in a potentially infinite list. For $x_0$ we define $\#x_0:= 1$ for the $i$'th variable $x_i$ we define $\#x_{i+1}:= \#x_i11$. Now given any string of symbols in the language associated with $\mathcal{L}$, $\xi := \xi_0\xi_1\dots\xi_n$ we define 
$$\#\xi:= \#\xi_00\#\xi_10\dots 0\#\xi_n,$$
i.e. we recursively define the encoding on strings through recursion in the length of the string. Now we define an order on strings $\#\xi$, where $\#\xi<\#\xi'$ if $\vert \#\xi\vert < \vert \#\xi'\vert$ ($\vert\bullet\vert$ is the length of a string) and when $\vert \#\xi\vert =\vert \#\xi'\vert$ if $\#\xi <_{\mathrm{lex}} \#\xi'$, where $<_{\mathrm{lex}}$ is the lexicographic ordering with $0<1<2$. We assume WLOG that every relational or functional symbol uses Polish notation. We now obtain the desired universal list of $\mathcal{L}$-sentences, which is the potentially infinite list 
$$\Lambda_\mathcal{L}:=[\sigma_0,\sigma_1,\dots]$$
of $\mathcal{L}$-sentences where 
$$\#\sigma_i <\#\sigma_j$$
for $i<j$. To see that is potentially infinite, note that there is a finite time algorithm for generating the next element in the list of possible strings,
\begin{enumerate}
    \item Let $\#\sigma_n$ be some element generated in the list. Set $l:= \vert\#\sigma_n\vert$. There are only finitely many possible strings of the form $\#\sigma$ of length $l$ (since there are only finitely many strings consisting of $0$'s, $1$'s and $2$'s) where $\sigma$ is a sentence. 
    \item If $\#\sigma_n$ is the largest of these strings compute the sentence $\sigma$ with the smallest $\#\sigma$ of length $l+1$ otherwise compute the $\sigma$ with the smallest $\#\sigma>\#\sigma_n$.  
\end{enumerate}
\subsubsection{Lindenbaum's Lemma}
\begin{theorem}(Lindenbaum's Lemma) Let $\mathcal{L}$ be a countable signature and $T$ a consistent $\mathcal{L}$-theory. If there is a sentence $\sigma_0$ with $T\not\vdash \sigma_0$. Then there is a maximally consistent theory $T^\ast_{\sigma_0}\supset T+\neg\sigma_0$. 
\end{theorem}
\begin{proof}
    Consider the universal list of $\mathcal{L}$-sentences
    $$\Lambda_\mathcal{L} = [\sigma_1,\sigma_2,\dots].$$    
    We place $\neg \sigma_0$ at the beginning of this list to obtain a list
    $$\Lambda_\mathcal{L}' := [\neg\sigma_0,\sigma_1,\sigma_2,\dots].$$
    Set $T_0 := [\neg\sigma_0]$ and recursively define 
    $$
        T_{i+1} := 
            \begin{cases}
                T_n + [\sigma_{n+1}] & \text{if } \mathrm{Con}(T+T_n+\sigma_{n+1})\\
                T_n & \text{otherwise}
            \end{cases}
    $$
    As a result of this process we obtain a potentially infinite list 
    $$T_{\sigma_0}^\ast := [\neg\sigma_0, \sigma_{i_1},\sigma_{i_2},\dots],$$
    Where we assume $\mathrm{Con}(T+T_n+\sigma_{n+1})$ or $\neg\mathrm{Con}(T+T_n+\sigma_{n+1})$. For this we could use law of excluded middle, which we already use in model theory. A weaker assumption that one can take is the weak König's Lemma whose statement is 
    $$\text{An infinite 0-1-tree has an infinite branch}$$
    where a 0-1-tree is a certain type of subtree of a binary tree.\\
    We now prove that $T^\ast_{\sigma_0}$ is a maximally consistent theory containing $T+\neg\sigma_0$. Clearly $\neg\sigma_0$ is in $T^\ast_{\sigma_0}$. Note that for every sentence $\sigma$ there is an $n$ such that $\sigma_n\equiv \sigma$. Hence it is sufficient to prove for each $n$ if $\sigma_n\in T$, then $\sigma_n\in T^\ast_{\sigma_0}$. We prove this by induction in $n$. For $n=1$, note that if $\sigma_1\in T$, then $\mathrm{Con}(T+\neg\sigma_0)$. Indeed, since $T\not\vdash \sigma_0$, we have that $\mathrm{Con}(T+\neg \sigma_0)$ by Remark~\ref{ConsistentTheoryIsInconsistentUponBeingExtendedByOneFormulaIffItCanProveSaidTheory}. It thus follows that $\mathrm{Con}(T+\sigma_1)$, hence $T_1=T_0+\sigma = [\neg\sigma_0,\sigma_1]$, hence $\sigma_1\in T^\ast_{\sigma_0}$. Suppose $\sigma_m\in T$ implies $\sigma\in T^\ast_{\sigma_0}$ for every $m\leq n$. Suppose $\sigma_{n+1}\in T$. Pick $N$ to be the largest $N\leq n$ such that $\sigma_M\in T^\ast_{\sigma_0}$. Then $T_N=T_n$. Suppose for a contradiction that $\neg\mathrm{Con}(T+T_n+\sigma_{n+1})$. Then since $\sigma_{n+1}\in T$, $\neg\mathrm{Con}(T+T_n)$, which means $\neg\mathrm{Con}(T+T_N)$, but then $\sigma_N\notin T^\ast_{\sigma_0}$. Then $\mathrm{Con}(T+T_n+\sigma_{n+1})$, hence $T_{n+1}=T_n+\sigma_{n+1}\ni \sigma_{n+1}$. It thus follows that $\sigma_{n+1}\in T^\ast_{\sigma_0}$.\\
    To show $\mathrm{Con}(T^\ast_{\sigma_0})$, by the Compactness Theorem it is sufficient to prove that every finite subset of $T_{\sigma_0}^\ast$ is consistent. Any such finite subset is contained in $T_n$ for some $n$. So it is sufficient to prove $\mathrm{Con}(T_n)$. Note that since $T\not\vdash \sigma_0$, we get $\mathrm{Con}(T+\neg\sigma_0)$, hence using the Compactness Theorem, $T_0=\{\neg\sigma_0\}$ is consistent. Suppose $\mathrm{Con}(T_m)$ for each $m\leq n$ for some $n$. If $\neg\mathrm{Con}(T+T_{n}+\sigma_{n+1})$, then $T_n=T_{n+1}$ and we are done. Suppose then that $\mathrm{Con}(T+T_n+\sigma_{n+1})$. Then since $T_{n+1}=T_n+\sigma_{n+1}$ is a finite subset $T+T_n+\sigma_{n+1}$, we get $\mathrm{Con}(T_{n+1})$.\\
    To conclude that $T^\ast_{\sigma_0}$ is maximally consistent, we prove that $\sigma\in T^\ast_{\sigma_0}$ or $\neg\mathrm{Con}(T^\ast_{\sigma_0}+\sigma)$. Let an $\mathcal{L}$-sentence be given. Such a sentence is $\sigma_n$ for some $n\geq 1$. Suppose $\mathrm{Con}(T+T_{n-1}+\sigma_n)$, then $\sigma_n\in T_{n-1}+\sigma_n=T_n$. Suppose $\neg\mathrm{Con}(T+T_{n-1}+\sigma_n)$. By Lemma~\ref{TheoryWithInconsistentSubTheoryIsInconsistent}, $\neg\mathrm{Con}(T^\ast_{\sigma_0}+\sigma_n)$. 
\end{proof}
\begin{remark}\label{RemarkAboutMaximallyConsistentExtensions}
    \begin{enumerate}
        \item Note that if $T \vdash \sigma$, then $\neg\mathrm{Con}(T+\neg\sigma)$. Note that if for a consistent theory $\Phi$ and a formula $\varphi\in \Phi$, then $\mathrm{Con}(\Phi+\varphi)$. Then since $T^\ast_{\sigma_0}$ is maximally consistent, we first have that $\neg \sigma\notin T^\ast_{\sigma_0}$ and then that $\sigma \in T^\ast_{\sigma_0}$.
        \item $T^\ast_{\sigma_0}\vdash \sigma \metaiff \sigma\in T^\ast_{\sigma_0}$:\\ 
        "$\implies$": Using maximal consistency, if $T^\ast_{\sigma_0}\vdash \sigma$, then $\mathrm{Con}(T^\ast_{\sigma_0}+\sigma)$. Note that $\sigma\equiv \sigma_n$. Then every finite subtheory of $T^\ast_{\sigma_0}+\sigma$ is consistent, hence any finite subtheory of $T+T_{n-1}+\sigma_n$ is consistent, which means $\mathrm{Con}(T+T_{n-1}+\sigma_n)$. By construction $\sigma\equiv \sigma_n\in T_n\subset T^\ast_{\sigma_0}$.\\
        "$\impliedby$":  This is obvious. 
        \item Note also $\sigma\iff \sigma'$ then $\sigma\in T^\ast_{\sigma_0} \metaiff \sigma'\in T^\ast_{\sigma_0}$. 
    \end{enumerate}
\end{remark}
One readily verifies the following lemma
\begin{lemma}\label{SentencesInMaximalExtensionBehaveLikeValidSentencesInAModel}
    Let $T$ be an $\mathcal{L}$-theory where $\mathcal{L}$ is countable and $\sigma_0$ be given as in Lindenbaum's Lemma. For any $\mathcal{L}$-sentences $\sigma,\sigma_1,\sigma_2$,
    \begin{enumerate}
        \item $\neg \sigma\in T^\ast_{\sigma_0}\metaiff \text{not } \sigma\in T^\ast_{\sigma_0}$.
        \item $\wedge \sigma_1\sigma_2\in T^\ast_{\sigma_0} \metaiff \sigma_1\in T^\ast_{\sigma_0}\text{ and }\sigma_2\in T^\ast_{\sigma_0}$
        \item $\vee\sigma_1\sigma_2 \in T^\ast_{\sigma_0} \metaiff \sigma_1\in T^\ast_{\sigma_0} \text{ or } \sigma_2\in T^\ast_{\sigma_0}$.
        \item $\to\sigma_1\sigma_2\in T^\ast_{\sigma_0} \metaiff \sigma_1\in T^\ast_{\sigma_0} \text{ then } \sigma_2\in T^\ast_{\sigma_0}$.
    \end{enumerate}
\end{lemma}
\subsubsection{An Extension of a Signature \& of a Theory}
\begin{definition}
    Given a countable $\mathcal{L}$-signature a \emph{term constant} is the term that results from applying one of these three rules recursively finitely many times.
    \begin{align*}
    \mathrm{C0}\colon\quad & \text{A variable-free } \mathcal{L}\text{-term is a term constant}.\\
    \mathrm{C1}\colon\quad & \text{if } \tau_0,\dots,\tau_{n-1} \text{ are term constants and } F \text{ is an } n \text{-ary function, then}\\ 
    & F\tau_1\cdots\tau_{n-1} \text{ is a term constant}.\\
    \mathrm{C2}\colon\quad & \text{If } i,n \text{ are natural numbers and }\tau_0,\dots,\tau_{n-1} \text{ are term constants, then }\\ 
    &(i,\tau_0,\dots,\tau_{n-1},n) \text{ is a term constant.} 
\end{align*}
The strings of the form $(i,\tau_0,\dots,\tau_{n-1},n)$ are called \emph{special constants}. In particular $(i,0)$ is a special constant. 
\end{definition}
\begin{remark}
    Note that term constants build with only $\mathrm{C0}$ and $\mathrm{C1}$ are in the language induced by $\mathcal{L}$ and are variable free, while those build with $\mathrm{C2}$ may not be.
\end{remark}
\begin{definition}
    Let a countable signature $\mathcal{L}$ be given. Then we define $\mathcal{L}_c$ to be $\mathcal{L}$ together with the special constants (of which there are countably many). 
\end{definition}
We now encode variable terms using the same encoding as we did to construct the universal list of sentences. Moreover, $\#( := 3$, $\#0$ the empty string, $\#(i+1):=\#i1$, $\#,:=4$ and $\#):= 5$. So given a special constant $c \equiv (i,\tau_0,\dots,\tau_{n-1},n)$, we define an ecoding
$$\#c := \#(\#i\#,\#\tau_0\#,\dots \#,\#\tau_{n-1}\#,\#n\#)$$
As before we order strings build from $\mathcal{L}_c$ first by their length and lexicographically with $0<1<2<3<4<5$. We define potentially infinite ordered lists 
$$
    \Lambda_\tau := [\tau_0,\tau_1,\dots] \text{ and } \Lambda_c := [c_0,c_1,\dots]
$$  
of terms constants and special constants respectively.
\begin{definition}
    Let $\sigma_i\in T^\ast_{\sigma_0}$ and $c_j\equiv (i,\tau_0,\dots,\tau_{n-1},n)$ a special constant. We say that \emph{$c_j$ is a witness of $\sigma_i$} if 
    \begin{enumerate}
        \item $i\geq 1$ and $\sigma_i$ is in special Prenex Normal Form.
        \item $\exists x_n$ appears in $\sigma_i$.
        \item For each $m<n$: If $\exists x_m$ appears in $\sigma_i$ then $\tau_m \equiv (i,\tau_0,\dots,\tau_{m-1},m)$.  
    \end{enumerate}
\end{definition} 
We replace $\neg\sigma_0$ by its equivalent formula $\sigma_k\in T^\ast_{\sigma_0}$, which is in sPNF.\\
We now construct $T^\ast_{\sigma_0}(c)$. Suppose we have a sentence $\sigma_i\in T^\ast_{\sigma_0}$ that is in sPNF and $\forall x_n$ or $\exists x_n$ appear in $\sigma_i$, then it is on the form 
$$\quantifs_0 x_0\quantifs_1x_1\dots \quantifs x_n\dots \quantifs x_m \sigma_i(x_0,x_1,\dots,x_n,\dots,x_m)'$$
where $\sigma_i$ is quantifier free and in which $x_1,\dots,x_m$ appear free, hence 
$$\sigma_i \equiv \quantifs_0 x_0\quantifs_1 x_1\dots \quantifs x_n \sigma_{i,n}(x_0,\dots,x_n)$$ 
where $x_1,\dots, x_n$ appear free in $\sigma_{i,n}(x_0,\dots,x_n)$. Then if $c_j \equiv (i,\tau_0,\dots,\tau_{n-1},n)$ witnesses $\sigma_i$, 
$$\sigma_i \equiv \quantifs_0 x_0\quantifs_1 x_1\dots \exists x_n \sigma_{i,n}(x_0,\dots,x_n).$$
Set 
$$\sigma_{i,n}[c_j]:\equiv \sigma_{i,n}(x_0/\tau_0,\dots,x_{n-1}/\tau_{n-1},x_n/c_j).$$
Set $T^\ast_0 := T^\ast_{\sigma_0}$. Suppose $T_j^\ast$ is already defined and $c_j \equiv (i,\tau_0,\dots,\tau_{n-1},n)$ is a special constant Then 
$$
    T^\ast_{j+1}:= \begin{cases}
        T_j^\ast & \text{if } c_j \text{ does not witness } \sigma\in T^\ast_{\sigma_0}\\
        \text{Insert } \sigma_{i,n}[c_j] \text{ into } T_j^\ast\\ \text{ such that the resulting list is ordered } & \text{otherwise} \\ \text{  with respect to the } \#\text{-encodings}  
    \end{cases}
$$
\begin{definition}
    We set $T^\ast_{\sigma_0}(c) := \bigcup T_j^\ast$
\end{definition}
\begin{definition}
We define the \emph{height} of term constants, denoted $\mathrm{height}(\bullet)$ by cases in the following way: 
\begin{enumerate}
    \item If $\tau$ is a closed term, then $\mathrm{height}(\tau):= 0$.
    \item If $\tau_0,\dots,\tau_{n-1}$ are term constants and $F\in\mathcal{L}$ is an $n$-ary function symbol, then 
    $$
        \mathrm{height}(F(\tau_0,\dots,\tau_{n-1})) := \max(\mathrm{height}(\tau_0),\dots,\mathrm{height}(\tau_{n-1})). 
    $$
    \item If $\tau\equiv (n,\tau_0,\dots,\tau_{n-1},i)$ is a special constant, define 
    $$
        \mathrm{height}(\tau):= 1 + \max(\mathrm{height}(\tau_0),\dots,\mathrm{height}(\tau_{n-1})).
    $$
\end{enumerate}
\end{definition}
\begin{lemma}
    The $\mathcal{L}_c$-theory $T^\ast_{\sigma_0}(c)$ is consistent.
\end{lemma}
\begin{proof}
    \textbf{Step 1: $T^\ast_{\sigma_0}$ is consistent with respect to $\mathcal{L}_c$:}\\
    Suppose for a contradiction that in $\mathcal{L}_c$
    $$
        T^\ast_{\sigma_0} \vdash \mathrm{False}.
    $$
    Fix a formal $\mathcal{L}_c$-proof $\sigma_1,\dots,\sigma_n$ of $\mathrm{False}$ from $T^\ast_{\sigma_0}$  
    for each special constant $c$ fix a unique variable $x_c$ that does not appear in any $\varphi_i$. Note that given an instance of a logical axiom $\mathrm{L}_i$ with respect to $\mathcal{L}_c$, if we replace each special constant with its $x_c$ we get another instance of $\mathrm{L}_i$. The same is the case for instances of modus ponens and generalization, since special constants never appear after quantifiers. This means that we obtain a formal proof of $\mathcal{L}$-sentences $\sigma_1',\dots,\sigma_n'$ of $\mathrm{False}$ from $T_{\sigma_0}^\ast$ by replacing special constants by variables in the way we just described. But then $\neg\mathrm{Con}(T^\ast_{\sigma_0})$ with respect to $\mathcal{L}$, which contradicts the fact that with respect to the signature $\mathcal{L}$ the extension $T_{\sigma_0}^\ast$ is maximally consistent and then in particular consistent.\\
    \textbf{Step 2: $T^\ast_{\sigma_0}(c)$ is consistent with respect to $\mathcal{L}_c$:}\\
    Suppose for a contradiction that $T_{\sigma_0}^\ast(c)$ is inconsistent. Using the compactness theorem there is a finite list of $\mathcal{L}_c$-sentences in $T^\ast_{\sigma_0}(c)$ that are inconsistent. In particular we get 
    $$
        \neg\mathrm{Con}(T^\ast_{\sigma_0}+\{\sigma_{i_1,n_1}[c_{j_1}],\dots,\sigma_{i_m,n_m}[c_{j_m}]\})
    $$
    for suitable distinct $\sigma_{i_1,n_1}[c_{j_1}],\dots,\sigma_{i_m,n_m}[c_{j_m}]$. Choose these sentences such that $n_1+\dots+n_m+m$ is minimal. WLOG 
    $$\mathrm{height}(c_{j_m}) \geq \mathrm{height}(c_{j_k})$$
    for each $k$. We thus have that $c_{j_m}$ does not appear in $c_{j_k}$ for $k<m$ for otherwise $c_{j_k}<c_{j_m}$ {\Large elaborate at some point}. Set
    $$
        \Sigma := \{\sigma_{i_1,n_1}[c_{j_1}],\dots,\sigma_{i_{m-1},n_{m-1}}[c_{j_{m-1}}]\}.
    $$
    Write 
    $$
        c_{j_m} \equiv (i_m,\tau_0,\dots,\tau_{n-1},n_m),
    $$  
    hence $\sigma_{i_m,n_m}[c_{j_m}]\equiv \sigma_{i_m}(x_0/\tau_0,\dots,x_{n-1}/\tau_{n-1},x_n/c_{j_m})$. Note that $\exists x_n$ appears in $\sigma_{i_m}$ since $c_{j_m}$ witnesses $\sigma_{i_m}$, hence
    $$\sigma_{i_m,n_m-1}(x_0,\dots,x_{n-1})\equiv \exists x_n \sigma_{i_m,n_m}(x_0,\dots,x_n).$$
    Define
    $$
        \widetilde{\sigma}(x_n):=\sigma_{i_m,n_m}(x_0/\tau_0,\dots,x_{n-1}/\tau_{n-1},x_n)
    $$
    and see that only $x_n$ appears free in this formula.
    \textbf{Claim:}
    $$
        \neg\mathrm{Con}(T^\ast_{\sigma_0}+\Sigma + \sigma_{i_m,n_m}[c_{j_m}])\metaimplies \neg\mathrm{Con}(T^\ast_{\sigma_0}+\Sigma+\exists x_n\widetilde{\sigma}(x_n)).
    $$
    Suppose $T^\ast_{\sigma_0}+\Sigma + \sigma_{i_m,n_m}[c_{j_m}]$ is not consistent. Then we get 
    \begin{gather}\label{Deduction1}
        T^\ast_{\sigma_0}+\Sigma + \sigma_{i_m,n_m}[c_{j_m}]\vdash \mathrm{False},
    \end{gather}
    hence by the deduction theorem
    $$
        T^\ast_{\sigma_0}+\Sigma \vdash \sigma_{i_m,n_m}[c_{j_m}] \to \mathrm{False}.
    $$
    In this proof replace $c_{j_m}$ by $x$; a variable that does not occur in $\sigma_{i_m,n_m}$ or in any of the formulae comprising (\ref{Deduction1}). Note that an instance of a logical axiom or an inference rule is still an instance of the same axiom or inference rule after this replacement. Any sentence in $T^\ast_{\sigma_0}$ remains unchanged upon such a replacement since these contain no special constants. The same is the case for the sentences in $\Sigma$ since $c_{j_m}$ is distinct from any other $c_{j_k}$. This means we get 
    $$
         T^\ast_{\sigma_0}+\Sigma \vdash \widetilde{\sigma}(x)\to \mathrm{False}.
    $$
    With generalization we get 
    $$
        T^\ast_{\sigma_0}+\Sigma \vdash \forall x(\widetilde{\sigma}(x)\to \mathrm{False}).
    $$
    From $\mathrm{L}_{13}$ we get 
    $$
        T^\ast_{\sigma_0}+\Sigma \vdash \forall x(\widetilde{\sigma}(x)\to \mathrm{False}) \to (\exists x\widetilde{\sigma}(x)\to \mathrm{False}).
    $$
    By modus ponens 
    $$
        T^\ast_{\sigma_0}+\Sigma \vdash (\exists x\widetilde{\sigma}(x)\to \mathrm{False}).  
    $$
    Since $x_n$ does not appear in $\widetilde{\sigma}(x)\to\mathrm{False}$ it is a tautology (which will be used without proof) that 
    $$
        T^\ast_{\sigma_0}+\Sigma \vdash (\exists x_n\widetilde{\sigma}(x_{n_m})\to \mathrm{False}),  
    $$
    which means $\neg \mathrm{Con}( T^\ast_{\sigma_0}+\Sigma + \exists x_{n_m}\widetilde{\sigma}(x_{n_m}))$.
    Set $i:= i_{m}$. Let $p\leq n$ be the largest integer such that for every $1\leq l\leq p$, $\forall x_{n-l}$ appears in $\sigma_i$. Then $\sigma_i$ is on the form
    \begin{gather*}\label{ImportantLogicalFormula}
        \sigma_i \equiv \quantifs_0 x_0 \dots \quantifs_{n_m-p-2} x_{n_m-p-2} \exists x_{n_m-p-1} \forall x_{n_m-p}\dots \forall x_{n-1}\exists x_n \sigma_{i_m,n_m}(x_0,\dots,x_n)
    \end{gather*} 
    Define 
    $$
        \widetilde{\sigma}_p :\equiv \sigma_{i_m,n_m-p-1}(x_0/\tau_0,\dots,x_{n_m-p-1}/\tau_{n_m-p-1})
    $$
    for $p\leq n_m-1$. When $p=n_m$, $\quantifs_{n'}\equiv \forall$ for each $n'<n_m$. In this case we set $\widetilde{\sigma}_p:\equiv \sigma_i\in T^\ast_{\sigma_0}$. When $p<n_m$, $\exists x_{n_m-p-1}$ appears in $\sigma_{i}$.\\
    In the first case,
    $$
        \widetilde{\sigma}_p \equiv \forall x_0\dots\forall x_{n_m-1}\exists x_{n_m} \sigma_{i,n_m}(x_0,\dots,x_{n_m})
    $$ 
    then since $\tau_0,\dots, \tau_{n_m-1}$ are variable free, it follows that $T^\ast_{\sigma_0} \vdash \exists x_{n_m} \widetilde{\sigma}(x_{n_m})$ by repeated use of $\mathrm{L}_{10}$. The claim together with what we assumed towards contradiction shows that $\neg\mathrm{Con}(T^\ast_{\sigma_0}+\Sigma +\exists x_{n_m}\widetilde{\sigma})$ and then $\neg\mathrm{Con}(T^\ast_{\sigma}+\Sigma)$, i.e. since we can deduce $\exists x_{n_m}\widetilde{\sigma}(x_{n_m})$, we on only need to add $\Sigma$ to $T^\ast_{\sigma_0}$ to get an inconsistent theory. But this contradicts the minimality of $n_1+\dots+n_m+m$.\\
    In the second case, since $c_{j_m}$ witnesses $\sigma_i$ and $\exists x_{n_m-p-1}$ appears in $\sigma_i$, we get 
    $$\tau_{n_m-p-1} \equiv (i,\tau_0,\dots,\tau_{n_m-p-2},n_m-p-1)$$
    which also by definition witnesses $\sigma_i$. Note 
    \begin{align*}
        \sigma_{i,n_m-p-1}[\tau_{n_m-p-1}] &\equiv \sigma_{i,n_m-p-1}(x_0/\tau_0,\dots,x_{n_m-p-2}/\tau_{n_m-p-1},x_{n_m-p-1}/\tau_{n_m-p-1}).
    \end{align*}
    From $T^\ast_{\sigma_0}$ and the above formula, we get 
    $$\forall x_{n_m-p}\dots \forall x_{n_m-1}\exists x_{n_m} \sigma_{i,n_m-p-1}[\tau_{n_m-p-1}].$$
    Using $\mathrm{L}_{10}$ we thus get 
    $$T^\ast_{\sigma_0}+\sigma_{i,n_m-p-1}[\tau_{n_m-p-1}]\vdash \exists x_{n_m} \widetilde{\sigma}(x_{n_m}) \metaiff T^\ast_{\sigma_0} \vdash \sigma_{i,n_m-p-1}[\tau_{n_m-p-1}]\to \exists x_{n_m} \widetilde{\sigma}(x_{n_m}).$$
    So if we derive $\mathrm{False}$ from $T^\ast_{\sigma_0}+\Sigma+\exists$, then we also derive it from $T^\ast_{\sigma_0}+\Sigma+\sigma_{i,n_m-p-1}[\tau_{n_m-p-1}]$. But 
    $$n_0+\dots+n_{m-1}+n_m-p-1 +m< n_0+\dots+n_{m}+m,$$
    leading to a contradiction.\\
    We thus conclude that $T^\ast_{\sigma_0}(c)$ is a consistent $\mathcal{L}_c$-theory.  
\end{proof}
\subsubsection{Gödel's Completeness Theorem (for Countable Signatures)}
\begin{lemma}
    $T^\ast_{\sigma_0}(c)$ can be extended to a consistent list $\widetilde{T}$ of $\mathcal{L}_c$-sentences such that the additional sentences contain no quantifiers or free variables and such that for each variable free $\mathcal{L}_c$-sentence $\sigma$,
    $$\sigma\in \widetilde{T} \text{ or } \neg\sigma\in \widetilde{T}.$$
\end{lemma}
\begin{proof}
    We construct the universal list for variable free $\mathcal{L}_c$-sentences of $T^\ast_{\sigma_0}(c)$ and go through list, setting $\Theta_0 := T^\ast_{\sigma}(c)$ and recursively add such a $\sigma$ to $\Theta_{n+1}$ if $\mathrm{Con}(T^\ast_{\sigma_0}(c)+\Theta_n+\sigma)$. The same argument as in Lindenbaum's lemma shows that this theory, denoted $\widetilde{T}$, is consistent and has the property that 
    $$
        \sigma \in \widetilde{T} \vee \neg \sigma\in \widetilde{T}.
    $$
\end{proof}
\begin{remark}
    Note that $\widetilde{T}$ is a maximally consistent theory. Indeed if 
\end{remark}
We now define a $\mathcal{L}_c$-structure $M$ which we will show in the next is a model of $\widetilde{T}$ and therefor also of $T$ and indeed $T+\neg \sigma_0$.

\begin{lemma}\label{LemmaForCase1InHenkinsProof}
    Suppose 
    $$
        \sigma\equiv \forall x_k\forall \dots\forall x_{n-1}\exists x_n\sigma_{i,n}(x_0/\tau_0,\dots,x_{k-1}/\tau_{k-1},x_k,\dots,x_n)
    $$
    is in $T_{\sigma_0}^\ast(c)$. Let $\tau_k,\dots,\tau_{n-1}$ be arbitrary term constants. Then for 
    $$
        c :\equiv (i,\tau_0,\dots,\tau_{n-1},n),
    $$
    we have that $\sigma_i[c]\in T^\ast_{\sigma_0}(c)\subset \widetilde{T}$.
\end{lemma}
\begin{proof}
Note that the special constant $c$ will witness $\sigma_i$. Indeed, 
$$
    \sigma \equiv \sigma_{i,k}[(i,\tau_0,\dots,\tau_{k-1},k)],
$$
hence $ (i,\tau_0,\dots,\tau_{k-1},k)$ witnesses $\sigma_i$, and this together with the fact that $\exists x_m$ does not appear in $\sigma_i$ for $k\leq m\leq n-1$ shows that $(i,\tau_0,\dots, \tau_{n-1},n)$ witnesses $\sigma_i$, hence 
$$
    \sigma_{i,n}(x_0/\tau_0,\dots,x_{n-1}/\tau_{n-1},x_n/c)
$$
is in $T_{\sigma_0}^\ast(c)\subset \widetilde{T}$ by construction.
\end{proof}
\begin{lemma}\label{LemmaForCase2InHenkinsProof}
    Suppose 
    $$
        \sigma\equiv \forall x_k\forall \dots\forall x_n\sigma_{i,n}(x_0/\tau_0,\dots,x_{k-1}/\tau_{k-1},x_k,\dots,x_n)
    $$
    is in $T_{\sigma_0}^\ast(c)$ where $\sigma_{i,n}$ is quantifier free. Let $\tau_k,\dots,\tau_n$ be arbitrary term constants. Then 
    $$
        \sigma' :\equiv \sigma_{i,n}(x_0/\tau_0,\dots,x_n/\tau_n)
    $$
    is in $\widetilde{T}$. 
\end{lemma}
\begin{proof}
    By $\mathrm{L}_{10}$ we get $T_{\sigma_0}^\ast(c)\vdash \sigma'$, hence since $\widetilde{T}\vdash \sigma'$, which by maximal consistency of $\widetilde{T}$ shows that $\sigma'\in \widetilde{T}$.
\end{proof}
\begin{remark}\label{ReallyImportantFactAboutExistentialWitnesses}
    Consider a $\sigma_{i,n}[c_j]\in T^\ast_{\sigma_0}(c)$ with $c_j\equiv (i,\tau_0,\dots,\tau_{n-1},n)$. Then the next existential quantifier $\quantifs_m \equiv \exists$ with $m<n$, then $\tau_m\equiv (i,\tau_0,\dots,\tau_{m-1})$ witnesses $\sigma_i$, hence $\sigma_{i,m}[c_j]\in T^\ast_{\sigma_0}(c)$ and recall that 
    $$
        \sigma_{i,m} \equiv \quantifs_{m+1} x_{m+1}\dots \exists x_n \quantifs_{n+1} x_{n+1}\dots \quantifs_{N} x_{N} \sigma_{i,N}(x_0,\dots,x_N).
    $$  
    Then using $L_{10}$ a potential bunch of times as we did in a proof above we can derive
    $$\sigma_{i,N}(x_0/\tau_0,\dots, x_k/\tau_k, x_{k+1},\dots,x_N)$$
    for any $m\leq k \leq N$ and using that $\widetilde{T}$ is maximally consistent, it follows that this sentence is a an element of said theory. Of no such existential quantifier appear we obviously are in a position where we can just apply $\mathrm{L}_{10}$ a potential bunch of times to obtain the same conclusion. The upshot is that for each $\sigma\in T^\ast_{\sigma_0}(c)$, there is an sPNF sentence $\sigma'\in T^\ast_{\sigma_0}(c)$,
    $$\sigma'\equiv \quantifs_0 x_0\dots \quantifs_n x_n \sigma_{i,n}(x_0,\dots,x_n)$$
    such that 
    $$
        \sigma \equiv \quantifs_{k+1}x_{k+1} \dots \quantifs_{n}x_n \sigma_{i,n}(x_0/\tau_0,\dots,x_k/\tau_k,x_{k+1},\dots,x_n)
    $$
    for some $k$ and term constants $\tau_0,\dots,\tau_k$ with $\tau_k\equiv (i,\tau_0,\dots,\tau_{k-1},k)$. This fact WILL BE VERY IMPORTANT SOON AND I AM VERY HAPPY THAT I FINALLY KNOW WHY IT IS TRUE!
\end{remark}
\subsubsection{Construction of a Model of $\widetilde{T}$ \& the Proof of Gödel's Completeness Theorem}
First we construct the domain of the $\mathcal{L}_c$-structure. Recall that $\Lambda_\tau=[\tau_0,\tau_1,\dots]$ is the potentially infinite list of term constants with respect to $\mathcal{L}$. Set 
$$
    A_0 := [].
$$
If non of the sentences $\tau_n=\tau_0,\dots,\tau_n=\tau_{n-1}$ are in $\widetilde{T}$ we set 
$$A_{n+1}:= A_n+[t_n].$$
If this is not the case, for each $0\leq m\leq n $ for which $\tau_n = \tau_m$ is in $\widetilde{T}$, append $\tau_n$ to the list(s) in $A_n$ which contain $t_m$. The resulting list, $A$, will be a potentially infinite list of potentially infinite lists of the form 
$$A := [[\tau_{n_0},\dots],[\tau_{n_1},\dots],\dots].$$
A simple induction argument shows that each term constant $\tau$ shows up in at most one list in $A$. If so call this list $\widetilde{\tau}$. We define a mapping from $\mathcal{L}_c$ to $A$, and this mapping will comprise the desired $\mathcal{L}_c$-structure which will be a model of $\widetilde{T}$. 
\begin{itemize}
    \item For constants of $\mathcal{L}$ and special constants of $\mathcal{L}_c$, $c$
    $$
        c^M := \widetilde{c}
    $$ 
    \item For a function symbol $F$ in $\mathcal{L}_c$, we define
    $$
        F^M(\widetilde{\tau_1},\dots,\widetilde{\tau_n}) := \widetilde{F(\tau_1,\dots,\tau_n)} \quad \left(\widetilde{\tau_1},\dots\widetilde{\tau_n}\in A\right).
    $$
    \item For a relation symbol $R$ in $\mathcal{L}_c$, we define 
    $$
        (\widetilde{\tau_1},\dots,\widetilde{\tau_n})\in R^M  :\metaiff R(\tau_1,\dots,\tau_n)\in \widetilde{T} \quad 
    $$ 
\end{itemize}   
\begin{remark}
    Note that this mapping is well-defined {\Large Check!}
\end{remark}
\begin{theorem}
    The $\mathcal{L}_c$-structure $M$ defined above defines a model for $\widetilde{T}$ and since $T$ is a subset of $\widetilde{T}$, when we restrict $\bullet^M$ to $\mathcal{L}$, it follows that this restriction is a model of $T+\neg\sigma_0$. 
\end{theorem}
\begin{proof}
    Note that the converse of $M\vDash \sigma \metaimplies \sigma\in \widetilde{T}$ since if $M\vDash \sigma$ then $\sigma\in \mathrm{Th}(T)\supset \widetilde{T}$.\\
    Consider first the case where $\sigma$ is variable free. We prove the statement for such $\sigma$ by induction in the number of logical operators. In the base case $\sigma\equiv \tau = \tau'$ or, in general, $\sigma \equiv R(\tau_1,\dots,\tau_n)$ for some $n$-ary relational symbol $R$ and terms $\tau_1,\dots,\tau_n,\tau,\tau'$. By definition, given any assignment $j$, for $I=(M,j)$, $\tau=\tau'$ if and only if $I(\tau)=I(\tau')$ and $R(\tau_1,\dots,\tau_n) \in \widetilde{T}$ if and only if $(\widetilde{\tau_1},\dots,\widetilde{\tau_n})\in R^M$.\\
    Now fix $\sigma$ to be a non-atomic sentence and assume that for every sentence $\sigma'$ containing fewer logical operators than $\sigma$ that $\sigma \in \widetilde{\sigma'}\metaiff M\vDash \sigma'$. It is a fact that one can find a sentence equivalent to $\sigma$ which only uses the logical operators $\wedge,\neg, \exists$. By the soundness theorem we may WLOG assume that any sentence is of this form. Since $\sigma$ is variable free, we need only consider two case:\\
    \textbf{$\sigma\equiv \neg\sigma''$:} By induction $\sigma''\in \widetilde{T}$ if and only if $M\vDash \sigma''$. By definition we thus have 
    $$
        \neg\sigma'' \notin\widetilde{T}\metaiff M\not\vDash \neg\sigma'', 
    $$ 
    which is equivalent to 
    $$
        \sigma \in \widetilde{T}\metaiff M\vDash \sigma 
    $$
    \textbf{$\sigma \equiv \sigma_1 \wedge \sigma_2$:} By induction $\sigma_i\in \widetilde{T} \metaiff M\vDash \sigma_i$ for $i=1,2$, hence by elementary considerations in the metatheory it follows that 
    $$
        \sigma\equiv\sigma_1 \wedge \sigma_2 \in \widetilde{T} \metaiff M\vDash \sigma\equiv\sigma_1\wedge \sigma_2.
    $$   
    We now allow $\sigma$ to contain variables and prove the statement by induction in the number of variables used in $\sigma$. We covered the base case above. So assume that $\sigma$ is not variable free and suppose that $\sigma'\in \widetilde{T} \metaimplies M\vDash\sigma'$ for every sentence with less variables than $\sigma$. By construction every element in $\widetilde{T}$ not in $T^\ast_{\sigma_0}(c)$ is a variable free $\mathcal{L}_c$-sentence. Then $\sigma\in T^\ast_{\sigma_0}(c)$. WLOG $\sigma$ is in sPNF. There is a $\widetilde{\sigma}\in T_{\sigma_0}^\ast(c)$ in special prenex normal form,
    $$\widetilde{\sigma} \equiv \quantifs_0 x_0\quantifs_1 x_1\dots\quantifs_n x_n \sigma_{i,n}(x_0,\dots,x_n)$$
    such that for some $k\leq k$ and term constants $\tau_0,\dots,\tau_{k-1}$,
    $$
        \sigma\equiv \quantifs_k x_k\dots \quantifs_n x_n \sigma_{i,n}(x_0/\tau_0,\dots, x_{k-1}/\tau_{k-1},x_k,\dots,x_n)
    $$   
    by Remark~\ref{ReallyImportantFactAboutExistentialWitnesses}.\\
    \textbf{ Case $\quantifs_k \equiv \exists$:} Since $\sigma$ has variables it must be an element of $T^\ast_{\sigma_0}(c)$. Note that 
    $$
        \sigma' \equiv \quantifs_{k+1} x_{k+1}\dots \quantifs_n x_n \sigma_{i,n}(x_0/t_0,\dots,x_k/t_k,x_{k+1},\dots,x_n)\in T_{\sigma_0}^\ast(c)\subset \widetilde{T}.
    $$
    By induction $M\vDash \sigma'$ and by $\mathrm{L}_{11}$, $\sigma ' \vdash \sigma$, hence by the Soundness Theorem $M\vDash \sigma$.\\
    \textbf{Case $\quantifs_k\equiv \forall$:}  
    We split this case into two cases:\\
    \textbf{Case 1: There is an existential quantifier appearing after $\quantifs_k$:}\\
    In this case for some $k< m\leq n$, 
    $$\sigma \equiv \forall x_k \dots \forall x_{m-1} \exists x_m \sigma_{i,m}(x_0/\tau_0,\dots,x_{k-1}/\tau_{k-1},x_k,\dots,x_m).$$
    Then setting 
    $$ 
        c_j \equiv (i,\tau_0,\dots,\tau_{m-1},m),
    $$
    where $\tau_k,\dots,\tau_{m}$ are arbitrary term constants,
    by Lemma~\ref{LemmaForCase1InHenkinsProof},
    $$
        \sigma'':\equiv\sigma_{i,m}(x_0/\tau_0, \dots,x_{m-1}/\tau_{m-1},x_m/c_j)\in \widetilde{T}.
    $$
    By induction it follows that $M\vDash \sigma''$. Take an arbitrary assignment $j$ and consider $I=(M,j)$. Note that then 
    \begin{align*}
        &I \vDash \sigma_{i,m}(x_0/\tau_0, \dots,x_{m-1}/\tau_{m-1},x_m/c_j) \metaiff I [x_m\mapsto I(c_j)] \sigma_{i,m}(x_0/\tau_0, \dots,x_{m-1}/\tau_{m-1},x_m)\\ 
        &\metaiff I \vDash \exists x_m \sigma_{i,m}(x_0/\tau_0, \dots,x_{m-1}/\tau_{m-1},x_m)\\
        &\metaiff I[x_k\mapsto \widetilde{\tau_k}]\dots [x_{m-1}\mapsto \widetilde{\tau_{m-1}}] \vDash \exists x_m\sigma_{i,m}(x_0/\tau_0,\dots,x_{k-1}/\tau_{k-1},x_k,\dots,x_m) 
    \end{align*}
    Since $\tau_k,\dots,\tau_{m-1}$ were taken arbitrarily it follows that 
    $$
        I \vDash \forall x_k\dots \forall x_{m-1}\exists x_m \sigma_{i,m}(x_0/\tau_0,\dots,x_{k-1}/\tau_{k-1},x_k,\dots, x_m) \metaiff I \vDash \sigma \metaiff M \vDash \sigma
    $$
    \textbf{Case 2: Every quantifier appearing after $\quantifs_k$ is universal:}\\
    In this case 
    $$\sigma \equiv \forall x_k\dots \forall x_n \sigma_{i,n}(x_k,\dots,x_n).$$
    Then using Lemma~\ref{LemmaForCase2InHenkinsProof},
    $$\sigma'' \equiv \sigma_{i,n}(x_0/\tau,\dots, x_n/\tau_n)\in \widetilde{T}$$
    for arbitrary term constants $\tau_k,\dots,\tau_n$. By induction $M\vDash \sigma''$. Following a similar argument to that of case 1 we obtain
    $$M\vDash \sigma.$$
    So we conclude that $M\vDash \sigma$ for any $\sigma \in \widetilde{T}$. It thus follows that $M\vDash \widetilde{T}$ and hence that $M\vDash T+\sigma_0$.
\end{proof}
\begin{theorem}(Henkin's Existence Theorem). If $\mathcal{L}$ is a countable signature and $T$ is a consistent then it has a model $M$.  
Moreover, if $\sigma_0$ is a sentence such that $T\not\vdash \sigma_0$, then there is a model of $T+\neg \sigma_0$ and in particular a model of $T$.  
\end{theorem}

\begin{corollary}
    Every theory that can be completed has a model.
\end{corollary}
\begin{definition}
    Let $T$ be an $\mathcal{L}$-theory and $\sigma$ an $\mathcal{L}$-sentence, then \emph{$T$ is a model of $\sigma$}, denoted $T\vDash \sigma$, if for every model $M$ of $T$ we have $M\vDash \sigma$
\end{definition}
\begin{corollary}(Gödel's Completeness Theorem)
    Let $\mathcal{L}$ be a countable signature. Consider an $\mathcal{L}$-theory $T$ and an $\mathcal{L}$-sentence $\sigma$. Then 
    $$T\vDash \sigma \metaiff T\vdash \sigma.$$
\end{corollary}
\begin{proof}
    $T\vdash \sigma\metaimplies T\vDash \sigma$ is the soundness theorem. Suppose $T\vDash \sigma$. If $T$ is inconsistent, then $T\vdash \sigma$. So suppose $\mathrm{Con}(T)$. Then if $T\not\vdash \sigma$, by Henkin's Existence Theorem we can find a model of $T+\neg \sigma$, i.e. 
    $$\text{Exists } M(M\vDash T+\neg \sigma)$$
    which is equivalent to 
    $$\text{Exists } M( M \vDash T \text{ and } M\vDash \neg\sigma) \metaiff \text{not}(\text{ Forall } M(M\vDash T \metaimplies M \vDash \sigma))$$
\end{proof}
\begin{corollary}
    Let $T$ be a theory. If every finite subset of $T$ has a model then $T$ has a model. 
\end{corollary}
\begin{proof}
    Indeed if every finite subset of $T$ has a model, then every finite subset of $T$ is consistent, hence by the Compactness Theorem $T$ is consistent, so by Henkin's Existence Theorem, $T$ has a model.   
\end{proof}
The above corollary is very useful for doing mathematics. Once we have argued for the formal proofs of basic theorems in ZF(C) and when we have a standard model of ZF(C), we have a way to prove mathematical statements without needing to do formal proofs. For instance instead of doing a formal proof of some sentence $\sigma$ from the axioms of Group Theory ($\verb|GT|$), Ring Theory($\verb|RT|$), Topology ($\verb|TOP|$, which is an extension of the axioms of ZF(C)), or (almost) whatever set of axioms we desire, we simply prove $\verb|GT|\vDash \sigma$, $\verb|RT|\vDash \sigma$ or $\verb|TOP|\vDash \sigma$. For example in group theory to prove that the neutral element is unique, we can take an arbitrary model of group theory and prove the result in this model.  
\subsection{The Axioms and Standard Model of Peano Arithmetic}

\subsubsection{The Axioms}
We introduce to the language of first order predicate logic the following symbols, $\mathcal{L}_{\mathrm{PA}}:=\{0,\mathrm{s},+,\cdot\}$ and these symbols make up the language of Peano Arithmetic ($\verb|PA|)$. The axioms of Peano arithmetic are as follows. 
    \begin{axioms}
        \begin{align*}
            \mathrm{PA}_0\colon\quad & \neg \exists x (\mathrm{s} x = 0) \\
            \mathrm{PA}_1\colon\quad & \forall x \forall y (\sverb x = \sverb y \rightarrow x = y) \\
            \mathrm{PA}_2\colon\quad & \forall x (x + 0 = x) \\
            \mathrm{PA}_3\colon\quad & \forall x \forall y (x + \sverb y = \sverb(x + y)) \\
            \mathrm{PA}_4\colon\quad & \forall x (x \cdot 0 = 0) \\
            \mathrm{PA}_5\colon\quad & \forall x \forall y (x \cdot \sverb y = (x \cdot y) + x) \\
            \mathrm{PA}_6\colon\quad & \left( \varphi(0) \land \forall x (\varphi(x) \rightarrow \varphi(\sverb (x))) \right) \rightarrow \forall x \varphi(x)
        \end{align*}
    \end{axioms}
\subsubsection{The Standard Model of $\mathrm{PA}$}
For the standard model we let the domain be $\N$, i.e. the naive notion of natural numbers we described earlier consisting of $\mathbf{0}$ and finite strings of the form $\mathbf{s}\dots \mathbf{s}\mathbf{0}$. 
\subsubsection{Gödels Incompleteness Theorem}

