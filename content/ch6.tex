% !TEX root = ../main.tex
\section{Algebraic Geometry using Schemes}

\subsection{Sheaves \& Presheaves}

\subsubsection{Presheaves}

\begin{definition}
    Consider a category $\mathcal{C}$. A \emph{presheaf} on $\mathcal{C}$ is a contravariant functor $\pazocal{F}: \mathcal{C}\rightarrow \mathrm{Set}$.\\
    For a subcategory $\mathcal{C}'$ of $\mathrm{Set}$, a \emph{$\mathcal{C}'$-valued presheaf} on $\mathcal{C}$ is a contravariant functor $\pazocal{F} : \mathcal{C}\rightarrow \mathcal{C}'$. We denote the image of a morphism $\varphi: V \rightarrow U$ under $\pazocal{F}$ by $\varphi_{UV}$.
\end{definition}
\begin{remark}
    We will mainly be interested in the case of $\mathcal{C}$ being obtained from a topological space: Consider a topological space $(X,\tau_X)$. We consider the category $(X,\tau_X)$ induced by the preorder on $(\tau_X,\subset)$. This will be called a \emph{presheaf on $X$} and denoted $X\rightarrow \mathrm{Set}$. Moreover we will mainly be interested in presheaves valued in algebraic categories such as $\mathrm{Group},\mathrm{AbGroup},\mathrm{Ring}$, etc.\\  
\end{remark}
\begin{example}
    Let $X$ be a variety. For an open subvariety $U\subset X$, set 
    $$
        \pazocal{O}(U) := \{ f\in K(X) : f \text{ is regular on some open subset } U'\subset U\},
    $$ 
    which is readily verified to be a subring of $K(X)$. Consider open subsets $V\subset U\subset X$, we define a ring homomorphism (in fact a $K$-algebra homomorphism)
    \begin{gather*}
        \pazocal{O}(U) \rightarrow \pazocal{O}(V)\\
        f\mapsto \left. f \right|_V
    \end{gather*}
    Indeed, if $f$ is regular on some $U'\subset U$, then $\left.f\right|_V$ is regular on $U'\cap V$. Then 
    \begin{gather*}
        \pazocal{O} : X \rightarrow \mathrm{CRing}\op\\
        U \mapsto \pazocal{O}(U)\\
        V\subset U \mapsto \pazocal{O}(U)\rightarrow \pazocal{O}(V) 
    \end{gather*} 
    defines a presheaf on $X$ valued in the category of commutative rings. 
\end{example}
\begin{definition}\label{RestrictionInImageOfPresheaf}
    For objects $U,V$ in $\pazocal{C}$ and a morphism $\varphi: V\rightarrow U$, for an  $s\in \pazocal{F}(U)$, we define \emph{restriction of $s$ to $V$} to be $$\left. s\right|_{V}:= \varphi_{UV}.$$
\end{definition}
\begin{lemma}\label{PresheafFromQuotientGroups}
    Consider a presheaf on a topological space $\pazocal{F}:X\rightarrow \mathrm{grp}$. Consider moreover a collection of ideals $\{I_U\trianglelefteq\pazocal{F}(U) : U\in \tau_X\}$ with $\varphi_{UV}(I_U) \subset I_V$ for each $U\supset V$. For each $U\subset X$ open, set $\pazocal{F}'(U):=\pazocal{F}(U)/I_U$. We have a well-defined map of groups
    \begin{gather*}
        \varphi'_{UV} : \pazocal{F}'(U)\rightarrow \pazocal{F}'(V)\\
        s + I_U \mapsto \left.s\right|_V + I_V
    \end{gather*}
    for $U\supset V$. Then 
    \begin{gather*}
        \pazocal{F}': X\rightarrow \mathrm{grp}\\
        U \mapsto \pazocal{F}'(U)\\
        U\supset V \mapsto \varphi'_{UV}
    \end{gather*}
\end{lemma}
\begin{proof}
    Indeed, for $U\supset V\supset W$,
    $$
        \varphi_{VW}'\varphi_{UV}'(s+I_U) = \varphi_{VW}\varphi_{UV}(s)+I_W=\varphi_{UW}(s)+I_W = \varphi'_{UW}(s+I_U),
    $$
    for each $s+I_U\in \pazocal{F}'(U)$. Moreover,
    $$
        \varphi_{UU}'(s+I_U)=\varphi_{UU}(s)+I_U = s+I_U.
    $$
\end{proof}
\subsubsection{Sheaves}
\begin{definition}
    Consider a topological space $X$. consider a presheaf $\pazocal{F}: X\rightarrow \mathrm{Set}$ on $X$. $\pazocal{F}$ is called a \emph{sheaf on $X$} if 
    \begin{enumerate}
        \item For every open $U\subset X$ and every open cover $\{U_i\}$ of $U$ and every $\phi,\varphi\in \pazocal{F}(U)$, if $\left.\varphi\right|_{U_i}= \left.\phi\right|_{U_i}$ for every $i$, then $\varphi=\phi$.
        \item For every open $U\subset X$ and every open covering $\{U_i\}$ and for every collection of elements (functions) 
        $$\left\{s_i\in \pazocal{F}(U_i): \left.s_i\right|_{U_i\cap U_j}=\left. s_j\right|_{U_i\cap U_j} \text{ for every }j\right\}_{i}$$
        there is an element $s\in \pazocal{F}(U)$ such that $\left.s\right|_{U_i}=s_i$ for each $i$.   
    \end{enumerate}
    A sheaf valued in a subcategory of $\mathrm{Set}$ is defined in a similar way for sheaves to how it was defined for presheaves. 
\end{definition}
\begin{remark}
    By condition 1. an element $s\in \pazocal{F}(U)$ obtained as described in 2. is unique.
\end{remark}

\begin{lemma}
    Consider a topological space $X$ and a presheaf $\pazocal{F}$ on $X$ valued in the category of abelian groups that satisfies the second condition of being a sheaf. Then $\pazocal{F}$ is a sheaf on $X$ if and only if for every open $U\subset X$ every open covering $\{U_i\}$ and every $s\in \pazocal{F}(U)$, if $\left.s\right|_{U_i}=0$ for every $i$, then $s=0$.   
\end{lemma}
\begin{proof}
    "$\implies$": If $\pazocal{F}$ is a sheaf on $X$, then since $\left. s\right|_{U_i}=0=\left.0\right|_{0}$ for every $i$, then $s=0$.\\
    "$\impliedby$": If $\left. s\right|_{U_i} = \left. t\right|_{U_i}$ for every $i$, hence 
    $$0=\left. s\right|_{U_i} - \left. t\right|_{U_i} = \left. s-t\right|_{U_i}$$
    for every $i$, which by assumption means $s-t=0$. 
\end{proof}
\begin{example}
    The presheaf $\pazocal{O}$ on a variety $X$ is a sheaf: Consider an open subvariety $U\subset X$ and an open covering $\{U_i\}$ of $U$ and suppose there is a rational function $f\in \pazocal{O}(U)$ such that $\left. f\right|_{U_i}=0$ for every $i$. Viewing $f$ as a partial function on $U$ it follows that given a point $x\in \dom\ f$, since $x\in \dom\ f \cap U_i$ for some $i$, 
    $$f(x)=\left.f\right|_{U_i}(x)= 0 \implies f = 0.$$
    Consider an open covering, $\{U_i\}$ of a subvariety $U\subset X$ and a collection of rational functions 
    $$\left\{f_i\in \pazocal{O}(U_i): \left.f_i\right|_{U_i\cap U_j}= \left. f_j\right|_{U_i\cap U_j} \right\}_i,$$
    then the function 
    \begin{gather*}
        f : U \rightarrow K \\
         x\mapsto f_i(x) \quad \text{if } x\in U_i
    \end{gather*} 
    is a rational function on $U$ such that $\left. f\right|_{U_i} = f_i$. Indeed, restriction in the sense of Definition~\ref{RestrictionInImageOfPresheaf} is in this instance literally restriction of partial functions in the usual sence, so the definition of $f$ is indenpendent of the choice of $U_i$.
\end{example}
\begin{lemma}
    Let $\pazocal{F} : X\rightarrow \mathrm{Set}$ be a sheaf. Let $U\subset X$ be an open subset. Then we define 
    \begin{gather*}
        \left.\pazocal{F}\right|_U : U \rightarrow \mathrm{ab}\\
    \end{gather*}
    to be the presheaf given by $\pazocal{F}$ restricted to the subspace topology on $U$ of open sets in $X$. This is a sheaf on $U$.
\end{lemma}
\begin{proof}
    Let $V\subset U$ be open and $\{V_i\}$ an open cover of $V$. Consider $\varphi,\phi\in \left.\pazocal{F}\right|_U(V)=\pazocal{F}(V)$ with 
    $$
        \left.\varphi\right|{V_i}=\left.\pazocal{F}\right|_{U}(V_i\subset V)(\varphi)=\left.\pazocal{F}\right|_{U}(V_i\subset V)(\phi)=\left.\phi\right|_{V_i}
    $$ 
    for each $i$. Then since $\pazocal{F}$ is a sheaf, we get $\varphi = \phi$, hence $\left.\pazocal{F}\right|_{V}$ satisfies the first condition for being a sheaf.\\
    Let $V\subset U$ be open with an open covering $\{V_i\}$ and consider a collection of elements $\{s_i\in \left.\pazocal{F}\right|_{U}(V_i)\}$ with 
    $$
        \left. s_i\right|_{V_i\cap V_j} = \left.\pazocal{F}\right|_{U}(V_i\cap V_j\subset V)(s_i) = \left.\pazocal{F}\right|_{U}(V_i\cap V_j\subset V)(s_j)= \left. s_j\right|_{V_i\cap V_j}
    $$ for each $i$ and $j$. Then there is an $s\in \pazocal{F}(V)=\left.\pazocal{F}\right|_U(V)$ such that 
    $$
        \left.\pazocal{F}\right|_{U}(V_i \subset V)(s) = \left. s\right|_{V_i} = s_i. 
    $$
    We thus conclude that $\left.\pazocal{F}\right|_{U}$ is a sheaf. 
\end{proof}
Recall from \ref{DirectLimits} the definition of a direct limit. We shall explicitly construct the direct limit in the category of groups
\begin{definition}
    Let $G_\bullet: (I,\leq) \rightarrow \mathrm{grp}$ be a direct system with associated collection of objects and morphisms $(\{G_i\},\{f_{ij}\})$. Given elements $(x_i,i),(x_j,j)$ in  
    $$
        \bigsqcup_{i\in I} G_i
    $$
    we define an equivalence relation $\sim$, by postulating $x_i\sim x_j$ if there exists $k\in I$ with $i,j\leq k$ and 
    $$
        f_{ik}(x_i) = f_{jk}(x_j).
    $$
    The \emph{direct limit of $G_\bullet$} is defined to be the set 
    $$
        \varinjlim G_i := \bigsqcup_{i\in I} G_i/\sim.
    $$
    Given a pair of elements $[x_i],[x_j]\in \varinjlim G_i$ we define 
    $$
        [x_i][x_j] := [f_{ik}(x_i)f_{jk}(x_j)]
    $$
    where $k\in I$ is an element such that $i,j\leq k$. There is a canonical natural transformation $\phi: G_\bullet \Rightarrow \varinjlim G_i$ given by
    $$
        \{G_i \rightarrow \varinjlim G_i, x_i \mapsto [x_i]\}
    $$
\end{definition}
\begin{remark}
    By functioriality the definition of the binary operation of the direct limit of $\{G_i\}$ is independent of the choice of $k$. Since for $i\leq l$, 
    $$
        f_{il}(x_i) = f_{ll}f_{il}(x_i) \implies [x_i]=[f_{il}(x_i)].
    $$
    This means that for every $h\geq k$, 
    $$
        [f_{ik}(x_i)f_{jk}(x_j)] = [f_{kh}(f_{ik}(x_i)f_{jk}(x_j))] =[(f_{kh}f_{ik})(x_i)(f_{kh}f_{ik})(x_j)]=[f_{ih}(x_i)f_{jh}(x_j)].
    $$
\end{remark}
\begin{lemma}
    Keeping the notation from the prior definition, the set $\varinjlim G_i$ with the described binary operation is a group. When the direct system is valued in $\mathrm{ab}$, the direct limit is an abelian group
\end{lemma}
\begin{proof}
    first we check that the operation is well defined. Let $([x_i],[x_j]),([x_p],[x_q])$ in the direct limit of $\{G_i\}$. Then $f_{ik}(x_i)=f_{pk}(x_p)$ and $f_{jh}(x_j)=f_{qh}(x_q)$ for suitable $k,h\in I$ with $i,p\leq k$ and $j,q\leq h$. Pick an $l\in I$ with $h,k\leq l$. Then 
    $$
        [x_i][x_j]=[f_{il}(x_i)f_{jl}(x_j)]=[f_{pl}(x_p)f_{ql}(x_q)] = [x_p][x_q].
    $$
    One readily verifies that this operation is associative.
    Note that since each $f_{ij}$ is a group homomorphism, the identity of each $G_i$ is in the same equivalence class and one sees that $e:=[e_i]$ is neutral with respect to the binary operation in $\varinjlim G_i$. Lastly defining $[x_i]^{-1} = [x_i^{-1}]$, one readily verifies that this is independent of the choice of representative and that it is inverse element of $[x_i]$.
\end{proof}
\begin{lemma}
    Consider a direct system $(\{G_i\},\{f_{ij}\})$ of groups. Then $\varinjlim G_i$ with $\phi: G_\bullet \Rightarrow \varinjlim G_i$ is the direct limit in the category of groups.
\end{lemma}
\begin{proof}
    Let a cocone $(G,\lambda)$ of the direct system be given. We define 
    \begin{gather*}
        u : \varinjlim G_k \rightarrow G\\
        [x_i] \mapsto \lambda_i(x_i)
    \end{gather*}
    This is well defined by the naturality of $\{\lambda_i\}$. Let $i,j\in I$ with $i\leq j$ be given. Clearly, $u$ is the unique homomorphism making 
    $$
        \begin{tikzcd}
            G_i \arrow[rr,"f_{ij}"] \arrow[rd, "\phi_i"]\arrow[rdd, "\lambda_i",bend right = 20] && G_j \arrow[ld,"\phi_j"]\arrow[ldd,"\lambda_j",bend left = 20]\\
            & \varinjlim G_k \arrow[d," u"]\\
            & G
        \end{tikzcd}
    $$    
    commute. We thus conclude that our construction is indeed the direct limit in the categorical sense. 
\end{proof}
\begin{remark}
    One can extend this result to direct systems in subcategories of $\mathrm{grp}$ such as $\mathrm{ab}$, $\mathrm{CRing}$ and $\mathrm{Mod}_R$. 
\end{remark}
\begin{definition}
    Consider a topological space $X$ and a presheaf $\pazocal{F}$ on $X$ valued in $\mathrm{ab}$. Let $P\in X$. We define the \emph{stalk of $\pazocal{F}$ at $P$} to be  
    $$
        \pazocal{F}_P := \varinjlim_{U\ni P} \pazocal{F}\in \mathrm{ab}. 
    $$
    To be precise, $N_P(X)$ denote the directed set (wrt. $\supset$) of open neighborhoods of $P$. $\pazocal{F}_P$ is then the direct limit of the direct system given by $\pazocal{F}$ restricted to $N_P(X)$ Given an element $s\in \pazocal{F}(U)$ we call $[s]_P := [s]$ a \emph{germ} of $f$ at $P$. 
\end{definition}
\begin{remark}
    Note that by construction $[s_U],[t_V]\in \pazocal{F}_P$ are equal if and only if there is some open neighborhood of $P$,  $W\subset U\cap V$ say, with 
    $$ 
        \left. s\right|_{W} = \left. t\right|_W.
    $$
    There is a natural transformation 
    \begin{gather*}
        \phi : \pazocal{F}(U) \Rightarrow \pazocal{F}_p\\
        s\mapsto [(s,U)]_P
    \end{gather*}
\end{remark}
\begin{example}
    Given a variety $X$ over a field $K$, $\pazocal{O}_P$ consists exactly of rational functions that are regular on some open neighborhood of $P$, i.e. 
    $$
        \pazocal{O}_P = \pazocal{O}_P(X).
    $$
\end{example}
\begin{definition}
    Given a category $\pazocal{C}$ \emph{a morphism of presheaves} is just a morphism in the functor category $\mathrm{Set}^{\pazocal{C}\op}$, i.e. a morphism $\Phi: \pazocal{F} \rightarrow \pazocal{G}$ of parallel presheaves on $\pazocal{C}$ is just a natural transformation $\Phi : \pazocal{F}\Rightarrow \pazocal{G}$.\\
    An isomorphism of presheaves is therefor just a natural isomorphism of presheaves. \\
    \emph{A morphism of sheaves} is just a morphism of presheaves in the category of sheaves, $\mathrm{Sh}(X)$, which is a subcategory of $\mathrm{Set}^{\pazocal{T}(X)\op}$.   
\end{definition}
\begin{definition}
    Consider a parallel pair of presheaves $\pazocal{F},\pazocal{G}: X\rightarrow \mathrm{ab}$ and fix a point $P\in X$. For a morphism of presheaves $\Phi : \pazocal{F}\Rightarrow \pazocal{G}$, there is a unique induced morphism of stalks at $P$ given by
    \begin{gather*}
        \Phi_P : \pazocal{F}_P \rightarrow \pazocal{G}_P\\
        (U,s)\mapsto (U,\Phi_U(s))
    \end{gather*}
\end{definition}
\begin{remark}
    Explicitly, we can describe the above map as taking a germ $[s_U]_P$ to $[\Phi_U(s)]_P$ where $U$ is any open neighborhood of $P$. 
\end{remark}
\begin{proposition}\label{EquivalentFormulationOfNaturalIsoInTermsOfStalkMorphisms}
    Consider a parallel pair of sheaves $\pazocal{F},\pazocal{G}: X\rightarrow \mathrm{ab}$ and morphism of sheaves $\Phi: \pazocal{F}\Rightarrow \pazocal{G}$. Then $\Phi$ is an isomorphism if and only if $\Phi_P:\pazocal{F}_P\rightarrow \pazocal{G}_P$ is an isomorphism for each $P\in X$.  
\end{proposition}
\begin{proof}    
    Suppose $\Phi$ is an isomorphism. Then $\Phi_P$ is an isomorphism for each $P$ by functoriality of the assignment $\pazocal{D}\mapsto \lim \ \pazocal{D}$ for diagrams indexed by some category $\pazocal{I}$ in a category that admits limits of such diagrams.\\
    Conversely suppose $\Phi_P$ is an isomorphism for each $P$. It suffices to show that each component of $\Phi$ is an isomorphism of abelian groups. Fix an open set $U$ in $X$.\\
    \textbf{Injectivity:} Suppose $s\in \pazocal{F}(U)$ is a section with $\Phi_U(s)=0$. Then in particular for each $P\in U$,
    $$
        \Phi_P([(s,U)]_P) = [(\Phi_U(s),U)]_P = [0]_P.
    $$
    Then on some open neighborhood $U_P$ of $P$, $\left. s\right|_{U_p} =0$. This with the fact that $\{U_P\}$ defines an open cover of $U$, implies $s = 0$ by the first sheaf property.\\
    \textbf{Surjectivity:} Consider a section $t\in \pazocal{G}(U)$. Let $P\in U$. Then 
    $$
        [\Phi_{W_P}(s_{W_P})]_P=\Phi_P([s_{W_P}]_P) = [(t,U)]_P
    $$
    for a suitable section $s_{W_P}\in\pazocal{F}(W_P)$ for some open neighborhood $W_P\subset U$ of $P$. We may assume that $\Phi_{W_P}(s_{W_P}) = \left. t\right|_{W_P}$. For pairs of points $P,Q\in U$, by naturality,
    $$
        \Phi_{W_P\cap W_Q}\left(\left.s_{W_P}\right|_{W_P\cap W_Q}\right)=\left. t\right|_{W_P\cap W_Q} = \Phi_{W_P\cap W_Q}\left(\left.s_{W_Q}\right|_{W_P\cap W_Q}\right),
    $$  
    and since $\Phi_{W_P\cap W_Q}$ is injective,
    $$
        \left.s_{W_P}\right|_{W_P\cap W_Q} = \left.s_{W_Q}\right|_{W_P\cap W_Q}.
    $$
    The second sheaf property implies the existence of a section $s\in\pazocal{F}(U)$ with $\left. s\right|_{W_P} = s_{W_P}$ for each $P\in U$. We aim to show that $\Phi_U(s)=t$. Again by naturality,
    $$
        \left.\Phi_{U}(s)\right|_{W_P} = \Phi_{W_P}\left(s_{W_P}\right) = \left. t\right|_{W_P} \implies \left.\left(\Phi_U(s)-t\right)\right|_{W_P} =0 
    $$
    for each $P\in U$. Then by the first sheaf property $\Phi_U(s)-t=0$.
\end{proof}

\begin{definition}
     Consider a parallel pair of presheaves $\pazocal{F},\pazocal{G}: \pazocal{C}\rightarrow \mathrm{ab}$ and morphism of sheaves $\Phi: \pazocal{F}\Rightarrow \pazocal{G}$. 
    \begin{itemize}
        \item We define \emph{the presheaf kernel of $\Phi$} to be the presheaf 
    \begin{gather*}
        \ker \ \Phi: \pazocal{C} \rightarrow \mathrm{ab}\\
        U \mapsto \ker\ \Phi_U\\
        U\rightarrow V \mapsto \left.\varphi_{UV}\right|_{\ker\ \Phi_V} : \ker \ \Phi_V \rightarrow \ker \ \Phi_U
    \end{gather*}
    \item  We define \emph{the presheaf cokernel of $\Phi$} to be the presheaf 
    \begin{gather*}
        \mathrm{pscoker} \  \Phi: \pazocal{C}\rightarrow \mathrm{ab}\\
        U \mapsto \coker\ \Phi_U\\
        U\rightarrow V \mapsto \coker\ \Phi_V \rightarrow \coker\ \Phi_U, x+ \im \ \Phi_V \mapsto \pi_{UV}(x) + \im\ \Phi_U  
    \end{gather*}
    \item We define \emph{the presheaf image of $\Phi$} to be the presheaf
    \begin{gather*}
        \mathrm{psim}\ \Phi : \pazocal{C}\rightarrow \mathrm{ab}\\
        U\mapsto \im \ \Phi_U\\
        U\rightarrow V\mapsto \left.\pi_{UV}\right|_{\im \ \Phi_V} \im \ \Phi_V\rightarrow \im \ \Phi_U, x\mapsto \pi_{UV}(x) 
    \end{gather*}
    \end{itemize}
\end{definition}
\begin{remark}
    Unsurprisingly, the welldefinedness of the above presheaves are ensured by naturality. Functorialty follows from the functioriality of $\pazocal{F}$ and $\pazocal{G}$.
\end{remark}
\begin{lemma}
    The presheaf kernel of a morphism of sheaves is a sheaf.
\end{lemma}
\begin{proof}
    Consider presheaves $\pazocal{F},\pazocal{G}:X\rightarrow \mathrm{ab}$ and a morphism of presheaves $\Phi:\pazocal{F}\Rightarrow \pazocal{G}$. Let $U\subset X$ be an open subset and let $s\in \ker\ \Phi(U)$ be given with an open covering $\{U_i\}$ of $U$ such that $\left. s\right|_{U_i}=0$ for each $i$, then since $s\in \pazocal{F}(U)$ and $\left. s \right|_{U_i}\in \pazocal{F}(U_i)$, it follows that $s=0$. Suppose $\{s_i\in\ker\ \Phi(U_i)\}$ is a family of sections with $\left. s_i\right|_{U_i\cap U_j} = \left. s_j\right|_{U_i\cap U_j}$. Then there is an $s\in \pazocal{F}(U)$ with $\left.s\right|_{U_i} = s_i$ for each $i$. We claim that $s \in \ker\ \Phi(U)$. Indeed, note that 
    $$
        \left.\Phi_U(s)\right|_{U_i}= \Phi_{U_i}\left(s_i\right)=0 
    $$
    for each $i$, hence $\Phi_U(s)=0$ by the first sheaf property. 
\end{proof}
\begin{remark}
    The same result cannot be attained. The failure of the image presheaf to be a sheaf comes from the fact that while we can paste local sections $\Phi_{U_i}(s)$ together to form some global section $t$ in $\pazocal{G}(U)$, it may be that the image of $\Phi_U$ does not contain $t$ {\Large Example?} {\Large Why does cokernel presheaf fail}
\end{remark}
\begin{lemma}
    Consider a presheaf $\pazocal{F}:X\rightarrow \mathrm{ab}$ on topological space $X$. There is a sheaf $\pazocal{F}^{+}: X\rightarrow \mathrm{ab}$ and morphism of presheaves $\Theta : \pazocal{F}\Rightarrow\pazocal{F}^{+}$ such that for any sheaf $\pazocal{G}:X\rightarrow \mathrm{ab}$ and any morphism of presheaves $\Phi : \pazocal{F} \Rightarrow \pazocal{G}$ there is a unique morphism of sheaves $\Psi : \pazocal{F}^{+}\Rightarrow \pazocal{G}$ such that $\Phi = \Psi \Theta$. Furthermore $(\pazocal{F}^{+},\Theta)$ is unique up to unique isomorphism. 
\end{lemma}
\begin{proof}
    We construct a presheaf
    \begin{gather*}
        X\rightarrow \mathrm{ab}
    \end{gather*}
    Given an open subset $U$ in $X$, we define $\pazocal{F}^{+}(U)$ to be the group of elements $([s]_P)$ in 
    $$\prod_{P\in U} \pazocal{F}_P$$
    such that for each $P\in U$ there is an open neighborhood of $P$, $W\subset U$ with an element $t\in \pazocal{F}(W)$ such that for each $Q\in V$ the germ $[t]_Q\in \pazocal{F}_Q$ of $t$ at $Q$ satisfies
    $$([s]_P)(Q) = [t]_Q.$$
    Given $U\subset V$ we define $\varphi_UV : \pazocal{F}^{+}(V)\rightarrow \pazocal{F}^{+}(U)$ to be given by restriction. It thus readily follows that $\pazocal{F}^{+}$ is a sheaf. We have a natural transformation
    \begin{gather*}
        \Theta : \pazocal{F}\Rightarrow \pazocal{F}^{+}\\
        \{\Theta_U : \pazocal{F}(U)\rightarrow \pazocal{F}^{+}(U), s \mapsto ([s]_P)\}.
    \end{gather*}
    Indeed, $(U,s)$ is a pair witnessing the fact that $[s]_P\in \pazocal{F}^{+}$, so this is well-defined. Let a natural transformation of presheaves $\Phi : \pazocal{F}\Rightarrow \pazocal{G}$ be given. We define a natural transformation of sheaves 
    \begin{gather*}
        \Psi: \pazocal{F}^{+} \Rightarrow \pazocal{G}
    \end{gather*}
    defined by the collection of morphisms  
    $$
        \left.\left\{ \Psi_U: \pazocal{F}^{+}(U) \rightarrow \pazocal{G}(U), ([s]_P)\mapsto \Psi_U(s) \ \right| \   U\subset X \text{ open}\right\}
    $$
    This is well-defined since $\pazocal{G} \cong \pazocal{G}^{+}$ or in other words since $\pazocal{G}(U)\simeq \pazocal{G}^{+}(U)$. $\Psi$ is the unique morphism of sheaves making the diagram
    $$
        \begin{tikzcd}
            \pazocal{F} \arrow[rd,Rightarrow, "\Phi"] \arrow[d,"\Theta",Rightarrow]\\
            \pazocal{F}^{+} \arrow[r, dashrightarrow] & \pazocal{G}
        \end{tikzcd}
    $$
    commute. Note that $(\pazocal{F}^{+},\Theta)$ is thus the coequalizer two copies of the identity transformation of $\pazocal{F}$ (cf. Example~\ref{EqualisersAndCoequalisers}) which means it is unique up to unique isomorphism.
\end{proof}
\begin{remark}
    We call $\pazocal{F}^{+}$ \emph{the sheaf associated with $\pazocal{F}$}
\end{remark}
\begin{definition}
    A \emph{subsheaf} of a sheaf $\pazocal{F}$ is a sheaf $\pazocal{F}'$ such that for every open set $U$ in $X$, $\pazocal{F}'(U)\leq \pazocal{F}(U)$ and, letting $\varphi_{UV}:\pazocal{F}(V)\rightarrow \pazocal{F}(U)$ denote the image of $U\subset V$ under $\pazocal{F}$, the image of $U\subset V$ under $\pazocal{F}'$ is the restriction of $\varphi_{UV}$ to $\pazocal{F}(V)$.
\end{definition}
\begin{remark}
    Note that given $(U,s)\in \pazocal{F}'_P$, we have that $s\in \pazocal{F}'(U)\subset \pazocal{F}(U)$, hence $(U,s)\in \pazocal{F}_P$. 
\end{remark}
\begin{example}
    The presheaf kernel of morphism of sheaves $\Phi:\pazocal{F}\Rightarrow \pazocal{G}$ is sheaf called the \emph{sheaf kernel}. By construction $(\ker \ \Phi)(U) = \ker \ \Phi_U \leq \pazocal{F}(U)$ and 
    $$(\ker \ \Phi)(U\subset V) = \left. \varphi_{UV}\right|_{\ker\ \Phi_U},$$
    hence $\ker \ \Phi$ is a subsheaf of $\pazocal{F}$. 
\end{example}
\begin{definition}
    A morphism of sheaves $\Phi:\pazocal{F}\Rightarrow \pazocal{G}$ is \emph{injective} if $\ker \ \Phi=0$, i.e. the sheaf that takes every open subset of $X$ to the trivial group.
\end{definition}
\begin{remark}
    $\Phi$ being injective is equivalent to $\Phi_U$ being injective for each open set $U$. 
\end{remark}
\begin{definition}
    For a morphism of sheaves $\Phi: \pazocal{F}\Rightarrow \pazocal{G}$, we define the \emph{image of $\Phi$} to be $\im \ \Phi:= (\mathrm{psim}\ \Phi)^{+}$
\end{definition}
\begin{proposition}
    Given a morphism of sheaves $\Phi:\pazocal{F}\Rightarrow \pazocal{G}$, the canonical morphism $\im\ \Phi\Rightarrow \pazocal{G}$ is injective, hence it can be indentified with a subsheaf of $\pazocal{G}$. 
\end{proposition}
\begin{proof}
    
\end{proof}
\begin{definition}
    A morphism of sheaves $\Phi:\pazocal{F}\Rightarrow \pazocal{G}$ is \emph{surjective} if $\im\ \Phi = \pazocal{G}$.
\end{definition}
\begin{definition}
    A sequence 
    $$
        \begin{tikzcd}
            \dots \arrow[r, Rightarrow, "\Phi^{i-1}"] & \pazocal{F}^{i-1} \arrow[r, Rightarrow, "\Phi^i"] & \pazocal{F}^i \arrow[r, Rightarrow, "\Phi^{i+1}"] & \pazocal{F}^{i+1} \arrow[r, Rightarrow] & \dots
        \end{tikzcd}
    $$
    is \emph{exact}, if $\ker\ \Phi^i = \im \ \Phi^{i-1}$ for each $i \in \Z$. 
\end{definition}
\begin{definition}
    Consider a sheaf $\pazocal{F}$ and subsheaf $\pazocal{F}'\subset \pazocal{F}$. We define the \emph{quotient sheaf $\pazocal{F}/\pazocal{F}'$} to be the sheaf associated with 
    \begin{gather*}
        X\rightarrow \mathrm{ab}\\
        U\mapsto \pazocal{F}(U)/\pazocal{F}'(U)\\
        U\subset V \mapsto \pazocal{F}(V)/\pazocal{F}'(V) \rightarrow \pazocal{F}(U)/\pazocal{F}'(U), s + \pazocal{F}'(V) \mapsto \left. s\right|_{U}+\pazocal{F}'(U)
    \end{gather*}
\end{definition}
\begin{remark}
    Fix a point $P\in X$ and consider $(U,s)\in (\pazocal{F}/\pazocal{F}')_P$. Note that then $s\in \pazocal{F}(U)/\pazocal{F}'(U)$, hence $(U,s)\in \pazocal{F}_P/\pazocal{F}'_P$.
\end{remark}
With the above in mind we can make the following definition
\begin{definition}
    For a morphism of sheaves $\Phi : \pazocal{F}\Rightarrow \pazocal{G}$, we define the \emph{cokernel of $\Phi$} to be $\coker\ \Phi := (\mathrm{pscoker}\ \Phi)^{+}$.
\end{definition}
\begin{definition}
    For a continuous map $f : X\rightarrow Y$ and a sheaf $\pazocal{F}$ on $X$, we define the \emph{direct image sheaf}, $f_\star\pazocal{F}$, on $Y$, to be the sheaf
    \begin{gather*}
        Y \rightarrow \mathrm{ab}\\
        U \mapsto \pazocal{F}(f^{-1}(U))\\
        U\subset V \mapsto \varphi_{f^{-1}(U)f^{-1}(V)}
    \end{gather*}
\end{definition}
\begin{remark}
    We check that this indeed a sheaf {\Large Check!}
\end{remark}
\begin{definition}
    For a continuous map $f : X\rightarrow Y$ and a sheaf $\pazocal{F}$ on $Y$, we define the \emph{inverse image sheaf}, $f^{-1}\pazocal{G}$ on $X$ to be the sheaf associated with the presheaf
    \begin{gather*}
        X\rightarrow \mathrm{ab}\\
        U\mapsto \varinjlim_{V\supset f(U)}\pazocal{G}(V)\\
        U\subset V\mapsto \varphi_{\lim_{W\supset f(U)}\pazocal{G}(W)\lim_{W\supset f(V)}\pazocal{G}(W)}
    \end{gather*}
\end{definition}
\begin{definition}
    Consider a topological space $X$, a subspace $Z\subset X$ and a sheaf $\pazocal{F}$ on $X$. Let $\iota$ denote the inclusion map $Z\hookrightarrow X$. We define \emph{the restriction of $\pazocal{F}$ to $Z$} to be the sheaf $\left.\pazocal{F}\right|_{Z}:= \iota^{-1}\pazocal{F}$.
\end{definition}
\begin{remark}
    Fix a point $P\in Z$. Let $(U,s)\in \left(\left. \pazocal{F}\right|_Z\right)_P$... {\Large Do!}
\end{remark}
\subsection{Schemes}
We adopt the convention that all rings are commutative.
\subsubsection{Affine Schemes}
\begin{definition}
    For a ring $R$, we define \emph{the spectrum of $R$} to be the set
    $$
        \Spec\ R := \{ I \subset R : I \text{ is a prime ideal}\}
    $$
    For an ideal $I\subset R$, we define 
    $$
        V(I) := \{ J\in \Spec \ R : I\subset J\}
    $$
\end{definition}
\begin{remark}
    In the special case of $R=K[x_1,\dots,x_n]$ for a field $K$, we denote the vanishing set of an ideal $I\subset K[x_1,\dots,x_n]$ by $Z(I)$ instead of $V(I)$ to avoid confusion.
\end{remark}
\begin{lemma}
    If $I\subset J\subset R$ are ideals, then $V(I)\supset V(J)$.
\end{lemma}
\begin{proof}
    This is trivial. 
\end{proof}
\begin{lemma}
    For ideals $I,J\subset R$
    $$V(IJ) = V(I)\cup V(J).$$
\end{lemma}
\begin{proof}
    Consider a prime ideal $\mathfrak{p}\in \Spec \ R$ containing $IJ$. If $\mathfrak{p}\supset J$ we are done. Otherwise given any $a\in I$ and choosing $b\in J\setminus \mathfrak{p}$ we have that $a \in \mathfrak{p}$ hence $\mathfrak{p}\in V(I)$. The converse inclusion follows from $I,J\supset IJ$ by the prior lemma. 
\end{proof}
\begin{lemma}
    For a family of ideals $\{I_\mu\}$ in $R$, 
    $$V\left(\sum_\mu I_\mu\right) = \bigcap_\mu V(I_\mu).$$ 
\end{lemma}
\begin{proof}
    Again one inclusion follows from $I_\nu\subset \sum_\mu I_\mu$. Suppose conversely that $\mathfrak{P}\in \Spec \ R$ is given such that it contains every $I_\mu$, then since ideals are closed under addition it contains $\sum_\mu I_\mu$.  
\end{proof}
\begin{proposition}
    For a ring $R$, the family of sets 
    $$
        \{V(I) : I \text{ is an ideal in } R\}
    $$
    are the closed set in a topology on $\Spec \ R$. 
\end{proposition}
\begin{proof}
    Indeed, note that 
    $$\bigcap_{I\subset R} V(I) = V\left(\sum_{I\subset R} I\right)= V(R) = \emptyset.$$
    and that $\Spec\ R = V(0)$. The two prior lemmas show that this family is closed under countable union and intersection over arbitrary index sets. 
\end{proof}
\begin{remark}
    The topology in question is unsurprisingly called the \emph{Zariski topology on $\Spec \ R$}. For an algebraically closed field $K$ and an affine variety $V\subset \A^n$, the points of $V$ correspond to the maximal ideals of $\Gamma(V)$. In this way we can identify $V$ with the subspace of maximal ideals of $\Gamma(V)$ in $\Spec \ \Gamma(V)$. In this sense $\Spec \ \Gamma(V)$ is a refinement of the notion of point in a variety. Namely, \emph{point} in $V$ have by definition come to mean the \emph{subvariety} of $V$, when we take the view that our vantage point lies in $\Spec\ \Gamma(V)$ when studying $V$.
\end{remark}
\begin{lemma}
    If $I,J\subset R$ are ideals, then $V(I)\subset V(J)$ if and only if $\rad(I)\supset \rad(J)$.
\end{lemma}
\begin{proof}
    Note that $V(I)\subset V(J)$ if and only if every prime ideal that contains $I$ also contains $J$ which again is equivalent to 
    $$
        \bigcap_{\mathfrak{p}\supset I \text{ prime }} \mathfrak{p} \supset \bigcap_{\mathfrak{p}\supset J \text{ prime }} \mathfrak{p}, 
    $$
    which by lemma \ref{RadicalOfIdealIsIntersectionOfAllPrimeIdealsContainingIt} is equivalent to 
    $$
        \rad(I)\supset \rad(J).
    $$
\end{proof}
Given a ring $R$ and prime ideal $\mathfrak{p}$ in $R$. We denote the localization of $R$ with respect to $\mathfrak{p}$ by $R_\mathfrak{p}$.
\begin{definition}
    Given a ring $R$, we define a presheaf on $\Spec \ R$ by, 
    \begin{gather*}
        \pazocal{O}:(\Spec\ R, \tau_{\Spec\ R})\op\rightarrow \mathrm{Ring} 
    \end{gather*}
    that takes an open subset $U$ of $\Spec\ R$ to the subring of $\prod_{\mathfrak{p}\in U} R_\mathfrak{p}$, 
    $$
        \pazocal{O}(U) := \left\{ s\in \prod_{\mathfrak{p}\in U}  R_\mathfrak{p} : \forall\mathfrak{p}\in U \exists  V \text{ op. ngh. } \text{ of } \mathfrak{p} \text{ s.t. } \left.s\right|_V = \frac{a}{f}   \text{ w/ } f\notin \mathfrak{q} \text{ for each } \mathfrak{q}\in V \right\}.
    $$  
    Moreover, $V\subset U$ is mapped to the restriction map $\pazocal{O}(U)\rightarrow \pazocal{O}(V), s\mapsto \left. s\right|_V$.
\end{definition}
\begin{remark}
    Clearly this presheaf is a sheaf, since the elements of $\pazocal{O}(U)$ are functions and restriction is just set function restriction in the same we saw in the example discussing the presheaf $\Gamma$. 
\end{remark}
\begin{definition}
    For $f\in R$, we define 
    $$
        D(f) := (\Spec \ R)\setminus V(\langle f\rangle)
    $$
\end{definition}
\begin{lemma}
    For a ring $R$,
    $$
        \{D(f)\subset \Spec\ R : f\in A\}
    $$
    forms a basis of the Zariski topology on $\Spec\ R$.
\end{lemma}
\begin{proof}
    Indeed, for $\mathfrak{p}\notin V(I)$, there is an $f\in I$ not in $\mathfrak{p}$, hence $\mathfrak{p}\notin V(\langle f\rangle)$, meaning $\mathfrak{p}\in D(f)$. Then $V(I)\cap D(f) = \emptyset$. This means that every point in an open subset $U$ in $\Spec \ R$ has an open neighborhood contained in $U$. It follows that the collection in question forms an open basis for the Zariski topology. 
\end{proof}
\begin{lemma}
    Given a ring $R$ and an element $f\in R$, $D(f)$ is quasi-compact with respect to the Zariski subspace topology.
\end{lemma}
\begin{proof}
    Let $\{V_\alpha\}$ be an open cover of $\Spec\ R$. WLOG $V_\alpha=D(h_\alpha)$ for some $h_\alpha\in R$ for each $\alpha$. Then 
    $$
        V(\langle f\rangle)=\bigcap_\alpha V(\langle h_\alpha\rangle)=V\left(\sum_\alpha \langle h_\alpha\rangle\right),
    $$ 
    which implies $\rad(\langle f\rangle)= \rad\left(\sum_\alpha\langle h_\alpha\rangle\right).$
    Then $f^n = \sum_1^m h_i$ for some $n\geq 1$ and $\{h_1,\dots,h_m\}\subset \{h_\alpha\}$, hence 
    $$
        D(f)=D(f^n)= \bigcup_1^m D(h_i).
    $$  
\end{proof}
\begin{corollary}
    $\Spec\ R$ is quasi-compact.
\end{corollary}
\begin{lemma}\label{FiniteNumberOfRepresentationsOfSection}
    Consider a ring $R$, an element $f\in R$ and a section $s\in \pazocal{O}(D(f))$. There are $a_1,\dots,a_m,h_1,\dots,h_m\in R$ such that 
    $$
        D(f) = \bigcup_1^m D(h_i)
    $$
    and 
    $$
        s(\mathfrak{p}) = \frac{a_i}{h_i}\in R_\mathfrak{p}
    $$
    for every $\mathfrak{p}\in D(h_i)$ for each $i$. 
\end{lemma}
\begin{proof}
    First of all, there are $\{\lambda_\alpha\}$ in $R$ such that 
    $$
        D(f) = \bigcup_\alpha D(\lambda_\alpha)
    $$
    and $s(\mathfrak{p})=\frac{a_\alpha'}{g_\alpha}$ for some $a_\alpha',g_\alpha \in R$ with $g_\alpha\notin \mathfrak{p}$ for each $\mathfrak{p}\in D(\lambda_\alpha)$. Since $D(\lambda_\alpha)\subset D(g_\alpha)$, it follows that $\rad(\langle \lambda_\alpha \rangle)\subset \rad(\langle g_\alpha\rangle)$, hence for some $k_\alpha \geq 1$, 
    $$
        \lambda_\alpha^{k_\alpha} = l_\alpha g_\alpha \implies \frac{l_\alpha a_\alpha'}{\lambda_\alpha^{k_\alpha}} = \frac{a_\alpha'}{g_\alpha}.
    $$
    Setting $h_\alpha := \lambda_\alpha^{k_\alpha}$ and $a_\alpha := l_\alpha a_\alpha'$. We get, upon noting that $D\left(h_\alpha^{k_\alpha}\right) = D(\lambda_\alpha)$, a cover $\{h_\alpha\}$ of $D(f)$ with 
    $$
        s(\mathfrak{p})= \frac{a_\alpha}{h_\alpha}
    $$
    with $h_\alpha\notin \mathfrak{p}$.
    Since $D(f)$ is quasi-compact we obtain a finite cover of such $D(h_\alpha)$.
\end{proof}


\begin{lemma}\label{StalkCorrespondToPrimeLocalizationSheafOnBasisSetCorrespondToLocalization}
    Consider a ring $R$. Then 
    \begin{enumerate}
        \item For every point $\mathfrak{p}\in \Spec\ R$
        $$
            \pazocal{O}_\mathfrak{p} \simeq R_\mathfrak{p}.
        $$
        \item For any $f\in R$,
        $$
            \pazocal{O}(D(f)) \simeq R_f.
        $$
        \item In particular,
        $$
            \Gamma(\Spec\ R,\pazocal{O}):= \pazocal{O}(\Spec\ R) = \pazocal{O}(D(1)) \simeq R_1 = R.
        $$
    \end{enumerate}    
\end{lemma}
\begin{proof}
    1. Consider the ring map
    \begin{gather*}
        \sigma:\pazocal{O}_\mathfrak{p} \rightarrow R_\mathfrak{p}\\
        [s_U]\mapsto s_U(\mathfrak{p})
    \end{gather*}
    Which exists upon applying the universal property for $\pazocal{O}_\mathfrak{p}$ on $\ev_\mathfrak{p} : \pazocal{O}\Rightarrow R_\mathfrak{p}$. To check surjectivity, let $\frac{a}{f}\in R_\mathfrak{p}$ be given. By definition $f\notin \mathfrak{p}$, hence $\mathfrak{p}\not\supset \langle f\rangle$. We thus get that $D(f)$ is an open neighborhood of $\mathfrak{p}$. Let $\frac{a}{f}: D(f)\rightarrow \bigcup_{\mathfrak{q}\in D(f)} R_\mathfrak{q}$ denote the constant function on $D(f)$ taking $\mathfrak{q}$ to $\frac{a}{f}$. Note that $\mathfrak{q}\not\supset \langle f\rangle$ for each $\mathfrak{q}\in D(f)$, so $\frac{a}{f}$ is a section in $\pazocal{O}(D(f))$. In particular this is an element of $\pazocal{O}_\mathfrak{p}$ and 
    $$
        \sigma\left(\frac{a}{f}\right) = \frac{a}{f}.
    $$
    To check injectivity, consider $s,t\in \pazocal{O}_\mathfrak{p}$ with $s(\mathfrak{p})=r(\mathfrak{p})$. On a sufficiently small neighborhood $U$ of $\mathfrak{p}$, we have that $\left. s\right|_U = \frac{a}{f}\in R_\mathfrak{q}$ and $\left. t\right|_U = \frac{b}{g}\in R_\mathfrak{q}$ for each $q\in U$. There is some $h\in R\setminus \mathfrak{p}$ such that $h(ag-bf)=0$, hence on $D(f)\cap D(g)\cap D(h)$, $s$ and $t$ agree. This implies $s=t$. We conclude that $\sigma$ is an isomorphism of rings.\\
    2. We define a homomorphism 
    \begin{gather*}
        \tau: R_f \rightarrow \pazocal{O}_P
    \end{gather*} 
    That takes $\frac{a}{f^n}\in R_f$ to the function $s\in \prod_{\mathfrak{p}\in D(f)}$ that takes $\mathfrak{p}\in D(f)$ to $\frac{a}{f^n}$, which is well-defined since $\mathfrak{p}\not\supset  \langle f^n\rangle $ for every $n\geq 0$. We claim that this is an isomorphism.\\
    To show injectivity, suppose 
    $$
        \tau\left(\frac{a}{f^n}\right) = \tau\left(\frac{b}{f^m} \right)
    $$
    for a pair $\frac{a}{f^n},\frac{b}{f^m}\in R_f$. Then for every $\mathfrak{p}\in D(f)$  
    $$
        h_\mathfrak{p}\left(af^m-bf^n\right) = 0
    $$
    for some $h_\mathfrak{p}\in R\setminus \mathfrak{p}$. Set $\mathfrak{a}:= \mathrm{Ann}(af^m-bf^n)$. Note then that $\mathfrak{p}\not\supset \mathfrak{a}$, since $h_\mathfrak{p}\in \mathfrak{a}$. We get that $V(\mathfrak{a})\cap D(f)=\emptyset$, hence $V(\mathfrak{a})\subset V(\langle f\rangle)$, which is equivalent to $\langle f\rangle \subset \rad(\mathfrak{a})$. Then $f^l(af^m-bf^n)=0 $ for some $l\geq0$. It thus follows that $\frac{a}{f^n}=\frac{b}{f^m}$ in $R_f$.\\
    To show surjectivity, note first that $D(f)$ has a finite cover $\{D(h_i)\}_1^m$ with $s(\mathfrak{p})=\frac{a_i}{h_i}\in R_\mathfrak{p}$ for some $a_i\in R$ for each $\mathfrak{p}\in D(h_i)$ for each $i$. Consider a pair of indeces $i$ and $j$ and note that under 
    $$
        \tau' : R_{h_ih_j} \rightarrow \pazocal{O}(D(h_ih_j))=\pazocal{O}(D(h_i)\cap D(h_j)), 
    $$
    which takes a fraction to its associated constant section on $D(h_ih_j)$, $\frac{a_i}{h_i}$ and $\frac{a_j}{h_j}$ agree, then $\frac{a_i}{h_i}=\frac{a_j}{h_j}$ in $R_{h_ih_j}$. Then for some $k\geq 0$
    $$
        \left(h_ih_j\right)^k(a_ih_j-a_jh_i)=0 \iff h_j^{k+1}\left(h_i^ka_i\right)= h_i^{k+1}\left(h_j^ka_j\right).
    $$
    Note that $D(f)=\bigcup_1^m D(h_i^{k+1})$ and $s(\mathfrak{p})=\frac{a_i}{h_i}=\frac{h_i^ka_i}{h_i^{k+1}}$ for each $\mathfrak{p}\in D(h_i)=D(h_i^{k+1})$. For the sake of simplifying notation set $\lambda_i := h_i^{k+1}$ and $c_i := h_i^ka_i$. With this notation we find $\lambda_jc_i = \lambda_ic_j$ for each $i$ and $j$. Since $f\in \rad\left(\sum_1^m \lambda_i\right)$, there is an $n\geq 1$ and $b_1,\dots,b_m\in R$ with 
    $$
        f^n = \sum_1^m b_i\lambda_i. 
    $$ 
    Set 
    $$
        a:= \sum_1^m b_ic_i.
    $$
    Then we find  
    $$
        \lambda_j a= \sum_1^m \lambda_j b_i c_i = c_j\sum_1^m \lambda_ib_i = c_jf
    $$
    Then on each $D(h_i)$ for each $\mathfrak{p}\in D(h_i)$,
    $$
        \tau\left(\frac{a}{f^n}\right)(\mathfrak{p})=\frac{c_i}{\lambda_i} = s(\mathfrak{p}) \implies \tau\left(\frac{a}{f^n}\right) = s,
    $$
    hence $\tau$ is also surjective. We thus conclude that $\tau$ is a ring isomorphism.\\
    3. Follows immediately from 2.
\end{proof}
\begin{definition}
    A \emph{ringed space} is a topological space $X$ together with a sheaf 
    $$
        \pazocal{O}_X : X \rightarrow \mathrm{CRing}.
    $$
    A \emph{locally ringed space} is a ringed space $(X,\pazocal{O}_X)$ such that each stalk $\pazocal{O}_{X,P}$ for $P\in X$ is a local ring
\end{definition}
\begin{definition}
    A morphism of ringed spaces, $(X,\pazocal{O}_X)\rightarrow (Y,\pazocal{O}_Y)$ is a continuous map $f: X\rightarrow Y$ and a natural transformation $f^\# : \pazocal{O}_Y \Rightarrow f_\star \pazocal{O}_X$.
\end{definition}
We wish to define a morphism between locally ringed spaces. To do this we first show how to construct a ring homomorphism between the stalks $\pazocal{O}_{Y,f(P)}$ and $\pazocal{O}_{X,P}$ for each $P\in X$. Indeed, $f^\#$ induces a ring homomorphism 
$$
    \pazocal{O}_{Y,f(P)} = \varinjlim_{U\ni f(P)} \pazocal{O}(U)\rightarrow \varinjlim_{U\ni f(P)} \pazocal{O}_{X}\left(f^{-1}(U)\right) = \varinjlim_{f^{-1}(U)\ni P} \pazocal{O}_X\left(f^{-1}(U)\right)
$$
Composing this with 
$$
    \varinjlim_{f^{-1}(U)\ni P} \pazocal{O}_X\left(f^{-1}(U)\right)\hookrightarrow \varinjlim_{W\ni P} \pazocal{O}_X(W) = \pazocal{O}_{X,P},
$$
we obtain a map 
\begin{gather*}
    f^\#_P : \pazocal{O}_{Y,f(P)}\rightarrow \pazocal{O}_{X,P}
\end{gather*}
\begin{definition}\label{MorphismOfRingedSpaces}
    A morphism of locally ringed spaces is a morphism of ringed spaces $(f,f^\#):(X,\pazocal{O}_X)\rightarrow (Y,\pazocal{O}_Y)$ such that $f^\#_P$ is local ring homomorphism (cf. Definition~\ref{LocalRingDefinition}) for each $P\in X$. 
\end{definition}
\begin{lemma}
    Consider a ringed space $(X,\pazocal{O})$ and an open subset $U\subset X$. Denote the from $X$ inherited sheaf by $\pazocal{O}_U$. The map $\iota : U\subset X$ together with 
    $$
        \iota^\#: \iota_\star\pazocal{O} \rightarrow \pazocal{O}_U
    $$
    given by 
    $$
        \left\{ \varphi_{U U\cap V} : \pazocal{O}(V) \rightarrow  \pazocal{O}_U(U\cap V)=\pazocal{O}(U\cap V) : V\subset Y \text{ open}\right\}.
    $$
    defines a morphism of ringed spaces. If $(X,\pazocal{O})$ is a locally ringed space, then this morphism is a morphism of locally ringed spaces.
\end{lemma}
\begin{proof}
    $\iota$ is a continuous map and the naturality of $\iota^\#$ follows immediately from the functorialty of $\pazocal{O}$. The map $\iota^\#_P$ is just the identity map for each $P\in U$, $(\iota,\iota^\#)$ defines a morphism of locally ringed spaces
\end{proof}
\begin{lemma}
    Consider a morphism of ringed spaces $\varphi: (X,\pazocal{O}_X)\rightarrow (Y,\pazocal{O}_Y)$. Let $U\subset X$ and $U'\subset Y$ open be given with $\varphi(U)\subset U'$. Consider $\iota: U\hookrightarrow X$. Then $\left.\varphi\right|_U$ together with
    $$
        \iota^\#\left.\varphi^\#\right|_{\left.\pazocal{O}_Y\right|_{U'}} : \left.\pazocal{O}_Y\right|_{U'} \Rightarrow \left(\left.\varphi\right|_{U}\right)_\star \left.\pazocal{O}_X\right|_U
    $$
    is a morphism of ringed spaces. If ringed spaces in question are locally ringed spaces, then this morphism is a morphism of locally ringed spaces. 
\end{lemma}
\begin{proof}
    We already know that $\left.\varphi\right|_{U}$ is a continuous map. $\iota^\#\left.\varphi^\#\right|_{\left.\pazocal{O}_Y\right|_{U'}}$ is a natural transformation. With the assumption of locally ringed spaces, note that the square  
    $$
        \begin{tikzcd}
            \left(\pazocal{O}_Y\right)_{\varphi(P)}\arrow[d]\arrow[rr,"\varphi^\#_P"]&& \left(\pazocal{O}_X\right)_{P} \arrow[d]\\
            \left(\left.\pazocal{O}_Y\right|_{U'}\right)_{\left.\varphi\right|_{U}(P)} \arrow[rr,"\left(\iota^\#\left.\varphi^\#\right|_{\left.\pazocal{O}_Y\right|_{U'}}\right)_{P}"] && \left(\left.\pazocal{O}_X\right|_U\right)_{P}
        \end{tikzcd}
    $$  
    where the vertical arrows are isomorphisms, commutes. It thus follows that $\left(\iota^\#\left.\varphi^\#\right|_{\left.\pazocal{O}_Y\right|_{U'}}\right)_{P}$ is a local homomorphism for each $P\in U$.
\end{proof}
\begin{lemma}
    \begin{enumerate}
        \item $\Spec \ R$ is a locally ringed space for each ring $R$.
        \item If $\sigma: A\rightarrow B$ is a map of rings, then it induces a morphism of locally ringed spaces
        \begin{gather*}
            (f,f^\#)_\sigma : (\Spec\ B, \pazocal{O}_{\Spec\ B}) \rightarrow (\Spec\ A, \pazocal{O}_{\Spec\ A})
        \end{gather*}
        \item The functor
        \begin{gather*}
            \Spec : \mathrm{CRing} \rightarrow \mathrm{LRS}\\
            R\mapsto \Spec\ R\\
            \sigma : A\rightarrow B \mapsto (f,f^\#)_\sigma
        \end{gather*}
    \end{enumerate}
    is fully faithful. Here $\mathrm{LRS}$ denotes the category of locally ringed spaces. 
\end{lemma}
\begin{proof}
    1. Follows from Lemma~\ref{StalkCorrespondToPrimeLocalizationSheafOnBasisSetCorrespondToLocalization} 1.\\
    2. We define 
    \begin{gather*}
        f : \Spec\ B\rightarrow \Spec \ A\\
        \mathfrak{p} \mapsto \sigma^{-1}(\mathfrak{p})
    \end{gather*}
    which is well-defined by Proposition~\ref{PreImageOfPrimeIdealUnderRingMapIsPrime}. For an ideal $I\subset A$ one readily verifies that $f^{-1}(V(I))=V(\sigma(I))$ {\Large I haven't actually defined what this means: DO!}, which shows that $f$ is continuous. For an open set $U\subset \Spec \ A $, we take 
    $$
        f^\#_U :\pazocal{O}_{\Spec \ A}(U)\rightarrow \left(f_\star\pazocal{O}_{\Spec\ B}\right)(U)= \pazocal{O}_{\Spec\ B}\left(f^{-1}(U)\right)
    $$
    to be ring map taking a section $s\in \pazocal{O}_{\Spec\ A}$ to the section
    \begin{gather*}
        f^{-1}(U) \rightarrow \bigcup_{\mathfrak{p}\in f^{-1}(U)} B_\mathfrak{p}\\
        \mathfrak{p} \mapsto \sigma_\mathfrak{p}(s\circ f)(\mathfrak{p})
    \end{gather*}
    where $\sigma_\mathfrak{p}: A_{f(\mathfrak{p})}\rightarrow B_{\mathfrak{p}}$ is the localization of $\sigma$ with respect to $\mathfrak{p}$. We check the naturality condition: Suppose $U\subset V$. Then for $s\in \pazocal{O}_{\Spec\ A}(U)$ and $\mathfrak{p}\in f^{-1}(V)$, 
    \begin{align*}
        \left.f^\#_U\right|(s)_{f^{-1}(V)}(\mathfrak{p})=\sigma_\mathfrak{p}((s\circ f)(\mathfrak{p}))= \sigma_\mathfrak{p}\left(\left(\left.s\right|_V\circ f\right)(\mathfrak{p})\right)=f^\#_V\left(\left.s\right|_V\right)(\mathfrak{p}).
    \end{align*}
    Let $\mathfrak{p}\in \Spec \ B $ and $(U,s)\in \pazocal{O}_{\Spec A,\mathfrak{p}}$. Then 
    $$
        \left(f^\#_{f(\mathfrak{p})}(s)\right)(\mathfrak{p})= f^\#_U(s)(\mathfrak{p})= \sigma_\mathfrak{p}(s(f(\mathfrak{p}))) =  \sigma\left(s\left(\sigma^{-1}(\mathfrak{p})\right)\right),
    $$
    hence the square
    $$
        \begin{tikzcd}
            \pazocal{O}_{\Spec\ A, f(\mathfrak{p})} \arrow[r,"f^\#_{\mathfrak{p}}"]\arrow[d,"\ev_{f(\mathfrak{p})}"] & \pazocal{O}_{\Spec\ B, \mathfrak{p}} \arrow[d,"\ev_{\mathfrak{p}}"]\\
            A_{\sigma^{-1}(\mathfrak{p})} \arrow[r,"\sigma_\mathfrak{p}"] & B_\mathfrak{p}
        \end{tikzcd}
    $$
    commutes. Since $\ev_{f(\mathfrak{p})}$ and $\ev_\mathfrak{p}$ are isomorphisms, it follows that $f^\#_\mathfrak{p}$ is a local ring homomorphism. We thus conclude that $(f,f^\#)_\sigma$ is a morphism of locally ringed spaces.\\
    3. Let a morphism of locally ringed spaces $(f,f^\#): (\Spec\ B,\pazocal{O}_{\Spec \ B})\rightarrow (\Spec \ A, \pazocal{O}_{\Spec\ A})$. Then there is a ring map 
    $$
        \sigma:=f^\#_{\Spec\ A} : A\simeq \Gamma(\Spec\ A, \pazocal{O}_{\Spec\ A})\rightarrow \Gamma(\Spec \ A, \pazocal{O}_{\Spec\ B})\simeq B
    $$
    We check that $(g, g^\#)_{\sigma}=(f,f^\#)$. 
    For a $\mathfrak{p}\in \Spec\ B$, consider the local ring homomorphism (cf. Definition~{MorphismOfRingedSpaces})
    $$
        \sigma_\mathfrak{p} : A_{f(\mathfrak{p})} \rightarrow B_\mathfrak{p}
    $$
    Note that
    $$
        \sigma^{-1}(\mathfrak{p})=\sigma_\mathfrak{p}^{-1}(\mathfrak{m}_\mathfrak{p}\cap B) = \mathfrak{m}_{f(\mathfrak{p})}\cap A = f(\mathfrak{p})
    $$
    hence $g$ and $f$ agree. Now note that given a global section $s$ on $\Spec \ A$ and a point $\mathfrak{p}\in \Spec\ B$,
    $$
        g^\#_{\Spec A}(s)(p)= \sigma_\mathfrak{p}(s\circ f(\mathfrak{p})) = f^\#_\mathfrak{p}(s)(\mathfrak{p})= f^\#_\mathfrak{p}(s)(\mathfrak{p})
    $$
    hence by naturality $f^\#_U = g^\#_U$ for every open subset $U$ of $\Spec \ A$. We thus conclude 
    $$
        (f,f^\#)=(g,g^\#)_\sigma.
    $$
\end{proof}
\begin{remark}\label{ConstructionOfNaturalTransformationOfSheavesFrom}
    Note that the construction of $f^\#$ only relies on the existence of local ring maps $A_{f(\mathfrak{p})}\rightarrow B_\mathfrak{p}$
\end{remark}
\begin{example}
    Consider a field $K$. The spectrum contains only one point, namely $0$. The sections over $\Spec \ K=\{\ast\}$ are thus the constant functions $\{\ast\}\rightarrow K$, hence the structure sheaf of $\Spec \ K$ is $K$ itself.  
\end{example}
\begin{example}
    Consider a DVR $R$. This two prime ideals: $0$ and the maximal ideal $\langle t\rangle\subset R$, where $t$ is the uniformizing parameter of $R$. Note that $0\notin V(\langle t\rangle)$, which implies $D(t)=\{0\}$. It follows that $0$ is an \emph{open point} (i.e. a point that is Zariski open), hence $\langle t\rangle$ is a closed point in $\Spec\ R$. Moreover $0$ is a dense point, since $V(0)=\Spec \ R$ is the only smallest closed set containing $0$, since any other closed set is of the form $V(\langle t^k\rangle)$ and no such set contains $0$. The local ring of $0$ is $Q(R)$ and the local ring of $\langle t\rangle$ is just $R$, since points away from $\langle t\rangle$ are just the units of $R$.\\
    The map 
    $$
       \iota: R\hookrightarrow Q(R)
    $$  
    induces a morphism of ringed spaces $(f,f^\#)_\iota$ with
    \begin{gather*}
        f: \Spec \ Q(R) \rightarrow \Spec \ R\\
        0 \mapsto 0
    \end{gather*}  
    Note that $(g,f^\#_\iota)$ with 
    \begin{gather*}
        g: \Spec \ Q(R) \rightarrow \Spec \ R\\
        0 \mapsto \langle t\rangle
    \end{gather*}
    is another morphism of ringed spaces. Since $f^\#_{0} : R\rightarrow Q(R)$ corresponds to $\iota$ and $\iota^{-1}(0)=0\neq \langle t\rangle = g(0)$, it follows that $(g,f^\#_\iota)$ is not a morphism of locally ringed spaces. It is thus not induced by any ring map $R\rightarrow Q(R)$, since any ring map would induce a morphism of locally ringed spaces.  
\end{example}
The above example shows the necessity of morphisms of locally ringed spaces having induced maps on stalks that are local ring homomorphisms.
\begin{example}
    For a field $K$, we define \emph{the affine line over $K$}, denoted $\A^1_K$ to be $\Spec\ K[x]$. We denote the zero ideal in $\A^1_K$ by $\xi$. This (as with DVR's) is a dense point, since $\xi \notin V(f)$ for any $f\in K[x]$ of degree$>0$, which is equivalent to $\{x\}\cap D(f) \neq \emptyset $ for any $f\in K[x]$ with $\deg\ f>0$. In particular, $\xi$ is not a closed point. All other points of $\A_K^1$ are the maximal ideals of $K[x]$ which are in one-to-one correspondence with monic irreducible polynomials in $K[x]$. The points in $\mathfrak{p}\in\A^1_K\setminus \{\xi\}$ are thus all closed since $\mathfrak{p}=\langle f\rangle$ is a maximal ideal for some monic irreducible $f\in K[x]$, which implies $\{p\} = V(\langle f\rangle)$.    
\end{example}
\begin{example}
    When $K$ is algebraically closed, closed points in $\A_K^1$ correspond to points in $K$. It is then apparent that $\A^1(K)$ and $\A^1_K$ are different constructions, since all points $\A^1(K)$ are closed, which means there is no dense point in $A^1(K)$.  
\end{example}
\begin{definition}
    In general, we define \emph{affine space over $K$} to be
    $$
        \A^n_K := \Spec\ K[x_1,\dots,x_n]
    $$
\end{definition}
\begin{definition}
    For a topological space $X$, \emph{a generic point} is point $x$ with 
    $$
        \overline{(\{x\})}=X.
    $$
    I.e. a point whose singleton set is dense in $X$.
\end{definition}
\begin{example}
    Let $K$ be an algebraically closed field. Then the maximal ideals of $K[x_1,x_2]$ correspond to points in $\A^2(K)$. The closed points of $\A^2_K$ thus correspond to the points of $\A^2(K)$ (cf. Hilbert's Nullstellensatz). Denote these by 
    $$
        \mathfrak{m}_{a,b} \quad \left((a,b)\in \A^2(K)\right)
    $$  
    Secondly, again the point $\xi\in \A^2_K$, the zero ideal in $K[x_1,x_2]$, is a generic point.\\
    Thirdly, an irreducible polynomial $f\in K[x_1,x_2]$, corresponds to $\eta_f:=\langle f\rangle\in \A^2_K$. Suppose $(a,b)\in f$ and let $D(g)\ni \mathfrak{m}_{a,b}$ be given. Then $g(a,b)\neq 0$ for otherwise $\mathfrak{a,b}\supset \langle g\rangle$. Then $Z(g)\not\supset Z(f)$ hence $\eta_f \not\supset \langle g\rangle$. We thus conclude that $\eta_f\in D(f)$, hence $\mathfrak{m}_{a,b}$ is a point of closure of $\{\eta_f\}$. On the other hand consider a point $\mathfrak{p}\in \A^2_K$ with $Z(\mathfrak{p})\not\subset f$. Then $\eta_f\not\supset \mathfrak{p}$, hence there is a $g\in \eta_f\setminus \mathfrak{p}$. Then $\mathfrak{p}\in D(g)$ while $\eta_f\notin D(g)$, implying $\mathfrak{p}$ is not a point of closure of $\{\eta_f\}$. It follows that 
    $$
        \overline{\{\eta_f\}}= \{\eta_f\}\cup \{\mathfrak{m}_{a,b} : (a,b)\in f\}.
    $$   
    A point of $\A^2_K$ is thus the generic point $\xi$, a closed point $\mathfrak{m}_{a,b}$ or a generic point of an affine curve $f\subset \A^2(K)$. The map
    \begin{gather*}
        \Phi:\A^2(K)\rightarrow \mathrm{Specm}\ K[x_1,x_2]:= \{m_{a,b} : (a,b)\in \A^2(K)\}\subset \A^2_K\\
        (a,b)\mapsto \mathfrak{m}_{a,b}        
    \end{gather*}
    thus defines a homeomorphism onto its image: bijection is ensured by points in $\A^2(K)$ being in bijection with maximal ideals in $K[x_1,x_2]$ (cf. HNS). To check continuity it is sufficient to check that $\Phi^{-1}(V(f))$ is closed in $\A^2$ when $f$ is an affine plane curve over $K$. Note that 
    $$
        V(f) = \overline{\{\eta_f\}} = \{\eta_f\} \cup \{\mathrm{m}_{a,b}: (a,b)\in Z(f)\},
    $$
    hence $\Phi^{-1}(V(f))=Z(f)$. Moreover $\Phi$ takes an affine plane curve $f$ to its closed points in $\A^2_K$, i.e. 
    $$ 
        \Phi(Z(f))= V(f)\cap \mathrm{Specm}\ K[x_1,x_2],
    $$
    which means $\Phi$ is bicontinuous. So while the notion of affine plane discussed here and the classical notion are not the same they are at least related. The relation between varieties $V$ and $\Spec\ \Gamma(V)$ will be made precise later.  
\end{example}
We have now seen some examples of the geometric objects that we are interested in studying, namely schemes, which are locally ringed spaces that locally look the spectrum of a ring. To give a precise definition we will first give a definition of affine schemes.
\begin{definition}
    An \emph{affine scheme} is a locally ringed space $(X,\pazocal{O}_X)$ isomorphic to $(\Spec \ R, \pazocal{O})$ for some ring $R$ as locally ringed spaces. 
\end{definition}
\begin{example}
    $\Spec \ R$ is trivially an example of an affine scheme, so $\A^n_K$ are all examples of affine schemes.
\end{example}
\subsubsection{Schemes}
\begin{definition}
    A \emph{scheme} is a locally ringed space $(X,\pazocal{O}_X)$ where for each $P\in X$ there is an open neighborhood $U$ of $P$ such that $(U,\left.\pazocal{O}_X\right|_U)$ is an affine scheme. I.e. there is an open covering of $X$ where each open set in the covering is an affine scheme. 
\end{definition}
\begin{definition}
    A morphism of schemes is a morphism of local rings between schemes 
\end{definition}
\begin{lemma}\label{RestrictionOfLocallyRingedSpaceIsLocallyRingedSpace}
    Let $(X,\pazocal{O})$ be a locally ringed space. Let $U$ be an open set in $X$. Then $(U,\left.\pazocal{O}\right|_U)$ is a locally ringed space.  
\end{lemma}
\begin{proof}
    We already proved that $\left.\pazocal{O}\right|_{U}$ is a sheaf in an earlier lemma. It thus suffices to prove that for each $P\in U$, $\left(\left.\pazocal{O}\right|_U\right)_P$ is local. Note that the set of open neighborhoods of $P$ contained in $U$ is coinitial in the set of open neighborhoods of $P$. It thus follows that 
    $$
        \left(\left.\pazocal{O}\right|_{U}\right)_P \simeq \pazocal{O}_P,
    $$
    hence $\left(\left.\pazocal{O}\right|_{U}\right)_P$ is local.
\end{proof}
\begin{lemma}
    Let $R$ be a ring and consider $f\in R$. Then 
    $$
        (D(f),\left.\pazocal{O}_{\Spec\ R}\right|_{D(f)})\simeq \Spec \ R_f,
    $$
    Hence $D(f)$ is an affine scheme.
\end{lemma}
\begin{proof}
    The canonical injection $R\hookrightarrow R_f$ induces a continuous map 
    \begin{gather*}
        \varphi: \Spec \ R_f\rightarrow \Spec \ R\\
        \mathfrak{p}\mapsto \mathfrak{p}\cap R
    \end{gather*}
    Note that the image of $\varphi$ is $D(f)$. Indeed, if $I$ is an ideal containing $f$, then $\frac{f}{1}\in R_fI$, which is a unit, hence $R_f I$ is not prime. We thus conclude that if $\mathfrak{p}\subset R_f$ is prime, then $\mathfrak{p}$ does not contain $f$. Note that $\varphi$ is injective by properties of localizations of ideals. Note then that
    \begin{gather*}
        \phi : \Spec \ R_f \rightarrow D(f)\\
        \mathfrak{p}\mapsto \varphi(\mathfrak{p})
    \end{gather*}
    is a continuous, bijection that is closed. Indeed, $\phi(V(I))=V(I\cap R)$. It follows that $\phi$ is a homeomorphism. Restricting $\varphi^\#$ to $\left.\pazocal{O}_{\Spec\ R}\right|_{D(f)}$ we obtain a natural transformation
    $$
        \phi^\# : \left.\pazocal{O}_{\Spec\ R}\right|_{D(f)}\Rightarrow \phi_\star\pazocal{O}_{\Spec \ R_f}
    $$
    Let $\mathfrak{p}\in \Spec \ R_f$. Note that 
    $$
        \left(\left.\pazocal{O}_{\Spec\ R}\right|_{D(f)}\right)_{\phi(\mathfrak{p})} = \left(\left.\pazocal{O}_{\Spec\ R}\right|_{D(f)}\right)_{\varphi(\mathfrak{p})} \simeq \pazocal{O}_{\Spec\ R, \mathfrak{p}}\simeq R_{\varphi(\mathfrak{p})}=R_{\phi(\mathfrak{p})}.
    $$
    Here the first isomorphism follows from the proof of Lemma~\ref{CofinalSubsetRestrictionDirectLimitIsLimitOfDirectSystemOfSuperSet}. The commutative diagram 
    $$
        \begin{tikzcd} 
            \left(\left.\pazocal{O}_{\Spec\ R}\right|_{D(f)}\right)_{\phi(\mathfrak{p})} \arrow[r, "\phi^\#_\mathfrak{p}"]\arrow[d,"\ev_{\phi(\mathfrak{p})}"] & \pazocal{O}_{\Spec\ R_f,\mathfrak{p}}\arrow[d, "\ev_\mathfrak{p}"]\\
            R_{\phi(\mathfrak{p})} \arrow[r,"\sigma_\mathfrak{p}"] & (R_f)_\mathfrak{p}
        \end{tikzcd}
    $$
    where $\sigma_\mathfrak{p}$ is the canonical isomorphism $R_{\phi(\mathfrak{p})}\simeq (R_f)_\mathfrak{p}$, shows that $\phi^\#_\mathfrak{p}$ is an isomorphism of local rings. Then $\phi^\#$ is a natural isomorphism (cf. Proposition~\ref{EquivalentFormulationOfNaturalIsoInTermsOfStalkMorphisms}). 
\end{proof}
\begin{lemma}
    Let $(X,\pazocal{O})$ be a scheme and $U\subset X$ an open set. Then $(U,\left.\pazocal{O}\right|_U)$ is a scheme. 
\end{lemma}
\begin{proof}
    It remains to find an open cover of affine schemes on $U$. Let $P\in U$. There is an open neighborhood $V$ of $P$ in $X$ with $\varphi: V\simeq \Spec R$ for some ring $R$. Then $\varphi(U\cap V)$ is an open neighborhood of $\varphi(P)$ in $\Spec \R$. We can find an open cover of $\varphi(U\cap V)$ consisting of sets $D(f)$. Pick an $f\in R$ with $\varphi(P)\in D(f)\subset \varphi(U\cap V)$. Then 
    $$(\varphi^{-1}(D(f)),\left.\pazocal{O}\right|_{\varphi^{-1}(D(f))})\simeq (D(f),\left. \pazocal{O}_{\Spec\ R}\right|_{D(f)})\simeq (\Spec\ R_f, \pazocal{O}_{\Spec\ R_f}). $$
    For the last isomorphism we use the prior lemma.
\end{proof}
\begin{example}
    The scheme $\Spec\ \Z$ consists of a generic point corresponding to the zero ideal in $\Z$ and the remaining point are closed points corresponding to ideals generated by a prime. Note that $\Spec \ \Z$ has cofinite topology. Indeed, given an ideal $I=\langle p_1^{r_1}\cdots p_n^{r_n}\rangle \subset \Z$, 
    $$
        V(\langle I \rangle) = \{\langle p_1\rangle ,\dots,\langle p_n\rangle \}
    $$   
\end{example}
\begin{lemma}
    $\Spec\ \Z$ is the terminal object in the category of affine schemes. 
\end{lemma}
\begin{proof}
    Since $R\mapsto \Spec \ R$ defines an equivalence of categories $\mathrm{CRing}\rightarrow \mathrm{AffSch}\op$ and $\Z$ is the initial object in $\mathrm{CRing}$, it follows that $\Spec\ \Z$ is the terminal object in $\mathrm{AffSch}$
\end{proof}
\begin{remark}
    Note that $\Spec \ R\rightarrow \Spec \ \Z$ is the morphism induced by $\Z\rightarrow R$.  
\end{remark}
\begin{proposition}
    The scheme $(\Spec\ \Z, \pazocal{O}_{\Spec\ Z})$ is the final object for the category of schemes.
\end{proposition}
\begin{proof}
    Let $(X,\pazocal{O})$ be a scheme. Let $\{U_i\}$ denote an affine cover of $X$. There are morphisms of affine schemes $\varphi_i : (U_i,\left.\pazocal{O}\right|_{U_i})\rightarrow \Spec \ \Z$. Now we check that $\varphi_i$ and $\varphi_j$ agree on the domains of there intersection (when these are non-empty). To check that $\left. \varphi_i\right|_{U_i\cap U_j}= \left. \varphi_j\right|_{U_i\cap U_j}$ it suffices to check that it agrees on each open affine set $W\subset U_i\cap U_j$. Now note that $\left.\varphi_i\right|_W,\left.\varphi_j\right|_W$ is a pair of morphisms of affine schemes, hence they must be equal since $\Spec \ Z$ is the terminal object in the category of affine schemes. It follows that 
    \begin{gather*}
        \varphi : X\rightarrow \Spec \ Z
    \end{gather*}  
    where $\varphi(P)=\varphi_i(P)$ if $P\in U_i$ is a well-defined morphism of schemes.\\
    Now consider a morphism of schemes $\phi: X\rightarrow \Spec \ \Z$. Since for each $i$, $\left.\phi\right|_{U_i}: U_i\rightarrow \Spec \ \Z$ defines a morphism of affine schemes it must be equal to $\varphi_i$, hence $\phi=\varphi$.  
\end{proof}
\begin{example}
    Consider schemes $X_1$ and $X_2$ with open subsets $U_1\subset X_1$ and $U_2\subset X_2$ and an isomorphism of schemes 
    $$
        \varphi: (U_1, \left.\pazocal{O}_{X_1}\right|_{U_1})\Rightarrow (U_2,\left.\pazocal{O}_{X_2}\right|_{U_2})
    $$
    A homeomorphism
    $$
        \varphi: U_1\rightarrow U_2
    $$
    gives rise to an equivalent relation on the disjoint union of $U_1$ with $U_2$: For $x,y\in X_1\sqcup X_2$, $x\sim_\varphi y$ if $x=y$ or $x\in U_1$ and $y=\varphi(x)$. We define the scheme obtained by \emph{gluing} $X_1$ and $X_2$ along $U_1$ and $U_2$ via $\varphi$ to be 
    $$
        (X_1\sqcup X_2)/\sim.
    $$
    For the purposes of this example denote this space by $X$. Consider the maps 
    \begin{gather*}
        \iota_k : X_k\rightarrow X\\
        x \mapsto [x]_{\sim_\varphi}
    \end{gather*}
    for $k=1,2$. Let $U\subset X$. Then $U$ is open if and only if $\iota_k^{-1}(U)$ is open for each $k$. Indeed if $U$ is open if and only if $q^{-1}(U)\subset X_1\sqcup X_2$ is open which is the case if and only if 
    $$
        \iota_k^{-1}(U)=\left(q\circ i_k\right)^{-1}(U)=i_k^{-1}\left(q^{-1}(U)\right)
    $$
    is open. Here $q$ denotes the quotient map with respect to $\sim_\varphi$ and $i_k$ is the canonical embedding of $X_k$ in $X_1\sqcup X_2$ for $k=1,2$. Recall that $X_1\sqcup X_2$ is equipped with the final topology of $\{i_k : k=1,2\}$. We define a sheaf 
    \begin{gather*}
        \pazocal{O}_{X}: X\rightarrow \mathrm{CRing}
    \end{gather*}
    that takes an open set $U\subset X$ to 
    $$
        \left.\Big\{(s_1,s_2)\in \pazocal{O}_{X_1}\left(\iota_1^{-1}(U)\right)\times \pazocal{O}_{X_2}\left(\iota_2^{-1}(U)\right)\ \right| \ \varphi^\#_{\iota_2^{-1}(U)\cap U_2}\left(\left.s_2\right|_{\iota_2^{-1}(U)\cap U_2}\right)=\left.s_1\right|_{\iota_1^{-1}(U)\cap U_1}\Big\}.
    $$
    For $U\subset V$, we define restriction of a pair of sections on $\iota_1^{-1(V)}$ and $\iota_2^{-1}(U)$ to a pair of sections on $\iota_1^{-1}(U)$ respectively $\iota_2^{-1}(U)$ by
    \begin{gather*}
        \left.\bullet\right|_{U}:\pazocal{O}_{X_1}(\iota^{-1}_1(V))\times \pazocal{O}_{X_2}(\iota_2^{-1}(V))\rightarrow \pazocal{O}_{X_1}(\iota^{-1}_1(U))\times \pazocal{O}_{X_2}(\iota_2^{-1}(U))\\
        (s_1,s_2)\mapsto \left(\left.s_1\right|_{\iota_1^{-1}(U)\cap U_1},\left.s_2\right|_{\iota_2^{-1}(U)\cap U_2}\right).
    \end{gather*}
    Suppose $(s_1,s_2)\in \pazocal{O}_X(V)$. Set $t_1: = \left.s_1\right|_{\iota_1^{-1}(V)\cap U_1}$ and $t_2:=\left.s_2\right|_{\iota_2^{-1}(V)\cap U_2}$. Then 
    \begin{align*}
        \left.s_1\right|_{\iota_1^{-1}(U)\cap U} &= \left.t_1\right|_{\iota_1^{-1}(U)\cap U_1} = \left.\left(\varphi^\#_{\iota_2^{-1}(V)\cap U_2}(t_2)\right)\right|_{\iota_1^{-1}(U)\cap U_1}\\
        &=  \varphi^\#_{\iota_2^{-1}(U)\cap U_2}\left(\left. t_2\right|_{\iota_2^{-1}(U)\cap U_2} \right) = \varphi^\#_{\iota_2^{-1}(U)\cap U_2}\left(\left.s_2\right|_{\iota_2^{-1}(U)\cap U_2}\right),
    \end{align*}
    hence $\left.\left(s_1,s_2\right)\right|_{U}\in \pazocal{O}_X(U)$. This assignment is obviously functorial. One also readily verifies that $\pazocal{O}_X$ is a sheaf using the that an open cover of $X$ or more generally an open subset of $X$ gives rise to open covers on $X_k$ or $\iota_k^{-1}(U)$, together with the structure sheaf on $X_k$ for $k=1,2$.\\
    \textbf{$X$ is a scheme:} Let $[x]\in X$. Then $x=\iota_k^{-1}([x])$ for some $k=1,2$. There is an open neighborhood $U\ni x$ in $X_k$ with $\phi:U\simeq \Spec \ A $ for some ring $A$. Note that $\iota_k : X_k\rightarrow X$ defines an injective morphism of locally ringed spaces with 
    \begin{gather*}
        {(\iota_k)}^\#_V : \pazocal{O}_X(V) \rightarrow \pazocal{O}_{X_1}\left(\iota_k^{-1}(V)\right)\\
        (s_1,s_2)\mapsto s_k
    \end{gather*}  
    $\left.\iota_k\right|_{U}: U\rightarrow \iota_k(U)$ defines an isomorphism of locally ringed spaces. Hence there is a commutative diagram
    $$
        \begin{tikzcd}
            U \arrow[r,"\phi"] \arrow[d,"\left.\iota_k\right|_U"'] & \Spec \ A\\
            \iota_k(U) \arrow[ur, dashrightarrow]
        \end{tikzcd}
    $$
    hence $\iota_k(U)\simeq \Spec \ A$.\\
\end{example}
We use the above example to construct a scheme that is not affine 
\begin{example}
    We use the prior example to construct a scheme that is not affine. Set $X_1=X_2=\A^1_K$ and $U_1=U_2=\A^1_K\setminus\{\mathfrak{p}\}$ with $\mathfrak{p}=\langle x\rangle$, $\varphi = \id_{\A^1_K\setminus \{\mathfrak{p}\}}$. Consider a morphism $\phi: X\rightarrow \Spec \ A$. Then we have a commutative diagram 
    $$
        \begin{tikzcd}
            X_1 \arrow[dr]\arrow[ddr,"\iota_1",bend right=25]\arrow[rr, "\id"] && X_2\arrow[dl]\arrow[ddl,"\iota_2",bend left=25]\\
            & \Spec \ A\\
            & X \arrow[u,"\phi"]
        \end{tikzcd}
    $$
    Denote $\mathfrak{p}_1:= \iota_1(\mathfrak{p})$ and $\mathfrak{p}_2:=\iota_2(\mathfrak{p})$. Then  
    $$
        \phi(\mathfrak{p}_1)= \phi(\iota_1(\mathfrak{p}))=\phi(\iota_2(\mathfrak{p}))=\phi(\mathfrak{p}_2)
    $$
    and since $\mathfrak{p}_1\neq \mathfrak{p}_2$ it follows that $\phi$ cannot be an isomorphism of locally ringed spaces. 
\end{example}
\subsubsection{Projective Schemes}
\begin{definition}
    Let $R$ be a graded (commutative) ring. We define the \emph{projective spectrum of $R$} to be 
    $$
        \Proj\ R := \left\{ \mathfrak{p}\in \Spec\ R : \mathfrak{p} \text{ and homogeneous}, R_{+}\not\subset\mathfrak{p} \right\}.
    $$ 
\end{definition}
\begin{definition}
    Given a homogeneous ideal $I$ in a graded ring $R$, we define 
    $$
        V(I):=V^\Pp(I) := \{\mathfrak{p}\in \Proj\ R: \mathfrak{p}\supset I\}.
    $$
\end{definition}
\begin{lemma}
    Let $R$ be a graded ring, then for each pair of homogeneous ideals $I,J$ in $R$
    $$
        V^\Pp(IJ)=V^\Pp(I)\cup V^\Pp(J) 
    $$
\end{lemma}
\begin{proof}
    The proof is the exact same as for $\Spec$ (keeping in mind that the product of homogeneous ideals is homogeneous)
\end{proof}
\begin{lemma}
    Let $R$ be a graded ring and $\{I_\mu\}$ a family of homogeneous ideals then 
    $$
        V\left(\sum_{\mu} I_\mu\right) = \bigcap_\mu V(I_\mu)
    $$
\end{lemma}
\begin{proof}
    Again, the proof is the same as for the affine case.
\end{proof}
\begin{corollary}
    For a graded ring $R$, the family of sets  
    $$\{V(I) : I\subset R, \text{ homogeneous}\}$$
    defines a closed basis for a topology on $\Proj\ R$.
\end{corollary}
There are "strengthenings" of these two statements
\begin{lemma}
    Let $\mathfrak{p}\in \Proj\ R$ and homogeneous ideals $I$ and $J$ in $R$ be given. The following are equivalent
    \begin{enumerate}
        \item $\mathfrak{p}\in V(IJ)$.
        \item For every pair of homogeneous elements $a,b\in R$, if $a\in I$ and $b\in J$ then $ab\in \mathfrak{p}$.
        \item $\mathfrak{p}\in V(I)\cup V(J)$. 
    \end{enumerate}
\end{lemma}
\begin{proof}
    obviously 1. is equivalent to 3. and clearly 2. implies 3. We also have that 2. implies 3.  by Lemma~\ref{EquivalentPrimeHomogeneousIdealCondition}
\end{proof}
\begin{lemma}
    Let $\mathfrak{p}\in \Proj\ R$ and a family of homogeneous ideals $\{I_\mu\}$ be given. The following are equivalent
    \begin{enumerate}
        \item $\mathfrak{p}\in V\left(\sum_{\mu}I_\mu\right)$.
        \item For every $(a_\mu)\in \bigoplus_\mu I_\mu$ with $a_\mu$ homogeneous for each $\mu$, we have $\sum_\mu a_\mu \in \mathfrak{p}$.
        \item $\mathfrak{p}\in \bigcap_\mu V(I_\mu)$. 
    \end{enumerate}
\end{lemma}
\begin{proof}
    The argument is parallel to the one given in the prior lemma. 
\end{proof}
Fix a graded ring $R$. Let a prime ideal $\mathfrak{p}\in \Proj\ R$ 
$$
    H_\mathfrak{p} := \{x \in R : x \text{ homogeneous, } x\notin \mathfrak{p}\}.
$$
This is a multiplicative set in $R$. Let $R_{(\mathfrak{p})}\subset H_\mathfrak{p}^{-1}R$ denote the ring of degree zero elements (cf. Lemma~\ref{LocalizationOfGradedRing}). For an open subset $U\subset \Proj \ R$ we define $\pazocal{O}_{\Proj\ R}(U)$ to be the set of function $s\in \prod_{\mathfrak{p}\in U} R_{(\mathfrak{p})}$ such that for each $\mathfrak{p}\in U$ there is an open neighborhood $V$ of $\mathfrak{p}$ such that for some homogeneous $a,f\in R$ for every $\mathfrak{q}\in V$, $f\notin \mathfrak{q}$ and
$$
    s(\mathfrak{q}) = \frac{f}{a}\in S_{(\mathfrak{q})}.
$$
Then we get a sheaf 
\begin{gather*}
    \pazocal{O}_{\Proj\ R} : \Proj \ R \rightarrow \mathrm{CRing}
\end{gather*}
where $\pazocal{O}_{\Proj\ R}(U)\rightarrow \pazocal{O}_{\Proj\ R}(V)$ induced by $U\supset V$ is given by function restriction to $V$.
\begin{lemma}
    For a graded ring $R$, $\Proj \ R$ is a locally ringed space
\end{lemma}
\begin{proof}
    Consider a point $\mathfrak{p}\in \Proj \ R$. We prove that 
    $$
        \pazocal{O}_\mathfrak{p} \simeq R_{(p)}.
    $$
    We check that 
    \begin{gather*}
        \ev_\mathfrak{p} : \pazocal{O}_\mathfrak{p} \rightarrow R_{(\mathfrak{p})}\\
        s\mapsto s(\mathfrak{p})
    \end{gather*}
    defines an isomorphism. The prove of this fact is almost identical to the affine case. One just replace $R_\mathfrak{p}$ with $R_{(\mathfrak{p})}$ in the proof.   
\end{proof}
\begin{definition}
    Let $R$ be a graded ring and $f\in R_{+}$ a homogeneous element. Let $D_{+}(f)$ denote $\Proj\ R\setminus V(\langle f\rangle)$.
\end{definition}
\begin{lemma}
    Let $R$ be a graded ring and $f\in R_{+}$ a homogeneous element. $D_{+}(f)$ is an open set in $\Proj \ R$ and 
    $$
        \{D_{+}(f) : f\in R_{+}\}
    $$
    is a cover of $\Proj\ R$.
    Furthermore, $(D_{+}(f),\left.\pazocal{O}\right|_{D_{+}(f)})$ is isomorphic to $\Spec\ R_{(f)}$ as locally ringed spaces. Here $R_{(f)}$ is the subring of degree $0$ elements in $R_{f}$. It follows that $(\Proj\ R,\pazocal{O}_{\Proj\ R})$ is a scheme. 
\end{lemma}
\begin{proof}
    By definition $D_{+}(f)$ is open since it is the complement of a closed set in $\Proj\ R$. Let $\mathfrak{p}\in \Proj\ R$. Then, since $\mathfrak{p}\not\supset R_{+}$, we can pick an $g\in R_{+}$ not in $\mathfrak{p}$, hence $\mathfrak{p}\in D_{+}(g)$, hence the sets $D_{+}(f)$, for $f\in R_{+}$ homogeneous, define an open cover of $\Proj\ R$.\\ 
    Fix an $f\in R_{+}$. We define 
    \begin{gather*}
        \varphi : D_{+}(f)\rightarrow \Spec \ R_{(f)}\\
        \mathfrak{p} \mapsto R_f\mathfrak{p}\cap R_{(f)}
    \end{gather*}  
    This is trivially well-defined, once one recognizes that the operations we apply to $\mathfrak{p}$ all preserve being a prime (see e.g. Corollary~\ref{LocalizationPreservePrimeness}).This map has an inverse given by 
    \begin{gather*}
        \varphi^{-1} : \Spec\ R_{(f)}\rightarrow D_{+}(f) \\
        \mathfrak{p} \mapsto R_f \mathfrak{p} \cap R
    \end{gather*}
    We can extend these maps to maps from all homogeneous ideals in $R$ and all (homogeneous) ideals in $R_{(f)}$ and vice versa. Note that both these maps are order preserving with respect to inclusion hence
    $$
        \varphi(V(I)) = V(\varphi(I)) \text{ and } \varphi^{-1}(V(J))=V(\varphi^{-1}(J))
    $$
    for homogeneous ideals $I$ in $R$ and ideals $J$ in $R_{(f)}$, which proves that $\varphi$ is a homeomorphism. Canonically, $R_{(f)}\subset R_{(\mathfrak{p})}$ hence $\sigma_{\mathfrak{p}}:\left(R_{(f)}\right)_{\varphi(\mathfrak{p})} \simeq R_{(\mathfrak{p})}$, since any element in $\left(R_{(f)}\right)_{\varphi(\mathfrak{p})}$ are of the form 
    $$
        \frac{\frac{a}{f^n}}{\frac{x}{f^m}}
    $$
    with appropriate restrictions on degrees (i.e. that $\deg \ a -\deg \ f^n = \deg \ x -\deg \ f^m$) and in $R_{(\mathfrak{p})}$ this is equal to 
    $$
        \frac{af^m}{xf^n},
    $$
    and trivially any element in $R_{(\mathfrak{p})}$ can be written on this form. We can then define a natural isomorphism $\varphi^\#: \pazocal{O}_{\Spec\ R_{(f)}}\Rightarrow \varphi_\star \left.\pazocal{O}_{\Proj\ R}\right|_{D_{+}(f)}$. Indeed, for $U\subset \Spec \ R{(f)}$ open we let the component $\varphi^\#_U$ be given by 
    $$
        \varphi^\#_U(s): \mathfrak{p}\mapsto \sigma_\mathfrak{p}(s\circ \varphi)(\mathfrak{p})\in R_{(p)},
    $$ 
    whose inverse is given by 
    $$
        \left(\varphi^\#_U\right)^{-1}(s) : \mathfrak{p}\mapsto \sigma_{\mathfrak{p}}^{-1}(s\circ\varphi^{-1})(\mathfrak{p}).
    $$ 
    Note that this construction is same as mentioned in Remark~\ref{ConstructionOfNaturalTransformationOfSheavesFrom} so naturality follows. We thus conclude that $D_{+}(f)\simeq \Spec\ R_{(f)}$.
\end{proof}
\subsubsection{Reduced Schemes}
\begin{definition}
    A scheme $(X,\pazocal{O})$ is \emph{reduced} if for each $U\subset X$ open, $\pazocal{O}(U)$ is reduced. 
\end{definition}
\begin{definition}
    Given a scheme $(X,\pazocal{O})$, we can form the \emph{reduced scheme induced by $X$}, to be the pair $X_\mathrm{red}:=(X,\pazocal{O}_\mathrm{red})$ where (cf. Lemma~\ref{PresheafFromQuotientGroups})
    \begin{gather*}
        \pazocal{O}_\mathrm{red}: X\rightarrow \mathrm{CRing}
    \end{gather*}
    is the presheaf induced by $\pazocal{O}$ and $\{\mathrm{nil}(\pazocal{O}(U)) : U\subset X\}$. 
\end{definition}
\begin{remark}
    Note that $\pazocal{O}_\mathrm{red}$ defines a sheaf of rings, since it is just the quotient sheaf of $\pazocal{O}$, with $U\mapsto \mathrm{nil}(\pazocal{O}(U))$, hence $X_\mathrm{red}$ is a ringed space. It is a locally ringed space since for each $P$ by {\Large more general lemma about locally ringed spaces}.
\end{remark}
\begin{lemma}
    A scheme $(X,\pazocal{O})$ is reduced if and only if $\pazocal{O}_P$ is reduced for each $P\in X$.
\end{lemma}
\begin{proof}
    Suppose $(X,\pazocal{O})$ is reduced. Let $P\in X$ and  $[s_U]\in \pazocal{O}_P\setminus 0$. For any open neighborhood $V$ of $P$, $\left. s_U\right|_{U\cap V}^n \neq 0$ for any $n\geq 1$, hence $[s_U]\neq 0\in \pazocal{O}_p$. It thus follows that $\pazocal{O}_P$ is reduced.\\
    Suppose conversely that $(X,\pazocal{O})$ is not reduced. Then there is a point $P$ in an open set $U$ with $\pazocal{O}(U)$ not reduced. I.e. there is a section $s\in \pazocal{O}(U)\setminus 0$ and an $n\geq 1$ with $s^n = 0$. Then $[s]^n = 0\in \pazocal{O}_P$, hence $\pazocal{O}_P$ is not reduced.   
\end{proof}
\begin{lemma}
    Let $R$ be a ring. Then 
    \begin{enumerate}
        \item $\Spec\ R$ is reduced if and only if $R$ is reduced.
        \item $\Spec\ R_{\mathrm{red}}\simeq (\Spec\ R)_\mathrm{red}$
    \end{enumerate}
\end{lemma}
\begin{proof}
    1. If $\Spec\ R$ is reduced, then $\pazocal{O}(\Spec\ R)\simeq R$ is reduced. Conversely, suppose $R$ is reduced. Then $\pazocal{O}_{\mathfrak{p}}\simeq R_\mathfrak{p}$ is reduced for each $\mathfrak{p}\in\Spec\ R$.\\
    2. Consider the map $R\rightarrow R_\mathrm{red}$. It induces a homeomorphism $\varphi:\Spec\ R_\mathrm{red}\rightarrow \Spec \ R$ taking prime ideal $\mathfrak{p}/\mathrm{nil}(R)$ to its corresponding prime ideal $\mathfrak{p}$ in $\Spec\ R$ and a natural transformation $\varphi^\#: \pazocal{O}_{\Spec \ R}\Rightarrow \varphi_\star\pazocal{O}_{\Spec \ R_{\mathrm{red}}}$. Consider the canonical natural transformation $\pi : \pazocal{O}_{\Spec \ R}\Rightarrow (\pazocal{O}_{\Spec \ R})_{red}$ defined by $\pazocal{O}_{\Spec\ R}(U)\twoheadrightarrow (\pazocal{O}_{\Spec\ R})_\mathrm{red}(U)$. Then we get a natural transformation 
    $$
        \overline{\varphi^\#}:(\pazocal{O}_{\Spec\ R})_\mathrm{red} \Rightarrow \varphi_\star\pazocal{O}_{\Spec \ R_\mathrm{red}}
    $$  
    Explicitly for $U\subset \Spec \ R$, $s +\mathrm{nil}(\pazocal{O}_{\Spec\ R}(U))\mapsto \varphi^\#_U(s).$ under $\overline{\varphi^\#}_U$. From $\sigma_{\mathfrak{p}}:R_{\varphi(\mathfrak{p})}\twoheadrightarrow (R_\mathrm{red})_{\mathfrak{p}}$ we get the isomorphism $\widetilde{\sigma_\mathfrak{p}} : R_\mathfrak{\varphi(p)}/\mathrm{nil}(R_\mathfrak{\varphi(p)})\rightarrow (R_\mathrm{red})_\mathfrak{p}$. Consider the map
    \begin{gather*}
        \overline{\ev_\mathfrak{q}}:\pazocal{O}_{\Spec\ R,\mathfrak{q}}\rightarrow (R_\mathfrak{q})_\mathrm{red}\\
        [s_U]\mapsto s_U(\mathfrak{q})+\mathrm{nil}(R_\mathfrak{q})  
    \end{gather*}
    Note that if $s_U(\mathfrak{q})=\frac{a}{f}\in \mathrm{nil}(R_\mathfrak{q})$, then since $s$ is equal to the constant section $\frac{a}{f}$ in a sufficiently small neighborhood around $\mathfrak{q}$, it follows that $[s_U]\in \mathrm{nil}(\pazocal{O}_{\Spec\ R,\mathfrak{q}})$, hence 
    \begin{gather*}
        \widetilde{\ev_\mathfrak{q}} : ((\pazocal{O}_{\Spec\ R})_{\mathrm{red}})_\mathfrak{q}\simeq \pazocal{O}_{\Spec\ R,\mathfrak{q}}/\mathrm{nil}(\pazocal{O}_{\Spec\ R,\mathfrak{q}})\rightarrow (R_\mathfrak{q})_\mathrm{red}\\
        [s_U] + \mathrm{nil}(\pazocal{O}_{\Spec\ R,\mathfrak{q}}) \mapsto s_U(\mathfrak{q})+ \mathrm{nil}(R_\mathfrak{q})
    \end{gather*}
    The implicit isomorphism given in the domain of this map is readily verified. Indeed, we know it to be a fact in general that $(\pazocal{F}/\pazocal{G})_P \simeq \pazocal{F}_P/\pazocal{G}_P$. It is also clear that $\mathrm{nil}(\pazocal{O}_{\Spec \ R})_\mathfrak{q}=\mathrm{nil}(\pazocal{O}_{\Spec\ R, \mathfrak{q}})$. One then readily verfies that the square 
    $$
        \begin{tikzcd}
            \pazocal{O}_{\Spec \ R,\varphi(\mathfrak{p})}/\mathrm{nil}(\pazocal{O}_{\Spec\ R, \varphi(\mathfrak{p})}) \arrow[r,"\overline{\varphi^\#}_\mathfrak{p}"]\arrow[d,"\widetilde{\ev_{\varphi(\mathfrak{p})}}"] & \pazocal{O}_{\Spec\ R_\mathrm{red},\mathfrak{p}}\arrow[d,"\ev_{\mathfrak{p}}"]\\
            (R_{\varphi(\mathfrak{p})})_\mathrm{red}\arrow[r,"\widetilde{\sigma_\mathfrak{p}}"] & (R_\mathrm{red})_\mathfrak{p}
        \end{tikzcd}
    $$
    commutes. Then $\overline{\varphi^\#}_\mathfrak{p}$ is an isomorphism for each $\mathfrak{p}\in \Spec \ R_\mathrm{red}$, hence $\overline{\varphi^\#}$ is a natural isomorphism, hence $(\varphi, \overline{\varphi^\#}): (\Spec\ R)_\mathrm{red} \rightarrow \Spec \ R_\mathrm{red}$ defines an isomorphism of locally ringed spaces.  
\end{proof}
\begin{lemma}
    The assignment
    $$
        (X,\pazocal{O})\mapsto (X,\pazocal{O}_\mathrm{red})
    $$
    is functorial in the category of ringed spaces, hence if $(X,\pazocal{O}_X)\simeq (Y,\pazocal{O}_Y)$, then 
    $$
        X_\mathrm{red} \simeq Y_\mathrm{red}.
    $$ 
\end{lemma}
\begin{proof}
    Let a morphism of locally ringed spaces $\varphi: (X,\pazocal{O}_X)\rightarrow (Y,\pazocal{O}_Y)$ be given. Since the assignment $R\mapsto R_\mathrm{red}$ is functorial it follows readily that the assignment, 
    $$
        \varphi^\#_U\mapsto (s+\mathrm{nil}(\pazocal{O}_Y(U))\mapsto \varphi^\#_U(s)+\mathrm{nil}(\varphi_\star\pazocal{O}_X(U)))
    $$
    defines a natural transformation $(\pazocal{O}_Y)_\mathrm{red}\Rightarrow \varphi_\star(\pazocal{O}_X)\mathrm{red}$ and this assignment is functorial.  
\end{proof}
\begin{lemma}
    Given a scheme $(X,\pazocal{O})$, $X_\mathrm{red}$ is a reduced scheme. 
\end{lemma}
\begin{proof}
    It remains to check that $X_\mathrm{red}$ admits a cover of affine schemes. $(X,\pazocal{O})$ admits a cover of affine schemes. Let $U$ be such an open set. Then $(U,\left.\pazocal{O}\right|_{U})\simeq \Spec \ R$ for some ring $R$. Hence 
    $$
        (U,\left.\pazocal{O}_\mathrm{red}\right|_{U})=(U,\left.\pazocal{O}\right|_{U})_\mathrm{red}\simeq (\Spec\ R)_\mathrm{red}=\Spec\ R_\mathrm{red}.
    $$
    Thus $X_\mathrm{red}$ is a scheme. 
\end{proof}
Consider a scheme $(X,\pazocal{O})$. Let $\pi_{X_\mathrm{red}} :X_\mathrm{red}\rightarrow (X,\pazocal{O})$ be the morphism of schemes, whose underlying continuous function is the identity map and whose natural transformation has components 
\begin{gather*}
    (\pi_{X_\mathrm{red}})^\#_U : \pazocal{O}(U)\twoheadrightarrow \pazocal{O}_\mathrm{red}(U)\\
    s\mapsto s+ \mathrm{nil}(\pazocal{O}(U))
\end{gather*}
\begin{proposition}
     Consider a morphism of schemes $\varphi : X\rightarrow Y$ and suppose $X$ is reduced. Then there is a unique morphism of schemes $\phi : X\rightarrow Y_\mathrm{red}$ such that 
     $$
        \begin{tikzcd}
            X \arrow[r,"\phi"]\arrow[rd,"\varphi"] & Y_\mathrm{red}\arrow[d,"\pi_{Y_{\mathrm{red}}}"]\\
            & Y
        \end{tikzcd}
     $$
\end{proposition}
\begin{proof}
    Let $U\subset Y$ be open, then, using that $\mathrm{nil}(\pazocal{O}_{X}(U))=0$, we get a unique $\phi^\#_U: \pazocal{O}_{Y,\mathrm{red}}(U)\rightarrow \varphi_\star\pazocal{O}_X(U)$ such that 
    $$
        \begin{tikzcd}
            \pazocal{O}_{Y,\mathrm{red}}(U) & \varphi_\star\pazocal{O}_{X}(U)\arrow[l,"\phi^\#_U"']\\
            & \pazocal{O}_Y(U)\arrow[u, "(\pi_{Y_\mathrm{red}})^\#_U"']\arrow[ul,"\varphi^\#_U"]
        \end{tikzcd}
    $$
    commutes. One easily checks that these form the components of a natural transformation satisfying the condition. Uniqueness follows from using the uniqueness on the components of $\phi^\#$.  
\end{proof}
\subsubsection{Varieties \& Schemes}
\begin{definition}
    Given a scheme $S$, \emph{a scheme over $S$} is a scheme $X$ together with a morphism $X\rightarrow S$. A morphism of schemes over $S$ from $X$ to $Y$ is a morphism of schemes $X\rightarrow Y$ such that 
    $$
        \begin{tikzcd}
            X \arrow[r]\arrow[rd] & Y\arrow[d]\\
            & S 
        \end{tikzcd}
    $$ 
    commutes. We denote the category of schemes over $S$ by $\mathfrak{Sch}(S)$. And the category of schemes over the spectrum of a ring $R$ by $\mathfrak{Sch}(R)$.
\end{definition}
\begin{theorem}
    Fix an algebraically closed field $K$. Then there is fully faithful functor 
    $$
        t : \mathrm{Var}_K \rightarrow \mathfrak{Sch}(K)
    $$
    such that each variety $V$ is homeomorphic to the closed points of the underlying space of $t(V)$, $\mathrm{sp}(t(V))$.
\end{theorem}
\begin{proof}
    We first construct an endofunctor $t$ on $\mathrm{Top}$ and that this restricts to a fully faithful functor $t: \mathrm{Var}_K\rightarrow \mathfrak{Sch}(K)$. For a space $X$, let $t(X)$ denote the irreducible closed subsets of $X$. Note that if $Y\subset X$ is closed then $t(Y)\subset t(X)$, if $Y_1,Y_2\subset X$ are closed, then $t(Y_1\cup Y_2) = t(Y_1)\cup t(Y_2)$ and if $\{Y_\alpha\}$ is a family of closed subsets of $X$, then $t\left(\bigcap_\alpha Y_\alpha\right) = \bigcap_\alpha t(Y_\alpha)$. It follows that the sets of the form $t(Y)$ form the closed sets of a topology on $t(X)$. Consider a continuous map $f: X_1\rightarrow X_2$. Then we define
    \begin{gather*}
        t(f) : t(X_1)\rightarrow t(X_2)\\
        Y \mapsto \mathrm{cl}(f(Y))
    \end{gather*}
    Let $t(Z)\subset t(X_2)$. Then $t(f)^{-1}(t(Z))= t(f^{-1}(Z))$, which implies $t(f)$ is continuous. Indeed, if $\mathrm{cl}(f(V))\in t(Z)$, then $f(V)\subset \cl(f(V))\subset Z$, hence $V\subset f^{-1}(f(V))\subset f^{-1}(Z)$, hence $V\in t(f^{-1})$. Conversely, if $V\subset f^{-1}(Z)$ is irreducible, then $V\subset X$ is irreducible and $f(V)\subset Z$, hence $t(V)=\cl(f(V))\subset Z$, so $V\in t(f)^{-1}(t(Z))$. The assignment, $f\mapsto t(f)$ is clearly functorial, since for a continuous function $g$, $\cl(g(A))=\cl(g(\cl(A)))$ for any subset $A$ of $\dom \ g$. \\
    Consider the map 
    \begin{gather*}
        \alpha: X\rightarrow t(X)\\
        P\mapsto \cl(\{P\})
    \end{gather*}
    This is clearly continuous, since for $t(Y)\subset t(X)$ closed, 
    $$
        \alpha^{-1}(t(Y))=Y.
    $$
    There is a bijection between the closed sets in $X$ and the closed sets in $t(X)$, given by 
    $$
        Y\subset X \mapsto t(Y)
    $$
    with inverse 
    $$
        t(Y)\mapsto \alpha^{-1}(t(Y))=Y.
    $$
    Thus there is a bijection between open sets of $X$ and open sets of $t(X)$.\\
    Consider a variety $V$ over $K$ as a locally ringed space with the usual sheaf of regular functions $\pazocal{O}$. We claim that $(t(V),\alpha_\star\Gamma)$ is a schemes. Since any variety has an open cover of affine subvarieties, it suffices to prove the claim for $V$ affine {\Large cf. trivial lemma not yet written}. Now not that the maximal ideals of $\Gamma(V)$ are in one-to-one correspondence with the points $V$. Note moreover that $I_P(V) := \{f\in \Gamma(V): f(P)=0\}\subset \Gamma(V)$ is a maximal ideal in $\Gamma(V)$ distinct from every other maximal ideal in $\Gamma(V)$ (reference to some lemmas). It thus follows that 
    \begin{gather*}
        \beta: V\rightarrow \mathrm{Specm}\ \Gamma(V)\\
        P\mapsto I_P(V)
    \end{gather*}  
    is a bijection of $V$ with the maximal ideals of $\Gamma(V)$, which we denote $\mathrm{Specm}\ \Gamma(V)$. Now note that $\beta^{-1}(V(I)\cap \mathrm{Specm}\ \Gamma(V)) = Z(I)\subset V$ for each closed set $V(I)\cap \mathrm{Specm}\ \Gamma(V)\subset\mathrm{Specm}\ \Gamma(V)$. One also easily sees that $\beta$ is a closed map, hence it follows that $\beta$ is a homeomorphism. Now note that stalk of $\Spec \ \Gamma(V)$ of one these maximal ideals $I_P(V)$ is isomorphic to the local ring at $P$, $\pazocal{O}_P(V)$. Let $U\subset \Spec \ \Gamma(V)$ be open. Consider a section $s\in \pazocal{O}_{\Spec\ \Gamma(V)}(U)$. Let $P\in \beta^{-1}(U)$. Then we can map $s$ to some regular function $\overline{s}\in \pazocal{O}_P(V)$. It follows that 
    \begin{gather*}
        \beta_U^\#: \pazocal{O}_{\Spec\ \Gamma(V)}(U)\rightarrow \beta_\star\pazocal{O}(U)\\
        s\mapsto \overline{s}
    \end{gather*}
    Consider some regular function $z$ in $\pazocal{O}(\beta^{-1}(U))$. On some open set $W$. This can be represented as some rational function $\frac{a}{f}$ on $V$ defined on $W$, i.e. with $f\notin I_Q(V)$ for each $Q\in W$. Then $\beta(W)\subset U$ is an open set with $\frac{a}{f}\in \pazocal{O}_{\Spec \ \Gamma(V)}(U)$ (i.e. the constant map $\mathfrak{p}\mapsto \frac{a}{f}$) being a section that maps to $z$, hence $\beta_U^\#$ is surjective. Suppose $\overline{s}=0$. For some open subset $W\subset U$, $s$ is represented by a rational function $\frac{a}{f}$ defined at each $Q\in W$. Shrinking $W$ sufficiently, we see that $a\in \Gamma(V)$ agrees with $0\in \Gamma(V)$ on $W$. Since $W$ is dense in $V$ it follows that $a = 0$. We thus conclude that $s = 0$, hence $\beta_U^\#$ is injective. We thus get a natural isomorphism $\pazocal{O}_X \cong \beta_\star\pazocal{O}$. Consider the bijective map 
    \begin{gather*}
        \gamma : t(V)\rightarrow \Spec \ \Gamma(V)\\
        W\mapsto I_V(W)
    \end{gather*} 
    one readily verifies that this is in fact a homeomorphism. Note that 
    $$
        \begin{tikzcd}
            V\arrow [r,"\beta"]\arrow[d,"\alpha"] & \Spec \ \Gamma(V)\\
            t(V)\arrow[ur,"\gamma"]
        \end{tikzcd}
    $$
    commutes. Hence $\pazocal{O}_{\Spec\ \Gamma(V)}\simeq \beta_\star\pazocal{O} = (\gamma\alpha)_\star \pazocal{O}= \alpha_\star \gamma_\star \pazocal{O}$. Thus setting $\pazocal{O}_{t(V)} := \alpha_\star\pazocal{O}$, we thus get an isomorphism of ringed spaces $\Spec \ \Gamma(V)\simeq (t(V),\pazocal{O}_{t(V)})$. It follows that $(t(V),\pazocal{O}_{t(V)})$ is an affine scheme. There is a canonical ring map $K\rightarrow \Gamma(V)$, hence we get a morphism of schemes $(t(V),\pazocal{O}_{t(V)})\cong \Spec\ \Gamma(V)\rightarrow \Spec \ K$, hence $(t(V),\pazocal{O}_{t(v)})$ is a scheme over $K$.\\
    Let $V$ and $W$ be varieties over $K$. We show that 
    \begin{gather*}
        t : \mathrm{Var}_K(V,W) \rightarrow \mathfrak{Sch}(K)(t(V),t(W))\\
        \varphi \mapsto t(\varphi)
    \end{gather*}  
    is bijective 
\end{proof}

\section*{Research statement Bristol}

{\Large Intro/Big Picture}\\
My main interest is in algebraic geometry, particularly in computational algebraic geometry. In first introductions to algebraic geometry and commutative one is presented with results, which shows the existence of objects and in a lot of cases it is not explicitly made clear if and how this can be computed explicitly. I am interested in how with going into detail about how to compute these objects.  

{\Large master's Thesis Talk}\\
My master's thesis, \emph{Two approaches to an effective Nullstellensatz} studied constructive versions of the weak Nullstellensatz. The weak Nullstellensatz, asserts that if $f_1,\dots,f_m\in K[x_1,\dots,x_n]$ have no common zeroes over an algebraically closed field $K$, then there are $g_1,\dots,g_m\in K[x_1,\dots,x_n]$ so that 
$$
    g_1f_1+\dots + g_mf_m = 1.
$$
The effective Nullstellensatz concerns finding an algorithm for computing the polynomials $g_1,\dots,g_m$. As the title of my thesis suggests, I presented two approaches to solving this problem.

One way to make the Nullstellensatz effective is to bound the degrees of the
polynomials $g_i$ in such a representation. If the input polynomials $f_i$ have degree at most $d$, an upper bound $B(d)$ on the degrees of the $g_i$ turns the problem, into solving a system of linear equations. In my thesis I followed an approach due to Jelonek {\Large ref}, which uses methods from classical algebraic geometry to find a bound of $d^n$.

What I appreciated in this part of the project was that it relied on concrete and rather elementary results. I found the way that these results  

The second approach in my thesis is based on Gröbner basis theory. The effective Nullstellensatz can be reformulated in terms of
so-called final polynomials. A result of Sturmfels {\Large ref} claims that such a final polynomial appears as an element of a Gröbner basis of a certain ideal with respect to a suitable term order. During this work it became clear that the statement is not correct as stated, since counterexamples were identified. This led to a augmented version of the result, which I proved and which does yield a Gröbner basis construction of final polynomials. A joint preprint on this work is available on arXiv~{\Large Ref.}

This part of the project was formative for me as forced me to reflect on what made the original approach fail and how to mend this failure. I see this as an essential aspect of doing original mathematical research.
{\Large Wishes for future research }
For