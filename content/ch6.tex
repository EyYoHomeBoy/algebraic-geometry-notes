% !TEX root = ../main.tex
\section{Algebraic Geometry using Schemes}

\subsection{Sheaves \& Presheaves}

\subsubsection{Presheaves}

\begin{definition}
    Consider a category $\mathcal{C}$. A \emph{presheaf} on $\mathcal{C}$ is a contravariant functor $\pazocal{F}: \mathcal{C}\rightarrow \mathrm{Set}$.\\
    For a subcategory $\mathcal{C}'$ of $\mathrm{Set}$, a \emph{$\mathcal{C}'$-valued presheaf} on $\mathcal{C}$ is a contravariant functor $\pazocal{F} : \mathcal{C}\rightarrow \mathcal{C}'$. We denote the image of a morphism $\varphi: V \rightarrow U$ under $\pazocal{F}$ by $\varphi_{UV}$.
\end{definition}
\begin{remark}
    We will mainly be interested in the case of $\mathcal{C}$ being obtained from a topological space: Consider a topological space $(X,\tau_X)$. We consider the category $\pazocal{T}(X)$ induced by the preorder on $\tau_X$. This will be called a \emph{presheaf on $X$}. Moreover we will mainly be interested in presheaves valued in algebraic categories such as $\mathrm{Group},\mathrm{AbGroup},\mathrm{Ring}$, etc.\\  
\end{remark}
\begin{example}
    Let $X$ be a variety. Then $$\Gamma: (\tau_X,\subset) \rightarrow \mathrm{FinAlg}_K, (U, W\subset V)\mapsto (\Gamma(U), \widetilde{\varphi}: \Gamma(V)\rightarrow \Gamma(W), f \mapsto \left.f\right|_{W})$$
    defines a presheaf on $\pazocal{T}(X)$ respect to the Zariski topology. 
\end{example}
\begin{definition}\label{RestrictionInImageOfPresheaf}
    For objects $U,V$ in $\pazocal{C}$ and a morphism $\varphi: V\rightarrow U$, for an  $s\in \pazocal{F}(U)$, we define \emph{restriction of $s$ to $V$} to be $$\left. s\right|_{V}:= \varphi_{UV}.$$
\end{definition}
\subsubsection{Sheaves}
\begin{definition}
    Consider a topological space $X$. Let $\pazocal{F}$ be a presheaf on a category $\pazocal{T}(X)$. $\pazocal{F}$ is called a \emph{sheaf on $X$} if moreover, 
    \begin{enumerate}
        \item For every open $U\subset X$ and every open cover $\{U_i\}$ of $U$ and every $\phi,\varphi\in \pazocal{F}(U)$, if $\left.\varphi\right|_{U_i}= \left.\phi\right|_{U_i}$ for every $i$, then $\varphi=\phi$.
        \item For every open $U\subset X$ and every open covering $\{U_i\}$ and for every collection of elements (functions) 
        $$\left\{s_i\in \pazocal{F}(U_i): \left.s_i\right|_{U_i\cap U_j}=\left. s_j\right|_{U_i\cap U_j} \text{ for every }j\right\}_{i}$$
        there is an element $s\in \pazocal{F}(U)$ such that $\left.s\right|_{U_i}=s_i$ for each $i$.   
    \end{enumerate}
    A sheaf valued in a subcategory of $\mathrm{Set}$ is defined in a similar way for sheaves to how it was defined for presheaves. 
\end{definition}
\begin{remark}
    By condition 1. an element $s\in \pazocal{F}(U)$ obtained as described in 2. is unique.
\end{remark}

\begin{lemma}
    Consider a topological space $X$ and a presheaf $\pazocal{F}$ on $X$ valued in the category of abelian groups that satisfies the second condition of being a sheaf. Then $\pazocal{F}$ is a sheaf on $X$ if and only if for every open $U\subset X$ every open covering $\{U_i\}$ and every $s\in \pazocal{F}(U)$, if $\left.s\right|_{U_i}=0$ for every $i$, then $s=0$.   
\end{lemma}
\begin{proof}
    "$\implies$": If $\pazocal{F}$ is a sheaf on $X$, then since $\left. s\right|_{U_i}=0=\left.0\right|_{0}$ for every $i$, then $s=0$.\\
    "$\impliedby$": If $\left. s\right|_{U_i} = \left. t\right|_{U_i}$ for every $i$, hence 
    $$0=\left. s\right|_{U_i} - \left. t\right|_{U_i} = \left. s-t\right|_{U_i}$$
    for every $i$, which by assumption means $s-t=0$. 
\end{proof}
\begin{example}
    The presheaf $\Gamma$ on a variety $X$ seen as a space equipped with the Zariski topology is a sheaf. Consider an subvariety $U\subset X$ and open covering $\{U_i\}$ of $U$ and suppose there is a regular function $f\in \Gamma(U)$ such that $\left. f\right|_{U_i}=0$ for every $i$. Viewing $f$ as a function $U$ it follows that given a point $x\in U$ since $x\in U_i$ for some $i$, 
    $$f(x)=\left.f\right|_{U_i}(x)= 0 \implies f = 0.$$
    Consider an open covering, $\{U_i\}$ of a subvariety $U\subset X$ and a collection of regular functions 
    $$\left\{f_i\in \Gamma(U_i): \left.f_i\right|_{U_i\cap U_j}= \left. f_j\right|_{U_i\cap U_j} \right\}_i,$$
    then the function 
    \begin{gather*}
        f : U \rightarrow K \\
         x\mapsto f_i(x) \quad \text{if } x\in U_i
    \end{gather*} 
    is a regular function on $U$ such that $\left. f\right|_{U_i} = f_i$. Indeed, restriction in the sense of Definition~\ref{RestrictionInImageOfPresheaf} is in this instance literally restriction in the usual sence, so the definition of $f$ is indenpendent of the choice of $U_i$.
\end{example}
Recall the following definition
Recall from \ref{DirectLimits} the definition of a direct limit. We shall construct the direct limit in the category of groups
\begin{definition}
    Let $G_\bullet: (I,\leq) \rightarrow \mathrm{grp}$ be a direct system and the associated collections of objects and morphisms $(\{G_i\},\{f_{ij}\})$. Given elements $(x_i,i),(x_j,j)$ in  
    $$
        \bigsqcup_{i\in I} G_i
    $$
    we define an equivalence relation $\sim$, by postulating $x_i\sim x_j$ if there exists $k\in I$ with $i,j\leq k$ and 
    $$
        f_{ik}(x_i) = f_{jk}(x_j).
    $$
    The \emph{direct limit of $G_\bullet$} is defined to be the set 
    $$
        \varinjlim G_i := \bigcup_{i\in I} G_i/\sim.
    $$
    Given a pair of elements $[x_i],[x_j]\in \varinjlim G_i$ we define 
    $$
        [x_i][x_j] := [f_{ik}(x_i)f_{jk}(x_j)]
    $$
    where $k\in I$ is an element such that $i,j\leq k$. There is a canonical natural transformation $\phi: G_\bullet \Rightarrow \varinjlim G_i$ given by
    $$\{G_i \rightarrow \varinjlim G_i, x_i \mapsto [x_i]\}$$
\end{definition}
\begin{remark}
    By functioriality the definition of the binary operation of the direct limit of $\{G_i\}$ is independent of the choice of $k$. Since for $i\leq l$, 
    $$
        f_{il}(x_i) = f_{ll}f_{il}(x_i) \implies [x_i]=[f_{il}(x_i)].
    $$
    This means that for every $h\geq k$, 
    $$
        [f_{ik}(x_i)f_{jk}(x_j)] = [f_{kh}(f_{ik}(x_i)f_{jk}(x_j))] =[(f_{kh}f_{ik})(x_i)(f_{kh}f_{ik})(x_j)]=[f_{ih}(x_i)f_{jh}(x_j)].
    $$
\end{remark}
\begin{lemma}
    Keeping the notation from the prior definition, the set $\varinjlim G_i$ with the described binary operation is a group. When the direct system is valued in $\mathrm{ab}$, the direct limit is an abelian group
\end{lemma}
\begin{proof}
    first we check that the operation is well defined. Let $([x_i],[x_j]),([x_p],[x_q])$ in the direct limit of $\{G_i\}$. Then $f_{ik}(x_i)=f_{pk}(x_p)$ and $f_{jh}(x_j)=f_{qh}(x_q)$ for suitable $k,h\in I$ with $i,p\leq k$ and $j,q\leq h$. Pick an $l\in I$ with $h,k\leq l$. Then 
    $$
        [x_i][x_j]=[f_{il}(x_i)f_{jl}(x_j)]=[f_{pl}(x_p)f_{ql}(x_q)] = [x_p][x_q].
    $$
    One readily verifies that this operation is associative.
    Note that since each $f_{ij}$ is a group homomorphism, the identity of each $G_i$ is in the same equivalence class and one sees that $e:=[e_i]$ is neutral with respect to the binary operation in $\varinjlim G_i$. Lastly defining $[x_i]^{-1} = [x_i^{-1}]$, one readily verifies that this is independent of the choice of representative and that it is inverse element of $[x_i]$.
\end{proof}
\begin{lemma}
    Consider a direct system $(\{G_i\},\{f_{ij}\})$ of groups. Then $\varinjlim G_i$ with $\phi: G_\bullet \Rightarrow \varinjlim G_i$ is the direct limit in the category theoretic sense.
\end{lemma}
\begin{proof}
    Let a cocone $(G,\lambda)$ of the direct system be given. We define 
    \begin{gather*}
        u : \varinjlim G_k \rightarrow G\\
        [x_i] \mapsto \lambda_i(x_i)
    \end{gather*}
    This is well defined by the naturality of $\{\lambda_i\}$. Let $i,j\in I$ with $i\leq j$ be given. Clearly, $u$ is the unique homomorphism making 
    $$
        \begin{tikzcd}
            G_i \arrow[rr,"f_{ij}"] \arrow[rd, "\phi_i"]\arrow[rdd, "\lambda_i",bend right = 20] && G_j \arrow[ld,"\phi_j"]\arrow[ldd,"\lambda_j",bend left = 20]\\
            & \varinjlim G_k \arrow[d," u"]\\
            & G
        \end{tikzcd}
    $$    
    commute. We thus conclude that our construction is indeed the direct limit in the categorical sense. 
\end{proof}
\begin{definition}
    Consider a topological space $X$ and a presheaf $\pazocal{F}$ on $X$ valued in $\mathrm{ab}$. Let $P\in X$. We define the \emph{stalk of $\pazocal{F}$ at $P$} to be  
    $$
        \pazocal{F}_P := \varinjlim_{U\ni P} \pazocal{F}\in \mathrm{ab}. 
    $$
    given an element $s\in \pazocal{F}(U)$ we call $[s]_P := [s]$ a \emph{germ} of $f$ at $P$. 
\end{definition}
\begin{remark}
    Note that by construction $(U,s),(V,t)\in \pazocal{F}_P$ are equal if and only if there is some open neighborhood of $P$,  $W\subset U\cap V$ with 
    $$\left. s\right|_{W} = \left. t\right|_W.$$
    There is a natural transformation 
    \begin{gather*}
        \phi : \pazocal{F}(U) \Rightarrow \pazocal{F}_p\\
        s\mapsto (U,s)
    \end{gather*}
\end{remark}
\begin{definition}
    Given a category $\pazocal{C}$ \emph{a morphism of presheaves} is just a morphism in the functor category $\mathrm{Set}^{\pazocal{C}\op}$, i.e. a morphism $\Phi: \pazocal{F} \rightarrow \pazocal{G}$ of parallel presheaves on $\pazocal{C}$ is just a natural transformation $\Phi : \pazocal{F}\Rightarrow \pazocal{G}$.\\
    An isomorphism of presheaves is therefor just a natural isomorphism of presheaves. \\
    \emph{A morphism of sheaves} is just a morphism of presheaves in the category of sheaves, $\mathrm{Sh}(X)$, which is a subcategory of $\mathrm{Set}^{\pazocal{T}(X)\op}$.   
\end{definition}
\begin{definition}
    Consider a parallel pair of presheaves $\pazocal{F},\pazocal{G}: X\rightarrow \mathrm{ab}$ and fix a point $P\in X$. For a morphism of presheaves $\Phi : \pazocal{F}\Rightarrow \pazocal{G}$, there is an induced morphism of stalks at $P$ given by
    \begin{gather*}
        \Phi_P : \pazocal{F}_P \rightarrow \pazocal{G}_P\\
        (U,s)\mapsto (U,\Phi_U(s))
    \end{gather*}
\end{definition}
\begin{remark}
    Alternatively, we can describe the above map as taking a germ $[s]_P$ to $[\Phi_U(s)]_P$ where $U$ is any open neighborhood of $P$. 
\end{remark}
\begin{proposition}
    Consider a parallel pair of sheaves $\pazocal{F},\pazocal{G}: X\rightarrow \mathrm{ab}$ and morphism of sheaves $\Phi: \pazocal{F}\Rightarrow \pazocal{G}$. Then $\Phi$ is an isomorphism if and only if $\Phi_P:\pazocal{F}_P\rightarrow \pazocal{G}_P$ is an isomorphism for each $P\in X$.  
\end{proposition}
\begin{proof}    
\end{proof}

\begin{definition}
     Consider a parallel pair of presheaves $\pazocal{F},\pazocal{G}: \pazocal{C}\rightarrow \mathrm{ab}$ and morphism of sheaves $\Phi: \pazocal{F}\Rightarrow \pazocal{G}$. 
    \begin{itemize}
        \item We define \emph{the presheaf kernel of $\Phi$} to be the presheaf 
    \begin{gather*}
        \ker \ \Phi: \pazocal{C} \rightarrow \mathrm{ab}\\
        U \mapsto \ker\ \Phi_U\\
        U\rightarrow V \mapsto \left.\varphi_{UV}\right|_{\ker\ \Phi_V} : \ker \ \Phi_V \rightarrow \ker \ \Phi_U
    \end{gather*}
    \item  We define \emph{the presheaf cokernel of $\Phi$} to be the presheaf 
    \begin{gather*}
        \mathrm{pscoker} \  \Phi: \pazocal{C}\rightarrow \mathrm{ab}\\
        U \mapsto \coker\ \Phi_U\\
        U\rightarrow V \mapsto \coker\ \Phi_V \rightarrow \coker\ \Phi_U, x+ \im \ \Phi_V \mapsto \pi_{UV}(x) + \im\ \Phi_U  
    \end{gather*}
    \item We define \emph{the presheaf image of $\Phi$} to be the presheaf
    \begin{gather*}
        \mathrm{psim}\ \Phi : \pazocal{C}\rightarrow \mathrm{ab}\\
        U\mapsto \im \ \Phi_U\\
        U\rightarrow V\mapsto \left.\pi_{UV}\right|_{\im \ \Phi_V} \im \ \Phi_V\rightarrow \im \ \Phi_U, x\mapsto \pi_{UV}(x) 
    \end{gather*}
    \end{itemize}
\end{definition}
\begin{remark}
    Unsurprisingly, the welldefinedness of the above presheaves are ensured by naturality. Functorialty follows from the functioriality of $\pazocal{F}$ and $\pazocal{G}$.
\end{remark}
\begin{lemma}
    The presheaf kernel of a morphism of presheaves is sheaf.
\end{lemma}
\begin{proof}
    
\end{proof}
\begin{lemma}
    Consider a presheaf $\pazocal{F}:X\rightarrow \mathrm{ab}$ on topological space $X$. There is a sheaf $\pazocal{F}^{+}: X\rightarrow \mathrm{ab}$ and morphism of presheaves $\Theta : \pazocal{F}\Rightarrow\pazocal{F}^{+}$ such that for any sheaf $\pazocal{G}:X\rightarrow \mathrm{ab}$ and any morphism of presheaves $\Phi : \pazocal{F} \Rightarrow \pazocal{G}$ there is a unique morphism of sheaves $\Psi : \pazocal{F}^{+}\Rightarrow \pazocal{G}$ such that $\Phi = \Psi \Theta$. Furthermore $(\pazocal{F}^{+},\Theta)$ is unique up to unique isomorphism. 
\end{lemma}
\begin{proof}
    We construct a presheaf
    \begin{gather*}
        X\rightarrow \mathrm{ab}
    \end{gather*}
    Given an open subset $U$ in $X$, we define $\pazocal{F}^{+}(U)$ to be the group of elements $([s]_P)$ in 
    $$\prod_{P\in U} \pazocal{F}_P$$
    such that for each $P\in U$ there is an open neighborhood of $P$, $W\subset U$ with an element $t\in \pazocal{F}(W)$ such that for each $Q\in V$ the germ $[t]_Q\in \pazocal{F}_Q$ of $t$ at $Q$ satisfies
    $$([s]_P)(Q) = [t]_Q.$$
    Given $U\subset V$ we define $\varphi_UV : \pazocal{F}^{+}(V)\rightarrow \pazocal{F}^{+}(U)$ to be given by restriction. It thus readily follows that $\pazocal{F}^{+}$ is a sheaf. We have a natural transformation
    \begin{gather*}
        \Theta : \pazocal{F}\Rightarrow \pazocal{F}^{+}\\
        \{\Theta_U : \pazocal{F}(U)\rightarrow \pazocal{F}^{+}(U), s \mapsto ([s]_P)\}.
    \end{gather*}
    Indeed, $(U,s)$ is a pair witnessing the fact that $[s]_P\in \pazocal{F}^{+}$, so this is well-defined. Let a natural transformation of presheaves $\Phi : \pazocal{F}\Rightarrow \pazocal{G}$ be given. We define a natural transformation of sheaves 
    \begin{gather*}
        \Psi: \pazocal{F}^{+} \Rightarrow \pazocal{G}
    \end{gather*}
    defined by the collection of morphisms  
    $$
        \left.\left\{ \Psi_U: \pazocal{F}^{+}(U) \rightarrow \pazocal{G}(U), ([s]_P)\mapsto \Psi_U(s) \ \right| \   U\subset X \text{ open}\right\}
    $$
    This is well-defined since $\pazocal{G} \cong \pazocal{G}^{+}$ or in other words since $\pazocal{G}(U)\simeq \pazocal{G}^{+}(U)$. $\Psi$ is the unique morphism of sheaves making the diagram
    $$
        \begin{tikzcd}
            \pazocal{F} \arrow[rd,Rightarrow, "\Phi"] \arrow[d,"\Theta",Rightarrow]\\
            \pazocal{F}^{+} \arrow[r, dashrightarrow] & \pazocal{G}
        \end{tikzcd}
    $$
    commute. Note that $(\pazocal{F}^{+},\Theta)$ is thus the coequalizer two copies of the identity transformation of $\pazocal{F}$ (cf. Example~\ref{EqualisersAndCoequalisers}) which means it is unique up to unique isomorphism.
\end{proof}
\begin{remark}
    We call $\pazocal{F}^{+}$ \emph{the sheaf associated with $\pazocal{F}$}
\end{remark}
\begin{definition}
    A \emph{subsheaf} of a sheaf $\pazocal{F}$ is a sheaf $\pazocal{F}'$ such that for every open set $U$ in $X$, $\pazocal{F}'(U)\leq \pazocal{F}(U)$ and, letting $\varphi_{UV}:\pazocal{F}(V)\rightarrow \pazocal{F}(U)$ denote the image of $U\subset V$ under $\pazocal{F}$, the image of $U\subset V$ under $\pazocal{F}'$ is the restriction of $\varphi_{UV}$ to $\pazocal{F}(V)$.
\end{definition}
\begin{remark}
    Note that given $(U,s)\in \pazocal{F}'_P$, we have that $s\in \pazocal{F}'(U)\subset \pazocal{F}(U)$, hence $(U,s)\in \pazocal{F}_P$. 
\end{remark}
\begin{example}
    The presheaf kernel of morphism of sheaves $\Phi:\pazocal{F}\Rightarrow \pazocal{G}$ is sheaf called the \emph{sheaf kernel}. By construction $(\ker \ \Phi)(U) = \ker \ \Phi_U \leq \pazocal{F}(U)$ and 
    $$(\ker \ \Phi)(U\subset V) = \left. \varphi_{UV}\right|_{\ker\ \Phi_U},$$
    hence $\ker \ \Phi$ is a subsheaf of $\pazocal{F}$. 
\end{example}
\begin{definition}
    A morphism of sheaves $\Phi:\pazocal{F}\Rightarrow \pazocal{G}$ is \emph{injective} if $\ker \ \Phi=0$, i.e. the sheaf that takes every open subset of $X$ to the trivial group.
\end{definition}
\begin{remark}
    $\Phi$ being injective is equivalent to $\Phi_U$ being injective for each open set $U$. 
\end{remark}
\begin{definition}
    For a morphism of sheaves $\Phi: \pazocal{F}\Rightarrow \pazocal{G}$, we define the \emph{image of $\Phi$} to be $\im \ \Phi:= (\mathrm{psim}\ \Phi)^{+}$
\end{definition}
\begin{proposition}
    Given a morphism of sheaves $\Phi:\pazocal{F}\Rightarrow \pazocal{G}$, the canonical morphism $\im\ \Phi\Rightarrow \pazocal{G}$ is injective, hence it can be indentified with a subsheaf of $\pazocal{G}$. 
\end{proposition}
\begin{proof}
    
\end{proof}
\begin{definition}
    A morphism of sheaves $\Phi:\pazocal{F}\Rightarrow \pazocal{G}$ is \emph{surjective} if $\im\ \Phi = \pazocal{G}$.
\end{definition}
\begin{definition}
    A sequence 
    $$
        \begin{tikzcd}
            \dots \arrow[r, Rightarrow, "\Phi^{i-1}"] & \pazocal{F}^{i-1} \arrow[r, Rightarrow, "\Phi^i"] & \pazocal{F}^i \arrow[r, Rightarrow, "\Phi^{i+1}"] & \pazocal{F}^{i+1} \arrow[r, Rightarrow] & \dots
        \end{tikzcd}
    $$
    is \emph{exact}, if $\ker\ \Phi^i = \im \ \Phi^{i-1}$ for each $i \in \Z$. 
\end{definition}
\begin{definition}
    Consider a sheaf $\pazocal{F}$ and subsheaf $\pazocal{F}'\subset \pazocal{F}$. We define the \emph{quotient sheaf $\pazocal{F}/\pazocal{F}'$} to be the sheaf associated with 
    \begin{gather*}
        X\rightarrow \mathrm{ab}\\
        U\mapsto \pazocal{F}(U)/\pazocal{F}'(U)\\
        U\subset V \mapsto \pazocal{F}(V)/\pazocal{F}'(V) \rightarrow \pazocal{F}(U)/\pazocal{F}'(U), s + \pazocal{F}'(V) \mapsto \left. s\right|_{U}+\pazocal{F}'(U)
    \end{gather*}
\end{definition}
\begin{remark}
    Fix a point $P\in X$ and consider $(U,s)\in (\pazocal{F}/\pazocal{F}')_P$. Note that then $s\in \pazocal{F}(U)/\pazocal{F}'(U)$, hence $(U,s)\in \pazocal{F}_P/\pazocal{F}'_P$.
\end{remark}
With the above in mind we can make the following definition
\begin{definition}
    For a morphism of sheaves $\Phi : \pazocal{F}\Rightarrow \pazocal{G}$, we define the \emph{cokernel of $\Phi$} to be $\coker\ \Phi := (\mathrm{pscoker}\ \Phi)^{+}$.
\end{definition}
\begin{definition}
    For a continuous map $f : X\rightarrow Y$ and a sheaf $\pazocal{F}$ on $X$, we define the \emph{direct image sheaf}, $f_\star\pazocal{F}$, on $Y$, to be the sheaf
    \begin{gather*}
        Y \rightarrow \mathrm{ab}\\
        U \mapsto \pazocal{F}(f^{-1}(U))\\
        U\subset V \mapsto \varphi_{f^{-1}(U)f^{-1}(V)}
    \end{gather*}
\end{definition}
\begin{remark}
    We check that this indeed a sheaf {\Large Check!}
\end{remark}
\begin{definition}
    For a continuous map $f : X\rightarrow Y$ and a sheaf $\pazocal{F}$ on $Y$, we define the \emph{inverse image sheaf}, $f^{-1}\pazocal{G}$ on $X$ to be the sheaf associated with the presheaf
    \begin{gather*}
        X\rightarrow \mathrm{ab}\\
        U\mapsto \varinjlim_{V\supset f(U)}\pazocal{G}(V)\\
        U\subset V\mapsto \varphi_{\lim_{W\supset f(U)}\pazocal{G}(W)\lim_{W\supset f(V)}\pazocal{G}(W)}
    \end{gather*}
\end{definition}
\begin{definition}
    Consider a topological space $X$, a subspace $Z\subset X$ and a sheaf $\pazocal{F}$ on $X$. Let $iota$ denote the inclusion map $Z\hookrightarrow X$. We define \emph{the restriction of $\pazocal{F}$ to $Z$} to be the sheaf $\left.\pazocal{F}\right|_{Z}:= \iota^{-1}\pazocal{F}$.
\end{definition}
\begin{remark}
    Fix a point $P\in Z$. Let $(U,s)\in \left(\left. \pazocal{F}\right|_Z\right)_P$... {\Large Do!}
\end{remark}
\subsection{Schemes}
We adopt the convention that all rings are commutative.
\subsubsection{Affine Schemes}
\begin{definition}
    For a ring $R$, we define \emph{the spectrum of $R$} to be the set
    $$
        \Spec\ R := \{ I \subset R : I \text{ is a prime ideal}\}
    $$
    For an ideal $I\subset R$, we define 
    $$
        V(I) := \{ J\in \Spec \ R : I\subset J\}
    $$
\end{definition}
\begin{lemma}
    If $I\subset J\subset R$ are ideals, then $V(I)\supset V(J)$.
\end{lemma}
\begin{proof}
    This is trivial. 
\end{proof}
\begin{lemma}
    For ideals $I,J\subset R$
    $$V(IJ) = V(I)\cup V(J).$$
\end{lemma}
\begin{proof}
    Consider a prime ideal $\mathfrak{P}\in \Spec \ R$ containing $IJ$. Then given $ab\in IJ$ with $a\in I$ and $b\in J$ we have that $a \in \mathfrak{P}$ or $b\in \mathfrak{P}$, hence $\mathfrak{P}\in V(I)\cup V(J)$. The converse inclusion follows from $I,J\supset IJ$ by the prior lemma. 
\end{proof}
\begin{lemma}
    For a family of ideals $\{I_\mu\}$ in $R$, 
    $$V(\sum_\mu I_\mu) = \bigcap_\mu V(I_\mu).$$ 
\end{lemma}
\begin{proof}
    Again one inclusion follows from $I_\nu\subset \sum_\mu I_\mu$. Suppose conversely that $\mathfrak{P}\in \Spec \ R$ is given such that it contains every $I_\mu$, then since ideals are closed under addition it contains $\sum_\mu I_\mu$.  
\end{proof}
\begin{proposition}
    For a ring $R$, the family of sets 
    $$
        \{V(I) : I \text{ is an ideal in } R\}
    $$
    are the closed set in a topology on $\Spec \ R$. 
\end{proposition}
\begin{proof}
    Indeed, note that 
    $$\bigcap_{I\subset R} V(I) = V\left(\sum_{I\subset R} I\right)= V(R) = \emptyset.$$
    and that $\Spec\ R = V(0)$. The two prior lemmas show that this family is closed under countable union and intersection over arbitrary index sets. 
\end{proof}
\begin{remark}
    The topology in question is unsurprisingly called the \emph{Zariski topology on $\Spec \ R$}. For an algebraically closed field $K$ and an affine variety $V\subset \A^n$, the points of $V$ correspond to the maximal ideals of $\Gamma(V)$. In this way we can identify $V$ with the subspace of maximal ideals of $\Gamma(V)$ in $\Spec \ \Gamma(V)$. In this sense $\Spec \ \Gamma(V)$ is a refinement of the notion of point in a variety. Namely, \emph{point} in $V$ have by definition come to mean the \emph{subvariety} of $V$, when we take the view that our vantage point lies in $\Spec\ \Gamma(V)$ when studying $V$.
\end{remark}
\begin{lemma}
    If $I,J\subset R$ are ideals, then $V(I)\subset V(J)$ if and only if $\rad(I)\supset \rad(J)$.
\end{lemma}
\begin{proof}
    Note that $V(I)\subset V(J)$ if and only if every prime ideal that contains $I$ also contains $J$ which again is equivalent to 
    $$
        \bigcap_{\mathfrak{p}\supset I \text{ prime }} \mathfrak{p} \supset \bigcap_{\mathfrak{p}\supset J \text{ prime }} \mathfrak{p}, 
    $$
    which by lemma \ref{RadicalOfIdealIsIntersectionOfAllPrimeIdealsContainingIt} is equivalent to 
    $$
        \rad(I)\supset \rad(J).
    $$
\end{proof}
Given a ring $R$ and prime ideal $\mathfrak{p}$ in $R$. We denote the localization of $R$ with respect to $\mathfrak{p}$ by $R_\mathfrak{p}$.
\begin{definition}
    Given a ring $R$, we define a presheaf on $\Spec \ R$ by, 
    \begin{gather*}
        \pazocal{O}:(\Spec\ R, \tau_{\Spec\ R})\op\rightarrow \mathrm{Ring}\\
        U \mapsto \prod_{\mathfrak{p}\in U} R_\mathfrak{p}\\
        U\subset V \mapsto (\prod_{\mathfrak{p}\in V} R_\mathfrak{p}\rightarrow \prod_{\mathfrak{p}\in U} R_\mathfrak{p}, s\mapsto \left. s\right|_U) 
    \end{gather*}
\end{definition}
\begin{remark}
    Clearly this presheaf is a sheaf, since the elements of $\pazocal{O}(U)$ are functions and restriction is just set function restriction in the same we saw in the example discussing the presheaf $\Gamma$. 
\end{remark}
\begin{definition}
    For $f\in R$, we define 
    $$
        D(f) := (\Spec \ R)\setminus V(\langle f\rangle)
    $$
\end{definition}
\begin{lemma}
    For a ring $R$,
    $$
        \{D(f)\subset \Spec\ R : f\in A\}
    $$
    forms a basis of the Zariski topology on $\Spec\ R$.
\end{lemma}
\begin{proof}
    Indeed, for $\mathfrak{p}\notin V(I)$, there is an $f\in I$ not in $\mathfrak{p}$, hence $\mathfrak{p}\notin V(\langle f\rangle)$, meaning $\mathfrak{p}\in D(f)$. Then $V(I)\cap D(f) = \emptyset$. This means that every point in an open subset $U$ in $\Spec \ R$ has an open neighborhood contained in $U$. It follows that the collection in question forms an open basis for the Zariski topology. 
\end{proof}
\begin{lemma}
    Consider a ring $R$. Then 
    \begin{enumerate}
        \item For every point $\mathfrak{p}\in \Spec\ R$
        $$
            \pazocal{O}_\mathfrak{p} \simeq R_\mathfrak{p}.
        $$
        \item For any $f\in R$,
        $$
            \pazocal{O}(D(f)) \simeq R_f.
        $$
        \item In particular,
        $$
            \Gamma(\Spec\ R,\pazocal{O}):= \pazocal{O}(\Spec\ R) = D(1) \simeq R_1 = R.
        $$
    \end{enumerate}    
\end{lemma}
\begin{proof}
    
\end{proof}
\begin{definition}
    A \emph{ringed space} is a topological space $X$ together with a sheaf 
    $$
        \pazocal{O}_X : X \rightarrow \mathrm{CRing}.
    $$
    A \emph{locally ringed space} is a ringed space $(X,\pazocal{O}_X)$ such that each stalk $\pazocal{O}_{X,P}$ for $P\in X$ is a local ring
\end{definition}
\begin{definition}
    A morphism of ringed spaces, $(X,\pazocal{O}_X)\rightarrow (Y,\pazocal{O}_Y)$ is a continuous map $f: X\rightarrow Y$ and a natural transformation $f^\# : \pazocal{O}_Y \Rightarrow f_\star \pazocal{O}_X$.
\end{definition}
We wish to define a morphism between locally ringed spaces. To do this we first show how to construct a ring homomorphism between the stalks $\pazocal{O}_{Y,f(P)}$ and $\pazocal{O}_{X,P}$ for each $P\in X$. Indeed, $f^\#$ induces a ring homomorphism 
$$
    \pazocal{O}_{Y,f(P)} = \varinjlim_{U\ni f(P)} \pazocal{O}(U)\rightarrow \varinjlim_{U\ni f(P)} \pazocal{O}_{X}\left(f^{-1}(U)\right) = \varinjlim_{f^{-1}(U)\ni P} \pazocal{O}_X\left(f^{-1}(U)\right)
$$
Composing this with 
$$
    \varinjlim_{f^{-1}(U)\ni P} \pazocal{O}_X\left(f^{-1}(U)\right)\hookrightarrow \varinjlim_{W\ni P} \pazocal{O}_X(W) = \pazocal{O}_{X,P}
$$
We denote this composition by $f^\#_P$.
\begin{definition}
    A morphism of locally ringed spaces is a morphism of ringed spaces $(f,f^\#):(X,\pazocal{O}_X)\rightarrow (Y,\pazocal{O}_Y)$ such that $f^\#_P$ is local ring homomorphism (cf. Definition~\ref{LocalRingDefinition}) for each $P\in X$. 
\end{definition}
\begin{lemma}
    \begin{enumerate}
        \item $\Spec \ R$ is a locally ringed space for each ring $R$.
        \item If $\sigma: A\rightarrow B$ is a map of rings, then it induces a morphism of locally ringed spaces
        \begin{gather*}
            (f,f^\#)_\sigma : (\Spec\ B, \pazocal{O}_{\Spec\ B}) \rightarrow (\Spec\ A, \pazocal{O}_{\Spec\ A})
        \end{gather*}
        \item The functor
        \begin{gather*}
            \Spec : \mathrm{CRing} \rightarrow \mathrm{LRS}\\
            R\mapsto \Spec\ R\\
            \sigma : A\rightarrow B \mapsto (f,f^\#)_\sigma
        \end{gather*}
    \end{enumerate}
    is fully faithful. Here $\mathrm{LRS}$ denotes the category of locally ringed spaces. 
\end{lemma}
\begin{proof}
    
\end{proof}
\begin{definition}
    An \emph{affine scheme} is a locally ringed space $(X,\pazocal{O}_X)$ isomorphic to $(\Spec \ R, \pazocal{O})$ for some ring $R$ as locally ringed spaces. 
\end{definition}
\subsubsection{Schemes}
\begin{definition}
    A \emph{scheme} is a locally ringed space $(X,\pazocal{O}_X)$ where for each $P\in X$ there is an open neighborhood of $P$ such that $(U,\left.\pazocal{O}_X\right|_U)$ is an affine scheme. I.e. there is an open covering of $X$ where each open set in the covering is an affine scheme. 
\end{definition}

