\appendix
\section{First Order Predicate Logic}
The purpose of this section is to introduce a framework for doing mathematics. We will choose first order logic \verb|FOL|.
\subsection{The Metatheory of First Order Logic}
\subsubsection{Classical Metatheory}
We assume a notion of \emph{finiteness}. With this notion consider symbols $\mathbf{0}$ and $\mathbf{s}$, from which we build a \emph{potentially infinite} list $\mathbf{N}$ of \emph{natural numbers} of finite strings formed by appending $\mathbf{s}$ to the left to the prior string in the list, where the list starts with $\mathbf{0}$. In other words the have a list
$$\mathbf{N} = [\mathbf{0},\mathbf{s0},\mathbf{ss0},\dots]$$ 
where $\dots$ signify that to look up the next string in the list, we concatenate from the right $\mathbf{s}$ the last formed string in the list. Each string is either the string $\mathbf{0}$ or $\sigma\mathbf{0}$, where $\sigma$ is some non-empty finite string of $\mathbf{s}$'s. We thus have the following for possibly empty strings of $\mathbf{s}$'s, $\sigma, \pi,\rho$
\begin{gather*}
    \sigma\pi \mathbf{0} = \pi\sigma \mathbf{0},\\
    \mathbf{s}\sigma\pi\mathbf{0} \equiv \mathbf{s}\pi\sigma\mathbf{0},\\
    \mathbf{s}\sigma\pi\mathbf{0}\equiv \sigma \mathbf{s}\pi\mathbf{0},\\
    \sigma\mathbf{0} \equiv \pi\mathbf{0} \metaiff \mathbf{s}\sigma\mathbf{0}\equiv \mathbf{s}\pi\mathbf{0},\\
    \sigma\mathbf{0}\equiv \pi\mathbf{0} \metaiff \sigma\rho\mathbf{0}\equiv \sigma\rho\mathbf{0}, 
\end{gather*}
where "$\equiv$" means "identical to". For two elements in $\mathbf{N}$, $n,m$ we write $n<m$ if $n$ is formed earlier in the list than $m$ and $n>m$ if $n$ is formed later in the list than $m$. By $n+1$ we mean the next element to be formed in the list. And if $n$ is not $\mathbf{0}$ we write $n-1$ to be the string formed Immediately before $n$. We define 
$$\sigma\mathbf{0}+\mathbf{0} :\equiv \sigma\mathbf{0} \text{ and } 0 + \pi\mathbf{0} :\equiv \pi\mathbf{0}$$
and 
$$\sigma\mathbf{0} + \pi\mathbf{0} :\equiv \sigma\pi \mathbf{0}.$$
We also have a principle of induction. I.e. if some statement $P$ holds for $\mathbf{0}$ and whenever $P$ holds for $n$, then it holds for $n+1$, then $P$ holds for all strings in $\mathbf{N}$.\\ 
We also have a principle of recursion. I.e. If we define $X_0$ and whenever $X_n$ is defined, so is $X_{n+1}$, then $X_m$ is defined for every string $m$ in $\mathbf{N}$.\\
Note here that we have implicitly introduced the notion of indexing. Given a finite list of objects, initialize a point on $\mathbf{N}$ at $\mathbf{0}$, pick out an element in the list of objects and assign the value point at in $\mathbf{N}$, move the point to the next element in $\mathbf{N}$ and move pick out the next element in the list of objects. This lets us consider a length $n$ list of object, e.g. $t_1,\dots,t_n$. These informal metamathematical concepts form a sufficient \emph{classical metatheory} for fist order logic. 
\subsubsection{Strong Metatheory}
Sometimes we need to strengthen our metatheory with \emph{the law of excluded middle} and \emph{naive set theory}. The first of these notions is that a statement is either true or false. The last of which is the notion that we can consider a sufficiently small (always finite) collection of objects and ask whether something is a member of such a collection. 
\subsection{The Alphabet of First Order Logic}
\subsubsection*{The Logical Symbols}
\verb|FOL| is a framework for formalizing different theories. It is therefor has to flexible enough to be augmented to work in a large variety of contexts and domains. We therefor make distinctions between the role of the symbols that comprise \verb|FOL|. The \emph{logical symbols} in the language of \verb|FOL| are those symbols that are not context specific. These are 
\begin{enumerate}
    \item \textbf{Variables}: These are play the role of placeholders for objects in the domain to which we want to apply \verb|FOL|. Eg. $x$ might be a placeholder for a set in Zermol-Fraenkel set theory or $n$ might be a placeholder for a natural number in the standard model of Peano arithemetic or $\mathcal{C}$ could be a placeholder for a category in Category Theory. We presuppose a potentially infinite list of such symbols. That is we may always generate sufficiently large number of distinct variables.
    \item \textbf{Logical operators:} $\neg$ (not), $\wedge$ (and), $\vee$ (or), $\to$ (implies).
    \item \textbf{Logical quantifiers:} $\forall$ (universal quantifier, to be read as "for all" or "for every") and $\exists$ (existential quantifier, to be read as "exists" or "there is").
    \item \textbf{Equality symbol:} $=$ a \emph{relational symbol} (scroll down a little for a definition) which is not context specific.    
\end{enumerate}
\subsubsection{The Non-logical Symbols}
Those symbols that are introduced in some context to use \verb|FOL| to talk about some domain. these are 
\begin{enumerate}
    \item \textbf{Constant symbols:} These are symbols that signify a specific object in a specific domain. Examples are $\emptyset$ in for example Zermelo-Fraenkel set theory or $0$ in Peano arithmetic.
    \item \textbf{Function symbols:} A symbol that is a placeholder for an object that depends on finite list of objects in a domain called arguments. For instance if $F$ is a function symbol that takes $n$ arguments, then given variables $x_1,\dots, x_n$, $F(x_1,\dots,x_n)$ is some object. A function symbol has a natural number attached to them called an \emph{arity}. A function symbol that takes $n$ arguments is an $n$-ary function. Examples are $+$ in Peano Arithmetic or $\cup$ in Zermelo-fraenkel set theory. 
    \item \textbf{Relation symbols:} Symbol that signify relations of objects. Like function symbols these have a natural number arity and given a relation symbol $R$ and objects $x_1,\dots, x_n$, $R(x_1,\dots,x_n)$ signifies that $x_1,\dots,x_n$ are in relation by $R$.  
\end{enumerate}
To actually give these syntactical meaning, we need to define rules for building well-formed. I.e. we need to define what strings of symbols we are allowed to write in the language. A set of non-logical symbols is called a \textit{signature} often denoted $\mathcal{L}$. 
\subsubsection{Terms and Formulae}
For a signature $\mathcal{L}$, a string of symbols is an \emph{$\mathcal{L}$-term} if it is a result of finitely applying these rules in some order
\begin{align*}
    \mathrm{T0}\colon\quad & \text{Each variable is an } \mathcal{L}\text{-term}.\\
    \mathrm{T1}\colon\quad & \text{Each constant symbol in }\mathcal{L} \text{ is an } \mathcal{L}\text{-term}.\\
    \mathrm{T2}\colon\quad & \text{If } \tau_1,\dots,\tau_n \text{ are any } \mathcal{L}\text{-terms and }F \text{ is an } n \text{-ary function symbol in } \mathcal{L}\text{,}\\
    &\text{then } F(\tau_1,\dots,\tau_n) \text{ is an } \mathcal{L}\text{-term}. 
\end{align*} 
When the symbols are not involving any non-logical symbols, it is called a \emph{term}. Those terms of the form $\mathrm{T0}$ or $\mathrm{T1}$ are called \emph{atomic terms}. Note that in the above, the $\tau$'s are placeholders for objects in the language \textbf{not} for objects in a domain of interest. If we want to prove a property $\Phi$ of terms, then we do that by proving that each of the three categories of terms satisfy $\Phi$, which is called the \emph{induction on term construction}.\\
An \emph{$\mathcal{L}$-formula} is a string of symbols resulting from finite application of the rules
\begin{align*}
    \mathrm{F0}\colon\quad & \text{For } \mathcal{L}\text{-terms } \tau_1,\tau_2\text{ then } \tau_1 = \tau_2\text{ is an } \mathcal{L}\text{-formula.}\\
    \mathrm{F1}\colon\quad & \text{For } \mathcal{L}\text{-terms } \tau_1,\dots,\tau_n \text{ and } R \text{ a non-logical } n\text{-ary relation symbol in }\mathcal{L},\\
     &\text{then } R(\tau_1,\dots,\tau_n) \text{ is an } \mathcal{L}\text{-formula.}\\
     \mathrm{F2}\colon\quad & \text{if }\varphi \text{ is an }\mathcal{L}\text{-formula, then } \neg\varphi \text{ is an }\mathcal{L}\text{-formula.}\\
     \mathrm{F3}\colon\quad & \text{if }\varphi \text{ and } \psi \text{ are } \mathcal{L}\text{-formulae, then } \varphi\to \psi, \varphi\wedge \psi \text{ and } \varphi\vee \psi \text{ are } \mathcal{L}\text{-formulae.}\\
     \mathrm{F4}\colon\quad &  \text{if }\varphi \text{ is an }\mathcal{L}\text{-formula, then given a variable }x\text{, } \exists x\varphi \text{ and } \forall x\varphi \text{ are } \mathcal{L}\text{-formulae.}  
\end{align*}
When an $\mathcal{L}$-formula is build from terms, it is simply called a \emph{formula}. $\mathrm{F0}$- and $\mathrm{F1}$-formulae are called \emph{atomic formulae}. The \emph{induction on formula construction} refers to the fact that proving a property $\Phi$ is satisfied by formulae, we can prove that it is satisfied for atomic formulae, and that for $\mathcal{L}$-formulae $\varphi,\psi$ satisfying $\Phi$, given a variable $x$, then $\neg\varphi,\varphi \wedge \psi,\varphi\vee\psi,\varphi\to \psi,\exists x\varphi,\forall x\varphi$ also satisfy $\Phi$.\\
Consider a variable $x$ and a formula $\varphi$ of the form $\forall x\psi$ or $\exists x\psi$. If $x$ is used to construct $\psi$ and does not immediately appear after a logical quantifier at some position in $\psi$, it is said to be in the \emph{range of} said logical quantifier. Such a variable is said to \emph{bound at that position} by the last quantifier that it is in the range of. A variable that is not bound by a quantifier at a particular position is said to be \emph{free} at position. A variable may appear in a formula as both bounded and free. For instance $(\forall x(x\neq x))\to \exists z(x=x)$. The first two occurences of $x$ are bound by the $\forall$ and the last two are bound by the $\exists$. The set of free variables of a formula (a rule given later will show that this set is uniquely defined) is denoted $\mathrm{free}(\varphi)$. A formula $\varphi$ is a \emph{sentence} if $\mathrm{free}(\varphi)=\emptyset$, so $x\neq x\to x=x$ is a formula while $\forall x(x\neq x\to x=x)$ is a sentence. 

\subsection{Axioms and Inference Rules}
\subsubsection{Logical Axioms: Assigning Truth Values to Formulae}
\subsubsection{Non-logical Axioms: Defining a Theory}
\subsubsection{Proofs}
\subsubsection{Tautology and Logical Equivalence}

