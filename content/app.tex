\appendix
\section{First Order Predicate Logic}
The purpose of this section is to introduce a framework for doing mathematics. We will choose first order logic \verb|FOL|.
\subsection{The Metatheory of First Order Logic}
\subsubsection{Classical Metatheory}
We assume a notion of \emph{finiteness}. With this notion consider symbols $\mathbf{0}$ and $\mathbf{s}$, from which we build a \emph{potentially infinite} list $\mathbf{N}$ of \emph{natural numbers} of finite strings formed by appending $\mathbf{s}$ to the left to the prior string in the list, where the list starts with $\mathbf{0}$. In other words the have a list
$$\mathbf{N} = [\mathbf{0},\mathbf{s0},\mathbf{ss0},\dots]$$ 
where $\dots$ signify that to look up the next string in the list, we concatenate from the right $\mathbf{s}$ the last formed string in the list. Each string is either the string $\mathbf{0}$ or $\sigma\mathbf{0}$, where $\sigma$ is some non-empty finite string of $\mathbf{s}$'s. We thus have the following for possibly empty strings of $\mathbf{s}$'s, $\sigma, \pi,\rho$
\begin{gather*}
    \sigma\pi \mathbf{0} = \pi\sigma \mathbf{0},\\
    \mathbf{s}\sigma\pi\mathbf{0} \equiv \mathbf{s}\pi\sigma\mathbf{0},\\
    \mathbf{s}\sigma\pi\mathbf{0}\equiv \sigma \mathbf{s}\pi\mathbf{0},\\
    \sigma\mathbf{0} \equiv \pi\mathbf{0} \metaiff \mathbf{s}\sigma\mathbf{0}\equiv \mathbf{s}\pi\mathbf{0},\\
    \sigma\mathbf{0}\equiv \pi\mathbf{0} \metaiff \sigma\rho\mathbf{0}\equiv \sigma\rho\mathbf{0}, 
\end{gather*}
where "$\equiv$" means "identical to". For two elements in $\mathbf{N}$, $n,m$ we write $n<m$ if $n$ is formed earlier in the list than $m$ and $n>m$ if $n$ is formed later in the list than $m$. By $n+1$ we mean the next element to be formed in the list. And if $n$ is not $\mathbf{0}$ we write $n-1$ to be the string formed Immediately before $n$. We define 
$$\sigma\mathbf{0}+\mathbf{0} :\equiv \sigma\mathbf{0} \text{ and } 0 + \pi\mathbf{0} :\equiv \pi\mathbf{0}$$
and 
$$\sigma\mathbf{0} + \pi\mathbf{0} :\equiv \sigma\pi \mathbf{0}.$$
We also have a principle of induction. I.e. if some statement $P$ holds for $\mathbf{0}$ and whenever $P$ holds for $n$, then it holds for $n+1$, then $P$ holds for all strings in $\mathbf{N}$.\\ 
We also have a principle of recursion. I.e. If we define $X_0$ and whenever $X_n$ is defined, so is $X_{n+1}$, then $X_m$ is defined for every string $m$ in $\mathbf{N}$.\\
Note here that we have implicitly introduced the notion of indexing. Given a finite list of objects, initialize a point on $\mathbf{N}$ at $\mathbf{0}$, pick out an element in the list of objects and assign the value point at in $\mathbf{N}$, move the point to the next element in $\mathbf{N}$ and move pick out the next element in the list of objects. This lets us consider a length $n$ list of object, e.g. $t_1,\dots,t_n$. These informal metamathematical concepts form a sufficient \emph{classical metatheory} for fist order logic. 
\subsubsection{Strong Metatheory}
Sometimes we need to strengthen our metatheory with \emph{the law of excluded middle} and \emph{naive set theory}. The first of these notions is that a statement is either true or false. The last of which is the notion that we can consider a sufficiently small (always finite) collection of objects and ask whether something is a member of such a collection. 
\subsection{The Alphabet of First Order Logic}
\subsubsection*{The Logical Symbols}
\verb|FOL| is a framework for formalizing different theories. It is therefor has to flexible enough to be augmented to work in a large variety of contexts and domains. We therefor make distinctions between the role of the symbols that comprise \verb|FOL|. The \emph{logical symbols} in the language of \verb|FOL| are those symbols that are not context specific. These are 
\begin{enumerate}
    \item \textbf{Variables}: These are play the role of placeholders for objects in the domain to which we want to apply \verb|FOL|. Eg. $x$ might be a placeholder for a set in Zermol-Fraenkel set theory or $n$ might be a placeholder for a natural number in the standard model of Peano arithemetic or $\mathcal{C}$ could be a placeholder for a category in Category Theory. We presuppose a potentially infinite list of such symbols. That is we may always generate sufficiently large number of distinct variables.
    \item \textbf{Logical operators:} $\neg$ (not), $\wedge$ (and), $\vee$ (or), $\to$ (implies).
    \item \textbf{Logical quantifiers:} $\forall$ (universal quantifier, to be read as "for all" or "for every") and $\exists$ (existential quantifier, to be read as "exists" or "there is").
    \item \textbf{Equality symbol:} $=$ a \emph{relational symbol} (scroll down a little for a definition) which is not context specific.    
\end{enumerate}
\subsubsection{The Non-logical Symbols}
Those symbols that are introduced in some context to use \verb|FOL| to talk about some domain. these are 
\begin{enumerate}
    \item \textbf{Constant symbols:} These are symbols that signify a specific object in a specific domain. Examples are $\emptyset$ in for example Zermelo-Fraenkel set theory or $0$ in Peano arithmetic.
    \item \textbf{Function symbols:} A symbol that is a placeholder for an object that depends on finite list of objects in a domain called arguments. For instance if $F$ is a function symbol that takes $n$ arguments, then given variables $x_1,\dots, x_n$, $F(x_1,\dots,x_n)$ is some object. A function symbol has a natural number attached to them called an \emph{arity}. A function symbol that takes $n$ arguments is an $n$-ary function. Examples are $+$ in Peano Arithmetic or $\cup$ in Zermelo-fraenkel set theory. 
    \item \textbf{Relation symbols:} Symbol that signify relations of objects. Like function symbols these have a natural number arity and given a relation symbol $R$ and objects $x_1,\dots, x_n$, $R(x_1,\dots,x_n)$ signifies that $x_1,\dots,x_n$ are in relation by $R$.  
\end{enumerate}
To actually give these syntactical meaning, we need to define rules for building well-formed. I.e. we need to define what strings of symbols we are allowed to write in the language. 
\subsubsection{Terms and Formulae}
\subsection{Axioms and Inference Rules}
\subsubsection{Logical Axioms: Assigning Truth Values to Formulae}
\subsubsection{Non-logical Axioms: Defining a Theory}
\subsubsection{Proofs}
\subsubsection{Tautology and Logical Equivalence}

